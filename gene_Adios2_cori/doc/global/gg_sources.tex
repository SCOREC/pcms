\chapter{Sources and Sinks}
This chapter treats the various sources and sinks which are currently implemented in \gene.

\section{Krook operator}

\section{Krook type heat source}

Similar to B.F. McMillan et al., {\em Long global gyrokinetic simulations: source terms and particle noise control}, Phys.~Plasmas {\bf 15} (2008) 052308.

\bea
\frac{\D g_{1\spec}}{\D t} = \mathcal{S}_{\rm KH} 
= - \gamma_{\rm KH} \left[\avg{f_{1\spec}^{\rm symm}}_{\rm FS} - \avg{F_{0\spec}}_{\rm FS} \frac{\avg{\int\D^3v\avg{f_{1\spec}^{\rm symm}}_{\rm FS}}_{\rm FS}}{\avg{\int\D^3v\avg{F_{0\spec}}_{\rm FS}}_{\rm FS}} \right] \label{eq:krookheat_def}
\eea

Here, $f_{1\spec}^{\rm symm} \equiv f_{1\spec}(\mvec{X},\abs{v_\|},\mu) \equiv (f_{1\spec}(\mvec{X},v_\|,\mu)+f_{1\spec}(\mvec{X},-v_\|,\mu))/2$

\section{Krook type particle source}

\begin{itemize}
\item Daniel's suggestion:
\bea
\frac{\D g_\spec}{\D t} = \mathcal{S}^{(1)}_{\rm KP} 
= - \gamma_{\rm KP} \left[\avg{f_{1\spec}^{\rm symm}}_{\rm FS} - \avg{F_{0\spec}}_{\rm FS} \frac{\sum_s q_s \avg{\int\D^3v\avg{f_{1s}^{\rm symm}}_{\rm FS}}_{\rm FS}/n_{\rm spec}}{q_\spec \avg{\int\D^3v\avg{F_{0\spec}}_{\rm FS}}_{\rm FS}} \right] \label{eq:krookpart1_def}
\eea

Heat being injected by this operator is compensated by dynamically adapting the Krook type heat source strength $\gamma_{\rm KH}$ (see above).

\item Minor modification (additional flux surface averages) Ben's suggestion for {\sc Orb}5/{\sc Gene} benchmark:
%Sp_i = -\gamma_p f_0i  \frac{ \Int_P \Sum_j \delta n_j  }{ \Int_P \Sum_j f_0j }

\bea
\frac{\D g_\spec}{\D t} = \mathcal{S}^{(2)}_{\rm KP} 
= - \gamma_{\rm KP} \avg{F_{0\spec}}_{\rm FS} \frac{\sum_s \avg{\int\D^3v \avg{f_{1s}^{\rm symm}}_{\rm FS}}_{\rm FS}}
{\sum_s \avg{\int\D^3v\avg{F_{0s}}_{\rm FS}}_{\rm FS}} \label{eq:krookpart2_def}
\eea

\end{itemize}


\section{Localized heat source}

The following section is closely following Y.~Sarazin et al., {\em Large scale dynamics in flux driven gyrokinetic turbulence}, Nucl.~Fusion {\bf 50} (2010), 054004. However, the normalization, for instance, slightly differs from {\sc Gysela}.\\

The heat source is added to the right hand side of the Vlasov equation as follows:

\bea
\frac{\D g}{\D t} = \mathcal{S}_H = \mathcal{S}_0 \hat{\mathcal{S}}_x \hat{\mathcal{S}}_E \label{eq:heat_src_def}
\eea
with the normalized radial source profile $\hat{\mathcal{S}}_x$, the normalized energy source term $\hat{\mathcal{S}}_E$, and an amplitude $\mathcal{S}_0$ which is given in units of $\frac{n_{0\spec}(\xp)}{v_{T\spec}^3(\xp)}\roL\frac{c_\rf}{L_\rf}$ as can be seen when normalizing Eq.~\ref{eq:heat_src_def}. Using the definition
\bea
E = \frac{1}{2}m_\spec v_\|^2 + \mu B_0 = T_{0\spec}(\xp) \left(\hat{v}_\|^2+\hat{\mu}\hat{B}_0\right) \equiv T_{0\spec}(\xp) \hat{E}
\eea
allows for evaluating the total injected power as follows
\bea
P_{\rm add} & = & \mathcal{S}_0 \int\!\D^3x \int\!\D^3v E \, \hat{\mathcal{S}}_x \hat{\mathcal{S}}_E \nonumber \\
&=& \mathcal{S}_0 \int\delta(\mvec{X}+\mvec{r}-\mvec{x}) E \, \hat{\mathcal{S}}_x \hat{\mathcal{S}}_E \frac{B_{0\|}^*}{m_\spec} \,\D^3X \D\vpar \D\mu \D\theta.
\eea
Since fluctuating quantities are absent, it is possible to identify $\mvec{r}\rightarrow 0$ and thus immediately perform the gyroangle integration which yields
\bea
P_{\rm add} & = & \frac{2\pi}{m_\spec} \mathcal{S}_0  \int\!\D^3x\,\hat{\mathcal{S}}_x \int\D\vpar\D\mu\, E \hat{\mathcal{S}}_E B_{0\|}^*
\eea
or, in the low-$\beta$ limit where $B_{0\|}^*\approx B_0$, 
\bea
P_{\rm add} & = & \frac{2\pi}{m_\spec}\mathcal{S}_0 \int\!\D^3x\,\hat{\mathcal{S}}_x B_0 \int\D\vpar\D\mu\, E \hat{\mathcal{S}}_E. \label{eq:p_low_beta}
\eea

The energy term is set to
\bea
\hat{S}_E = \frac{1}{\mathcal{N}_E} \left(\frac{\hat{E}}{\hat{T}_{p\spec}}-\frac{3}{2}\right) \hat{F}_{0\spec}
\eea
to represent a pure heat but no particle or momentum source as can be seen by computing the according moments
\bea
\int\D\hat{v}_\|\D\hat{\mu}\,\hat{S}_E & = & 0 \nonumber \\
\int\D\hat{v}_\|\D\hat{\mu}\,\hat{v}_\|\hat{S}_E & = & 0 \nonumber.
\eea
The normalization factor $\mathcal{N}_E$ is chosen such that
\bea
\pi \hat{B}_0 \hat{p}_{0\spec}(\xp)\, \int\D\hat{v}_\|\D\hat{\mu}\, \hat{E} \hat{S}_E = 1
\eea
which gives
\bea
\hat{S}_E = \frac{2}{3}\frac{1}{\hat{p}_{0\spec}(x)} \left(\frac{\hat{E}}{\hat{T}_{p\spec}}-\frac{3}{2}\right) \hat{F}_{0\spec}.
\eea
This result can be employed in the normalized version of Eq.~\ref{eq:p_low_beta},
\bea
P_{\rm add} & = & n_\rf T_\rf \rho_\rf^3\frac{c_\rf}{L_\rf}\,  \hat{\mathcal{S}}_0 \int\!\D^3\hat{x}\, \hat{\mathcal{S}}_x(\hat{x}) \hat{J}(\hat{x},\hat{z}) \underbrace{\pi \hat{B}_0\,\hat{p}_{0\spec}(\xp) \int\D\hat{\vpar}\D\hat{\mu}\, \hat{E} \hat{\mathcal{S}}_E}_{=1}.
\eea
In a next step, the radial profile is normalized such that
\bea
\int\!\D^3\hat{x}\, \hat{\mathcal{S}}_x(\hat{x}) \hat{J}(\hat{x},\hat{z}) = 1
\eea
which implies
\bea
\hat{\mathcal{S}}_x(\hat{x}) & = & \mathcal{S}_{x,in}(\hat{x})\,/\int\!\D^3\hat{x}\, \hat{\mathcal{S}}_{x,in}(\hat{x}) \hat{J}(\hat{x},\hat{z}) \nonumber \\
& \sim & \mathcal{S}_{x,in}(\hat{x}) \cdot \left[\frac{\hat{L}_x}{N_{x}} n_0 \hat{L}_y \frac{2\pi}{N_z} \sum_x \sum_z \hat{\mathcal{S}}_{x,in}(\hat{x}) \hat{J}(\hat{x},\hat{z})\right]^{-1}
\eea
Note, that we consider the whole flux surface, i.e. the total box length in the $y$ direction is $n_0 L_y$!
With this choice, the final relation between $P_{\rm add}$ and the source amplitude $\mathcal{S}_0$ evaluates to
\bea
P_{\rm add} & = &\hat{\mathcal{S}}_0 n_\rf T_\rf \rho_\rf^3\frac{c_\rf}{L_\rf}
\eea
which can alternatively be written as
\bea
P_{\rm add} & = &\hat{\mathcal{S}}_0 \cdot 1.6726\cdot 10^{-5} \cdot n_{e19} T_{\rf ,\mathrm{keV}}^3 \frac{m_\rf}{m_p}\frac{1}{B_{\rf,T}^3}\frac{1}{L_{\rf,m}}\, \mathrm{MW}
\eea
with electron density $n_{e19}$ in $10^{19}\mathrm{m}^{-3}$, $T_{\rf ,\mathrm{keV}}$ in units of $\mathrm{keV}$, $B_{\rf,\mathrm{T}}$ in Tesla, and the reference length $L_{\rf,\mathrm{m}}$ in $\mathrm{m}$. \\
\\
For an {\sc ITER}-like deuterium plasma with $n_\rf = 5\cdot 10^{19}\,\mathrm{m}^{-3}$, $T_\rf=5\,\mathrm{keV}$ (taken at mid-minor radius), $m_\rf/m_p=2$, $B_\rf = 5\,\mathrm{T}$ (taken at magnetic axis), and $L_\rf = R_0 = 6.21\,\mathrm{m}$, $\hat{\mathcal{S}}_0$ should be about $3.7\cdot 10^{5}$ to achieve a power injection of about $10\,\mathrm{MW}$.\\
For an {\sc Asdex}-Upgrade deuterium plasma, we have $n_\rf \approx 4\cdot 10^{19}\,\mathrm{m}^{-3}$, $T_\rf\approx 3\,\mathrm{keV}$ (taken at mid-minor radius), $m_\rf/m_p=2$, $B_\rf \approx 2.5\,\mathrm{T}$ (taken at magnetic axis), and $L_\rf = R_0 = 1.65\,\mathrm{m}$, $\hat{\mathcal{S}}_0$ should thus be about $7.1\cdot 10^{4}$ for a $10\,\mathrm{MW}$ power injection.

%%% Local Variables:
%%% mode: latex
%%% TeX-master: "globalgene"
%%% End:
