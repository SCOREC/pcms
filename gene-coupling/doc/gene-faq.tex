\newpage
\section{Frequently Asked Questions}

\begin{description}
\item[\hypertarget{hyp-advice}{What should I set the diffusivities to?}]\hspace*{1em}\\
If a simulation makes use of an adiabatic response, is electrostatic, or has no (or very small) shear,
\texttt{hyp\_z} should be set to the approximate linear growth rate;
otherwise, higher values are required and it is recommended to scale up \texttt{hyp\_z} with increasing resolution.
This is achieved by setting negative values, where
\texttt{hyp\_z}$=-0.5$ is a good starting point (larger absolute value increases the dissipation).
Especially, \texttt{hyp\_z}$=-1$ is equivalent to \texttt{hyp\_z}$=4/(3\Delta z)$,
which mimics 3rd order upwind dissipation, however neglecting the advection prefactor.
In simulations with collisions, these provide a physical velocity space diffusion;
if the collisionality is large enough, \texttt{hyp\_v} may thus be reduced or set to zero.
Otherwise, a small amount ({\tt hyp\_v}$\sim 0.2$) may be necessary.
Perpendicular hyperdiffusion is not needed in fluxtube simulations unless the maximum wave numbers do not
resolve the full driving range and a spectral pile-up may occur. In that case, the \hyperlink{gyroles}{\tt GyroLES}
method is recommended.

\item[How can I reset my forgotten password for the GENE repository?]\hspace*{1em}\\
Please send an email with a corresponding request and your user id to
\href{mailto:support@genecode.org}{support@genecode.org}.

\item[Repeating a nonlinear simulation with identical initial condition]\hspace*{1em}\\
{\em I did this exercise and got different time traces. Is this a bug?}\hspace*{1em}\\
Both, GENE and FFTW, run some internal optimization before actually starting the simulation. Hence, the cache or MPI distribution might not be completely identical but might yield differences on the last digits. However, in a turbulence simulation even tiny deviations in the initial state can have a visible influence after some time. Hence, as long as the time averaged observables are similar there is no need for any concern. For a rigorous test, you need to fix the parallelization, the \texttt{perf\_vec}, the \texttt{nblocks} value and switch off any optimization in the FFT library, e.g., by using MKL instead of FFTW.

\end{description}


%%% Local Variables:
%%% mode: latex
%%% TeX-master: "gene"
%%% End:
