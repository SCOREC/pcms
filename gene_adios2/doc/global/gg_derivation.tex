\chapter{Derivation of the equations for global gyrokinetic
  simulation}
\label{sec:globaleq}

\section{Derivation of the full-$f$ equations}
\label{sec:fullfeq}

If one wants to use a general $\delta f$ splitting without any
assumptions on $F_0$, one can also take the full-$f$ equations and
solve them. I will derive them here starting from the derivation in my
PhD thesis.

In the derivation of the Vlasov equation in my thesis I used the
following assumptions
\begin{eqnarray*}
  \omega\ll\Omega&\quad&\mbox{basic gyrokinetic assumption}\\
  \rho\ll L_B&\quad&\mbox{for trafo to guiding center variables
    needed}\\
  \frac{|A_1|}{\rho B_0}\sim\frac{e\Phi_1}{T_0}\sim\epsilon\ll 1&\quad&\mbox{small perturbations of the fields by the plasma}\\
  B_{1\|}\ll |\mathbf{B}_{1\bot}|&\quad&\mbox{low $\beta$
    assumption}\\
  k_\|\ll k_\bot&\quad&\mbox{extended structures along the field
    lines, flute character}
\end{eqnarray*}

The we arrive at the full-$F$ Vlasov equation
\begin{multline}
\label{eq:full_f_vlasov}
  \frac{\partial F}{\partial t}
  +\left(
    v_\|\mathbf{b}_0
    +\frac{B_0}{B_{0\|}^*}\left(
      \mathbf{v}_{E_\chi}
      +\mathbf{v}_{\nabla B_0}
      +\mathbf{v}_c
    \right)
  \right)\\
  \cdot\left(
    \nabla F
    +\frac{1}{mv_\|}\left(
      -e\nabla\bar{\Phi}_1-\frac{e}{c}\mathbf{b}_0\dot{\bar{A}}_{1\|}-\mu\nabla(B_0+\bar{B}_{1\|})
    \right)\frac{\partial F}{\partial v_\|}
  \right)=0
\end{multline}
To simplify the further calculations, I assume a low $\beta$, so we
can neglect $B_{1\|}$ and the curvature and $\nabla B_0$ drift can be
written together as
\begin{displaymath}
  \mathbf{v}_d = \mathbf{v}_{\nabla B_0}+\mathbf{v}_c
  = \frac{\mu}{m\Omega}\,\mathbf{b}_0\times\nabla B_0
  + \frac{v_\|^2}{\Omega B_0}\,\mathbf{b}_0\times\nabla B_0
  = \frac{\mu B_0+ mv_\|^2}{mB_0\Omega}
  \,\mathbf{b}_0\times\nabla B_0
\end{displaymath}
Then we have the equation
\begin{multline}
\label{eq:full_f_vlasov_lowbeta}
  \frac{\partial F}{\partial t}
  +\left(
    v_\|\mathbf{b}_0
    +\frac{B_0}{B_{0\|}^*}\left(
      \mathbf{v}_{E_\chi}
      +\mathbf{v}_d
    \right)
  \right)\\
  \cdot\left(
    \nabla F
    +\frac{1}{mv_\|}\left(
      -e\nabla\bar{\Phi}_1-\frac{e}{c}\mathbf{b}_0\dot{\bar{A}}_{1\|}-\mu\nabla B_0
    \right)\frac{\partial F}{\partial v_\|}
  \right)=0
\end{multline}
What happens to the prefactor $B_0/B_{0\|}^*$?
\begin{eqnarray*}
  \frac{B_0}{B_{0\|}^*} &=& \frac{B_0}{\mathbf{b}_0\cdot\mathbf{B}_0^*}
  =\frac{B_0}{\mathbf{b}_0\cdot\nabla\times\left(\mathbf{A}_0+\frac{mc}{e} v_\|\mathbf{b}_0\right)}
  =\frac{B_0}{\mathbf{b}_0\cdot\left(\mathbf{B}_0+\frac{mc}{e}
      v_\|\nabla\times\mathbf{b}_0\right)}\\
  &=&\frac{B_0}{\left(\mathbf{b}_0\cdot\mathbf{B}_0+\frac{mc}{e}
      v_\|\mathbf{b}_0\cdot\nabla\times\mathbf{b}_0\right)}
  =\frac{B_0}{\left(B_0+\frac{mc}{eB_0} v_\|
      \mathbf{b}_0\cdot\nabla\times\mathbf{B}_0\right)}\\
  &=&\frac{1}{\left(1+\frac{mc}{eB_0^2} v_\|
      \mathbf{b}_0\cdot\nabla\times\mathbf{B}_0\right)}
  =\frac{1}{\left(1+\beta_e\frac{m_jc}{4\pi e_jn_{e0}T_{e0}} v_\|
      \mathbf{b}_0\cdot\nabla\times\mathbf{B}_0\right)}
  =\frac{1}{1+\beta_e\frac{m_j}{e_jn_{e0}T_{e0}} v_\|j_{0\|}}
\end{eqnarray*}
where we used Amp\`eres law in the last step.
Now we normalize $v_\|=\hat{v}_\|v_{Tj}$ and $j_\|=\hat{j}_\|
en_{\mathrm{e}0}c_s$ and get
\begin{eqnarray*}
  \frac{B_0}{B_{0\|}^*} &=&
  \frac{1}{1+\beta_e\frac{em_j}{e_jT_{e0}} v_{Tj}c_s \hat{v}_\|\hat{j}_{0\|}}
  =\frac{1}{1+\beta_e\frac{e}{e_j} \sqrt{\frac{2T_{j0}m_j}{T_{e0}m_i}} \hat{v}_\|\hat{j}_{0\|}}
\end{eqnarray*}
Up to now the calculation was exact (despite the low-$\beta$
assumption and the neglection of the displacement current). Now we
estimate the term in the denominator for by setting the  parallel current
to the electron current (how to estimate the equilibrium parallel
current, which makes the equilibrium magnetic field?) and the velocity
to the thermal velocity of the species. Then we have
\begin{eqnarray*}
  \frac{B_0}{B_{0\|}^*} &=& \frac{1}{1-\frac{4\pi m_j}{e_jB_0^2}
    v_{Tj} en_{\mathrm{e}0} v_{T\mathrm{e}}}
  =\frac{1}{1-\frac{\beta_\mathrm{e} m_j e}{e_jT_{\mathrm{e}0}}
    \sqrt{\frac{4T_{j0}T_{\mathrm{e}0}}{m_jm_\mathrm{e}}}}
  =\frac{1}{1-2\beta_\mathrm{e} \frac{e}{e_j}
    \sqrt{\frac{T_{j0}m_j}{T_{\mathrm{e}0} m_\mathrm{e}}}}
\end{eqnarray*}
For electrons one has $B_0/B_{0\|}^*\approx 1/(1+2\beta_\mathrm{e})$,
for ions 
\begin{displaymath}
  \frac{B_0}{B_{0\|}^*} =\frac{1}{1-2\beta_\mathrm{e} \frac{e}{e_\mathrm{i}}
    \sqrt{\frac{T_{\mathrm{i}0}m_\mathrm{i}}{T_{\mathrm{e}0} m_\mathrm{e}}}}
\end{displaymath}
As long as $\beta_e$ is around $0.1\%$ this term can be approximated
safely by 1, but if $\beta_e$ goes to around $1\%$, at least for the
ions, this approximation is not good, and one has to take the correct
value, which can be calculated from a given magnetic field by
calculating $\nabla\times\mathbf{B}_0$. So we keep this factor in our
calculations.

Some reordering leads to 
\begin{multline*}
  \frac{\partial F}{\partial t}
  -\frac{e}{mc}\dot{\bar{A}}_{1\|}\frac{\partial F}{\partial v_\|}
  +v_\|\nabla_\| F
  +\frac{B_0}{B_{0\|}^*}\left(
    \mathbf{v}_{E_\chi}
    +\mathbf{v}_d
  \right)\cdot\nperp F
  -\frac{1}{m}\frac{\partial F}{\partial v_\|}\left(
    e\nabla_\|\bar{\Phi}_1
    +\mu\nabla_\| B_0
  \right)\\
  -\frac{1}{mv_\|}\frac{\partial F}{\partial v_\|}
  \frac{B_0}{B_{0\|}^*}\left(
    e\left(
      \mathbf{v}_{E_\chi}
      +\mathbf{v}_d
    \right)\cdot\nabla\bar{\Phi}_1
    +\mu\mathbf{v}_{E_\chi}\cdot\nabla B_0
  \right)
  =0
\end{multline*}
\begin{multline*}
  \frac{\partial F}{\partial t}
  -\frac{e}{mc}\dot{\bar{A}}_{1\|}\frac{\partial F}{\partial v_\|}
  +v_\|\nabla_\| F
  +\frac{B_0}{B_{0\|}^*}\left(
    \mathbf{v}_{E_\chi}
    +\mathbf{v}_d
  \right)\cdot\nperp F
  -\frac{1}{m}\frac{\partial F}{\partial v_\|}\left(
    e\nabla_\|\bar{\Phi}_1
    +\mu\nabla_\| B_0
  \right)\\
  -\frac{1}{mv_\|}\frac{\partial F}{\partial v_\|}
  \frac{B_0}{B_{0\|}^*}\left(
    e\left(
      \frac{v_\|}{B_0}\nabla\bar{A}_{1\|}\times\mathbf{b}_0
      +\mathbf{v}_d
    \right)\cdot\nabla\bar{\Phi}_1
    +\mu\mathbf{v}_{E_\chi}\cdot\nabla B_0
  \right)
  =0
\end{multline*}

\section{$\delta F$ splitting for the Vlasov equation}
\label{sec:vlasoveq}

We decided to stick to the usual $\delta F$ splitting with the
assumption $F_1\ll F_0$ but to take $F_0$ general, so not necessarily
a Maxwellian. This leads to a derivation which is very similar to what
I did for the beam ions. There I found (see the fast ion paper):
\begin{displaymath}
  \frac{\partial F}{\partial t}
  +\left(
    v_\|\mathbf{b}_0
    +\mathbf{v}_{E_\chi}
    +\mathbf{v}_{\nabla B_0}
    +\mathbf{v}_c
  \right)
  \cdot\left(
    \nabla F
    +\frac{1}{mv_\|}\left(
      -e\nabla\bar{\Phi}_1
      -\frac{e}{c}\dot{\bar{A}}_{\|1}\mathbf{b}_0
      -\mu\nabla B_0
    \right)\frac{\partial F}{\partial v_\|}
  \right)=0
\end{displaymath}
with the abbreviations for the drift velocities
\begin{displaymath}
  \mathbf{v}_{E_\chi}=\frac{c}{B_0}\frac{-\nabla
    \chi_1\times\mathbf{B}_0}{B_0}
  \qquad \mathbf{v}_{\nabla B_0} = \frac{\mu
    c}{eB_0}\,\mathbf{b}_0\times\nabla B_0
  \qquad \mathbf{v}_c = \frac{v_\|^2}{\Omega}(\nabla\times\mathbf{b}_0)_\bot.
\end{displaymath}
The overbars indicate a gyoraveraged quantity $\bar{\Phi}_1 =
J_0(\lambda)\Phi_1$ and analogously for $\bar{A}_{\|1}$.
For sake of simplicity we introduced the generalized potential
$\chi_1=\bar{\Phi}_1-\frac{v_\|}{c}\bar{A}_{\|1}$. 

In a next step we introduce the $\delta F$ splitting by writing
$F=F_0+F_1$ and require $F_1\ll F_1$. By taking the time derivative of
$F_0$ to be small (or zero) we can write the total equation which is
to solve (now no splitting into two equations is done, because of the
fact, that with a general $F_0$ the zeroth order equation is not
fulfilled in general). So we only drop all higher order terms (of the
order of $\epsilon^2\omega F_0$). This leads to the equation
\begin{multline*}
  \frac{\partial F}{\partial t}
  +\left(
    v_\|\mathbf{b}_0
    +\mathbf{v}_{E_\chi}
    +\mathbf{v}_{\nabla B_0}
    +\mathbf{v}_c
  \right)\cdot\nabla F\\
  +\frac{1}{mv_\|}\left(
    v_\|\mathbf{b}_0
    +\mathbf{v}_{E_\chi}
    +\mathbf{v}_{\nabla B_0}
    +\mathbf{v}_c
  \right)\cdot\left(
    -e\nabla\bar{\Phi}_1
    -\frac{e}{c}\dot{\bar{A}}_{\|1}\mathbf{b}_0
    -\mu\nabla B_0
  \right)\frac{\partial F}{\partial v_\|}
  =0
\end{multline*}
Estimating the drift velocities leads to
\begin{eqnarray*}
  v_{E_\chi}&=&\frac{c}{B_0}\mathbf{b}_0\times\nabla\chi
  \sim\frac{e_j}{m_j\Omega_j}k_\bot\rho_{Tj}\frac{\chi}{\rho_{Tj}}
  \sim\frac{e_j}{m_jv_{Tj}^2}v_{Tj}k_\bot\rho_{Tj}\chi
  \sim v_{Tj}k_\bot\rho_{Tj}\frac{e_j\chi}{T_j}\\
  v_d&=&v_c+v_{\nabla B_0}=\left(
    \frac{v_\|^2}{\Omega_jB_0}
    +\frac{\mu c}{e_jB_0}
  \right)\mathbf{b}_0\times\nabla B_0
  \sim\left(
    \frac{v_{Tj}^2}{\Omega_j B_0}
    +\frac{v_{Tj}^2}{\Omega_jB_0}
  \right)\frac{B_0}{L_B}
  \sim\frac{\rho_{Tj}v_{Tj}}{L_B}\sim\epsilon v_{Tj}
\end{eqnarray*}
Now estimating all terms leads to
\begin{eqnarray*}
  \frac{\partial F}{\partial t}&\sim&\omega F\\
  v_\|\mathbf{b}_0\cdot\nabla F&\sim&\frac{v_{Tj}}{L_\|}F\\
  \mathbf{v}_{E_\chi}\cdot\nabla F
  &\sim&v_{Tj}k_\bot\rho_{Tj}\frac{e_j\chi}{T_j}\nabla F\\
  \mathbf{v}_d\cdot\nabla F
  &\sim&\epsilon v_{Tj}\left(
    \frac{\rho_j}{L_\bot}\frac{F_0}{\rho_j}
    +\frac{F_1}{\rho_j}
  \right)
  \sim\omega\left(
    \epsilon F_0
    +F_1
  \right)\\
                                %
  -e\frac{1}{m}\mathbf{b}_0\cdot\nabla\bar{\Phi}_1
  \frac{\partial F}{\partial v_\|}
  &\sim&-\frac{e_j}{m_j}\frac{\bar{\Phi}_1}{L_\|}
  \frac{F}{v_{Tj}}
  \sim -\frac{v_{Tj}}{L_\|}\frac{e_j\bar{\Phi}_1}{T_j} F\\
                                %
  -\frac{e}{mc}\dot{\bar{A}}_{\|1}\frac{\partial F}{\partial v_\|}
  &\sim&-\Omega_j\omega\frac{\bar{A}_{\|1}}{B_0}\frac{F}{v_{Tj}}
  \sim -\epsilon\omega F\\
                                %
  -\frac{\mu}{m}\mathbf{b}_0\cdot\nabla B_0
  \frac{\partial F}{\partial v_\|}
  &\sim& -\frac{v_{Tj}}{L_\|} F\\
                                %
  -\frac{e}{mv_\|}\mathbf{v}_{E_\chi}\cdot\nabla\bar{\Phi}_1
  \frac{\partial F}{\partial v_\|}
  &=&-\frac{ec}{mv_\|B_0}
  \mathbf{b}_0\times\nabla\left(
    \bar\Phi_1
    -\frac{v_\|}{c}\bar{A}_{\|1}
  \right)\cdot\nabla\bar{\Phi}_1
  \frac{\partial F}{\partial v_\|}
  =\frac{e}{mB_0}
  \mathbf{b}_0\times\nabla\bar{A}_{\|1}
  \cdot\nabla\bar{\Phi}_1
  \frac{\partial F}{\partial v_\|}\\
  &\sim&\frac{e_j}{m_jB_0}k_\bot\bar{A}_{\|1}
  k_\bot\bar{\Phi}_1\frac{F}{v_{Tj}}
  \sim\frac{v_{Tj}^2}{\rho_j}k_\bot^2\rho_j^2\frac{\bar{A}_{\|1}}{\rho_jB_0}
  \frac{e_j\bar{\Phi}_1}{T_j}\frac{F}{v_{Tj}}
  \sim\epsilon\Omega_j \frac{\bar{A}_{\|1}}{\rho_jB_0} F
  \sim\epsilon\omega F\\
                                %
  -\frac{e}{mc\,v_\|}\mathbf{v}_{E_\chi}\cdot\mathbf{b}_0
  \dot{\bar{A}}_{\|1}\frac{\partial F}{\partial v_\|}
  &=&0\quad\mbox{geometrical reasons}\\
  -\mu\frac{1}{mv_\|}\mathbf{v}_{E_\chi}
  \cdot\nabla B_0\frac{\partial F}{\partial v_\|}
  &\sim&-k_\bot\frac{e_j\chi}{T_j}
  \frac{\rho_{Tj}}{L_B}\frac{F}{v_{Tj}}\\
                                %
  -e\frac{1}{mv_\|}\mathbf{v}_d\cdot\nabla\bar{\Phi}_1
  \frac{\partial F}{\partial v_\|}
  &\sim& -\frac{e_j}{m_jv_{Tj}}\epsilon v_{Tj} k_\bot\bar{\Phi}_1
  \frac{F}{v_{Tj}}\\
                                %
  -\frac{e}{mc\,v_\|}\mathbf{v}_d\cdot\mathbf{b}_0
  \dot{\bar{A}}_{\|1}\frac{\partial F}{\partial v_\|}
  &=&0\quad\mbox{geometrical reasons}\\
  -\frac{\mu}{mv_\|}\mathbf{v}_d\cdot\nabla B_0\frac{\partial F}{\partial v_\|}&\sim&
\end{eqnarray*}

Taking all terms, except the time derivative of $F_0$ we arrive at the
general Vlasov equation, we want to solve.
\begin{multline*}
  \frac{\partial F_1}{\partial t}
  -\frac{e}{mc}\frac{\partial\bar{A}_{\|1}}{\partial t}
  \frac{\partial F_0}{\partial v_\|}
  +\left(
    v_\|\mathbf{b}_0
    +\mathbf{v}_{E_\chi}
    +\mathbf{v}_d
  \right)\cdot\nabla F\\
  +\frac{1}{mv_\|}\left(
    v_\|\mathbf{b}_0
    +\mathbf{v}_{E_\chi}
    +\mathbf{v}_d
  \right)\cdot\left(
    -e\nabla\bar{\Phi}_1
    -\mu\nabla B_0
  \right)\frac{\partial F}{\partial v_\|}
  =0
\end{multline*}
where we used the low-$\beta$ assumption to rewrite
the curvature drift as $\mathbf{v}_c=v_\|^2/(\Omega B_0)
\mathbf{b}_0\times\nabla B_0$. Then we introduced the combined $\nabla
B_0$ and curvature drift to the perpendicular drift $\mathbf{v}_d$
which is also perpendicular to $\nabla B_0$. 

Furthermore we introduce now the modified distribution function
$g=F_1-\frac{e}{mc}\frac{\partial F_0}{\partial v_\|} \bar{A}_{\|1}$
and get 
\begin{displaymath}
  \frac{\partial g}{\partial t}
  +\left(
    v_\|\mathbf{b}_0
    +\mathbf{v}_{E_\chi}
    +\mathbf{v}_d
  \right)\cdot\nabla F
  +\frac{1}{mv_\|}\left(
    v_\|\mathbf{b}_0
    +\mathbf{v}_{E_\chi}
    +\mathbf{v}_d
  \right)\cdot\left(
    -e\nabla\bar{\Phi}_1
    -\mu\nabla B_0
  \right)\frac{\partial F}{\partial v_\|}
  =0
\end{displaymath}
Now I continue by rewriting the last equation with explicitly giving
the dependencies of all quantities (especially the radial
dependencies are of interest in a nonlocal code). First all
abbreviations are written out.
\begin{multline*}
  \frac{\partial g_j}{\partial t}
  +\left(
    v_\|\mathbf{b}_0
    +\frac{c}{B_0}\mathbf{b}_0\times\nabla\chi_j
    +\left(
      \frac{\mu c}{e_jB_0}
      +\frac{v_\|^2}{\Omega_j B_0}
    \right)\mathbf{b}_0\times\nabla B_0
  \right)\\
  \cdot\left(
    \nabla F_j
    -\frac{1}{m_jv_\|}\frac{\partial F_j}{\partial v_\|}\left(
      e_j\nabla\bar{\Phi}_1
      +\mu\nabla B_0
    \right)
  \right)
  =0
\end{multline*}
This is the general Vlasov equation in coordinate independent vector
form. Now it is to determine the coordinate system and to rewrite the
last equation in coordinate form, which is the only possible form for
numerical simulation. 


\section{Gyrokinetic equation in field aligned coordinates}


\label{sec:gk_field_al_coord}

\subsection{The field aligned coordinates system}
We start from the straight field line coordinate system $(\Psi, \chi, \phi)$, where $\Psi$ is the poloidal flux function, $\chi$ is the straight field line poloidal angle, and $\phi$ is the toroidal angle. Using these quantities one has
$$\mb{B}_0=\nabla \Psi \times \nabla (q \chi-\Phi).$$
We can therefore define the field aligned coordinate system $(x,y,z)$ by the following transformation,
$$x=C_x(\Psi) - x_0 \hspace{1cm} \mbox{and}  \hspace{1cm}\; y=C_y(\Psi) (q \chi-\Phi)-y_0.$$
Using this definition one obtains $\nabla x \times \nabla y=\frac{d C_x}{d \Psi} C_y \nabla \Psi \times \nabla (q \chi-\Phi)=\frac{d C_x}{d \Psi} C_y  \mb{B}_0 $,  namely $x$ and $y$ are directions perpendicular  to the magnetic field.\\

\subsection{The gyrokinetic equation} 
We have :
\begin{eqnarray}
\vE \cdot \nabla &=& \frac{1}{B_0 ^2} (\mb{B}_0 \times \nabla \chi _1) \cdot \nabla \\
\vgB \cdot \nabla &=& \frac{\mu}{m \Omega B_0}(\mb{B}_0 \times \nabla B_0) \cdot \nabla \\
\vc \cdot \nabla &=& \frac{\mu_0 \vpar ^2}{\Omega B_0^2}\mb{b}_0 \times \nabla \left (p+\frac{B_0^2}{2 \mu_0} \right) \cdot \nabla \\
\b _0 \cdot \nabla &=& \frac{\mb{B}_0}{B_0} \cdot \nabla \; \; \; \; ; \; \; \tilde{\b} \cdot \nabla = \b _0 \cdot \nabla  - \frac{\mb{B} _0 \times \nabla A_{1 \parallel}}{B_0 B_{0 \parallel} ^*} \cdot \nabla 
\end{eqnarray}
Thus we need to compute the following quantities
\[
\mb{B}_0 \times \nabla \Phi \cdot \nabla \; \; \; ; \; \; \; \mb{B}_0 \cdot \nabla \; \; \mbox{and} \; \; \nabla _\perp ^2 \Phi
\]
where $\Phi$ is any scalar. Introducing $\C=1/(\frac{d f}{d \Psi} C_y)$, $\mb{B}_0=\C(x) \nabla x \times \nabla y$, we get
\begin{eqnarray*}
\mb{B}_0 \times \nabla \Phi \cdot \nabla &=& \C (\nabla x \times \nabla y) \times \nabla \Phi \cdot \nabla \\
&=&\C \left [ (\nabla \Phi \cdot \nabla x) \nabla y - ( \nabla \Phi \cdot \nabla y ) \nabla x \right ] \cdot \nabla\\
&=&\C \left [ (\d _i \Phi \nabla u^i \cdot \nabla x ) \nabla y - (\d _i \Phi \nabla u^i \cdot \nabla y ) \nabla x \right ] \cdot \nabla \\
&=&\C \left [ g^{1i}\nabla y - g^{i2}\nabla x \right ] \d _i \Phi \cdot \nabla u^j \d _j \\
&=&\C ( g^{1i} g^{2j}-g^{2i}g^{1j})\d _i \Phi \d_j
\end{eqnarray*}
with $u^i=(x,y,z)$. 
\begin{eqnarray*}
\mb{B}_0 \cdot \nabla &=& \C (\nabla x \times \nabla y) \cdot \nabla u^j \d _i \\
                      &=&\C  (\nabla x \times \nabla y) \cdot \nabla z \d _z \\
                      &=& \frac{\C}{J} \d _z
\end{eqnarray*}
and 
\begin{eqnarray*}
\nabla _\perp ^2 = (\nabla x \d _x + \nabla y \d _y)^2 = g^{11}\d _x ^2 + g^{22} \d _y ^2 + 2 g^{12} \d ^2 _{x y} 
\end{eqnarray*}

Using the ordering $k_\parallel \ll k_\perp$, terms $\frac{\d}{\d z}$ of perturbed quantities are neglected with respect to derivatives in the perpendicular plain. In addition, from the axisymmetry of the equilibria, all equilibrium quantities are independent of $y$.
Introducing the notation 
\begin{eqnarray*}
\gamma _1 &=& g^{11} g^{22}-(g^{21})^2 ,\\
\gamma _2 &=& g^{11} g^{23}-g^{21}g^{13} ,\\
\gamma _3 &=& g^{12} g^{23}-g^{22}g^{13},
\end{eqnarray*}
the different term of the vlasov equation can be written
\begin{eqnarray*}
\vE \cdot \nabla f_0 &=& \frac{\C}{B_0^2} \left ( \gamma _2 \dz f_0 \dx \chi _1
-\gamma _1 \dx f_0 \dy \chi _1 
+ \gamma _3 \dz f_0 \dy \chi _1 \right ) ,\\
%
\vE \cdot \nabla B_0 &=& \frac{\C}{B_0^2} \left ( \gamma _2 \dz B_0 \dx \chi _1
-\gamma _1 \dx B_0 \dy \chi _1 
+ \gamma _3 \dz B_0 \dy \chi _1 \right ) ,\\
%
\vE \cdot \G &=& \frac{\C}{B_0 ^2}(\gamma _1 \dx \chi_1 \G _y - \gamma _1 \dy \chi _1 \G _x),\\
%
\vgB \cdot \G &=& \frac{\C \mu}{m \Omega B_0} \left (  \gamma_1 \dx B_0 \G _y 
- \gamma _3 \dz B_0 \G _x - \gamma_2 \dz B_0 \G _y \right ) ,\\
%
\vc \cdot \G &=& \frac{\C \vpar ^2}{\Omega B_0^2}\left ( \gamma_1 \dx B_0 \G _y 
- \gamma _3 \dz B_0 \G _x - \gamma_2 \dz B_0 \G _y \right )
+ \frac{\mu_0 \vpar ^2 \C}{\Omega B_0^3} \gamma _1 \dx p \G _y  ,\\
%
\b _0 \cdot \G &=& \frac{\C}{B_0 J}\G _z, \\
%
\tilde{\b} \cdot \nabla B_0 &=& \frac{\C}{B_0 J} \dz B_0 + \mbox{neglected term},\\
%
\vc \cdot \nabla B_0 &=& 0 + \mathcal{O}(\rhor/L_\perp).
\end{eqnarray*}
where $\G_j=\d_j f_1-e_j /(m_j \vpar)\d _j \bar{\Phi}_1 \d f_0/\d \vpar$ for $j=(x,y,z)$.\\
The gyrokinetic vlasov equation becomes 
\begin{eqnarray*}
- \d _t g &=& \frac{\C}{\Bs B_0} \left [  \gamma _2 \dz f_0 \dx \chi _1
-\gamma _2 \dx f_0 \dy \chi _1 
+ \gamma _3 \dz f_0 \dy \chi _1  \right . \\
&-& \left . \frac{\C\mu}{m \vpar} (\gamma _2 \dz B_0 \dx \chi _1
-\gamma _2 \dx B_0 \dy \chi _1 
+ \gamma _3 \dz B_0 \dy \chi _1)  \right ] \\
%
&+& \frac{\C}{\Bs B_0}  (\gamma _1 \dx \chi_1 \G _y - \gamma _1 \dy \chi _1 \G _x) \\
%
&+& \C\frac{\mu B_0+m\vpar ^2}{m \Omega \Bs B_0 }\left (\Kx \G _x + \Ky \G _x\right) + \frac{\C \mu_0 \vpar ^2 }{\Omega B_0^2 \Bs}  \gamma _1  \dx p \G _y\\
%
&+& \frac{\C\vpar}{B_0 J}\G _z - \frac{\C\mu}{m B_0 J} \dz B_0 \frac{\d f_1}{\d \vpar} + \frac{df_0}{dt}
\end{eqnarray*}
where 
\begin{eqnarray*}
\Kx &=& - \gamma _3 \dz B_0,  \\
\Ky &=& \gamma_1 \dx B_0 - \gamma_2 \dz B_0
\end{eqnarray*}


\subsection{Choice of $f_0$}
Using a $f_0$ function of the form :
\begin{equation}
f_{0j}=\left ( \frac{m_j}{2\pi T_{0j}} \right )^\frac{3}{2} n_{0j} e^{-\frac{m\vpar^2/2+\mu B_0}{T_{0j}}}
\end{equation}
where $T_j$ and $n_j$ are function of $x$ and j stands for the different species,
\begin{eqnarray*}
\dx f_{0j} &=& \left (\frac{1}{n_0}\dx n_{0j} +(\frac{m_j \vpar^2}{2 T_{0j}}+\frac{\mu B_0}{T_{0j}}-\frac{3}{2})\frac{1}{T_{0j}}\dx T_{0j}- \frac{\mu}{T_{0j}} \dx B_0 \right ) f_{0j} \\
\dz f_{0j} &=& -\frac{\mu}{T_j}\dz B_0 f_{0j}\\
\d _{\vpar} f_{0j} &=& -\frac{m_j \vpar}{T_{0j}} f_{0j} \\
\d _\mu f_{0j}&=& -\frac{B_0}{T_{0j}} f_{0j}
\end{eqnarray*}

\subsection{Normalization}
\textit{This part has to be change for the general case where x and y doesn't have dimension of length}
\label{sec:normalization}

In this section, we will introduce the normalization. This is
presented before the derivation of the equations, to have the proper
formulas present when we need them later on.

The idea behind the normalization is to have in the end quantities of
roughly the same magnitude in the equations. Furthermore all
quantities should be dimensionless. Therefore we introduce first
dimensional quantities which are drawn out of the quantities in the
equations. The dimensional quantities are taken at some reference
point and are therefore neither space dependent nor species dependent.
We use $n_\rf, T_\rf, B_\rf,
m_\rf$ and $L_\rf$ as the base references. Derived
from these quantities, we can define
\begin{displaymath}
  c_\rf^2=\frac{T_\rf}{m_\rf}
  \qquad \Omega_\rf=\frac{eB_\rf}{m_\rf c}
  \qquad \rho_\rf=\frac{c_\rf}{\Omega_\rf}
\end{displaymath}

The normalization of the independent quantities, i.e. the coordinates
of our system are then
\begin{displaymath}
  t=\frac{L_\rf}{c_\rf}\hat{t}
  \qquad x=\hat{x}\rho_\rf
  \qquad y=\hat{y}\rho_\rf
  \qquad z=\hat{z}
  \qquad v_\|=v_{0j}\hat{v}_\|
  \qquad \mu = \frac{\refj{T}}{B_\rf}\hat{\mu}
\end{displaymath}
where we introduced the quantity
$v_{0j}=\sqrt{2\frac{\refj{T}}{m_j}}$. The advantage
of this normalization is that it is species dependent. As the
velocities of the species can be very different, it is a good idea to
use a normalization to the thermal velocities of the species. But the
temperature of a species is radial dependent, so we use
$\refj{T}$ which represents the temperature per species $j$ at a 
fixed point in the profile in the normalizing velocity to be spatial
independent (this choice is similar to the local version). The same
argument applies for the density where we introduce $\refj{n}$ as
reference value per species $j$.

For the dependent quantities we can then define the following
normalizations
\begin{displaymath}
  g_j=\frac{\rho_\rf}{L_\rf}\frac{\refj{n}}{v_{0j}^3}\hat{g}_j
  \qquad F_{0j}=\frac{\refj{n}}{v_{0j}^3}\hat{F}_{0j}
  \qquad \Phi=\frac{\rho_\rf}{L_\rf}\frac{T_\rf}{e}\hat{\Phi}
  \qquad A_\|=\frac{\rho_\rf}{L_\rf}B_\rf\rho_\rf\hat{A}_\|.
\end{displaymath}

The normalized temperature and density profiles are abbreviated by
\begin{displaymath}
  \Tpj = \frac{T_{j0}(x^1)}{\refj{T}}  
  \qquad \npj = \frac{n_{j0}(x^1)}{\refj{n}}.
\end{displaymath}
We define also 
\begin{displaymath}
  \omT = L_\rf\frac{d \ln T_{j0}}{dx}\, ,\qquad
  \omn = L_\rf\frac{d \ln n_{j0}}{dx}.
\end{displaymath}
The pressure term is normalized to $p=\pr \h{p}=\Tr \nr \h{p}$. \\
The coefficient $\C$ is normalized to $\hat{\C}=\C/\Br$. Note in the local version of the code, x and y are defined such that $\hat{\C}=1$.



\subsection{The normalized gyrokinetic equation}
Using the relation $B^2=\C^2 \gamma _1$ the normalized gyrokinetic equation reads

% \begin{eqnarray}
% \nonumber \frac{\d g_j}{\d t} &+& \frac{1}{\C}\left (\omn 
% + \omT (  \vpar ^2 +   \mu   B - 3/2 ) \right )   f _0 \frac{\d   \chi}{\d   y} 
% +  \frac{1}{\C}\{ \chi _1,   g _1\} \\ \nonumber
% &+& \frac{1}{B} \frac{  \mu   B + 2   \vpar ^2}{\sigma _j} \left (  K _x \G _x +   K_y \G _y \right )\\ \nonumber
% &+& \frac{\vpar ^2 \beta}{\sigma_j B^2 \C} \frac{d p}{d x} \G _y
% + \alpha _j \frac{\C  \vpar}{  J   B}\G _z \\ \nonumber
% &-& \frac{\C  \mu \alpha _j}{2   J   B}\frac{\d   f_1}{\d   \vpar} \frac{\d   B}{\d   z} \\
% &+& \left [\omn - \omT ( \vpar ^2 +  \mu  B - 3/2 ) \right ] \frac{1}{B} \frac{ \mu  B + 2  \vpar ^2}{\sigma _j} K _x  f _0 = 0
% \end{eqnarray} 
% where :
% \begin{eqnarray*}
% K_x &=&  -  \frac{1}{\C} \frac{\gamma _2}{\gamma _1} \frac{\d B}{\d z}, \\
% K_y &=&   \frac{1}{\C} \left [ \frac{\d B}{\d x} -  
% \frac{\gamma _3}{\gamma _1}\frac{\d B}{\d z} \right ].
% \end{eqnarray*}

\bea
\pderiv{\hat{g}_j}{\hat{t}} & = &
%
%pdchibardy
- \left\{\frac{1}{\hat{\cofac}}\frac{\hat{B}_0}{\hat{B}_{0\|}^*}
\left[\omega_{nj}+\omega_{Tj}\left(\frac{\hat{v}_\|^2
+\hat{\mu}\hat{B}_0}{\hat{T}_{0j}/\hat{T}_{0j}(x_0)}-\frac{3}{2}\right)\right]\hat{F}_{0j} \right. \nn \\
&&\left. +\frac{\hat{B}_0}{\hat{B}_{0\|}^*}\frac{\hat{T}_{0j}(x_0)}{\hat{T}_{0j}}\frac{\hat{\mu} \hat{B}_0 + 2 \hat{v}_\|^2}{\hat{B}_0}K_y\hat{F}_{0j}
+\frac{\hat{B}_0}{\hat{B}_{0\|}^*} \frac{{T}_{0j}(x_0)}{\hat{T}_{0j}}\frac{\hat{v}_\|^2}{\hat{\cofac}}\beta_\rf
\frac{\hat{p_0}}{\hat{B}_0^2}\omega_{p}\hat{F}_{0j}\right\}
\partial_{\hat{y}}\hat{\chi}_1 \nn \\
%
%pdchibardx
&&-\frac{\hat{B}_0}{\hat{B}_{0\|}^*}\frac{\hat{T}_{0j}(x_0)}{\hat{T}_{0j}}\frac{\hat{\mu} \hat{B}_0 + 2 \hat{v}_\|^2}{\hat{B}_0}
\hat{F}_{0j}K_x\partial_{\hat{x}}\hat{\chi}_1 \nn \\
%
%pdgdx
&&-\frac{\hat{B}_0}{\hat{B}_{0\|}^*}\frac{\hat{T}_{0j}(x_0)}{\hat{q}_j}\frac{\hat{\mu} \hat{B}_0 + 2 \hat{v}_\|^2}{\hat{B}_0}
K_x\partial_{\hat{x}} \hat{g}_{1j} \nn \\
%
%pdgdy
&&-\left\{\frac{\hat{B}_0}{\hat{B}_{0\|}^*}\frac{\hat{T}_{0j}(x_0)}{\hat{q}_j}\frac{\hat{\mu} \hat{B}_0 + 2 \hat{v}_\|^2}{\hat{B}_0}
K_y +\frac{\hat{B}_0}{\hat{B}_{0\|}^*} \frac{{T}_{0j}(x_0)}{\hat{q}_j}\frac{\hat{v}_\|^2}{\hat{\cofac}}\beta_\rf
\frac{\hat{p_0}}{\hat{B}_0^2}\omega_{p}\right\}\partial_{\hat{y}} \hat{g}_{1j} \nn \\
%
%pdfdz
&&-\hat{v}_{Tj}(x_0)\frac{\hat{\cofac}}{J\hat{B}_0}\hat{v}_\|\left( 
\partial_{\hat{z}}\hat{F}_{1j}
%pdphidz
+\frac{\hat{q}_j}{\hat{T}_{0j}}\hat{F}_{0j} \partial_{\hat{z}}\hat{\bar{\phi}}_1
+\frac{\hat{T}_{0j}(x_0)}{\hat{T}_{0j}}\hat{F}_{0j}\hat{\mu} \partial_{\hat{z}}\hat{\bar{B}}_{1\|}\right)\nn \\
%
%pnl (nonlinearity)
&&+\frac{\hat{B}_0}{\hat{B}_{0\|}^*}\frac{1}{\hat{\cofac}}\left(-
\partial_{\hat{x}}\hat{\chi}_1\partial_{\hat{y}} \hat{g}_{1j}+\partial_{\hat{y}}\hat{\chi}_1\partial_{\hat{x}} \hat{g}_{1j}\right) \nn \\
%
%trp
&&+\frac{\hat{v}_{Tj}(x_0)}{2}\frac{\hat{\cofac}}{J \hat{B}_0}\hat{\mu}\partial_{\hat{z}} \hat{B}_0\pderiv{\hat{F}_{1j}}{\hat{v}_\|} \nn \\
%f0 contribution
&&+\frac{\hat{B}_0}{\hat{B}_{0\|}^*}\hat{F}_{0j}\frac{\hat{T}_{0j}(x_0)}{\hat{q}_j}
\frac{\hat{\mu} \hat{B}_0 + 2 v_\|^2}{\hat{B}_0}K_x
\left[\omega_{nj}+\omega_{Tj}\left(\frac{\hat{v}_\|^2
+\hat{\mu}\hat{B}_0}{\hat{T}_{0j}/\hat{T}_{0j}(x_0)}-\frac{3}{2}\right)\right]
\eea

with

\bea
\omega_{nj} & = & - \frac{L_\rf}{n_{0j}(x)}\pderiv{n_{0j}(x)}{x} \qquad
 \omega_{Tj}  =  - \frac{L_\rf}{T_{0j}(x)}\pderiv{T_{0j}(x)}{x} \nn \\
K_x & = & - \frac{L_\rf}{B_\rf}\frac{\gamma_2}{\gamma_1}\pderiv{B_0}{z} \qquad 
K_y  =  \frac{L_\rf}{B_\rf}\left(\pderiv{B_0}{x}+\frac{\gamma_3}{\gamma_1}\pderiv{B_0}{z}\right) \nn
\eea

\subsubsection{Definition of prefactors used in \gene}
\bea
\pderiv{\hat{g}_j}{\hat{t}} & = &
%
\text{pdchibardy }\partial_{\hat{y}}\hat{\chi}_1 \nn \\
%
&&+\text{pdchibardx }\partial_{\hat{x}}\hat{\chi}_1 \nn \\
%
&&+\text{pdgdx }\partial_{\hat{x}} \hat{g}_{1j} \nn \\
%
&&+\text{pdgdy }\partial_{\hat{y}} \hat{g}_{1j} \nn \\
%
%parallel derivatives
&&+\text{pdfdz }\partial_{\hat{z}}\hat{F}_{1j}
+\text{pdphidz }\left(\partial_{\hat{z}}\hat{\bar{\phi}}_1+
\text{mu\_Tjqj }\partial_{\hat{z}}\hat{\bar{B}}_{1\|}\right)\nn \\
%
%nonlinearity
&&+\text{pnl }\left(-\partial_{\hat{x}}\hat{\chi}_1\partial_{\hat{y}} \hat{g}_{1j}+\partial_{\hat{y}}\hat{\chi}_1\partial_{\hat{x}} \hat{g}_{1j}\right) \nn \\
%
%trp
&&+\text{trp }\pderiv{\hat{F}_{1j}}{\hat{v}_\|} \nn \\
%f0 contribution
&&+\text{f0\_contr }
\eea
and
\bea
\hat{g}_j & = & \hat{F}_{1j} - \text{papbar } \hat{\bar{A}}_{1\|} \nn \\
\hat{\bar{\chi}}_1 & = & \hat{\bar{\phi}}_1 - \text{vTvpar } \hat{\bar{A}}_{1\|} + \text{mu\_Tjqj } \hat{\bar{B}}_{1\|}
\eea


%%%%%%%%%%%%%%%
\bea
\text{pdchibardy} & = & - \text{edr }-\text{curv } \text{qjTjF0 } K_y + \text{press } \text{qjTjF0 } \nn \\
%
\text{pdchibardx} & = & -\text{curv } \text{qjTjF0 }K_x \nn \\ 
%
\text{pdgdx} & = & -\text{cur } K_x \nn \\
%
\text{pdgdy} & = & -\text{cur }K_y + \text{press } \nn \\
%pdfdz
\text{pdfdz } & = & -\hat{v}_{Tj}(x_0)\frac{\hat{\cofac}}{J\hat{B}_0}\hat{v}_\| \nn \\
%pdphidz
\text{pdphidz } & = & \text{pdfdz } \frac{\hat{q}_j}{\hat{T}_{0j}}\hat{F}_{0j} \nn \\
%mu_Tjqj
\text{mu\_Tjqj} & = & \frac{\hat{T}_{0j}(x_0)}{\hat{q}_j}\hat{\mu} \nn \\
%pnl (nonlinearity)
\text{pnl } & = &\frac{\hat{B}_0}{\hat{B}_{0\|}^*}\frac{1}{\hat{\cofac}} \nn \\
%
%trp
\text{trp } &=&\frac{\hat{v}_{Tj}(x_0)}{2}\frac{\hat{\cofac}}{J \hat{B}_0}\hat{\mu}\partial_{\hat{z}} \hat{B}_0 \nn \\
%f0 contribution
\text{f0\_contr} &=& \text{curv }\text{edr }K_x \frac{\hat{B}_{0\|}^*}{\hat{B}_{0} \hat{\cofac} } \nn \\
%
%vTvpar
\text{vTvpar} & = & \sqrt{\frac{2 \hat{T}_{0j}(x_0)}{\hat{m}_j}} \hat{v}_\| \nn \\
%papbar
\text{papbar} & = & \text{vTvpar } \frac{\hat{q}_j}{\hat{T}_{0j}} \hat{F}_{0j}
\eea
where
\bea
\text{edr} & = & \frac{1}{\hat{\cofac}}\frac{\hat{B}_0}{\hat{B}_{0\|}^*}
\left[\omega_{nj}+\omega_{Tj}\left(\frac{\hat{v}_\|^2
+\hat{\mu}\hat{B}_0}{\hat{T}_{0j}/\hat{T}_{0j}(x_0)}-\frac{3}{2}\right)\right]\hat{F}_{0j} \nn \\
%
\text{curv } & = & \frac{\hat{B}_0}{\hat{B}_{0\|}^*}\frac{\hat{T}_{0j}(x_0)}{\hat{q}_j}\frac{\hat{\mu} \hat{B}_0 + 2 \hat{v}_\|^2}{\hat{B}_0} \nn \\
%
\text{qjTjF0 } & = & \frac{\hat{q}_j}{\hat{T}_{0j}}\hat{F}_{0j} \nn \\
%
\text{press } & = & \beta_\rf \frac{\hat{B}_0}{\hat{B}_{0\|}^*} \frac{{T}_{0j}(x_0)}{\hat{q}_j}\frac{\hat{v}_\|^2}{\hat{\cofac}}
\frac{\partial_{\hat{x}}\hat{p}_0}{\hat{B}_0^2}
\eea

\section{Boundary conditions}
\label{sec:boundary}

The boundary conditions are periodic in $y$-direction, as this is
still in the negligible direction. The boundary conditions in radial
direction is matter of the subsection \ref{sec:xbound}, as this has carefully to
be chosen. First I will start with the quasiperiodic boundary
conditions in the parallel direction $z$. 

\subsection{Quasi-periodic boundary in parallel direction}
\label{sec:zbound}

\subsubsection{In angle-like coordinates}

The parallel boundary condition in angle-like coordinates for axisymmetric systems is given by
\bea
F(\Psi,\nu,\chi+2\pi) = F(\alpha,\nu-2\pi q,\chi)
\eea
Here, $\Psi$ is the flux surface label, $\chi$ the poloidal angle and $\nu=q\chi-\Phi$ the field-line label.
The angle coordinate $\nu$ is $2\pi$ periodic on the entire flux surface and can therefore be decomposed in Fourier components
\bea
F(\Psi,\nu,\chi) = \sum_l \tilde{F}_l(\Psi,\chi) \exp{\left[{\rm i}l(\nu-\nu_0)\right]}.
\eea
The parallel boundary condition thus becomes
\bea
\tilde{F}_l(\Psi,\chi+2\pi) = \tilde{F}_l(\Psi,\chi) \exp{\left[-2\pi {\rm i} l q\right]}.
\eea

\subsubsection{In flux tube coordinates}

Now, with the already defined flux tube coordinates
\bea
x=C_x(\Psi) - x_0 \qquad y=C_y(\Psi)\nu-y_0 \qquad z = C_z \chi
\eea 
we have
\bea
F(x,y,z+L_z) = F(x,y-2\pi q C_y(\Psi),z)
\eea
and therefore in Fourier space
\bea
\tilde{F}_l(x,z+L_z) &=& \tilde{F}_l(x,z) \exp{\left[-{\rm i} k_y \, 2\pi q(x) C_y(\Psi)\right]} \\
&=& \tilde{F}_l(x,z) \exp{\left[-2\pi{\rm i} \kymin C_y(\Psi) q(x) l\right]}. \label{eq:pbc1}
\eea
In the last step $k_y$ has been replaced by $\kymin l$.\\
By considering the toroidal mode number (number of flux tubes which fit on a flux surface in $y$ direction)
\bea
n_0 &=& \frac{2\pi}{\Delta\nu} \\
&=& \frac{2\pi C_y(\Psi)}{L_y} \\
&=& \kymin C_y(\Psi)
\eea
eq. \ref{eq:pbc1} can be rewritten to
\bea
\tilde{F}_l(x,z+L_z) &=& \tilde{F}_l(x,z) \exp{\left[-2\pi{\rm i} n_0 q(x) l\right]}.
\eea
The constraint on $\kymin$ can be rewritten to 
\bea
\kymin &=& k_{y0}^{\rm min} \frac{C_y(\Psi_0)}{C_y(\Psi)}.
\eea
if a reference wave number, e.g. in the center of the simulation domain 
$k_{y0}^{\rm min}=\frac{n_0}{C_y(\Psi_0)}$, shall be used as input (instead of $n_0$).

\paragraph{Application}
In the circular model we have $x\sim \Psi$ and $y$ shall be a physical length which implies
$C_y(x) = r/q(r)$. Hence, 
\bea
\kymin\rho_\rf &=& n_0 \frac{q(r)}{r} \rho_\rf \nn \\
 &=& k_{y0}^{\rm min} \frac{q(r)}{q_0}\frac{r_0}{r}
\eea
with $k_{y0}^{\rm min} \rho_\rf=\frac{n_0 q_0}{r_0/a}\rho^*$.

\subsubsection{The local limit}
In the local code, $q(x)$ is approximated by
\bea
q(x)&\approx&q_0\left(1+\frac{r_0}{q_0}\frac{{\rm d}q}{{\rm d}x}\frac{x}{r_0}\right) \\
 &=& q_0\left(1+\hat{s}\frac{x}{r_0}\right).
\eea
Hence, the parallel boundary condition becomes
\bea
\tilde{F}_l(x,z+L_z) &=& \tilde{F}_l(x,z) \exp{\left[-2\pi{\rm i} n_0 q_0 l\right]} \exp{\left[-2\pi{\rm i} k_y \hat{s} x \right]}. \label{eq:pbc_loc1}
\eea
For convenience, it is usually assumed that $n_0q_0$ is an integer as well, so that the corresponding term in 
\ref{eq:pbc_loc1} vanishes. 


\subsection{Boundary condition in radial direction}
\label{sec:xbound}




%%% Local Variables: 
%%% mode: latex
%%% TeX-master: "globalgene"
%%% End: 
