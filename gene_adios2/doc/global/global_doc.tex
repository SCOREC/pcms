\documentclass[12pt]{article}

%\usepackage{a4wide}
\usepackage{amsfonts}
\usepackage{amssymb}
\usepackage{hyperref}
\usepackage{graphicx}
\usepackage{../macros}

\newcommand{\mach}{$\langle$\textsl{my\_machine}$\rangle$}

\begin{document}

\title{The Gyrokinetic Plasma Turbulence Code {\sc Gene}: User Manual (global code)}
\vspace{1cm}
\date{\Large August 23, 2013}
\author{\gene Development Team}

\maketitle

\vspace{2cm}
\begin{center}
\includegraphics[width=\textwidth]{../gene_logo2.png}
\end{center}

\newpage

\subsubsection{Other parameters to be aware of when using global GENE}

\begin{description}
\item[\texttt{y\_local [logical]:}] (general namelist) flag for running y-global version of GENE.
\item[\texttt{mag\_prof [logical]:}] (geometry namelist) use radially dependent magnetic profiles . The \texttt{q\_coeffs} input parameter is then used to prescribe profiles in terms of a fifth order polynomial (unless numerical input from EFIT, CHEASE or GIST files is used). Currently, this switch must be set to true in order to run the x-global code. 
\end{description}

\subsubsection{Comments on parallelization}

As with the local version, species parallelization is most efficient and $\mu$ parallelization is also effective when collisions are not being used.  Even with collisions, $\mu$ parallelization is often competitive with other coordinates.  For x-global, radial parallelization is quite efficient.  Parallelization in the y-coordinate is not efficient and cannot be used in conjunction with radial parallelization.  Parallel velocity parallelization has the drawback that it does not reduce the size of the gyroaveraging matrix (which is independent of $v_{||}$) on each processor. 


% \begin{thebibliography}{99}
% 
% \bibitem{reference} Add references.
% 
% 
% \end{thebibliography}

%\newpage


%\clearpage
%\newpage
%\begin{figure}
%\includegraphics[width=15.0cm]{ky_combo.ps}
%\caption{}
%\label{figure:ky_combo}
%\end{figure}






\end{document}
