\documentclass[12pt]{report}
\usepackage{a4wide}
\usepackage{amsmath}
\usepackage{amsfonts}
\usepackage{lscape}
\newcommand{\bea}{\begin{eqnarray}}
\newcommand{\eea}{\end{eqnarray}}
\newcommand{\abs}[1]{\left| #1 \right|}
\newcommand{\avg}[1]{\left\langle #1 \right\rangle}
\newcommand{\tavg}[1]{\left\langle #1 \right\rangle_{\!\!t}}
\newcommand{\nperp}{\nabla\!\!_\bot}
\renewcommand{\div}{\mbox{div}\,}
\newcommand{\gene}{{\sc Gene}\,}
\newcommand{\dx}{\Delta x}
\newcommand{\sgn}{\mbox{sgn}\,}
\newcommand{\diag}{\mbox{diag}\,}
\newcommand{\rf}{\mathrm{ref}} %index for reference values in math mode
\newcommand{\nn}{\nonumber}

%apply on values which are species dependent normalizations
\newcommand{\refj}[2][j]{#2_{\rf,#1}} 

%species dependent profiles normalized to Tref,j
\newcommand{\Tpj}{\hat{T}_{j,x}}
\newcommand{\npj}{\hat{n}_{j,x}} 
\newcommand{\omT}{\omega _{Tj,x}}
\newcommand{\omn}{\omega _{nj,x}}

%\includeonly{gg_numerics,gg_derivation}
%\includeonly{gg_numerics}

%some other command definition used in gg_derivation
%commands definitions

\def\vpar{v_\parallel}
\def\d{\partial}
\def\D{{\rm d}}
\newcommand{\pderiv}[2]{\frac{\partial #1}{\partial #2}}
\def\b{\mathbf{b}}
\def\vE{\mathbf{v_E}}
\def\vgB{\mathbf{v_{\nabla B}}}
\def\vc{\mathbf{v_c}}
\def\Bs{B_{0 \parallel} ^*}
\newcommand{\mb}{\mathbf}
\newcommand{\h}{\hat}
\def\ra{\rightarrow}
\newcommand{\kymin}{k_y^{\rm min}}
\newcommand{\rhoref}{\rho_{\rm ref}}
% normalisation
\def\Br{B_{\mbox{\small ref\,}}}
\def\rhor{\rho_{\mbox{\small ref\,}}}
\def\crf{c _{\mbox{\small ref\,}}}
\def\wr{\Omega _{\mbox{\small ref\,}}}
\def\mr{m _{\mbox{\small ref\,}}}
\def\nr{n _{\mbox{\small ref\,}}}
\def\Tr{T _{\mbox{\small ref\,}}}
\def\qr{q _{\mbox{\small ref\,}}}
\def\pr{p _{\mbox{\small ref\,}}}
\def\Lp{L _{\perp\,}}
% derivative
\def\dx{\d _{x \,}}
\def\dy{\d _{y \,}}
\def\dz{\d _{z \,}}
\def\dfdv{\frac{\d f_1}{\d \vpar}}
\def\dchida{\frac{\d \chi_1}{\d \alpha}}
\def\dchidb{\frac{\d \chi_1}{\d \beta}}
\def\dfoda{\frac{\d f _0}{\d \alpha}}
\def\dfodv{\frac{\d f _0}{\d \vpar}}
\def\G{\mathbf{\mathcal{G}}}
\def\Kx{\mathcal{K}_x}
\def\Ky{\mathcal{K}_y}
\def\C{\mathcal{C}}
\def\cofac{\C\,}
\def\spec{j}
\newcommand{\xp}{x_0}		%radial reference position for profiles
\newcommand{\roL}{\frac{\rho_\rf}{L_\rf}} % rho_ref/L_ref = _r_ho _o_ver _L_
\newcommand{\mvec}[1]{\mathbf{#1}}                 %my vec


\title{Developer's manual for the global version of the GENE code}
\author{S.~Brunner, T.~Dannert, X.~Lapillonne\\Ecole Polytechnique
  F\'ed\'erale de Lausanne\\Centre de Recherches en Physique des
  Plasmas\\1015 Lausanne, Switzerland \and T.~G\"orler, F.~Jenko,
  F.~Merz, D.~Told\\Max-Planck Institut f\"ur
  Plasmaphysik\\Boltzmannstr. 2\\85748 Garching}


\begin{document}
\maketitle
\tableofcontents

\chapter{Derivation of the equations for global gyrokinetic
  simulation}
\label{sec:globaleq}

\section{Derivation of the full-$f$ equations}
\label{sec:fullfeq}

If one wants to use a general $\delta f$ splitting without any
assumptions on $F_0$, one can also take the full-$f$ equations and
solve them. I will derive them here starting from the derivation in my
PhD thesis.

In the derivation of the Vlasov equation in my thesis I used the
following assumptions
\begin{eqnarray*}
  \omega\ll\Omega&\quad&\mbox{basic gyrokinetic assumption}\\
  \rho\ll L_B&\quad&\mbox{for trafo to guiding center variables
    needed}\\
  \frac{|A_1|}{\rho B_0}\sim\frac{e\Phi_1}{T_0}\sim\epsilon\ll 1&\quad&\mbox{small perturbations of the fields by the plasma}\\
  B_{1\|}\ll |\mathbf{B}_{1\bot}|&\quad&\mbox{low $\beta$
    assumption}\\
  k_\|\ll k_\bot&\quad&\mbox{extended structures along the field
    lines, flute character}
\end{eqnarray*}

The we arrive at the full-$F$ Vlasov equation
\begin{multline}
\label{eq:full_f_vlasov}
  \frac{\partial F}{\partial t}
  +\left(
    v_\|\mathbf{b}_0
    +\frac{B_0}{B_{0\|}^*}\left(
      \mathbf{v}_{E_\chi}
      +\mathbf{v}_{\nabla B_0}
      +\mathbf{v}_c
    \right)
  \right)\\
  \cdot\left(
    \nabla F
    +\frac{1}{mv_\|}\left(
      -e\nabla\bar{\Phi}_1-\frac{e}{c}\mathbf{b}_0\dot{\bar{A}}_{1\|}-\mu\nabla(B_0+\bar{B}_{1\|})
    \right)\frac{\partial F}{\partial v_\|}
  \right)=0
\end{multline}
To simplify the further calculations, I assume a low $\beta$, so we
can neglect $B_{1\|}$ and the curvature and $\nabla B_0$ drift can be
written together as
\begin{displaymath}
  \mathbf{v}_d = \mathbf{v}_{\nabla B_0}+\mathbf{v}_c
  = \frac{\mu}{m\Omega}\,\mathbf{b}_0\times\nabla B_0
  + \frac{v_\|^2}{\Omega B_0}\,\mathbf{b}_0\times\nabla B_0
  = \frac{\mu B_0+ mv_\|^2}{mB_0\Omega}
  \,\mathbf{b}_0\times\nabla B_0
\end{displaymath}
Then we have the equation
\begin{multline}
\label{eq:full_f_vlasov_lowbeta}
  \frac{\partial F}{\partial t}
  +\left(
    v_\|\mathbf{b}_0
    +\frac{B_0}{B_{0\|}^*}\left(
      \mathbf{v}_{E_\chi}
      +\mathbf{v}_d
    \right)
  \right)\\
  \cdot\left(
    \nabla F
    +\frac{1}{mv_\|}\left(
      -e\nabla\bar{\Phi}_1-\frac{e}{c}\mathbf{b}_0\dot{\bar{A}}_{1\|}-\mu\nabla B_0
    \right)\frac{\partial F}{\partial v_\|}
  \right)=0
\end{multline}
What happens to the prefactor $B_0/B_{0\|}^*$?
\begin{eqnarray*}
  \frac{B_0}{B_{0\|}^*} &=& \frac{B_0}{\mathbf{b}_0\cdot\mathbf{B}_0^*}
  =\frac{B_0}{\mathbf{b}_0\cdot\nabla\times\left(\mathbf{A}_0+\frac{mc}{e} v_\|\mathbf{b}_0\right)}
  =\frac{B_0}{\mathbf{b}_0\cdot\left(\mathbf{B}_0+\frac{mc}{e}
      v_\|\nabla\times\mathbf{b}_0\right)}\\
  &=&\frac{B_0}{\left(\mathbf{b}_0\cdot\mathbf{B}_0+\frac{mc}{e}
      v_\|\mathbf{b}_0\cdot\nabla\times\mathbf{b}_0\right)}
  =\frac{B_0}{\left(B_0+\frac{mc}{eB_0} v_\|
      \mathbf{b}_0\cdot\nabla\times\mathbf{B}_0\right)}\\
  &=&\frac{1}{\left(1+\frac{mc}{eB_0^2} v_\|
      \mathbf{b}_0\cdot\nabla\times\mathbf{B}_0\right)}
  =\frac{1}{\left(1+\beta_e\frac{m_jc}{4\pi e_jn_{e0}T_{e0}} v_\|
      \mathbf{b}_0\cdot\nabla\times\mathbf{B}_0\right)}
  =\frac{1}{1+\beta_e\frac{m_j}{e_jn_{e0}T_{e0}} v_\|j_{0\|}}
\end{eqnarray*}
where we used Amp\`eres law in the last step.
Now we normalize $v_\|=\hat{v}_\|v_{Tj}$ and $j_\|=\hat{j}_\|
en_{\mathrm{e}0}c_s$ and get
\begin{eqnarray*}
  \frac{B_0}{B_{0\|}^*} &=&
  \frac{1}{1+\beta_e\frac{em_j}{e_jT_{e0}} v_{Tj}c_s \hat{v}_\|\hat{j}_{0\|}}
  =\frac{1}{1+\beta_e\frac{e}{e_j} \sqrt{\frac{2T_{j0}m_j}{T_{e0}m_i}} \hat{v}_\|\hat{j}_{0\|}}
\end{eqnarray*}
Up to now the calculation was exact (despite the low-$\beta$
assumption and the neglection of the displacement current). Now we
estimate the term in the denominator for by setting the  parallel current
to the electron current (how to estimate the equilibrium parallel
current, which makes the equilibrium magnetic field?) and the velocity
to the thermal velocity of the species. Then we have
\begin{eqnarray*}
  \frac{B_0}{B_{0\|}^*} &=& \frac{1}{1-\frac{4\pi m_j}{e_jB_0^2}
    v_{Tj} en_{\mathrm{e}0} v_{T\mathrm{e}}}
  =\frac{1}{1-\frac{\beta_\mathrm{e} m_j e}{e_jT_{\mathrm{e}0}}
    \sqrt{\frac{4T_{j0}T_{\mathrm{e}0}}{m_jm_\mathrm{e}}}}
  =\frac{1}{1-2\beta_\mathrm{e} \frac{e}{e_j}
    \sqrt{\frac{T_{j0}m_j}{T_{\mathrm{e}0} m_\mathrm{e}}}}
\end{eqnarray*}
For electrons one has $B_0/B_{0\|}^*\approx 1/(1+2\beta_\mathrm{e})$,
for ions 
\begin{displaymath}
  \frac{B_0}{B_{0\|}^*} =\frac{1}{1-2\beta_\mathrm{e} \frac{e}{e_\mathrm{i}}
    \sqrt{\frac{T_{\mathrm{i}0}m_\mathrm{i}}{T_{\mathrm{e}0} m_\mathrm{e}}}}
\end{displaymath}
As long as $\beta_e$ is around $0.1\%$ this term can be approximated
safely by 1, but if $\beta_e$ goes to around $1\%$, at least for the
ions, this approximation is not good, and one has to take the correct
value, which can be calculated from a given magnetic field by
calculating $\nabla\times\mathbf{B}_0$. So we keep this factor in our
calculations.

Some reordering leads to 
\begin{multline*}
  \frac{\partial F}{\partial t}
  -\frac{e}{mc}\dot{\bar{A}}_{1\|}\frac{\partial F}{\partial v_\|}
  +v_\|\nabla_\| F
  +\frac{B_0}{B_{0\|}^*}\left(
    \mathbf{v}_{E_\chi}
    +\mathbf{v}_d
  \right)\cdot\nperp F
  -\frac{1}{m}\frac{\partial F}{\partial v_\|}\left(
    e\nabla_\|\bar{\Phi}_1
    +\mu\nabla_\| B_0
  \right)\\
  -\frac{1}{mv_\|}\frac{\partial F}{\partial v_\|}
  \frac{B_0}{B_{0\|}^*}\left(
    e\left(
      \mathbf{v}_{E_\chi}
      +\mathbf{v}_d
    \right)\cdot\nabla\bar{\Phi}_1
    +\mu\mathbf{v}_{E_\chi}\cdot\nabla B_0
  \right)
  =0
\end{multline*}
\begin{multline*}
  \frac{\partial F}{\partial t}
  -\frac{e}{mc}\dot{\bar{A}}_{1\|}\frac{\partial F}{\partial v_\|}
  +v_\|\nabla_\| F
  +\frac{B_0}{B_{0\|}^*}\left(
    \mathbf{v}_{E_\chi}
    +\mathbf{v}_d
  \right)\cdot\nperp F
  -\frac{1}{m}\frac{\partial F}{\partial v_\|}\left(
    e\nabla_\|\bar{\Phi}_1
    +\mu\nabla_\| B_0
  \right)\\
  -\frac{1}{mv_\|}\frac{\partial F}{\partial v_\|}
  \frac{B_0}{B_{0\|}^*}\left(
    e\left(
      \frac{v_\|}{B_0}\nabla\bar{A}_{1\|}\times\mathbf{b}_0
      +\mathbf{v}_d
    \right)\cdot\nabla\bar{\Phi}_1
    +\mu\mathbf{v}_{E_\chi}\cdot\nabla B_0
  \right)
  =0
\end{multline*}

\section{$\delta F$ splitting for the Vlasov equation}
\label{sec:vlasoveq}

We decided to stick to the usual $\delta F$ splitting with the
assumption $F_1\ll F_0$ but to take $F_0$ general, so not necessarily
a Maxwellian. This leads to a derivation which is very similar to what
I did for the beam ions. There I found (see the fast ion paper):
\begin{displaymath}
  \frac{\partial F}{\partial t}
  +\left(
    v_\|\mathbf{b}_0
    +\mathbf{v}_{E_\chi}
    +\mathbf{v}_{\nabla B_0}
    +\mathbf{v}_c
  \right)
  \cdot\left(
    \nabla F
    +\frac{1}{mv_\|}\left(
      -e\nabla\bar{\Phi}_1
      -\frac{e}{c}\dot{\bar{A}}_{\|1}\mathbf{b}_0
      -\mu\nabla B_0
    \right)\frac{\partial F}{\partial v_\|}
  \right)=0
\end{displaymath}
with the abbreviations for the drift velocities
\begin{displaymath}
  \mathbf{v}_{E_\chi}=\frac{c}{B_0}\frac{-\nabla
    \chi_1\times\mathbf{B}_0}{B_0}
  \qquad \mathbf{v}_{\nabla B_0} = \frac{\mu
    c}{eB_0}\,\mathbf{b}_0\times\nabla B_0
  \qquad \mathbf{v}_c = \frac{v_\|^2}{\Omega}(\nabla\times\mathbf{b}_0)_\bot.
\end{displaymath}
The overbars indicate a gyoraveraged quantity $\bar{\Phi}_1 =
J_0(\lambda)\Phi_1$ and analogously for $\bar{A}_{\|1}$.
For sake of simplicity we introduced the generalized potential
$\chi_1=\bar{\Phi}_1-\frac{v_\|}{c}\bar{A}_{\|1}$. 

In a next step we introduce the $\delta F$ splitting by writing
$F=F_0+F_1$ and require $F_1\ll F_1$. By taking the time derivative of
$F_0$ to be small (or zero) we can write the total equation which is
to solve (now no splitting into two equations is done, because of the
fact, that with a general $F_0$ the zeroth order equation is not
fulfilled in general). So we only drop all higher order terms (of the
order of $\epsilon^2\omega F_0$). This leads to the equation
\begin{multline*}
  \frac{\partial F}{\partial t}
  +\left(
    v_\|\mathbf{b}_0
    +\mathbf{v}_{E_\chi}
    +\mathbf{v}_{\nabla B_0}
    +\mathbf{v}_c
  \right)\cdot\nabla F\\
  +\frac{1}{mv_\|}\left(
    v_\|\mathbf{b}_0
    +\mathbf{v}_{E_\chi}
    +\mathbf{v}_{\nabla B_0}
    +\mathbf{v}_c
  \right)\cdot\left(
    -e\nabla\bar{\Phi}_1
    -\frac{e}{c}\dot{\bar{A}}_{\|1}\mathbf{b}_0
    -\mu\nabla B_0
  \right)\frac{\partial F}{\partial v_\|}
  =0
\end{multline*}
Estimating the drift velocities leads to
\begin{eqnarray*}
  v_{E_\chi}&=&\frac{c}{B_0}\mathbf{b}_0\times\nabla\chi
  \sim\frac{e_j}{m_j\Omega_j}k_\bot\rho_{Tj}\frac{\chi}{\rho_{Tj}}
  \sim\frac{e_j}{m_jv_{Tj}^2}v_{Tj}k_\bot\rho_{Tj}\chi
  \sim v_{Tj}k_\bot\rho_{Tj}\frac{e_j\chi}{T_j}\\
  v_d&=&v_c+v_{\nabla B_0}=\left(
    \frac{v_\|^2}{\Omega_jB_0}
    +\frac{\mu c}{e_jB_0}
  \right)\mathbf{b}_0\times\nabla B_0
  \sim\left(
    \frac{v_{Tj}^2}{\Omega_j B_0}
    +\frac{v_{Tj}^2}{\Omega_jB_0}
  \right)\frac{B_0}{L_B}
  \sim\frac{\rho_{Tj}v_{Tj}}{L_B}\sim\epsilon v_{Tj}
\end{eqnarray*}
Now estimating all terms leads to
\begin{eqnarray*}
  \frac{\partial F}{\partial t}&\sim&\omega F\\
  v_\|\mathbf{b}_0\cdot\nabla F&\sim&\frac{v_{Tj}}{L_\|}F\\
  \mathbf{v}_{E_\chi}\cdot\nabla F
  &\sim&v_{Tj}k_\bot\rho_{Tj}\frac{e_j\chi}{T_j}\nabla F\\
  \mathbf{v}_d\cdot\nabla F
  &\sim&\epsilon v_{Tj}\left(
    \frac{\rho_j}{L_\bot}\frac{F_0}{\rho_j}
    +\frac{F_1}{\rho_j}
  \right)
  \sim\omega\left(
    \epsilon F_0
    +F_1
  \right)\\
                                %
  -e\frac{1}{m}\mathbf{b}_0\cdot\nabla\bar{\Phi}_1
  \frac{\partial F}{\partial v_\|}
  &\sim&-\frac{e_j}{m_j}\frac{\bar{\Phi}_1}{L_\|}
  \frac{F}{v_{Tj}}
  \sim -\frac{v_{Tj}}{L_\|}\frac{e_j\bar{\Phi}_1}{T_j} F\\
                                %
  -\frac{e}{mc}\dot{\bar{A}}_{\|1}\frac{\partial F}{\partial v_\|}
  &\sim&-\Omega_j\omega\frac{\bar{A}_{\|1}}{B_0}\frac{F}{v_{Tj}}
  \sim -\epsilon\omega F\\
                                %
  -\frac{\mu}{m}\mathbf{b}_0\cdot\nabla B_0
  \frac{\partial F}{\partial v_\|}
  &\sim& -\frac{v_{Tj}}{L_\|} F\\
                                %
  -\frac{e}{mv_\|}\mathbf{v}_{E_\chi}\cdot\nabla\bar{\Phi}_1
  \frac{\partial F}{\partial v_\|}
  &=&-\frac{ec}{mv_\|B_0}
  \mathbf{b}_0\times\nabla\left(
    \bar\Phi_1
    -\frac{v_\|}{c}\bar{A}_{\|1}
  \right)\cdot\nabla\bar{\Phi}_1
  \frac{\partial F}{\partial v_\|}
  =\frac{e}{mB_0}
  \mathbf{b}_0\times\nabla\bar{A}_{\|1}
  \cdot\nabla\bar{\Phi}_1
  \frac{\partial F}{\partial v_\|}\\
  &\sim&\frac{e_j}{m_jB_0}k_\bot\bar{A}_{\|1}
  k_\bot\bar{\Phi}_1\frac{F}{v_{Tj}}
  \sim\frac{v_{Tj}^2}{\rho_j}k_\bot^2\rho_j^2\frac{\bar{A}_{\|1}}{\rho_jB_0}
  \frac{e_j\bar{\Phi}_1}{T_j}\frac{F}{v_{Tj}}
  \sim\epsilon\Omega_j \frac{\bar{A}_{\|1}}{\rho_jB_0} F
  \sim\epsilon\omega F\\
                                %
  -\frac{e}{mc\,v_\|}\mathbf{v}_{E_\chi}\cdot\mathbf{b}_0
  \dot{\bar{A}}_{\|1}\frac{\partial F}{\partial v_\|}
  &=&0\quad\mbox{geometrical reasons}\\
  -\mu\frac{1}{mv_\|}\mathbf{v}_{E_\chi}
  \cdot\nabla B_0\frac{\partial F}{\partial v_\|}
  &\sim&-k_\bot\frac{e_j\chi}{T_j}
  \frac{\rho_{Tj}}{L_B}\frac{F}{v_{Tj}}\\
                                %
  -e\frac{1}{mv_\|}\mathbf{v}_d\cdot\nabla\bar{\Phi}_1
  \frac{\partial F}{\partial v_\|}
  &\sim& -\frac{e_j}{m_jv_{Tj}}\epsilon v_{Tj} k_\bot\bar{\Phi}_1
  \frac{F}{v_{Tj}}\\
                                %
  -\frac{e}{mc\,v_\|}\mathbf{v}_d\cdot\mathbf{b}_0
  \dot{\bar{A}}_{\|1}\frac{\partial F}{\partial v_\|}
  &=&0\quad\mbox{geometrical reasons}\\
  -\frac{\mu}{mv_\|}\mathbf{v}_d\cdot\nabla B_0\frac{\partial F}{\partial v_\|}&\sim&
\end{eqnarray*}

Taking all terms, except the time derivative of $F_0$ we arrive at the
general Vlasov equation, we want to solve.
\begin{multline*}
  \frac{\partial F_1}{\partial t}
  -\frac{e}{mc}\frac{\partial\bar{A}_{\|1}}{\partial t}
  \frac{\partial F_0}{\partial v_\|}
  +\left(
    v_\|\mathbf{b}_0
    +\mathbf{v}_{E_\chi}
    +\mathbf{v}_d
  \right)\cdot\nabla F\\
  +\frac{1}{mv_\|}\left(
    v_\|\mathbf{b}_0
    +\mathbf{v}_{E_\chi}
    +\mathbf{v}_d
  \right)\cdot\left(
    -e\nabla\bar{\Phi}_1
    -\mu\nabla B_0
  \right)\frac{\partial F}{\partial v_\|}
  =0
\end{multline*}
where we used the low-$\beta$ assumption to rewrite
the curvature drift as $\mathbf{v}_c=v_\|^2/(\Omega B_0)
\mathbf{b}_0\times\nabla B_0$. Then we introduced the combined $\nabla
B_0$ and curvature drift to the perpendicular drift $\mathbf{v}_d$
which is also perpendicular to $\nabla B_0$. 

Furthermore we introduce now the modified distribution function
$g=F_1-\frac{e}{mc}\frac{\partial F_0}{\partial v_\|} \bar{A}_{\|1}$
and get 
\begin{displaymath}
  \frac{\partial g}{\partial t}
  +\left(
    v_\|\mathbf{b}_0
    +\mathbf{v}_{E_\chi}
    +\mathbf{v}_d
  \right)\cdot\nabla F
  +\frac{1}{mv_\|}\left(
    v_\|\mathbf{b}_0
    +\mathbf{v}_{E_\chi}
    +\mathbf{v}_d
  \right)\cdot\left(
    -e\nabla\bar{\Phi}_1
    -\mu\nabla B_0
  \right)\frac{\partial F}{\partial v_\|}
  =0
\end{displaymath}
Now I continue by rewriting the last equation with explicitly giving
the dependencies of all quantities (especially the radial
dependencies are of interest in a nonlocal code). First all
abbreviations are written out.
\begin{multline*}
  \frac{\partial g_j}{\partial t}
  +\left(
    v_\|\mathbf{b}_0
    +\frac{c}{B_0}\mathbf{b}_0\times\nabla\chi_j
    +\left(
      \frac{\mu c}{e_jB_0}
      +\frac{v_\|^2}{\Omega_j B_0}
    \right)\mathbf{b}_0\times\nabla B_0
  \right)\\
  \cdot\left(
    \nabla F_j
    -\frac{1}{m_jv_\|}\frac{\partial F_j}{\partial v_\|}\left(
      e_j\nabla\bar{\Phi}_1
      +\mu\nabla B_0
    \right)
  \right)
  =0
\end{multline*}
This is the general Vlasov equation in coordinate independent vector
form. Now it is to determine the coordinate system and to rewrite the
last equation in coordinate form, which is the only possible form for
numerical simulation. 


\section{Gyrokinetic equation in field aligned coordinates}


\label{sec:gk_field_al_coord}

\subsection{The field aligned coordinates system}
We start from the straight field line coordinate system $(\Psi, \chi, \phi)$, where $\Psi$ is the poloidal flux function, $\chi$ is the straight field line poloidal angle, and $\phi$ is the toroidal angle. Using these quantities one has
$$\mb{B}_0=\nabla \Psi \times \nabla (q \chi-\Phi).$$
We can therefore define the field aligned coordinate system $(x,y,z)$ by the following transformation,
$$x=C_x(\Psi) - x_0 \hspace{1cm} \mbox{and}  \hspace{1cm}\; y=C_y(\Psi) (q \chi-\Phi)-y_0.$$
Using this definition one obtains $\nabla x \times \nabla y=\frac{d C_x}{d \Psi} C_y \nabla \Psi \times \nabla (q \chi-\Phi)=\frac{d C_x}{d \Psi} C_y  \mb{B}_0 $,  namely $x$ and $y$ are directions perpendicular  to the magnetic field.\\

\subsection{The gyrokinetic equation} 
We have :
\begin{eqnarray}
\vE \cdot \nabla &=& \frac{1}{B_0 ^2} (\mb{B}_0 \times \nabla \chi _1) \cdot \nabla \\
\vgB \cdot \nabla &=& \frac{\mu}{m \Omega B_0}(\mb{B}_0 \times \nabla B_0) \cdot \nabla \\
\vc \cdot \nabla &=& \frac{\mu_0 \vpar ^2}{\Omega B_0^2}\mb{b}_0 \times \nabla \left (p+\frac{B_0^2}{2 \mu_0} \right) \cdot \nabla \\
\b _0 \cdot \nabla &=& \frac{\mb{B}_0}{B_0} \cdot \nabla \; \; \; \; ; \; \; \tilde{\b} \cdot \nabla = \b _0 \cdot \nabla  - \frac{\mb{B} _0 \times \nabla A_{1 \parallel}}{B_0 B_{0 \parallel} ^*} \cdot \nabla 
\end{eqnarray}
Thus we need to compute the following quantities
\[
\mb{B}_0 \times \nabla \Phi \cdot \nabla \; \; \; ; \; \; \; \mb{B}_0 \cdot \nabla \; \; \mbox{and} \; \; \nabla _\perp ^2 \Phi
\]
where $\Phi$ is any scalar. Introducing $\C=1/(\frac{d f}{d \Psi} C_y)$, $\mb{B}_0=\C(x) \nabla x \times \nabla y$, we get
\begin{eqnarray*}
\mb{B}_0 \times \nabla \Phi \cdot \nabla &=& \C (\nabla x \times \nabla y) \times \nabla \Phi \cdot \nabla \\
&=&\C \left [ (\nabla \Phi \cdot \nabla x) \nabla y - ( \nabla \Phi \cdot \nabla y ) \nabla x \right ] \cdot \nabla\\
&=&\C \left [ (\d _i \Phi \nabla u^i \cdot \nabla x ) \nabla y - (\d _i \Phi \nabla u^i \cdot \nabla y ) \nabla x \right ] \cdot \nabla \\
&=&\C \left [ g^{1i}\nabla y - g^{i2}\nabla x \right ] \d _i \Phi \cdot \nabla u^j \d _j \\
&=&\C ( g^{1i} g^{2j}-g^{2i}g^{1j})\d _i \Phi \d_j
\end{eqnarray*}
with $u^i=(x,y,z)$. 
\begin{eqnarray*}
\mb{B}_0 \cdot \nabla &=& \C (\nabla x \times \nabla y) \cdot \nabla u^j \d _i \\
                      &=&\C  (\nabla x \times \nabla y) \cdot \nabla z \d _z \\
                      &=& \frac{\C}{J} \d _z
\end{eqnarray*}
and 
\begin{eqnarray*}
\nabla _\perp ^2 = (\nabla x \d _x + \nabla y \d _y)^2 = g^{11}\d _x ^2 + g^{22} \d _y ^2 + 2 g^{12} \d ^2 _{x y} 
\end{eqnarray*}

Using the ordering $k_\parallel \ll k_\perp$, terms $\frac{\d}{\d z}$ of perturbed quantities are neglected with respect to derivatives in the perpendicular plain. In addition, from the axisymmetry of the equilibria, all equilibrium quantities are independent of $y$.
Introducing the notation 
\begin{eqnarray*}
\gamma _1 &=& g^{11} g^{22}-(g^{21})^2 ,\\
\gamma _2 &=& g^{11} g^{23}-g^{21}g^{13} ,\\
\gamma _3 &=& g^{12} g^{23}-g^{22}g^{13},
\end{eqnarray*}
the different term of the vlasov equation can be written
\begin{eqnarray*}
\vE \cdot \nabla f_0 &=& \frac{\C}{B_0^2} \left ( \gamma _2 \dz f_0 \dx \chi _1
-\gamma _1 \dx f_0 \dy \chi _1 
+ \gamma _3 \dz f_0 \dy \chi _1 \right ) ,\\
%
\vE \cdot \nabla B_0 &=& \frac{\C}{B_0^2} \left ( \gamma _2 \dz B_0 \dx \chi _1
-\gamma _1 \dx B_0 \dy \chi _1 
+ \gamma _3 \dz B_0 \dy \chi _1 \right ) ,\\
%
\vE \cdot \G &=& \frac{\C}{B_0 ^2}(\gamma _1 \dx \chi_1 \G _y - \gamma _1 \dy \chi _1 \G _x),\\
%
\vgB \cdot \G &=& \frac{\C \mu}{m \Omega B_0} \left (  \gamma_1 \dx B_0 \G _y 
- \gamma _3 \dz B_0 \G _x - \gamma_2 \dz B_0 \G _y \right ) ,\\
%
\vc \cdot \G &=& \frac{\C \vpar ^2}{\Omega B_0^2}\left ( \gamma_1 \dx B_0 \G _y 
- \gamma _3 \dz B_0 \G _x - \gamma_2 \dz B_0 \G _y \right )
+ \frac{\mu_0 \vpar ^2 \C}{\Omega B_0^3} \gamma _1 \dx p \G _y  ,\\
%
\b _0 \cdot \G &=& \frac{\C}{B_0 J}\G _z, \\
%
\tilde{\b} \cdot \nabla B_0 &=& \frac{\C}{B_0 J} \dz B_0 + \mbox{neglected term},\\
%
\vc \cdot \nabla B_0 &=& 0 + \mathcal{O}(\rhor/L_\perp).
\end{eqnarray*}
where $\G_j=\d_j f_1-e_j /(m_j \vpar)\d _j \bar{\Phi}_1 \d f_0/\d \vpar$ for $j=(x,y,z)$.\\
The gyrokinetic vlasov equation becomes 
\begin{eqnarray*}
- \d _t g &=& \frac{\C}{\Bs B_0} \left [  \gamma _2 \dz f_0 \dx \chi _1
-\gamma _2 \dx f_0 \dy \chi _1 
+ \gamma _3 \dz f_0 \dy \chi _1  \right . \\
&-& \left . \frac{\C\mu}{m \vpar} (\gamma _2 \dz B_0 \dx \chi _1
-\gamma _2 \dx B_0 \dy \chi _1 
+ \gamma _3 \dz B_0 \dy \chi _1)  \right ] \\
%
&+& \frac{\C}{\Bs B_0}  (\gamma _1 \dx \chi_1 \G _y - \gamma _1 \dy \chi _1 \G _x) \\
%
&+& \C\frac{\mu B_0+m\vpar ^2}{m \Omega \Bs B_0 }\left (\Kx \G _x + \Ky \G _x\right) + \frac{\C \mu_0 \vpar ^2 }{\Omega B_0^2 \Bs}  \gamma _1  \dx p \G _y\\
%
&+& \frac{\C\vpar}{B_0 J}\G _z - \frac{\C\mu}{m B_0 J} \dz B_0 \frac{\d f_1}{\d \vpar} + \frac{df_0}{dt}
\end{eqnarray*}
where 
\begin{eqnarray*}
\Kx &=& - \gamma _3 \dz B_0,  \\
\Ky &=& \gamma_1 \dx B_0 - \gamma_2 \dz B_0
\end{eqnarray*}


\subsection{Choice of $f_0$}
Using a $f_0$ function of the form :
\begin{equation}
f_{0j}=\left ( \frac{m_j}{2\pi T_{0j}} \right )^\frac{3}{2} n_{0j} e^{-\frac{m\vpar^2/2+\mu B_0}{T_{0j}}}
\end{equation}
where $T_j$ and $n_j$ are function of $x$ and j stands for the different species,
\begin{eqnarray*}
\dx f_{0j} &=& \left (\frac{1}{n_0}\dx n_{0j} +(\frac{m_j \vpar^2}{2 T_{0j}}+\frac{\mu B_0}{T_{0j}}-\frac{3}{2})\frac{1}{T_{0j}}\dx T_{0j}- \frac{\mu}{T_{0j}} \dx B_0 \right ) f_{0j} \\
\dz f_{0j} &=& -\frac{\mu}{T_j}\dz B_0 f_{0j}\\
\d _{\vpar} f_{0j} &=& -\frac{m_j \vpar}{T_{0j}} f_{0j} \\
\d _\mu f_{0j}&=& -\frac{B_0}{T_{0j}} f_{0j}
\end{eqnarray*}

\subsection{Normalization}
\textit{This part has to be change for the general case where x and y doesn't have dimension of length}
\label{sec:normalization}

In this section, we will introduce the normalization. This is
presented before the derivation of the equations, to have the proper
formulas present when we need them later on.

The idea behind the normalization is to have in the end quantities of
roughly the same magnitude in the equations. Furthermore all
quantities should be dimensionless. Therefore we introduce first
dimensional quantities which are drawn out of the quantities in the
equations. The dimensional quantities are taken at some reference
point and are therefore neither space dependent nor species dependent.
We use $n_\rf, T_\rf, B_\rf,
m_\rf$ and $L_\rf$ as the base references. Derived
from these quantities, we can define
\begin{displaymath}
  c_\rf^2=\frac{T_\rf}{m_\rf}
  \qquad \Omega_\rf=\frac{eB_\rf}{m_\rf c}
  \qquad \rho_\rf=\frac{c_\rf}{\Omega_\rf}
\end{displaymath}

The normalization of the independent quantities, i.e. the coordinates
of our system are then
\begin{displaymath}
  t=\frac{L_\rf}{c_\rf}\hat{t}
  \qquad x=\hat{x}\rho_\rf
  \qquad y=\hat{y}\rho_\rf
  \qquad z=\hat{z}
  \qquad v_\|=v_{0j}\hat{v}_\|
  \qquad \mu = \frac{\refj{T}}{B_\rf}\hat{\mu}
\end{displaymath}
where we introduced the quantity
$v_{0j}=\sqrt{2\frac{\refj{T}}{m_j}}$. The advantage
of this normalization is that it is species dependent. As the
velocities of the species can be very different, it is a good idea to
use a normalization to the thermal velocities of the species. But the
temperature of a species is radial dependent, so we use
$\refj{T}$ which represents the temperature per species $j$ at a 
fixed point in the profile in the normalizing velocity to be spatial
independent (this choice is similar to the local version). The same
argument applies for the density where we introduce $\refj{n}$ as
reference value per species $j$.

For the dependent quantities we can then define the following
normalizations
\begin{displaymath}
  g_j=\frac{\rho_\rf}{L_\rf}\frac{\refj{n}}{v_{0j}^3}\hat{g}_j
  \qquad F_{0j}=\frac{\refj{n}}{v_{0j}^3}\hat{F}_{0j}
  \qquad \Phi=\frac{\rho_\rf}{L_\rf}\frac{T_\rf}{e}\hat{\Phi}
  \qquad A_\|=\frac{\rho_\rf}{L_\rf}B_\rf\rho_\rf\hat{A}_\|.
\end{displaymath}

The normalized temperature and density profiles are abbreviated by
\begin{displaymath}
  \Tpj = \frac{T_{j0}(x^1)}{\refj{T}}  
  \qquad \npj = \frac{n_{j0}(x^1)}{\refj{n}}.
\end{displaymath}
We define also 
\begin{displaymath}
  \omT = L_\rf\frac{d \ln T_{j0}}{dx}\, ,\qquad
  \omn = L_\rf\frac{d \ln n_{j0}}{dx}.
\end{displaymath}
The pressure term is normalized to $p=\pr \h{p}=\Tr \nr \h{p}$. \\
The coefficient $\C$ is normalized to $\hat{\C}=\C/\Br$. Note in the local version of the code, x and y are defined such that $\hat{\C}=1$.



\subsection{The normalized gyrokinetic equation}
Using the relation $B^2=\C^2 \gamma _1$ the normalized gyrokinetic equation reads

% \begin{eqnarray}
% \nonumber \frac{\d g_j}{\d t} &+& \frac{1}{\C}\left (\omn 
% + \omT (  \vpar ^2 +   \mu   B - 3/2 ) \right )   f _0 \frac{\d   \chi}{\d   y} 
% +  \frac{1}{\C}\{ \chi _1,   g _1\} \\ \nonumber
% &+& \frac{1}{B} \frac{  \mu   B + 2   \vpar ^2}{\sigma _j} \left (  K _x \G _x +   K_y \G _y \right )\\ \nonumber
% &+& \frac{\vpar ^2 \beta}{\sigma_j B^2 \C} \frac{d p}{d x} \G _y
% + \alpha _j \frac{\C  \vpar}{  J   B}\G _z \\ \nonumber
% &-& \frac{\C  \mu \alpha _j}{2   J   B}\frac{\d   f_1}{\d   \vpar} \frac{\d   B}{\d   z} \\
% &+& \left [\omn - \omT ( \vpar ^2 +  \mu  B - 3/2 ) \right ] \frac{1}{B} \frac{ \mu  B + 2  \vpar ^2}{\sigma _j} K _x  f _0 = 0
% \end{eqnarray} 
% where :
% \begin{eqnarray*}
% K_x &=&  -  \frac{1}{\C} \frac{\gamma _2}{\gamma _1} \frac{\d B}{\d z}, \\
% K_y &=&   \frac{1}{\C} \left [ \frac{\d B}{\d x} -  
% \frac{\gamma _3}{\gamma _1}\frac{\d B}{\d z} \right ].
% \end{eqnarray*}

\bea
\pderiv{\hat{g}_j}{\hat{t}} & = &
%
%pdchibardy
- \left\{\frac{1}{\hat{\cofac}}\frac{\hat{B}_0}{\hat{B}_{0\|}^*}
\left[\omega_{nj}+\omega_{Tj}\left(\frac{\hat{v}_\|^2
+\hat{\mu}\hat{B}_0}{\hat{T}_{0j}/\hat{T}_{0j}(x_0)}-\frac{3}{2}\right)\right]\hat{F}_{0j} \right. \nn \\
&&\left. +\frac{\hat{B}_0}{\hat{B}_{0\|}^*}\frac{\hat{T}_{0j}(x_0)}{\hat{T}_{0j}}\frac{\hat{\mu} \hat{B}_0 + 2 \hat{v}_\|^2}{\hat{B}_0}K_y\hat{F}_{0j}
+\frac{\hat{B}_0}{\hat{B}_{0\|}^*} \frac{{T}_{0j}(x_0)}{\hat{T}_{0j}}\frac{\hat{v}_\|^2}{\hat{\cofac}}\beta_\rf
\frac{\hat{p_0}}{\hat{B}_0^2}\omega_{p}\hat{F}_{0j}\right\}
\partial_{\hat{y}}\hat{\chi}_1 \nn \\
%
%pdchibardx
&&-\frac{\hat{B}_0}{\hat{B}_{0\|}^*}\frac{\hat{T}_{0j}(x_0)}{\hat{T}_{0j}}\frac{\hat{\mu} \hat{B}_0 + 2 \hat{v}_\|^2}{\hat{B}_0}
\hat{F}_{0j}K_x\partial_{\hat{x}}\hat{\chi}_1 \nn \\
%
%pdgdx
&&-\frac{\hat{B}_0}{\hat{B}_{0\|}^*}\frac{\hat{T}_{0j}(x_0)}{\hat{q}_j}\frac{\hat{\mu} \hat{B}_0 + 2 \hat{v}_\|^2}{\hat{B}_0}
K_x\partial_{\hat{x}} \hat{g}_{1j} \nn \\
%
%pdgdy
&&-\left\{\frac{\hat{B}_0}{\hat{B}_{0\|}^*}\frac{\hat{T}_{0j}(x_0)}{\hat{q}_j}\frac{\hat{\mu} \hat{B}_0 + 2 \hat{v}_\|^2}{\hat{B}_0}
K_y +\frac{\hat{B}_0}{\hat{B}_{0\|}^*} \frac{{T}_{0j}(x_0)}{\hat{q}_j}\frac{\hat{v}_\|^2}{\hat{\cofac}}\beta_\rf
\frac{\hat{p_0}}{\hat{B}_0^2}\omega_{p}\right\}\partial_{\hat{y}} \hat{g}_{1j} \nn \\
%
%pdfdz
&&-\hat{v}_{Tj}(x_0)\frac{\hat{\cofac}}{J\hat{B}_0}\hat{v}_\|\left( 
\partial_{\hat{z}}\hat{F}_{1j}
%pdphidz
+\frac{\hat{q}_j}{\hat{T}_{0j}}\hat{F}_{0j} \partial_{\hat{z}}\hat{\bar{\phi}}_1
+\frac{\hat{T}_{0j}(x_0)}{\hat{T}_{0j}}\hat{F}_{0j}\hat{\mu} \partial_{\hat{z}}\hat{\bar{B}}_{1\|}\right)\nn \\
%
%pnl (nonlinearity)
&&+\frac{\hat{B}_0}{\hat{B}_{0\|}^*}\frac{1}{\hat{\cofac}}\left(-
\partial_{\hat{x}}\hat{\chi}_1\partial_{\hat{y}} \hat{g}_{1j}+\partial_{\hat{y}}\hat{\chi}_1\partial_{\hat{x}} \hat{g}_{1j}\right) \nn \\
%
%trp
&&+\frac{\hat{v}_{Tj}(x_0)}{2}\frac{\hat{\cofac}}{J \hat{B}_0}\hat{\mu}\partial_{\hat{z}} \hat{B}_0\pderiv{\hat{F}_{1j}}{\hat{v}_\|} \nn \\
%f0 contribution
&&+\frac{\hat{B}_0}{\hat{B}_{0\|}^*}\hat{F}_{0j}\frac{\hat{T}_{0j}(x_0)}{\hat{q}_j}
\frac{\hat{\mu} \hat{B}_0 + 2 v_\|^2}{\hat{B}_0}K_x
\left[\omega_{nj}+\omega_{Tj}\left(\frac{\hat{v}_\|^2
+\hat{\mu}\hat{B}_0}{\hat{T}_{0j}/\hat{T}_{0j}(x_0)}-\frac{3}{2}\right)\right]
\eea

with

\bea
\omega_{nj} & = & - \frac{L_\rf}{n_{0j}(x)}\pderiv{n_{0j}(x)}{x} \qquad
 \omega_{Tj}  =  - \frac{L_\rf}{T_{0j}(x)}\pderiv{T_{0j}(x)}{x} \nn \\
K_x & = & - \frac{L_\rf}{B_\rf}\frac{\gamma_2}{\gamma_1}\pderiv{B_0}{z} \qquad 
K_y  =  \frac{L_\rf}{B_\rf}\left(\pderiv{B_0}{x}+\frac{\gamma_3}{\gamma_1}\pderiv{B_0}{z}\right) \nn
\eea

\subsubsection{Definition of prefactors used in \gene}
\bea
\pderiv{\hat{g}_j}{\hat{t}} & = &
%
\text{pdchibardy }\partial_{\hat{y}}\hat{\chi}_1 \nn \\
%
&&+\text{pdchibardx }\partial_{\hat{x}}\hat{\chi}_1 \nn \\
%
&&+\text{pdgdx }\partial_{\hat{x}} \hat{g}_{1j} \nn \\
%
&&+\text{pdgdy }\partial_{\hat{y}} \hat{g}_{1j} \nn \\
%
%parallel derivatives
&&+\text{pdfdz }\partial_{\hat{z}}\hat{F}_{1j}
+\text{pdphidz }\left(\partial_{\hat{z}}\hat{\bar{\phi}}_1+
\text{mu\_Tjqj }\partial_{\hat{z}}\hat{\bar{B}}_{1\|}\right)\nn \\
%
%nonlinearity
&&+\text{pnl }\left(-\partial_{\hat{x}}\hat{\chi}_1\partial_{\hat{y}} \hat{g}_{1j}+\partial_{\hat{y}}\hat{\chi}_1\partial_{\hat{x}} \hat{g}_{1j}\right) \nn \\
%
%trp
&&+\text{trp }\pderiv{\hat{F}_{1j}}{\hat{v}_\|} \nn \\
%f0 contribution
&&+\text{f0\_contr }
\eea
and
\bea
\hat{g}_j & = & \hat{F}_{1j} - \text{papbar } \hat{\bar{A}}_{1\|} \nn \\
\hat{\bar{\chi}}_1 & = & \hat{\bar{\phi}}_1 - \text{vTvpar } \hat{\bar{A}}_{1\|} + \text{mu\_Tjqj } \hat{\bar{B}}_{1\|}
\eea


%%%%%%%%%%%%%%%
\bea
\text{pdchibardy} & = & - \text{edr }-\text{curv } \text{qjTjF0 } K_y + \text{press } \text{qjTjF0 } \nn \\
%
\text{pdchibardx} & = & -\text{curv } \text{qjTjF0 }K_x \nn \\ 
%
\text{pdgdx} & = & -\text{cur } K_x \nn \\
%
\text{pdgdy} & = & -\text{cur }K_y + \text{press } \nn \\
%pdfdz
\text{pdfdz } & = & -\hat{v}_{Tj}(x_0)\frac{\hat{\cofac}}{J\hat{B}_0}\hat{v}_\| \nn \\
%pdphidz
\text{pdphidz } & = & \text{pdfdz } \frac{\hat{q}_j}{\hat{T}_{0j}}\hat{F}_{0j} \nn \\
%mu_Tjqj
\text{mu\_Tjqj} & = & \frac{\hat{T}_{0j}(x_0)}{\hat{q}_j}\hat{\mu} \nn \\
%pnl (nonlinearity)
\text{pnl } & = &\frac{\hat{B}_0}{\hat{B}_{0\|}^*}\frac{1}{\hat{\cofac}} \nn \\
%
%trp
\text{trp } &=&\frac{\hat{v}_{Tj}(x_0)}{2}\frac{\hat{\cofac}}{J \hat{B}_0}\hat{\mu}\partial_{\hat{z}} \hat{B}_0 \nn \\
%f0 contribution
\text{f0\_contr} &=& \text{curv }\text{edr }K_x \frac{\hat{B}_{0\|}^*}{\hat{B}_{0} \hat{\cofac} } \nn \\
%
%vTvpar
\text{vTvpar} & = & \sqrt{\frac{2 \hat{T}_{0j}(x_0)}{\hat{m}_j}} \hat{v}_\| \nn \\
%papbar
\text{papbar} & = & \text{vTvpar } \frac{\hat{q}_j}{\hat{T}_{0j}} \hat{F}_{0j}
\eea
where
\bea
\text{edr} & = & \frac{1}{\hat{\cofac}}\frac{\hat{B}_0}{\hat{B}_{0\|}^*}
\left[\omega_{nj}+\omega_{Tj}\left(\frac{\hat{v}_\|^2
+\hat{\mu}\hat{B}_0}{\hat{T}_{0j}/\hat{T}_{0j}(x_0)}-\frac{3}{2}\right)\right]\hat{F}_{0j} \nn \\
%
\text{curv } & = & \frac{\hat{B}_0}{\hat{B}_{0\|}^*}\frac{\hat{T}_{0j}(x_0)}{\hat{q}_j}\frac{\hat{\mu} \hat{B}_0 + 2 \hat{v}_\|^2}{\hat{B}_0} \nn \\
%
\text{qjTjF0 } & = & \frac{\hat{q}_j}{\hat{T}_{0j}}\hat{F}_{0j} \nn \\
%
\text{press } & = & \beta_\rf \frac{\hat{B}_0}{\hat{B}_{0\|}^*} \frac{{T}_{0j}(x_0)}{\hat{q}_j}\frac{\hat{v}_\|^2}{\hat{\cofac}}
\frac{\partial_{\hat{x}}\hat{p}_0}{\hat{B}_0^2}
\eea

\section{Boundary conditions}
\label{sec:boundary}

The boundary conditions are periodic in $y$-direction, as this is
still in the negligible direction. The boundary conditions in radial
direction is matter of the subsection \ref{sec:xbound}, as this has carefully to
be chosen. First I will start with the quasiperiodic boundary
conditions in the parallel direction $z$. 

\subsection{Quasi-periodic boundary in parallel direction}
\label{sec:zbound}

\subsubsection{In angle-like coordinates}

The parallel boundary condition in angle-like coordinates for axisymmetric systems is given by
\bea
F(\Psi,\nu,\chi+2\pi) = F(\alpha,\nu-2\pi q,\chi)
\eea
Here, $\Psi$ is the flux surface label, $\chi$ the poloidal angle and $\nu=q\chi-\Phi$ the field-line label.
The angle coordinate $\nu$ is $2\pi$ periodic on the entire flux surface and can therefore be decomposed in Fourier components
\bea
F(\Psi,\nu,\chi) = \sum_l \tilde{F}_l(\Psi,\chi) \exp{\left[{\rm i}l(\nu-\nu_0)\right]}.
\eea
The parallel boundary condition thus becomes
\bea
\tilde{F}_l(\Psi,\chi+2\pi) = \tilde{F}_l(\Psi,\chi) \exp{\left[-2\pi {\rm i} l q\right]}.
\eea

\subsubsection{In flux tube coordinates}

Now, with the already defined flux tube coordinates
\bea
x=C_x(\Psi) - x_0 \qquad y=C_y(\Psi)\nu-y_0 \qquad z = C_z \chi
\eea 
we have
\bea
F(x,y,z+L_z) = F(x,y-2\pi q C_y(\Psi),z)
\eea
and therefore in Fourier space
\bea
\tilde{F}_l(x,z+L_z) &=& \tilde{F}_l(x,z) \exp{\left[-{\rm i} k_y \, 2\pi q(x) C_y(\Psi)\right]} \\
&=& \tilde{F}_l(x,z) \exp{\left[-2\pi{\rm i} \kymin C_y(\Psi) q(x) l\right]}. \label{eq:pbc1}
\eea
In the last step $k_y$ has been replaced by $\kymin l$.\\
By considering the toroidal mode number (number of flux tubes which fit on a flux surface in $y$ direction)
\bea
n_0 &=& \frac{2\pi}{\Delta\nu} \\
&=& \frac{2\pi C_y(\Psi)}{L_y} \\
&=& \kymin C_y(\Psi)
\eea
eq. \ref{eq:pbc1} can be rewritten to
\bea
\tilde{F}_l(x,z+L_z) &=& \tilde{F}_l(x,z) \exp{\left[-2\pi{\rm i} n_0 q(x) l\right]}.
\eea
The constraint on $\kymin$ can be rewritten to 
\bea
\kymin &=& k_{y0}^{\rm min} \frac{C_y(\Psi_0)}{C_y(\Psi)}.
\eea
if a reference wave number, e.g. in the center of the simulation domain 
$k_{y0}^{\rm min}=\frac{n_0}{C_y(\Psi_0)}$, shall be used as input (instead of $n_0$).

\paragraph{Application}
In the circular model we have $x\sim \Psi$ and $y$ shall be a physical length which implies
$C_y(x) = r/q(r)$. Hence, 
\bea
\kymin\rho_\rf &=& n_0 \frac{q(r)}{r} \rho_\rf \nn \\
 &=& k_{y0}^{\rm min} \frac{q(r)}{q_0}\frac{r_0}{r}
\eea
with $k_{y0}^{\rm min} \rho_\rf=\frac{n_0 q_0}{r_0/a}\rho^*$.

\subsubsection{The local limit}
In the local code, $q(x)$ is approximated by
\bea
q(x)&\approx&q_0\left(1+\frac{r_0}{q_0}\frac{{\rm d}q}{{\rm d}x}\frac{x}{r_0}\right) \\
 &=& q_0\left(1+\hat{s}\frac{x}{r_0}\right).
\eea
Hence, the parallel boundary condition becomes
\bea
\tilde{F}_l(x,z+L_z) &=& \tilde{F}_l(x,z) \exp{\left[-2\pi{\rm i} n_0 q_0 l\right]} \exp{\left[-2\pi{\rm i} k_y \hat{s} x \right]}. \label{eq:pbc_loc1}
\eea
For convenience, it is usually assumed that $n_0q_0$ is an integer as well, so that the corresponding term in 
\ref{eq:pbc_loc1} vanishes. 


\subsection{Boundary condition in radial direction}
\label{sec:xbound}




%%% Local Variables: 
%%% mode: latex
%%% TeX-master: "globalgene"
%%% End: 

\chapter{Gyroaveraging and field equations}
\label{chap:numerics}
In the last chapter, we focused on the Vlasov equation. In the present
one, the field equations and the closely linked gyro-averaging are
treated. 



\section{Gyroaveraging}
\label{sec:gyroaveraging}

In the local version of the GENE code, we did the gyroaveraging by
Fourier transforming in both perpendicular directions and multiplying
with a Bessel function $J_0$. This is further not possible 
as we can not Fourier transform in radial ($x$) direction. So we have
to go back to the definition of the gyroaveraging. Gyro-averaging of a
function $\Phi$ just means to integrate the function along a gyro
trajectory. This operation can be written as
\begin{eqnarray*}
  \avg{\Phi(\mathbf{X}+\mathbf{r})} 
  = \frac{1}{2\pi}\int\limits_0^{2\pi} \Phi(\mathbf{X}+\mathbf{r}(\theta))\,d\theta
\end{eqnarray*}
with the gyroangle $\theta$, $\mathbf{X}$ the gyrocenter position and
$\mathbf{r}$ the vector pointing from the gyro center to the particle
position $\mathbf{x}$. So we have $\mathbf{x}=\mathbf{X}+\mathbf{r}$. 

We do not have the potential $\Phi$ in real space, but as Fourier
series in $k_y$ with the definition
\begin{displaymath}
  \Phi(\mathbf{x}) = \Phi(x,y,z) = \sum_m \Phi_m(x,z) e^{ik_m y}
\end{displaymath}
Putting both together, we arrive at
\begin{displaymath}
  \avg{\Phi(\mathbf{X}+\mathbf{r})} 
  = \frac{1}{2\pi}\int\limits_0^{2\pi} \sum_m
  \Phi_m(X_1+r_1,X_3+r_3) e^{ik_m(X_2+r_2)}\,d\theta
  = \sum_m e^{ik_mX_2} \frac{1}{2\pi}\int\limits_0^{2\pi}
  \Phi_m(X_1+r_1,X_3) e^{ik_m r_2}\,d\theta.
\end{displaymath}
In the last step, we also assumed that the gyroaveraging takes place
in the plane perpendicular to the magnetic field. Hence, $r_3$ is zero
then, as the gyro-orbit is only in the $x_1-x_2$-plane.
Now using $r_1 = \rho\cos\theta$ and $r_2=\rho\sin\theta$ leads to the
final expression
\begin{displaymath}
  \avg{\Phi(\mathbf{X}+\mathbf{r})} 
  = \sum_m e^{ik_mX_2} \frac{1}{2\pi}\int\limits_0^{2\pi}
  \Phi_m(X_1+\rho\cos\theta,X_3) e^{ik_m \rho\sin\theta}\,d\theta.
\end{displaymath}

To discretize in radial direction we use finite elements
$\Lambda_n(x)$. For the electrostatic potential, we can then write
\begin{displaymath}
  \Phi(x,k_y,z) = \sum_n \Phi_n(k_y,z)\Lambda_n(x).
\end{displaymath}
Using this in the above expression leads to
\begin{eqnarray*}
  \avg{\Phi(\mathbf{X}+\mathbf{r})} 
  &=& \sum_m e^{ik_mX_2} \frac{1}{2\pi}\int\limits_0^{2\pi}
  \sum_n \Phi_{n}(k_m,X_3)\Lambda_n(X_1+\rho\cos\theta) e^{ik_m \rho\sin\theta}\,d\theta\\
  &=& \sum_n \sum_m e^{ik_mX_2} \Phi_{n}(k_mX_3) \frac{1}{2\pi}\int\limits_0^{2\pi}
  \Lambda_n(X_1+\rho\cos\theta) e^{ik_m \rho\sin\theta}\,d\theta\\
  &=& \sum_m e^{ik_mX_2} \sum_n \Phi_{n}(k_m,X_3) \mathcal{M}'_{n}(X_1,k_m,\rho)
\end{eqnarray*}
with
\begin{displaymath}
  \mathcal{M}'_{n}(X_1,k_m,\rho)= \frac{1}{2\pi}\int\limits_0^{2\pi}
  \Lambda_n(X_1+\rho\cos\theta) e^{ik_m \rho\sin\theta}\,d\theta
\end{displaymath}
which can all be calculated in the initialization phase to arbitrary
accuracy by choosing as many $\theta$ points as wanted.
The equation can also be written in a matrix-vector form in the
following way.
\begin{displaymath}
  \bar{\mathbf{\Phi}}(k_m,x_3) = \mathcal{M}'(k_m,\rho)\mathbf{\Phi}(k_m,x_3) 
\end{displaymath}
with the matrix $\mathcal{M}'$ which is composed of the elements 
\begin{displaymath}
  \mathcal{M}'_{in}(k_m,x_3,\rho,\sigma) = \frac{1}{2\pi}\int\limits_0^{2\pi}
  \Lambda_n(X_{1,i}+\rho(X_{1,i},x_3,\sigma)\cos\theta) e^{ik_m \rho(X_{1,i},x_3,\sigma)\sin\theta}\,d\theta
\end{displaymath}
where $\sigma$ is the species index. A bold letter indicates a vector
in $x_1$ direction.

This quantity is only used temporarily, in the code it never comes up
as we have a grid on particle coordinates, not on gyrocenter
coordinates. What we really have are the following expressions.
\begin{eqnarray*}
  \avg{F(\mathbf{x}-\mathbf{r})} 
  &=& \frac{1}{2\pi}\int\limits_0^{2\pi} F(\mathbf{x}-\mathbf{r})\,d\theta
  = \frac{1}{2\pi}\int\limits_0^{2\pi} \sum_m
  F_m(x_1-r_1,x_3-r_3)\,e^{ik_m(x_2-r_2)}\,d\theta\\
  &=& \sum_m e^{ik_mx_2} \frac{1}{2\pi}\int\limits_0^{2\pi} 
  F_m(x_1-r_1,x_3)\,e^{-ik_m r_2}\,d\theta\\
  &=& \sum_m e^{ik_mx_2} \frac{1}{2\pi}\int\limits_0^{2\pi} 
  F_m(x_1-\rho\cos\theta,x_3)\,e^{-ik_m \rho\sin\theta}\,d\theta\\
  &=& \sum_m e^{ik_mx_2} \sum_n F_{n,m}(x_3) \frac{1}{2\pi}\int\limits_0^{2\pi} 
  \Lambda_n(x_1-\rho\cos\theta)\,e^{-ik_m \rho\sin\theta}\,d\theta\\
  &=& \sum_m e^{ik_mx_2} \sum_n F_{n,m}(x_3) \mathcal{M}_{n}(x_1,k_m,x_3,\rho)\\
\end{eqnarray*}
with 
\begin{displaymath}
  \mathcal{M}_{n}(x_1,k_m,x_3,\rho,\sigma)=\frac{1}{2\pi}\int\limits_0^{2\pi} 
  \Lambda_n(x_1-\rho(x_1,x_3,\sigma)\cos\theta)\,e^{-ik_m \rho(x_1,x_3,\sigma)\sin\theta}\,d\theta
\end{displaymath}
In matrix-vector form we get the analog to the result shown above.


\subsection{Gyro-averaging in non-orthogonal coordinates}
In the last section, we derived formulas for the $J_0$ operator
calculated in mixed twodimensional space. The gyroaveraging 
has been done by really sampling over the gyroorbit. In the last
section, we assumed a circular gyroorbit in the given coordinates $x$
and $y$, which is only true if $x$ and $y$ are orthogonal. In general
this is not fulfilled, especially if we go along the field line where
the metric element $g^{12}$ becomes nonzero. In this subsection, I
will calculate the gyroorbit in non-orthogonal coordinates. I will
assume only the knowledge of the metric tensor in the new coordinates,
which is denoted by $g^{ij}$. The starting coordinate system will
be the orthonormal Cartesian one (here the metric tensor is clearly
the unit matrix). In principle what we show in this subsection is how
to come to the transformation rules between two coordinate systems, if
one only has the metric tensors of both systems. The only
simplification done here is to start from a Cartesian coordinate system.

I want to represent the vector $\mathbf{v}$ with the old 
coordinates $v_j$ in the new system, where it has the 
coordinates $\bar v_j$. Both vectors represent the same physical
point, so we can expand the vector in the two different bases:
\begin{equation}
  \label{eq:vecrepr}
  \mathbf{v}=v_j\mathbf{e}^j=\bar v_j\bar{\mathbf{e}}^j
\end{equation}
For the new contravariant base vectors, we can write
\begin{displaymath}
  \bar{\mathbf{e}}^j = \nabla\bar x^j
  =\frac{\partial\bar x^j}{\partial x^r}\mathbf{e}^r
\end{displaymath}
with the transformation rule for the contravariant components of a
vector $\bar{x}^j=\bar{x}^j(x^r)$. 
From this we get a system of equations for the metric coefficients.
\begin{equation}
  \label{eq:metricnew}
  g^{ij} = \bar{\mathbf{e}}^i\cdot\bar{\mathbf{e}}^j
\end{equation}
As the metric tensor of the old Cartesian coordinate system is already
known to be the unit matrix and is therefore never explicitly used as
symbol, we denote the new metric tensor without an overbar to simplify
notation. Written eq.~(\ref{eq:metricnew}) in explicit form, we get
\begin{eqnarray*}
  \left(
    \frac{\partial\bar x^1}{\partial x^1}
  \right)^2+\left(
    \frac{\partial\bar x^1}{\partial x^2}
  \right)^2 &=& g^{11}\\
  \left(
    \frac{\partial\bar x^1}{\partial x^1}
  \right)\left(
    \frac{\partial\bar x^2}{\partial x^1}
  \right)
  + \left(
    \frac{\partial\bar x^1}{\partial x^2}
  \right)\left(
    \frac{\partial\bar x^2}{\partial x^2}
  \right) &=& g^{12}\\
  \left(
    \frac{\partial\bar x^2}{\partial x^1}
  \right)^2
  +\left(
    \frac{\partial\bar x^2}{\partial x^2}
  \right)^2 &=& g^{22}
\end{eqnarray*}
These are three equations for four unknowns. To determine the
derivatives we have to add another restriction. In general the
question is, what determines the transformation rules if one has the
metric of a new coordinate system. From the last equations we can see,
that there is still one degree of freedom. This degree is given by the
fact, that the metric is identical if one turns the whole coordinate
system around the origin or if one translates the origin. The metric
only describes ``inner'' relations between the base vectors. So to get
the real transformation rules, we can choose the direction of one base
vector (the origin is always the same for all coordinate systems). 

We choose arbitrarily that the first contravariant base vector of the
new system should be parallel to the first contravariant base vector
of the old system, and should point in the same direction. In formulas
\begin{displaymath}
  \bar{\mathbf{e}}^1=a \mathbf{e}^1
\end{displaymath}
with a proportionality constant $a>0$.
This can be written explicitly as
\begin{eqnarray*}
  a\mathbf{e}^1 = \bar{\mathbf{e}}^1 = \frac{\partial\bar{x}^1}{\partial x^1}\,\mathbf{e}^1
  + \frac{\partial\bar{x}^1}{\partial x^2}\,\mathbf{e}^2
\end{eqnarray*}
Therefore we can deduce
\begin{displaymath}
  a=\frac{\partial\bar{x}^1}{\partial x^1}\qquad\mbox{and}\qquad 
  \frac{\partial\bar{x}^1}{\partial x^2}=0.
\end{displaymath}

Using these expression, we get for the equations shown above
\begin{eqnarray*}
  \left(
    \frac{\partial\bar x^1}{\partial x^1}
  \right)^2 &=& g^{11}\\
  \left(
    \frac{\partial\bar x^1}{\partial x^1}
  \right)\left(
    \frac{\partial\bar x^2}{\partial x^1}
  \right) &=& g^{12}\\
  \left(
    \frac{\partial\bar x^2}{\partial x^1}
  \right)^2
  +\left(
    \frac{\partial\bar x^2}{\partial x^2}
  \right)^2 &=& g^{22}
\end{eqnarray*}
Hence, the solution is 
\begin{eqnarray*}
  \frac{\partial\bar x^1}{\partial x^1} = \sqrt{g^{11}}\qquad
  \frac{\partial\bar x^2}{\partial x^1} = \frac{g^{12}}{\sqrt{g^{11}}}\qquad
  \frac{\partial\bar x^2}{\partial x^2}
  = \pm\sqrt{g^{22}-\frac{(g^{12})^2}{g^{11}}}
\end{eqnarray*}
Using this result, we can write for the new contravariant base vectors
\begin{eqnarray*}
  \bar{\mathbf{e}}^1&=&\sqrt{g^{11}}\,\mathbf{e}^1\\
  \bar{\mathbf{e}}^2&=&\frac{g^{12}}{\sqrt{g^{11}}}\,\mathbf{e}^1
  \pm\sqrt{\frac{g}{g^{11}}}\,\mathbf{e}^2
\end{eqnarray*}
where we used $g=\mbox{det}(g^{ij})=g^{11}g^{22}-(g^{12})^2$. We have
still to choose the sign in the second expression. As one can see,
this sign stands in front of the Jacobian of the coordinate system. We
want that the Jacobian of the new system has the same sign as the
Jacobian of the old system, this means that the ``handedness''
(H\"andigkeit) is the same in both systems. Therefore we choose the
sign to be positive.

To come to the explicit transformation rules for the vector
components, we can expand eq.~(\ref{eq:vecrepr}) and get
\begin{displaymath}
  v_1\mathbf{e}^1+v_2\mathbf{e}^2=\bar v_1\bar{\mathbf{e}}^1+\bar v_2\bar{\mathbf{e}}^2
\end{displaymath}
This equation is now multiplied once with $\bar{\mathbf{e}}^1$ and
once with $\bar{\mathbf{e}}^2$ to get the following two equations for
two unknowns.
\begin{eqnarray*}
  v_1\mathbf{e}^1\cdot\bar{\mathbf{e}}^1
  +v_2\mathbf{e}^2\cdot\bar{\mathbf{e}}^1
  &=& \bar v_1g^{11}+\bar v_2g^{12}\\
  v_1\mathbf{e}^1\cdot\bar{\mathbf{e}}^2
  +v_2\mathbf{e}^2\cdot\bar{\mathbf{e}}^2
  &=& \bar v_1g^{12}+\bar v_2g^{22}
\end{eqnarray*}
Further the dot products on the left hand sides can be written explicitly to lead
to
\begin{eqnarray*}
  v_1 \frac{\partial\bar x^1}{\partial x^1}
  +v_2\frac{\partial\bar x^1}{\partial x^2}
  &=& \bar v_1g^{11}+\bar v_2g^{12}\\
  v_1 \frac{\partial\bar x^2}{\partial x^1}
  +v_2\frac{\partial\bar x^2}{\partial x^2}
  &=& \bar v_1g^{12}+\bar v_2g^{22}
\end{eqnarray*}
The two unknowns are the barred vector components, so written in usual
matrix vector form we get
\begin{displaymath}
  \left(
    \begin{array}{cc}
      g^{11} & g^{12}\\
      g^{21} & g^{22}
    \end{array}
  \right)\cdot \left(
    \begin{array}{c}
      \bar v_1\\ \bar{v}_2
    \end{array}
  \right)
  = \left(
    \begin{array}{cc}
      \frac{\partial\bar x^1}{\partial x^1} &
      \frac{\partial\bar x^1}{\partial x^2}\\
      \frac{\partial\bar x^2}{\partial x^1} &
      \frac{\partial\bar x^2}{\partial x^2}
    \end{array}\right)\cdot\left(
    \begin{array}{c}
      v_1 \\
      v_2
    \end{array}
  \right)
\end{displaymath}
Bringing the metric tensor to the other side, we have then
\begin{displaymath}
   \left(
    \begin{array}{c}
      \bar v_1\\ \bar{v}_2
    \end{array}
  \right)
  = \frac{1}{g}\left(
    \begin{array}{cc}
      g^{22} & -g^{12}\\
      -g^{21} & g^{11}
    \end{array}
  \right)\cdot\left(
    \begin{array}{cc}
      \sqrt{g^{11}} &
      0\\
      \frac{g^{12}}{\sqrt{g^{11}}} &
      \sqrt{\frac{g}{g^{11}}}
    \end{array}\right)\cdot\left(
    \begin{array}{c}
      v_1 \\
      v_2
    \end{array}
  \right).
\end{displaymath}
The two matrices can be multiplied which yields
\begin{displaymath}
   \left(
    \begin{array}{c}
      \bar v_1\\ \bar{v}_2
    \end{array}
  \right)
  = \frac{1}{\sqrt{g^{11}}}\left(
    \begin{array}{cc}
      1 & -\frac{g^{12}}{\sqrt{g}}\\
      0 & \frac{g^{11}}{\sqrt{g}}
    \end{array}
  \right)\cdot\left(
    \begin{array}{c}
      v_1 \\
      v_2
    \end{array}
  \right)
\end{displaymath}

\subsubsection{Final transformation system}
The final transformation rules are put together in this subsection.
\begin{align*}
  v_j\rightarrow\bar v_j: && \bar v_1 &= \frac{1}{\sqrt{g^{11}}}\left(
    v_1 
    -\frac{g^{12}}{\sqrt{g}} v_2
  \right) &
  \bar v_2 &= \frac{\sqrt{g^{11}}}{\sqrt{g}}v_2\\
  % 
  v^j\rightarrow \bar v^j: && \bar v^1 &= \sqrt{g^{11}} v^1 &
  \bar v^2 &= \frac{g^{21}}{\sqrt{g^{11}}} v^1 
  +\frac{\sqrt{g}}{\sqrt{g^{11}}} v^2\\
  \bar v^j\rightarrow v^j: &&
  v^1 &= \frac{\bar v^1}{\sqrt{g^{11}}} &
  v^2 &= -\frac{g^{12}}{\sqrt{g}\sqrt{g^{11}}}\bar v^1
  +\frac{\sqrt{g^{11}}}{\sqrt{g}}\bar v^2\\
  \bar v_j\rightarrow v_j: && 
  v_1 &= \sqrt{g^{11}}\bar v_1
  +\frac{g^{12}}{\sqrt{g^{11}}}\bar v_2 &
  v_2 &= \frac{\sqrt{g}}{\sqrt{g^{11}}}\bar v_2
\end{align*}
and the base vectors
\begin{align*}
  \mbox{contravariant:} && \mathbf{\bar e}^1 &= \sqrt{g^{11}}\mathbf{e}^1 &
  \mathbf{\bar e}^2 &= \frac{g^{21}}{\sqrt{g^{11}}}\mathbf{e}^1
  +\frac{\sqrt{g}}{\sqrt{g^{11}}}\mathbf{e}^2\\
  % 
  \mbox{covariant:} && \mathbf{\bar e}_1 &= \frac{1}{\sqrt{g^{11}}}\mathbf{e}_1
  -\frac{g^{12}}{\sqrt{g}\sqrt{g^{11}}}\mathbf{e}_2 &
  %
  \mathbf{\bar e}_2 &= \frac{\sqrt{g^{11}}}{\sqrt{g}}\mathbf{e}_2.
\end{align*}

\subsubsection{Application to the gyroaveraging}
Having this transformation rules, we now want to apply these results
to the gyro-averaging case. First we have a look on how Fourier
transforms are applied in one direction in a non-orthogonal coordinate
system. Our new coordinate system has been chosen in a way that the
$\bar{\mathbf{e}}^1 \| \mathbf{e}^1$ which leads also to
$\bar{\mathbf{e}}_2\|\mathbf{e}_2$, due to the orthogonality condition
$\bar{\mathbf{e}}_i\cdot\bar{\mathbf{e}}^j=
\delta_i^j$. For the Fourier transform in the ``2'' direction we need
the base functions, which are the Fourier exponentials
$e^{i\mathbf{k}\mathbf{x}}$. The wave front of these plane waves is
always parallel to $\bar{\mathbf{e}}_1$, so that the wave vector shows
in the $\bar{\mathbf{e}}^2$ direction. This leads to 
\begin{displaymath}
  \mathbf{k}=\bar{k}_2\bar{\mathbf{e}}^2
\end{displaymath}
without a component in the $\bar{\mathbf{e}}^1$ direction. 
Hence, we can represent a function of $\mathbf{x}$ in a Fourier series
in the second direction by the following rule
\begin{displaymath}
  \Phi(\bar{\mathbf{x}})=\sum_{m=-N_y/2+1}^{N_y/2}\Phi_m(\bar{x}^1)\,e^{i\bar{k}_{2,m}\bar{x}^2}
\end{displaymath}

Now we will use this rule for the application to gyro-averaging.
The idea behind the application to gyroaveraging is that the
coordinate system is sheared along the field line and becomes
non-orthogonal. But to calculate the gyroorbit, we have to go back to
the orthogonal one because only there the gyroorbit is a real
circle. To describe the gyroorbit in the new non-orthogonal 
coordinate system, we have to use the transformation rules. The
gyro-averaging is given by
\begin{eqnarray*}
  \avg{F(\mathbf{\bar x}-\mathbf{\bar r})} 
  &=& \frac{1}{2\pi}\int\limits_0^{2\pi} F(\mathbf{\bar
    x}-\mathbf{\bar r})\,d\theta
  = \frac{1}{2\pi}\int\limits_0^{2\pi} \sum_m
  F_m(\bar x^1-\bar r^1)\,e^{i\bar k_{2,m}(\bar x^2-\bar r^2)}\,d\theta\\
  &=& \sum_m e^{i\bar k_{2,m}\bar x^2} \frac{1}{2\pi}\int\limits_0^{2\pi} 
  F_m(\bar x^1-\bar r^1)\,e^{-i\bar k_{2,m} \bar r^2}\,d\theta\\
  &=& \sum_m e^{i\bar k_{2,m}\bar x^2} \frac{1}{2\pi}\int\limits_0^{2\pi} 
  F_m(\bar x^1-\sqrt{g^{11}} r^1)\,\exp\left\{
    -i\bar k_{2,m} \left(
      \frac{g^{21}}{\sqrt{g^{11}}} r^1 
      +\frac{\sqrt{g}}{\sqrt{g^{11}}} r^2
    \right)
  \right\}\,d\theta\\
  &=& \sum_m e^{i\bar k_{2,m}\bar x^2} \frac{1}{2\pi}\int\limits_0^{2\pi} 
  F_m(\bar x^1-\sqrt{g^{11}} \rho\cos\theta)\,\exp\left\{
    -i\bar k_{2,m} \left(
      \frac{g^{21}}{\sqrt{g^{11}}} \rho\cos\theta
      +\frac{\sqrt{g}}{\sqrt{g^{11}}} \rho\sin\theta
    \right)
  \right\}\,d\theta\\
  &=& \sum_m e^{i\bar k_{2,m}\bar x^2} \sum_n
  F_{n,m}\frac{1}{2\pi}\int\limits_0^{2\pi} 
  \Lambda_n(\bar x^1-\sqrt{g^{11}} \rho\cos\theta)\,e^{
    -i\bar k_{2,m} \left(
      g^{21} \rho\cos\theta
      +\sqrt{g} \rho\sin\theta
    \right)/\sqrt{g^{11}}
  }\,d\theta\\
  &=& \sum_m e^{i\bar k_{2,m}\bar x^2} \mathcal{M}(\bar{k}_{2,m},z,\mu)\cdot\tilde{\mathbf{F}}_m
\end{eqnarray*}
again with a gyro averaging matrix $\mathcal{M}$, whose elements are
\begin{displaymath}
  \mathcal{M}_{in}(\bar{k}_{2,m},z,\mu,\sigma)=\frac{1}{2\pi}\int\limits_0^{2\pi} 
  \Lambda_n(\bar x^1_i-\sqrt{g^{11}} \rho_\sigma(\bar{x}^1_i,z,\mu)\cos\theta)\,e^{
    -i\bar k_{2,m} \rho_\sigma(\bar{x}^1_i,z,\mu)\left(
      g^{21} \cos\theta
      +\sqrt{g} \sin\theta
    \right)/\sqrt{g^{11}}
  }\,d\theta
\end{displaymath}
Here, tilded quantities stand for base function coefficients, while untilded
for point values. The gyroradius is given by
\begin{displaymath}
  \rho_\sigma^2(z,\mu)=\frac{v_\bot^2}{\Omega_\sigma^2(z)}
  =\frac{v_\bot^2}{c_\rf^2}\frac{\Omega_\rf^2}{\Omega_\sigma^2(z)}\frac{c_\rf^2}{\Omega_\rf^2}
  =\frac{2\mu B_0(z)}{m_\sigma c_\rf^2}\frac{\Omega_\rf^2m_\sigma^2c^2}{e_\sigma^2B_0^2(z)}\rho_\rf^2
  =2\frac{\mu B_\rf}{\refj[\sigma]{T}}\frac{\refj[\sigma]{T}}{T_\rf} \frac{e^2}{e_\sigma^2} \frac{m_\sigma}{m_\rf}
  \frac{B_\rf}{B_0(z)}\rho_\rf^2.
\end{displaymath}


\subsubsection{Efficient calculation of the gyro matrix}
To efficiently calculate the elements of the last matrix we can write
(using $\rho=\rho_\sigma(x^1,z,\mu)$ and $\bar{r}^2(\theta)=\rho \left(
      g^{21} \cos\theta+\sqrt{g} \sin\theta
    \right)/\sqrt{g^{11}}$ and $\bar{r}^1(\theta)=\sqrt{g^{11}} \rho\cos\theta$)
\begin{eqnarray*}
  \mathcal{M}(\bar{k}_{2,m},z,\mu) &=&\frac{1}{2\pi}\int\limits_0^{2\pi} 
  \Lambda_n(\bar x^1-\bar{r}^1(\theta))\,e^{
    -i\bar k_{2,m}\bar{r}^2(\theta)
  }\,d\theta\\
  &=&\frac{1}{2\pi}\int\limits_0^{\pi} 
  \Lambda_n(\bar x^1-\bar{r}^1(\theta))\,e^{
    -i\bar k_{2,m}\bar{r}^2(\theta)
  }\,d\theta
  +\frac{1}{2\pi}\int\limits_\pi^{2\pi} 
  \Lambda_n(\bar x^1-\bar{r}^1(\theta))\,e^{
    -i\bar k_{2,m}\bar{r}^2(\theta)
  }\,d\theta\\
  &=&\frac{1}{2\pi}\int\limits_0^{\pi} 
  \Lambda_n(\bar x^1-\bar{r}^1(\theta))\,e^{
    -i\bar k_{2,m}\bar{r}^2(\theta)
  }\,d\theta
  +\frac{1}{2\pi}\int\limits_{-\pi}^{0} 
  \Lambda_n(\bar x^1-\bar{r}^1(\theta))\,e^{
    -i\bar k_{2,m}\bar{r}^2(\theta)
  }\,d\theta\\
  &=&\frac{1}{2\pi}\int\limits_0^{\pi} 
  \Lambda_n(\bar x^1-\bar{r}^1(\theta))\,e^{
    -i\bar k_{2,m}\bar{r}^2(\theta)
  }\,d\theta
  -\frac{1}{2\pi}\int\limits_0^{-\pi}
  \Lambda_n(\bar x^1-\bar{r}^1(\theta))\,e^{
    -i\bar k_{2,m}\bar{r}^2(\theta)
  }\,d\theta
\end{eqnarray*}
Now me make a variable substitution from $\theta$ to
$\theta'=-\theta$ in the second integral. This leaves the $\cos$ unchanged and changes the
sign of the sine.
\begin{eqnarray*}
  \mathcal{M}(\bar{k}_{2,m},z,\mu) 
  &=&\frac{1}{2\pi}\int\limits_0^{\pi} 
  \Lambda_n(\bar x^1-\bar{r}^1(\theta))\,e^{
    -i\bar k_{2,m}\rho \left(
      g^{21} \cos\theta
      +\sqrt{g} \sin\theta
    \right)/\sqrt{g^{11}}
  }\,d\theta\\
  &&+\frac{1}{2\pi}\int\limits_0^\pi
  \Lambda_n(\bar x^1-\bar{r}^1(\theta'))\,e^{
    -i\bar k_{2,m}\rho \left(
      g^{21} \cos\theta'
      -\sqrt{g} \sin\theta'
    \right)/\sqrt{g^{11}}
  }\,d\theta'\\
  &=&\frac{1}{2\pi}\int\limits_0^{\pi} 
  \Lambda_n(\bar x^1-\bar{r}^1(\theta))\left[
    \,e^{
      -i\bar k_{2,m}\rho \left(
        g^{21} \cos\theta
        +\sqrt{g} \sin\theta
      \right)/\sqrt{g^{11}}
    }
    +e^{
      -i\bar k_{2,m}\rho \left(
        g^{21} \cos\theta
        -\sqrt{g} \sin\theta
      \right)/\sqrt{g^{11}}
    }
  \right]\,d\theta\\
  &=&\frac{1}{2\pi}\int\limits_0^{\pi} 
  \Lambda_n(\bar x^1-\bar{r}^1(\theta))e^{
      -i\bar k_{2,m}g^{21} \rho\cos\theta/\sqrt{g^{11}}
    }\left[
    e^{
      -i\bar k_{2,m}\sqrt{g} \rho\sin\theta/\sqrt{g^{11}}
    }
    +e^{
      i\bar k_{2,m} \sqrt{g} \rho\sin\theta/\sqrt{g^{11}}
    }
  \right]\,d\theta\\
  &=&\frac{1}{\pi}\int\limits_0^{\pi} 
  \Lambda_n(\bar x^1-\bar{r}^1(\theta))e^{
      -i\bar k_{2,m}g^{21} \rho\cos\theta/\sqrt{g^{11}}
    }\cos\left(
      \bar k_{2,m}\sqrt{g} \rho\sin\theta/\sqrt{g^{11}}
    \right)\,d\theta
\end{eqnarray*}
Now discretizing the interval $[0,\pi]$ with $N+1$ points, with
boundary points on $0$ and $\pi$:
\begin{displaymath}
  \theta_j=j\cdot\frac{\pi}{N}\qquad j=\{0,\ldots,N\}
\end{displaymath}
The integral is then discretized with a sum as follows.
\begin{eqnarray*}
  I &=&\frac{1}{\pi}\int\limits_0^{\pi} 
  \Lambda_n(\bar x^1-\sqrt{g^{11}} \rho\cos\theta)e^{
    -i\bar k_{2,m}g^{21} \rho\cos\theta/\sqrt{g^{11}}
  }\cos\left(
    \bar k_{2,m}\sqrt{g} \rho\sin\theta/\sqrt{g^{11}}
  \right)\,d\theta\\
  &=& \frac{1}{\pi}\sum\limits_{j=0}^{N} 
  \Lambda_n(\bar x^1-\sqrt{g^{11}} \rho\cos\theta_j) e^{
    -i\bar k_{2,m}g^{21} \rho\cos\theta_j/\sqrt{g^{11}}
  }\cos\left(
    \bar k_{2,m}\sqrt{g} \rho\sin\theta_j/\sqrt{g^{11}}
  \right)\,\Delta\theta_j\\
  &=& \frac{1}{\pi}
  \Lambda_n(\bar x^1-\sqrt{g^{11}} \rho\cos\theta_0) e^{
    -i\bar k_{2,m}g^{21} \rho\cos\theta_0/\sqrt{g^{11}}
  }\cos\left(
    \bar k_{2,m}\sqrt{g} \rho\sin\theta_0/\sqrt{g^{11}}
  \right)\,\frac{\pi}{2N}\\
  &&+ \frac{1}{\pi}
  \Lambda_n(\bar x^1-\sqrt{g^{11}} \rho\cos\theta_N) e^{
    -i\bar k_{2,m}g^{21} \rho\cos\theta_N/\sqrt{g^{11}}
  }\cos\left(
    \bar k_{2,m}\sqrt{g} \rho\sin\theta_N/\sqrt{g^{11}}
  \right)\,\frac{\pi}{2N}\\
  &&+ \frac{1}{\pi}\sum\limits_{j=1}^{N-1} 
  \Lambda_n(\bar x^1-\sqrt{g^{11}} \rho\cos\theta_j) e^{
    -i\bar k_{2,m}g^{21} \rho\cos\theta_j/\sqrt{g^{11}}
  }\cos\left(
    \bar k_{2,m}\sqrt{g} \rho\sin\theta_j/\sqrt{g^{11}}
  \right)\,\frac{\pi}{N}\\
  &=& \frac{1}{2N}
  \Lambda_n(\bar x^1-\sqrt{g^{11}} \rho) e^{
    -i\bar k_{2,m}g^{21} \rho/\sqrt{g^{11}}
  }
  + \frac{1}{2N}
  \Lambda_n(\bar x^1+\sqrt{g^{11}} \rho) e^{
    i\bar k_{2,m}g^{21} \rho/\sqrt{g^{11}}
  }\\
  &&+ \frac{1}{N}\sum\limits_{j=1}^{N-1} 
  \Lambda_n(\bar x^1-\sqrt{g^{11}} \rho\cos\theta_j) e^{
    -i\bar k_{2,m}g^{21} \rho\cos\theta_j/\sqrt{g^{11}}
  }\cos\left(
    \bar k_{2,m}\sqrt{g} \rho\sin\theta_j/\sqrt{g^{11}}
  \right)
\end{eqnarray*}

The parallelization over all dimensions beside $x$ is
straightforward. If parallelized over $x$, one has to calculate a
matrix-vector multiplication with divided fields and matrices, which
can be done with \texttt{petsc}. At the moment is seems as if one
needs the node vector of the $x$ nodes on all processors in $x$
direction to set up the gyro-matrix. But this also should be
possible. 


\subsection{Efficient calculation of the gyro-average}
In the last subsection, we derived an efficient way to set up the gyro
matrix. This is important for convenience, but not so important for
production runs. There the gyro-averaging procedure becomes much more
important, as it is called quite often during a simulation. To make
the matrix-vector multiplication for the gyroaveraging efficient, we
must use the sparse (banded) structure of the matrix. In general the
matrix-vector multiplication of a matrix $\mathcal{M}$ with the
elements $m_{ij}$ with a vector $v$ with the elements $v_j$ can be
written as
\begin{equation}
  \label{eq:matvec}
  r_i=\sum_{j=1}^N m_{ij}\,v_j.
\end{equation}
But we know that the matrix is banded, so we save only the band and
leave out all of the zeros. Therefore we define a matrix, which has as
columns the band diagonals of the original matrix. 
\begin{displaymath}
  a_{r,k}=m_{r,r+k}\quad\forall\,k\in\{-b,\ldots,0,\ldots,b\}\quad\mbox{and}\quad\forall r\in\{1,\ldots,N\}
\end{displaymath}
where $b$ is the number of neighboring diagonals. Hence the total
number of diagonals is $2b+1$, which is also the number of columns of
the new matrix $\mathcal{A}$. So do we come back from the entries of
$\mathcal{A}$ to the entries of $\mathcal{M}$? The following relation
will do it.
\begin{displaymath}
  m_{ij}=
  \begin{cases}
    a_{i,j-i} & j\in[i-b,i+b]\\
    0 & \mbox{else}
  \end{cases}
\end{displaymath}
Using this relation in eq.~(\ref{eq:matvec}) and come to
\begin{displaymath}
  r_i=\sum_{j=i-b}^{i+b} a_{i,j-i}\,v_j
\end{displaymath}
Substituting the index $j$ by the rule $k=j-i$, we come to
\begin{displaymath}
  r_i=\sum_{k=-b}^{b} a_{i,k}\,v_{k+i}
\end{displaymath}





\section{Gyro-Mapping and the field equations}
\label{sec:gyromapping}

For the field equations we don't need a gyro-averaging operator but an
operator which does the gyro-mapping. The derivation of the field
equations needs an expression for the particle density and the
particle current. 

\subsection{Quasineutrality}
\label{sec:quasineutrality}
We have (as already often derived) for the density
\begin{eqnarray*}
  n_{1j}(\mathbf{x}) &=& \frac{B_0(x,z)}{m_j}\int f_{1j}\,dv_\|\,d\mu\,d\theta\\
  &=& \frac{B_0(x,z)}{m_j}\int \delta(\mathbf{X}+\mathbf{r}-\mathbf{x})\,T^*F_{1j}\,dv_\|\,d\mu\,d\theta\\
  &=& \frac{B_0(x,z)}{m_j}\int
  \delta(\mathbf{X}+\mathbf{r}-\mathbf{x})\,\left[
    F_{j1}(\mathbf{X})
    -\frac{e_j}{T_{0j}(x)}\tilde\Phi(\mathbf{X}+\mathbf{r}) F_{0j}(x,z)
  \right]\,d\mathbf{X}\,dv_\|\,d\mu\,d\theta\\
  &=& \frac{B_0(x,z)}{m_j}\int
  \delta(\mathbf{X}+\mathbf{r}-\mathbf{x})\,\Bigg[
    F_{j1}(\mathbf{X})\\
    &&-\frac{e_j}{T_{0j}(x)}\left(\Phi(\mathbf{X}+\mathbf{r})-\langle\Phi(\mathbf{X}+\mathbf{r})\rangle\right) F_{0j}(x,z)
  \Bigg]\,d\mathbf{X}\,dv_\|\,d\mu\,d\theta\\
  &=& \frac{B_0(x,z)}{m_j}\int F_{j1}(\mathbf{x}-\mathbf{r})\,dv_\|\,d\mu\,d\theta\\
  &&-\frac{e_j}{T_{0j}(x)}\left(
    \Phi(\mathbf{x}) n_{0j}(x)
    -\frac{B_0(x,z)}{m_j}\int\delta(\mathbf{X}+\mathbf{r}-\mathbf{x})\,\langle\Phi(\mathbf{X}+\mathbf{r})\rangle
    F_{0j}(x,z)\,d\mathbf{X}\,dv_\|\,d\mu\,d\theta
  \right) 
\end{eqnarray*}
This expression is now again gyro-averaged which leads to
\begin{eqnarray*}
  n_{1j}(\mathbf{x}) &=&\frac{1}{2\pi}\int\limits_0^{2\pi}
  \Bigg[
    \frac{B_0(x,z)}{m_j}\int F_{j1}(\mathbf{x}-\mathbf{r}'')\,dv_\|\,d\mu\,d\theta\\
    &&-\frac{e_j}{T_{0j}(x)}\left(
      \Phi(\mathbf{x}) n_{0j}(x)
      -\frac{B_0(x,z)}{m_j}\int\delta(\mathbf{X}+\mathbf{r}''-\mathbf{x})\,\langle\Phi(\mathbf{X}+\mathbf{r}')\rangle
      F_{0j}(x,z)\,d\mathbf{X}\,dv_\|\,d\mu\,d\theta
    \right) 
  \Bigg]\,d\theta''\\
  &=& 
  \frac{B_0(x,z)}{m_j}\int \frac{1}{2\pi}\int\limits_0^{2\pi} F_{j1}(\mathbf{x}-\mathbf{r}'')\,d\theta''\,dv_\|\,d\mu\,d\theta\\
  &&-\frac{e_j}{T_{0j}(x)}\Bigg(
    \Phi(\mathbf{x}) n_{0j}(x)\\
    &&-\frac{B_0(x,z)}{m_j}\int 
    \frac{1}{2\pi}\int\limits_0^{2\pi}\delta(\mathbf{X}+\mathbf{r}''-\mathbf{x})\,\langle\Phi(\mathbf{X}+\mathbf{r}')\rangle
    \,d\theta''\,d\mathbf{X} F_{0j}(x,z)\,dv_\|\,d\mu\,d\theta
  \Bigg)\\
  &=& \frac{B_0(x,z)}{m_j}\int \mathcal{G}[F_{j1}](\mathbf{x},v_\|,\mu)\,dv_\|\,d\mu\,d\theta\\
  &&-\frac{e_j}{T_{0j}(x)}\left(
    \Phi(\mathbf{x}) n_{0j}(x)
    -\frac{B_0(x,z)}{m_j}\int \mathfrak{G}[\Phi](\mathbf{x},\mu) F_{0j}(x,z)\,dv_\|\,d\mu\,d\theta
  \right) 
\end{eqnarray*}
with $\mathfrak{G}[\Phi]=\int
d\mathbf{X}\frac{1}{2\pi}\int\limits_0^{2\pi}d\theta''
\delta(\mathbf{X}+\mathbf{r}''-\mathbf{x})\,\langle\Phi(\mathbf{X}+\mathbf{r}')\rangle$. 
So we have to evaluate the two integrals. The first one is already
calculated in Sec.~\ref{sec:gyroaveraging}, the second will be
calculated in the present section. We start with 
\begin{eqnarray*}
  \mathfrak{G}[\Phi](\bar{\mathbf{x}},\mu) &=&
  \avg{\int\delta(\bar{\mathbf{X}}+\bar{\mathbf{r}}-\bar{\mathbf{x}})\bar{\Phi}(\bar{\mathbf{X}})\,d\bar{\mathbf{X}}}
  =\avg{\int\delta(\bar{\mathbf{X}}+\bar{\mathbf{r}}-\bar{\mathbf{x}})
    \avg{\Phi(\bar{\mathbf{X}}+\bar{\mathbf{r}})}\,d\bar{\mathbf{X}}} \\
  &=&\avg{\int\delta(\bar{\mathbf{X}}+\bar{\mathbf{r}}-\bar{\mathbf{x}})
    \sum_m e^{i\bar{k}_{2,m}\bar{X}^2}\sum_n\mathcal{M}_n(\bar{X}^1,\bar{k}_{2,m},z,\mu,\sigma)\tilde{\Phi}_n(\bar{k}_{2,m},z)\,d\mathbf{X}} \\
  &=&\frac{1}{2\pi}\int\limits_0^{2\pi}
  \sum_m e^{i\bar{k}_{2,m}(\bar{x}^2-\bar{r}^2)}
  \sum_n\mathcal{M}_n(\bar{x}^1-\bar{r}^1,\bar{k}_{2,m},z,\mu,\sigma)\tilde{\Phi}_n(\bar{k}_{2,m},z)\,d\theta \\
  &=&\sum_m e^{i\bar{k}_{2,m}\bar{x}^2} \sum_n \left[
    \frac{1}{2\pi}\int\limits_0^{2\pi}
    \mathcal{M}_n(\bar{x}^1-\bar{r}^1,\bar{k}_{2,m},z,\mu,\sigma)
    e^{-i\bar{k}_{2,m}\bar{r}^2}\,d\theta
  \right]\tilde{\Phi}_n(\bar{k}_{2,m},z)
\end{eqnarray*}
The term in the square brackets is a function of the radial coordinate
and can therefore be expanded in a series of base functions.
\begin{displaymath}
  \mathcal{M}_n(\bar{x}^1-\bar{r}^1,\bar{k}_{2,m},z,\mu,\sigma)
  =\sum_s \mathcal{M}_{sn}(\bar{k}_{2,m},z,\mu,\sigma)\Lambda_s(\bar{x}^1-\bar{r}^1)
\end{displaymath}
Using this, we can further write
\begin{eqnarray*}
  \mathfrak{G}[\Phi](\bar{\mathbf{x}},\mu)
  &=&\sum_m e^{i\bar{k}_{2,m}\bar{x}^2} \sum_n \left[
    \sum_s \mathcal{M}_{sn}(\bar{k}_{2,m},z,\mu,\sigma)
    \frac{1}{2\pi}\int\limits_0^{2\pi}\Lambda_s(\bar{x}^1-\bar{r}^1)
    e^{-i\bar{k}_{2,m}\bar{r}^2}\,d\theta
  \right]\tilde{\Phi}_n(\bar{k}_{2,m},z)\\
  &=&\sum_m e^{i\bar{k}_{2,m}\bar{x}^2} \sum_n \left[
    \sum_s \mathcal{M}_s(\bar{k}_{2,m},z,\mu,\sigma)\cdot\mathcal{M}_{sn}(\bar{k}_{2,m},z,\mu,\sigma)
  \right]\tilde{\Phi}_n(\bar{k}_{2,m},z)
\end{eqnarray*}
In a matrix vector form, this can also be written as
\begin{eqnarray*}
  \pmb{\mathfrak{G}}[\Phi](\bar{\mathbf{x}},\mu)
  &=&\sum_m e^{i\bar{k}_{2,m}\bar{x}^2} \mathcal{M}^2(\bar{k}_{2,m},z,\mu,\sigma)\cdot\tilde{\mathbf{\Phi}}(\bar{k}_{2,m},z)
\end{eqnarray*}
where bold quantities indicate a vector in $\bar{x}^1$ direction.

Another way to derive the double gyro-averaging operator is the
following, where we explicitly calculate the $\mathcal{P}$
matrix. This matrix is identical to the $\mathcal{M}^2$ matrix, as can
also be shown numerically. In the \gene\  code, we calculate the
gyro-matrix $\mathcal{M}$ and then square it to get $\mathcal{P}$. 
\begin{eqnarray*}
  \mathfrak{G}[\Phi]&=&\avg{\int\delta(\mathbf{X}+\mathbf{r}-\mathbf{x})\bar{\Phi}(\mathbf{X})\,d\mathbf{X}} 
  =\avg{\int\delta(\mathbf{X}+\mathbf{r}-\mathbf{x})\avg{\Phi(\mathbf{X}+\mathbf{r})}\,d\mathbf{X}} \\
  &=& \Bigg\langle
    \int\delta(\mathbf{X}+\mathbf{r}-\mathbf{x}) 
    \sum_m e^{i\bar k_{2,m}\bar X^2} \sum_n \Phi_{n,m}\\
    &&\cdot\,\frac{1}{2\pi}\int\limits_0^{2\pi}
    \Lambda_n(\bar{X}^1+\sqrt{g^{11}}\rho\cos\theta')
    e^{i\bar{k}_{2,m} \rho\left(
        g^{12}\cos\theta'+g\sin\theta'
      \right)/\sqrt{g^{11}}
    }\,d\theta'\,d\mathbf{X}
  \Bigg\rangle \\
  &=& \frac{1}{2\pi}\int\limits_0^{2\pi} 
    \int\delta(\mathbf{X}+\mathbf{r}-\mathbf{x}) 
    \sum_m e^{i\bar k_{2,m}\bar X^2} \sum_n \Phi_{n,m}\\
    &&\cdot\,\frac{1}{2\pi}\int\limits_0^{2\pi}
    \Lambda_n(\bar{X}^1+\sqrt{g^{11}}\rho\cos\theta')
    e^{i\bar{k}_{2,m} \rho\left(
        g^{12}\cos\theta'+g\sin\theta'
      \right)/\sqrt{g^{11}}
    }\,d\theta'\,d\mathbf{X}
  \,d\theta \\
  &=& \frac{1}{2\pi}\int\limits_0^{2\pi} 
  \sum_m e^{i\bar k_{2,m}(\bar x^2-\bar r^2)} 
    \sum_n \Phi_{n,m}\\
    &&\cdot\,\frac{1}{2\pi}\int\limits_0^{2\pi}
    \Lambda_n(\bar{x}^1-\bar{r}^1+\sqrt{g^{11}}\rho\cos\theta')
    e^{i\bar{k}_{2,m} \rho\left(
        g^{12}\cos\theta'+g\sin\theta'
      \right)/\sqrt{g^{11}}
    }\,d\theta' \,d\theta \\
  &=& \sum_m e^{i\bar k_{2,m}\bar x^2} \sum_n \Phi_{n,m}
  \frac{1}{2\pi}\int\limits_0^{2\pi} e^{-i\bar k_{2,m}\rho(g^{12}\cos\theta+g\sin\theta)/\sqrt{g^{11}}} \\
  &&\cdot\,\frac{1}{2\pi}\int\limits_0^{2\pi}
  \Lambda_n(\bar{x}^1-\sqrt{g^{11}}\rho\cos\theta+\sqrt{g^{11}}\rho\cos\theta')
  e^{i\bar{k}_{2,m} \rho\left(
      g^{12}\cos\theta'+g\sin\theta'
    \right)/\sqrt{g^{11}}
  }\,d\theta' \,d\theta \\
  &=& \sum_m e^{i\bar k_{2,m}\bar x^2} \sum_n \Phi_{n,m}
  \frac{1}{2\pi}\int\limits_0^{2\pi} e^{-i\bar k_{2,m}\rho(g^{12}\cos\theta+g\sin\theta)/\sqrt{g^{11}}} \\
  &&\cdot\,\frac{1}{2\pi}\int\limits_0^{2\pi}
  \Lambda_n(\bar{x}^1-\sqrt{g^{11}}\rho(\cos\theta-\cos\theta'))
  e^{i\bar{k}_{2,m} \rho\left(
      g^{12}\cos\theta'+g\sin\theta'
    \right)/\sqrt{g^{11}}
  }\,d\theta' \,d\theta \\
  &=& \sum_m e^{i\bar k_{2,m}\bar x^2} \sum_n \mathcal{P}(\bar k_{2,m},z,\mu) \Phi_{n,m}
\end{eqnarray*}
with
\begin{eqnarray*}
  \mathcal{P}(\bar{k}_{2,m},z,\mu) &=& 
  \frac{1}{2\pi}\int\limits_0^{2\pi} e^{-i\bar k_{2,m}\rho(g^{12}\cos\theta+g\sin\theta)/\sqrt{g^{11}}} \\
  &&\cdot\,\frac{1}{2\pi}\int\limits_0^{2\pi}
  \Lambda_n(\bar{x}^1-\sqrt{g^{11}}\rho(\cos\theta-\cos\theta'))
  e^{i\bar{k}_{2,m} \rho\left(
      g^{12}\cos\theta'+g\sin\theta'
    \right)/\sqrt{g^{11}}
  }\,d\theta' \,d\theta 
\end{eqnarray*}
This operator accords to $J_0^2$ operator in the double Fourier case.
If we also take the moment of this operator weighted with $F_0$ we
arrive at the correspondent to the $\Gamma_0$ function in the double
Fourier case. We can therefore write
\begin{eqnarray*}
  G[\Phi](\mathbf{x}) &=& \frac{B_0(x,z)}{m_j}\int
  \mathfrak{G}[\Phi](\mathbf{x},\mu) F_{0j}(x,z,v_\|,\mu)\,dv_\|\,d\mu\,d\theta\\
  &=& \frac{2n_{0j}(x)}{v_{Tj}^2(x)}\frac{B_0(x,z)}{m_j}\int\mathfrak{G}[\Phi](\mathbf{x},\mu)
  \,e^{-\mu B_0(x,z)/T_{0j}(x)}\,d\mu\\
  &=& \frac{2n_{0j}(x)}{v_{Tj}^2(x)}\frac{B_0(x,z)}{m_j}
  \int \sum_m e^{i\bar k_{2,m}\bar x^2} \sum_n \mathcal{P}_n(\bar{x}^1,\bar{k}_{2,m},z,\mu) \tilde{\Phi}_{n}(\bar{k}_{2,m},z)
  \,e^{-\mu B_0(x,z)/T_{0j}(x)}\,d\mu
\end{eqnarray*}
This can be written in matrix-vector form
\begin{eqnarray*}
  \mathbf{G}[\Phi](\bar{k}_{2,m},z) &=& 
  \frac{2n_{0j}(x)}{v_{Tj}^2(x)}\frac{B_0(x,z)}{m_j}\int \mathcal{P}(\bar k_{2,m},z,\mu)\cdot\tilde{\mathbf{\Phi}}_m
  \,e^{-\mu B_0(x,z)/T_{0j}(x)}\,d\mu\\
  &=& \left[\frac{2n_{0j}(x)}{v_{Tj}^2(x)}\frac{B_0(x,z)}{m_j}\int \mathcal{P}(\bar{k}_{2,m},z,\mu)
    \,e^{-\mu B_0(x,z)/T_{0j}(x)}\,d\mu\right] \cdot\tilde{\mathbf{\Phi}}_m
\end{eqnarray*}
where the expression in the brackets can also be calculated during the
initialization. The tilded boldface $\Phi$ stands for a vector of
base function coefficients, representing the electrostatic potential. Values
without a tilde are values at the grid points.
With this notation, we arrive at an expression for the density given
by
\begin{eqnarray*}
  n_{1j}(\mathbf{x}) &=&
  \frac{B_0(x,z)}{m_j}\int \mathcal{G}[F_{j1}](\mathbf{x},v_\|,\mu)\,dv_\|\,d\mu\,d\theta
  -\frac{e_j}{T_{0j}(x)}\left(
    \Phi(\mathbf{x}) n_{0j}(x)
    -G[\Phi](\mathbf{x})
  \right) 
\end{eqnarray*}

Using the quasineutrality equation, we can write for the first field
equation in unnormalized units
\begin{eqnarray*}
  0 &=&\sum_j e_jn_{1j}(\mathbf{x})\\
  &=&\sum_j e_j \left(
    \frac{B_0(x,z)}{m_j}\int \mathcal{G}[F_{j1}](\mathbf{x},v_\|,\mu)\,dv_\|\,d\mu\,d\theta
    -\frac{e_j}{T_{0j}(x)}\left(
      \Phi(\mathbf{x}) n_{0j}(x)
      -G[\Phi](\mathbf{x})
    \right)  
  \right)
\end{eqnarray*}
Reordered we come to
\begin{displaymath}
  \sum_j\frac{e_j^2}{T_{0j}(x)}\left(
    \Phi(\mathbf{x}) n_{0j}(x)
    -G[\Phi](\mathbf{x})
  \right) 
  = \sum_je_j\frac{B_0(x,z)}{m_j}\int \mathcal{G}[F_{j1}](\mathbf{x},v_\|,\mu)\,dv_\|\,d\mu\,d\theta
\end{displaymath}
To further normalize this equation, we use the expressions, derived
above to come to
\begin{multline*}
  \sum_j\frac{e_j^2}{T_{0j}(x)}\Bigg(
    \sum_m e^{i\bar k_{2,m}\bar x^2} \Phi(\bar{x}^1,\bar{k}_{2,m},z) n_{0j}(x)\\
    -\frac{2n_{0j}(x)}{v_{Tj}^2(x)}\frac{B_0(x,z)}{m_j}
    \int \sum_m e^{i\bar k_{2,m}\bar x^2} \sum_n \mathcal{P}_n(\bar{x}^1,\bar{k}_{2,m},z,\mu) \tilde{\Phi}_{n}(\bar{k}_{2,m},z)
    \,e^{-\mu B_0(x,z)/T_{0j}(x)}\,d\mu
  \Bigg) \\
  = \sum_j e_j\frac{B_0(x,z)}{m_j}\int \sum_m e^{i\bar{k}_{2,m}\bar{x}^2}
  \sum_n\mathcal{M}_n(\bar{x}^1,\bar{k}_{2,m},z,\mu)\tilde{F}_n(\bar{k}_{2,m},z,v_\|,\mu)\,dv_\|\,d\mu\,d\theta
\end{multline*}
Reordering
\begin{multline*}
  \sum_m e^{i\bar k_{2,m}\bar x^2} \sum_j\frac{e_j^2}{T_{0j}(x)}\Bigg(
    \Phi(\bar{x}^1,\bar{k}_{2,m},z) n_{0j}(x)\\
    -\frac{2n_{0j}(x)}{v_{Tj}^2(x)}\frac{B_0(x,z)}{m_j}
    \int  \sum_n \mathcal{P}_n(\bar{x}^1,\bar{k}_{2,m},z,\mu) \tilde{\Phi}_{n}(\bar{k}_{2,m},z)
    \,e^{-\mu B_0(x,z)/T_{0j}(x)}\,d\mu
  \Bigg) \\
  = \sum_m e^{i\bar{k}_{2,m}\bar{x}^2} \sum_j e_j\frac{B_0(x,z)}{m_j}\int 
  \sum_n\mathcal{M}_n(\bar{x}^1,\bar{k}_{2,m},z,\mu)\tilde{F}_n(\bar{k}_{2,m},z,v_\|,\mu)\,dv_\|\,d\mu\,d\theta
\end{multline*}
As both sides of the field equation are Fourier sums in the second
coordinate, we can write the field equation for all modes
($\forall m$)
\begin{multline*}
  \sum_j\frac{e_j^2}{T_{0j}(x)}\Bigg(
    \Phi(\bar{x}^1,\bar{k}_{2,m},z) n_{0j}(x)\\
    -\frac{2n_{0j}(x)}{v_{Tj}^2(x)}\frac{B_0(x,z)}{m_j}
    \int  \sum_n \mathcal{P}_n(\bar{x}^1,\bar{k}_{2,m},z,\mu) \tilde{\Phi}_{n}(\bar{k}_{2,m},z)
    \,e^{-\mu B_0(x,z)/T_{0j}(x)}\,d\mu
  \Bigg) \\
  = \sum_j 2\pi e_j\frac{B_0(x,z)}{m_j}\int 
  \sum_n\mathcal{M}_n(\bar{x}^1,\bar{k}_{2,m},z,\mu)\tilde{F}_n(\bar{k}_{2,m},z,v_\|,\mu)\,dv_\|\,d\mu
\end{multline*}
Now switching to the matrix-vector form, we can write
\begin{multline*}
  \sum_je_j^2 \diag\{\frac{n_{0j}(x)}{T_{0j}(x)}\}\Bigg(
    \mathcal{L}\\
    -\frac{2}{m_j}\diag\{\frac{B_0(x,z)}{v_{Tj}^2(x)}\}
    \int \mathcal{P}(\bar{k}_{2,m},z,\mu)
    \cdot\diag\{e^{-\mu B_0(x,z)/T_{0j}(x)}\}\,d\mu
  \Bigg)\cdot\tilde{\mathbf{\Phi}}(\bar{k}_{2,m},z) \\
  = \sum_j \frac{2\pi e_j}{m_j}\diag\{B_0(x,z)\}\int 
  \mathcal{M}(\bar{k}_{2,m},z,\mu)\cdot\tilde{\mathbf{F}}(\bar{k}_{2,m},z,v_\|,\mu)\,dv_\|\,d\mu
\end{multline*}
where we used a base function representation for the electrostatic potential also for
the first occurrence with
$\mathbf{\Phi}(\bar{k}_{2,m},z)=\mathcal{L}\cdot\tilde{\mathbf{\Phi}}(\bar{k}_{2,m},z)$.
We can reorganize the equation to introduce dimensionless quantities.
\begin{multline*}
  \frac{e}{T_\rf}n_\rf\sum_j\frac{e_j^2}{e^2} \diag\{\frac{n_{0j}(x)}{n_\rf}\frac{T_\rf}{T_{0j}(x)}\}\Bigg(
    \mathcal{L}\\
    -\diag\{\frac{B_0(x,z)}{B_\rf}\frac{\refj{T}}{T_{0j}(x)}\}\frac{B_\rf}{\refj{T}}\int \mathcal{P}(\bar{k}_{2,m},z,\mu)
    \cdot\diag\{e^{-\mu B_0(x,z)/T_{0j}(x)}\}\,d\mu
  \Bigg)\cdot\tilde{\mathbf{\Phi}}(\bar{k}_{2,m},z) \\
  = 2\pi\sum_j \frac{1}{m_j}\frac{e_j}{e}\diag\{B_0(x,z)\}\int 
  \mathcal{M}(\bar{k}_{2,m},z,\mu)\cdot\tilde{\mathbf{F}}(\bar{k}_{2,m},z,v_\|,\mu)\,dv_\|\,d\mu
\end{multline*}
Using 
\begin{multline*}
  \mathcal{P}''(\bar{k}_{2,m},z) = \sum_j\frac{e_j^2}{e^2}
  \diag\{\frac{n_{0j}(x)}{n_\rf}\frac{T_\rf}{T_{0j}(x)}\}\cdot\\
  \left(
    \mathcal{L}
    -\diag\{\frac{B_0(x,z)}{B_\rf}\frac{\refj{T}}{T_{0j}(x)}\}\frac{B_\rf}{\refj{T}}\int \mathcal{P}(\bar{k}_{2,m},z,\mu)
    \cdot\diag\{e^{-\mu B_0(x,z)/T_{0j}(x)}\}\,d\mu
  \right)
\end{multline*}
we can write the quasineutrality equation in the following form
\begin{multline*}
  \frac{e}{T_\rf} \mathcal{P}''(\bar{k}_{2,m},z)\cdot\tilde{\mathbf{\Phi}}(\bar{k}_{2,m},z) \\
  = \frac{2\pi}{n_\rf}\sum_j \frac{1}{m_j}\frac{e_j}{e}\diag\{B_0(x,z)\}\int 
  \mathcal{M}(\bar{k}_{2,m},z,\mu)\cdot\tilde{\mathbf{F}}(\bar{k}_{2,m},z,v_\|,\mu)\,dv_\|\,d\mu
\end{multline*}
with the dimensionless matrices $\mathcal{P}''$ and $\mathcal{M}$.

The matrix $\mathcal{P}''$ can be calculated during initialization and
then in each timestep, we have either to solve the resulting system of
equations or we invert the matrix in the initialization and multiply
just the inverted matrix to the right hand side. Which method is
faster, has to be seen.
As the $\mathcal{P}''$ matrix is also independent of the species (it is already a sum
over all species) and independent of velocity space variables, we can
combine its inverse with the gyro-averaging matrix and get the final
field equation (still unnormalized)
\begin{multline*}
  \frac{e}{T_\rf} \tilde{\mathbf{\Phi}}(\bar{k}_{2,m},z) \\
  = \frac{2\pi}{n_\rf}\sum_j \frac{1}{m_j}\frac{e_j}{e}\diag\{B_0(x,z)\}\int 
  \left[\mathcal{P}''(\bar{k}_{2,m},z)\right]^{-1}\cdot\mathcal{M}(\bar{k}_{2,m},z,\mu)\cdot\tilde{\mathbf{F}}(\bar{k}_{2,m},z,v_\|,\mu)\,dv_\|\,d\mu
\end{multline*}
This combined matrix can be calculated during initialization.

\subsubsection{Normalization}
\label{sec:norm_field1}
We are normalizing as described in sec.~\ref{sec:normalization}. The
$\mathcal{P}''$ matrix is already dimensionless, but it is calculated
in normalized units as follows.
\begin{multline*}
  \mathcal{P}''(\bar{k}_{2,m},z) = \sum_j\frac{e_j^2}{e^2}
  \diag\{\frac{n_{0j}(x)}{n_\rf}\frac{T_\rf}{T_{0j}(x)}\}\cdot\\
  \left(
    \mathcal{L}
    -\diag\{\frac{B_0(x,z)}{B_\rf}\frac{\refj{T}}{T_{0j}(x)}\} \int \mathcal{P}(\bar{k}_{2,m},z,\mu)
    \cdot\diag\{e^{-\hat{\mu}\refj{T}/T_{0j}(x)\, B_0(x,z)/B_\rf}\}\,d\hat{\mu}
  \right)
\end{multline*}
We leave the hats and use only normalized quantities in what follows in this
subsection. 
\begin{multline*}
  \tilde{\mathbf{\Phi}}(\bar{k}_{2,m},z)
  = \pi\sum_j \frac{e_j}{e}\frac{\refj{n}}{n_\rf}
  \diag\{\frac{B_0(x,z)}{B_\rf}\}\int \left[\mathcal{P}''(\bar{k}_{2,m},z)\right]^{-1}\cdot\mathcal{M}(\bar{k}_{2,m},z,\mu)
  \cdot\tilde{\mathbf{F}}(\bar{k}_{2,m},z,v_\|,\mu)\,dv_\|\,d\mu
\end{multline*}
For a more efficient implementation we can write
\begin{multline*}
  \tilde{\mathbf{\Phi}}(\bar{k}_{2,m},z) =\\
  \pi\sum_j \frac{e_j}{e}\frac{\refj{n}}{n_\rf}
  \diag\{\frac{B_0(x,z)}{B_\rf}\}\int \left[\mathcal{P}''(\bar{k}_{2,m},z)\right]^{-1}\cdot\mathcal{M}(\bar{k}_{2,m},z,\mu)
  \cdot\left[
    \int\tilde{\mathbf{F}}(\bar{k}_{2,m},z,v_\|,\mu)\,dv_\|
  \right]\,d\mu
\end{multline*}
Or with the $\mathcal{P}''$ matrix on the left hand side:
\begin{multline*}
  \mathcal{P}''(\bar{k}_{2,m},z)\cdot\tilde{\mathbf{\Phi}}(\bar{k}_{2,m},z) =\\
  \pi\sum_j \frac{e_j}{e}\frac{\refj{n}}{n_\rf}
  \diag\{\frac{B_0(x,z)}{B_\rf}\}\int \mathcal{M}(\bar{k}_{2,m},z,\mu)
  \cdot\left[
    \int\tilde{\mathbf{F}}(\bar{k}_{2,m},z,v_\|,\mu)\,dv_\|
  \right]\,d\mu
\end{multline*}


\subsection{Amp\`ere's law}
\label{sec:field2}
What we need for Amp\`ere's law is the parallel current density
$j_{\|1j}$. It is calculated according to
\begin{eqnarray*}
  j_{\|1j}(\mathbf{x}) &=& e_j\int v_\|f_1(\mathbf{x},v_\|,v_\bot,\theta)\,d^3v
  =e_j\frac{B_0}{m_j}\int v_\| f_1(\mathbf{x},v_\|,\mu,\theta)\,dv_\|\,d\mu\,d\theta\\
  &=&e_j\frac{B_0}{m_j}\int \delta(\mathbf{X}+\mathbf{r}-\mathbf{x}) v_\|
  T^*F_1(\mathbf{X},v_\|,\mu)\,d\mathbf{X}\,dv_\|\,d\mu\,d\theta\\
  &=&e_j\frac{B_0}{m_j}\int \delta(\mathbf{X}+\mathbf{r}-\mathbf{x}) v_\|
  \left(
    F_{j1}(\mathbf{X})
    -\frac{e_j}{T_{0j}}\tilde\Phi(\mathbf{X}+\mathbf{r}) F_{0j}
  \right)\,d\mathbf{X}\,dv_\|\,d\mu\,d\theta\\
  &=&e_j\frac{B_0}{m_j}\int \delta(\mathbf{X}+\mathbf{r}-\mathbf{x}) v_\|
  \left(
    F_{j1}(\mathbf{X})
    -\frac{e_j}{T_{0j}}\left(\Phi(\mathbf{X}+\mathbf{r})-\langle\Phi(\mathbf{X}+\mathbf{r})\rangle\right) F_{0j}
  \right)\,d\mathbf{X}\,dv_\|\,d\mu\,d\theta\\
  &=& e_j\frac{B_0}{m_j}\int \delta(\mathbf{X}+\mathbf{r}-\mathbf{x}) v_\|F_{j1}(\mathbf{X})\,d\mathbf{X}\,dv_\|\,d\mu\,d\theta\\
  &&-\frac{e_j}{T_{0j}}e_j\frac{B_0}{m_j}
  \int \delta(\mathbf{X}+\mathbf{r}-\mathbf{x}) v_\|\left(
    \Phi(\mathbf{X}+\mathbf{r})
    -\langle\Phi(\mathbf{X}+\mathbf{r})\rangle
  \right) F_{0j}\,d\mathbf{X}\,dv_\|\,d\mu\,d\theta\\
  &=& e_j\frac{B_0}{m_j}\int \delta(\mathbf{X}+\mathbf{r}-\mathbf{x}) v_\|F_{j1}(\mathbf{X})\,d\mathbf{X}\,dv_\|\,d\mu\,d\theta
\end{eqnarray*}
The second integral vanishes due to the $v_\|$ symmetry of
$F_{0j}$. At this point, we now use the modified distribution function
$g_j$ to replace
$F_{1j}(\mathbf{X})=g_j(\mathbf{X})-\frac{2e_jv_\|}{m_jcv_{Tj}^2}
F_{0j}\frac{1}{2\pi}\int A_{\|1}(\mathbf{X}+\mathbf{r})\,d\theta$. We
arrive at
\begin{eqnarray*}
  j_{\|1j}(\mathbf{x}) 
  &=& \frac{1}{2\pi}\int e_j\frac{B_0}{m_j}\int \delta(\mathbf{X}+\mathbf{r}''-\mathbf{x})
  v_\|\left[
    g_j(\mathbf{X})-\frac{2e_jv_\|}{m_jcv_{Tj}^2}
    F_{0j}\frac{1}{2\pi}\int A_{\|1}(\mathbf{X}+\mathbf{r}')\,d\theta'
  \right]\,d\mathbf{X}\,dv_\|\,d\mu\,d\theta\,d\theta''\\
  &=& \frac{1}{2\pi}\int e_j\frac{B_0}{m_j}\int \delta(\mathbf{X}+\mathbf{r}''-\mathbf{x})
  v_\|g_j(\mathbf{X})\,d\mathbf{X}\,dv_\|\,d\mu\,d\theta\,d\theta''\\
  &&-\frac{1}{2\pi}\int e_j\frac{B_0}{m_j}\int \delta(\mathbf{X}+\mathbf{r}''-\mathbf{x})
  v_\|\frac{2e_jv_\|}{m_jcv_{Tj}^2}
  F_{0j}\frac{1}{2\pi}\int A_{\|1}(\mathbf{X}+\mathbf{r}')\,d\theta'\,d\mathbf{X}\,dv_\|\,d\mu\,d\theta\,d\theta''\\
  &=& \frac{1}{2\pi}\int\limits_0^{2\pi} e_j
  \frac{B_0}{m_j}\int v_\|g_j(\mathbf{x}-\mathbf{r}'',v_\|,\mu)\,dv_\|\,d\mu\,d\theta\,d\theta''\\
  &&-\frac{2e_j^2}{m_jcv_{Tj}^2}\frac{1}{2\pi}\int\limits_0^{2\pi} \frac{B_0}{m_j}\int \delta(\mathbf{X}+\mathbf{r}''-\mathbf{x})
  v_\|^2 F_{0j}\frac{1}{2\pi}\int\limits_0^{2\pi} 
  A_{\|1}(\mathbf{X}+\mathbf{r}')\,d\theta'\,d\mathbf{X}\,dv_\|\,d\mu\,d\theta\,d\theta''\\
  &=&  2\pi e_j\frac{B_0}{m_j}\int v_\|
  \frac{1}{2\pi}\int\limits_0^{2\pi}g_j(\mathbf{x}-\mathbf{r}'',v_\|,\mu)\,d\theta''\,dv_\|\,d\mu\\
  &&-\frac{2e_j^2}{m_jcv_{Tj}^2}n_{0j} \frac{B_0}{m_j}\int e^{-\mu B_0/T_{0j}}
  \frac{1}{2\pi}\int\limits_0^{2\pi} \delta(\mathbf{X}+\mathbf{r}''-\mathbf{x})
  \frac{1}{2\pi}\int\limits_0^{2\pi} A_{\|1}(\mathbf{X}+\mathbf{r}')\,d\theta'\,d\mathbf{X}\,d\mu\,d\theta''\\
  &=&  2\pi e_j\frac{B_0}{m_j}\int v_\| \mathcal{G}[g_j](\mathbf{x},v_\|,\mu)\,dv_\|\,d\mu
  -\frac{2e_j^2}{m_jcv_{Tj}^2}n_{0j} \frac{B_0}{m_j}\int e^{-\mu B_0/T_{0j}}
  \mathfrak{G}[A_{\|1}](\mathbf{x},\mu)\,d\mu\\
  &=&  2\pi e_j\frac{B_0}{m_j}\int v_\| \mathcal{G}[g_j](\mathbf{x},v_\|,\mu)\,dv_\|\,d\mu
  -\frac{e_j^2}{m_jc} G[A_{\|1}](\mathbf{x})
\end{eqnarray*}
This is put into the field equation
\begin{eqnarray*}
  -\nabla_\bot^2 A_{\|1}(\mathbf{x}) 
  = \frac{4\pi}{c}\sum_j \left[
    2\pi e_j\frac{B_0}{m_j}\int v_\| \mathcal{G}[g_j](\mathbf{x},v_\|,\mu)\,dv_\|\,d\mu
    -\frac{e_j^2}{m_jc} G[A_{\|1}](\mathbf{x})
  \right]
\end{eqnarray*}
Reordering
\begin{eqnarray*}
  -\nabla_\bot^2 A_{\|1}(\mathbf{x}) 
  +\frac{4\pi}{c}\sum_j\frac{e_j^2}{m_jc} G[A_{\|1}](\mathbf{x})
  = \frac{4\pi}{c}\sum_j 2\pi e_j\frac{B_0}{m_j}\int v_\| \mathcal{G}[g_j](\mathbf{x},v_\|,\mu)\,dv_\|\,d\mu
\end{eqnarray*}
Representing the electromagnetic potential in terms of Fourier and
spline basis functions, we can write
\begin{displaymath}
  A_{\|1}(\mathbf{x})=\sum_m A_m(\bar x^1,z)\,e^{i\bar{k}_{2,m}\bar x^2}
  =\sum_m \sum_n \tilde{A}_{nm}(z)\Lambda_n(\bar x^1)\,e^{i\bar{k}_{2,m}\bar x^2}
\end{displaymath}
Now how to calculate the Laplacian? In general coordinates we have for
the $-\nabla_\bot^2$ part
\begin{eqnarray*}
  -\nabla_\bot^2 A_{\|1}(\mathbf{x}) &=& -\left(
    g^{11}\partial_1^2+2g^{12}\partial_1\partial_2+g^{22}\partial_2^2
  \right) \sum_m \sum_n \tilde{A}_{nm}(z)\Lambda_n(\bar x^1)\,e^{i\bar{k}_{2,m}\bar x^2}\\
  &=&-\sum_m \sum_n \tilde{A}_{nm}(z) \left(
    g^{11}\frac{\partial^2\Lambda_n(\bar x^1)}{\partial(\bar x^1)^2}\,
    +2i\bar{k}_{2,m} g^{12}\frac{\partial\Lambda_n(\bar x^1)}{\partial\bar x^1}
    -\Lambda_n(\bar x^1) g^{22}\bar{k}_{2,m}^2
  \right) \,e^{i\bar{k}_{2,m}\bar x^2}
\end{eqnarray*}
Put this result into the field equation, we come to
\begin{multline*}
  -\sum_m \sum_n \tilde{A}_{nm}(z) \left(
    g^{11}\frac{\partial^2\Lambda_n(\bar x^1)}{\partial(\bar x^1)^2}\,
    +2i\bar{k}_{2,m} g^{12}\frac{\partial\Lambda_n(\bar x^1)}{\partial\bar x^1}
    -\Lambda_n(\bar x^1) g^{22}\bar{k}_{2,m}^2
  \right) \,e^{i\bar{k}_{2,m}\bar x^2}\\
  +\frac{4\pi}{c}\sum_j\frac{e_j^2}{m_jc} G[A_{\|1}](\mathbf{x})\\
  = \frac{4\pi}{c}\sum_j 2\pi e_j\frac{B_0}{m_j}\int v_\| \mathcal{G}[g_j](\mathbf{x},v_\|,\mu)\,dv_\|\,d\mu
\end{multline*}
Using now the expressions for the gyro-averaging and the gyro-mapping,
we come to
\begin{multline*}
  -\sum_m \sum_n \tilde{A}_{nm}(z) \left(
    g^{11}\frac{\partial^2\Lambda_n(\bar x^1)}{\partial(\bar x^1)^2}\,
    +2i\bar{k}_{2,m} g^{12}\frac{\partial\Lambda_n(\bar x^1)}{\partial\bar x^1}
    -\Lambda_n(\bar x^1) g^{22}\bar{k}_{2,m}^2
  \right) \,e^{i\bar{k}_{2,m}\bar x^2}\\
  +\frac{4\pi}{c}\sum_j\frac{e_j^2}{m_jc} \sum_m e^{i\bar k_{2,m}\bar x^2} 
  \left[\frac{2n_{0j}}{v_{Tj}^2}\frac{B_0}{m_j}\int \mathcal{P}(\bar{k}_{2,m},z,\mu)
    \,e^{-\mu B_0/T_{0j}}\,d\mu\right] \cdot\tilde{\mathbf{A}}_m\\
  = \frac{4\pi}{c}\sum_j 2\pi e_j \frac{B_0}{m_j}\int v_\| 
  \sum_m e^{i\bar k_{2,m}\bar x^2} \mathcal{M}(\bar{k}_{2,m},z,\mu)\cdot\tilde{\mathbf{g}}_m(z,v_\|,\mu)\,dv_\|\,d\mu
\end{multline*}
Simplifying and reordering leads to
\begin{multline*}
  -\sum_m e^{i\bar{k}_{2,m}\bar x^2} \sum_n \tilde{A}_{nm}(z) \left(
    g^{11}\frac{\partial^2\Lambda_n(\bar x^1)}{\partial(\bar x^1)^2}\,
    +2i\bar{k}_{2,m} g^{12}\frac{\partial\Lambda_n(\bar x^1)}{\partial\bar x^1}
    -\Lambda_n(\bar x^1) g^{22}\bar{k}_{2,m}^2
  \right) \\
  +\sum_m e^{i\bar k_{2,m}\bar x^2} \frac{4\pi}{c}\sum_j\frac{e_j^2}{m_jc} 
  \left[\frac{2n_{0j}}{v_{Tj}^2}\frac{B_0}{m_j}\int \mathcal{P}(\bar{k}_{2,m},z,\mu)
    \,e^{-\mu B_0/T_{0j}}\,d\mu\right] \cdot\tilde{\mathbf{A}}_m\\
  = \sum_m e^{i\bar k_{2,m}\bar x^2}\frac{4\pi}{c}\sum_j 2\pi e_j \frac{B_0}{m_j}\int v_\| 
   \mathcal{M}(\bar{k}_{2,m},z,\mu)\cdot\tilde{\mathbf{g}}_m(z,v_\|,\mu)\,dv_\|\,d\mu
\end{multline*}
Two Fouriersums are only equivalent if all Fourier coefficients are
equal. Hence, we come to
\begin{multline*}
  -\sum_n \tilde{A}_{nm}(z) \left(
    g^{11}\frac{\partial^2\Lambda_n(\bar x^1)}{\partial(\bar x^1)^2}\,
    +2i\bar{k}_{2,m} g^{12}\frac{\partial\Lambda_n(\bar x^1)}{\partial\bar x^1}
    -\Lambda_n(\bar x^1) g^{22}\bar{k}_{2,m}^2
  \right) \\
  + \frac{4\pi}{c}\sum_j\frac{e_j^2}{m_jc} 
  \left[\frac{2n_{0j}}{v_{Tj}^2}\frac{B_0}{m_j}\int \mathcal{P}(\bar{k}_{2,m},z,\mu)
    \,e^{-\mu B_0/T_{0j}}\,d\mu\right] \cdot\tilde{\mathbf{A}}_m\\
  = \frac{4\pi}{c}\sum_j 2\pi e_j \frac{B_0}{m_j}\int v_\| 
   \mathcal{M}(\bar{k}_{2,m},z,\mu)\cdot\tilde{\mathbf{g}}_m(z,v_\|,\mu)\,dv_\|\,d\mu
\end{multline*}
Furthermore we can write the Laplacian as a matrix $\mathcal{D}(\bar
k_{2,m},z)$ which is defined as
\begin{displaymath}
  \mathcal{D}_{rn}(\bar k_{2,m},z) = 
  g^{11}(z)\frac{\partial^2\Lambda_n(\bar x^1)}{\partial(\bar x^1)^2}\Bigg|_{\bar x^1_r}
  +2i\bar{k}_{2,m} g^{12}(z)\frac{\partial\Lambda_n(\bar x^1)}{\partial\bar x^1}\Bigg|_{\bar x^1_r}
  -\Lambda_n(\bar x^1_r) g^{22}(z)\bar{k}_{2,m}^2.
\end{displaymath}
With this we can write the field equation 
\begin{multline*}
  -\mathcal{D}(\bar k_{2,m},z)\cdot \tilde{\mathbf{A}}_m(z)
  + \frac{4\pi}{c}\sum_j\frac{e_j^2}{m_jc} 
  \left[\frac{2n_{0j}}{v_{Tj}^2}\frac{B_0}{m_j}\int \mathcal{P}(\bar{k}_{2,m},z,\mu)
    \,e^{-\mu B_0/T_{0j}}\,d\mu\right] \cdot\tilde{\mathbf{A}}_m\\
  = \frac{4\pi}{c}\sum_j 2\pi e_j \frac{B_0}{m_j}\int v_\| 
   \mathcal{M}(\bar{k}_{2,m},z,\mu)\cdot\tilde{\mathbf{g}}_m(z,v_\|,\mu)\,dv_\|\,d\mu
\end{multline*}
And at last we combine the left hand side to have
\begin{multline*}
  \left[
    -\mathcal{D}(\bar k_{2,m},z)
    + \frac{4\pi}{c}\sum_j\frac{e_j^2}{m_jc} \frac{2n_{0j}}{v_{Tj}^2}\frac{B_0}{m_j}\int \mathcal{P}(\bar{k}_{2,m},z,\mu)
    \,e^{-\mu B_0/T_{0j}}\,d\mu
  \right] \cdot\tilde{\mathbf{A}}_m(z)\\
  = \frac{4\pi}{c}\sum_j 2\pi e_j \frac{B_0}{m_j}\int v_\| 
   \mathcal{M}(\bar{k}_{2,m},z,\mu)\cdot\tilde{\mathbf{g}}_m(z,v_\|,\mu)\,dv_\|\,d\mu
\end{multline*}


\subsubsection{Normalization}
\label{sec:norm_amp}
This second field equation has also to be normalized. We use the
normalization, defined in Section~\ref{sec:normalization}.
\begin{multline*}
  \frac{\rho_\rf}{L_\rf}B_\rf\rho_\rf\left[
    -\frac{1}{\rho_\rf^2}\hat{\mathcal{D}}(\bar k_{2,m},z)
    +\frac{4\pi}{c}\sum_j\frac{e_j^2}{m_jc} \frac{2n_{0j}}{v_{Tj}^2}
    \frac{\refj{T}}{B_\rf}\frac{B_0}{m_j}\int \mathcal{P}(\bar{k}_{2,m},z,\mu)
    \,e^{-\hat{\mu} B_0/B_\rf\,\refj{T}/T_{0j}}\,d\hat{\mu}
  \right] \cdot\hat{\tilde{\mathbf{A}}}_m(z)\\
  = \frac{\rho_\rf}{L_\rf}\frac{4\pi}{c}
  \sum_j 2\pi e_j \frac{\refj{n}}{v_{0j}^3}\frac{B_0}{m_j}v_{0j}^2\frac{\refj{T}}{B_\rf} \int \hat{v}_\| 
   \mathcal{M}(\bar{k}_{2,m},z,\mu)\cdot\hat{\tilde{\mathbf{g}}}_m(z,v_\|,\mu)\,d\hat{v}_\|\,d\hat{\mu}
\end{multline*}
Simplifications lead to
\begin{multline*}
  \left[
    -\hat{\mathcal{D}}(\bar k_{2,m},z)
    +\frac{\beta}{2}\sum_j\frac{e_j^2}{e^2}\frac{n_{0j}}{n_\mathrm{ref}}\frac{m_\rf}{m_j} 
    \frac{\refj{T}}{T_{0j}}\frac{B_0}{B_\rf}\int \mathcal{P}(\bar{k}_{2,m},z,\mu)
    \,e^{-\hat{\mu} B_0/B_\rf\,\refj{T}/T_{0j}}\,d\hat{\mu}
  \right] \cdot\hat{\tilde{\mathbf{A}}}_m(z)\\
  = \frac{\beta}{2}\sum_j \frac{e_j}{e} \frac{v_{0j}}{c_\rf}\pi\frac{B_0}{B_\rf}\int \hat{v}_\| 
   \mathcal{M}(\bar{k}_{2,m},z,\mu)\cdot\hat{\tilde{\mathbf{g}}}_m(z,v_\|,\mu)\,d\hat{v}_\|\,d\hat{\mu}
\end{multline*}
with $\beta = \frac{8\pi n_\rf T_\rf}{B_\rf^2}$.

\subsubsection{Efficient implementation}
The operator on the left hand side of the last field equation can be
calculated in advance, to give a combined matrix operator
$\mathcal{C}(\bar{k}_{2,m},z)$ which not any more species dependent,
as it is a sum over all species. The field equation is then
\begin{displaymath}
  \mathcal{C}(\bar{k}_{2,m},z) \cdot\hat{\tilde{\mathbf{A}}}_m(z)
  = \frac{\beta}{2}\sum_j \frac{e_j}{e} \frac{v_{0j}}{c_\rf}\pi\frac{B_0}{B_\mathrm{ref}}\int \hat{v}_\| 
  \mathcal{M}(\bar{k}_{2,m},z,\mu)\cdot\hat{\tilde{\mathbf{g}}}_m(z,v_\|,\mu)\,d\hat{v}_\|\,d\hat{\mu}
\end{displaymath}
This combined matrix operator can be brought to the right hand side by
multiplying with the inverse from the left. As the operator is species
and $\mu$ independent, we can combine it further with the matrix
operator on the right hand side $\mathcal{M}$. 
\begin{displaymath}
  \hat{\tilde{\mathbf{A}}}_m(z)
  = \frac{\beta}{2}\sum_j \frac{e_j}{e} \frac{v_{0j}}{c_\rf}\pi\frac{B_0}{B_\mathrm{ref}}\int \hat{v}_\| 
  \mathcal{C}(\bar{k}_{2,m},z)^{-1}\cdot\mathcal{M}(\bar{k}_{2,m},z,\mu)\cdot\hat{\tilde{\mathbf{g}}}_m(z,v_\|,\mu)\,d\hat{v}_\|\,d\hat{\mu}
\end{displaymath}
The then whole matrix operator is independent of $v_\|$, so we can
first do the integration over $v_\|$, then let the matrix operate on
the result and then do the $\mu$ integration.
\begin{displaymath}
  \hat{\tilde{\mathbf{A}}}_m(z)
  = \frac{\beta}{2}\sum_j \frac{e_j}{e} \frac{v_{0j}}{c_\rf}\frac{B_0}{B_\mathrm{ref}}\int
  \mathcal{C}(\bar{k}_{2,m},z)^{-1}\cdot\mathcal{M}(\bar{k}_{2,m},z,\mu)\cdot\left[
    \pi\int\hat{v}_\|\hat{\tilde{\mathbf{g}}}_m(z,v_\|,\mu)\,d\hat{v}_\|\right]\,d\hat{\mu}
\end{displaymath}


\subsection{Adiabatic electrons}
\label{sec:adiabatic_electrons}
The two field equations derived in the to preceeding sections are
valid for two species simulations. If we want to treat the electrons
adiabatically, we have to replace the expression for the perturbed
electron density by the following unnormalized expression:
\begin{displaymath}
  \frac{n_{1\mathrm{e}}}{n_{0\mathrm{e}}(x)} 
  = \frac{e\Phi}{T_{0\mathrm{e}}(x)}-\frac{e}{T_{0\mathrm{e}}(x)}\langle\Phi\rangle
\end{displaymath}
where $\langle\cdot\rangle$ is the flux surface average of the argument.
Using again the quasineutrality condition
\begin{eqnarray*}
  0&=&\sum_j e_jn_{1j}
  =-en_{1\mathrm{e}}+\sum_{j=\mbox{ions}}e_jn_{1j}\\
  &=&-\frac{e^2}{T_{0\mathrm{e}}(x)}n_{0\mathrm{e}}(x)\left[
    \Phi
    -\langle\Phi\rangle
  \right]
  +\sum_{j}e_j\left[
    \frac{B_0}{m_j}\int \mathcal{G}[F_{j1}](\mathbf{x},v_\|,\mu)\,dv_\|\,d\mu\,d\theta
    -\frac{e_j}{T_{0j}}\left(
      \Phi(\mathbf{x}) n_{0j}
      -G[\Phi](\mathbf{x})
    \right) 
  \right]
\end{eqnarray*}
We again reorder to have all field dependent parts on the left hand
side and the parts dependent on the distribution function on the right
hand side.
\begin{eqnarray*}
  \frac{e^2}{T_{0\mathrm{e}}}n_{0\mathrm{e}}\left[
    \Phi
    -\langle\Phi\rangle
  \right] 
  +\sum_{j}\frac{e_j^2}{T_{0j}}\left(
    \Phi(\mathbf{x}) n_{0j}
    -G[\Phi](\mathbf{x})
  \right) 
  &=& \sum_{j}e_j\frac{B_0}{m_j}\int \mathcal{G}[F_{j1}](\mathbf{x},v_\|,\mu)\,dv_\|\,d\mu\,d\theta
\end{eqnarray*}
The same procedure as for the nonadiabatic case is followed and in the
end we come to
\begin{multline*}
  \frac{e^2}{T_{0\mathrm{e}}}n_{0\mathrm{e}}\mathcal{L}\cdot\left[
    \tilde{\mathbf{\Phi}}(k_m,z)
    -\tilde{\mathbf{\langle\Phi\rangle}}
  \right] \\
  +\sum_{j}\frac{e_j^2}{T_{0j}}n_{0j}\left(
    \mathcal{L} 
    -\frac{B_0}{T_{0j}}\int \mathcal{P}(\bar{k}_{2,m},z,\mu)
    \,e^{-\mu B_0/T_{0j}}\,d\mu
  \right)\cdot\tilde{\mathbf{\Phi}}(k_m,z) \\
  = \sum_j e_j 2\pi\frac{B_0}{m_j}\int 
  \mathcal{M}(\bar{k}_{2,m},z,\mu)\cdot\tilde{\mathbf{F}}_m(z,v_\|,\mu)\,dv_\|\,d\mu
\end{multline*}
Normalization yields to
\begin{multline*}
  \frac{T_\rf}{e}e^2\frac{n_{0\mathrm{e}}}{T_{0\mathrm{e}}} \mathcal{L}\cdot\left[
    \hat{\tilde{\mathbf{\Phi}}}(k_m,z)-\hat{\tilde{\mathbf{\langle\Phi\rangle}}}\right]\\
  +\frac{T_\rf}{e} e^2 \sum_{j}\frac{e_j^2}{e^2}\frac{n_{0j}}{T_{0j}}\left(
    \mathcal{L}-\frac{B_0}{B_\rf}\frac{\refj{T}}{T_{0j}}\int \mathcal{P}(\bar{k}_{2,m},z,\mu)
    \,e^{-\hat{\mu} B_0/B_\rf \refj{T}/T_{0j}}\,d\hat{\mu}
  \right)\cdot\hat{\tilde{\mathbf{\Phi}}}(k_m,z) \\
  = \sum_j e_j 2\pi\frac{B_0}{m_j}\frac{\refj{n}}{v_{0j}^2}\frac{\refj{T}}{B_\rf}\int 
  \mathcal{M}(\bar{k}_{2,m},z,\mu)\cdot\hat{\tilde{\mathbf{F}}}_m(z,v_\|,\mu)\,d\hat{v}_\|\,d\hat{\mu}
\end{multline*} 
which can be simplified to
\begin{multline}
  \label{eq:qnfull}
  \frac{n_{0\mathrm{e}}}{n_\rf} \frac{T_\rf}{T_{0\mathrm{e}}} \mathcal{L}\cdot\left[
    \hat{\tilde{\mathbf{\Phi}}}(k_m,z)-\hat{\tilde{\mathbf{\langle\Phi\rangle}}}\right]
  +\mathcal{P}_{\mathrm{ions}}''\cdot\hat{\tilde{\mathbf{\Phi}}}(k_m,z) \\
  = \sum_j \frac{e_j}{e} \frac{\refj{n}}{n_\rf} \pi\hat{B}\int 
  \mathcal{M}(\bar{k}_{2,m},z,\mu)\cdot\hat{\tilde{\mathbf{F}}}_m(z,v_\|,\mu)\,d\hat{v}_\|\,d\hat{\mu}
\end{multline} 
using a similar abbreviation as before
\begin{multline*}
    \mathcal{P}_{\mathrm{ions}}'' = \sum_{j=\mathrm{ions}}\frac{e_j^2}{e^2}
  \diag\{\frac{n_{0j}(x)}{n_\rf}\frac{T_\rf}{T_{0j}(x)}\}\cdot\\
  \left(\mathcal{L}
    - \int \diag\{\frac{B_0(x,z)}{B_\rf}\frac{\refj{T}}{T_{0j}(x)} 
    e^{-\hat{\mu}\refj{T}/T_{0j}(x)\, B_0(x,z)/B_\rf}\} \cdot 
    \mathcal{P}(\bar{k}_{2,m},z,\mu)\,d\hat{\mu}\right).
\end{multline*}
To get an expression for the flux surface average of the electrostatic
potential, we calculate the flux surface average of the whole
equation which gives
\begin{multline*}
\left\langle\mathcal{P}_{\mathrm{ions}}''\cdot\hat{\tilde{\mathbf{\Phi}}}(k_m,z)\right\rangle
  = \left\langle\sum_j \frac{e_j}{e} \frac{\refj{n}}{n_\rf} \pi\hat{B}\int 
  \mathcal{M}(\bar{k}_{2,m},z,\mu)\cdot\hat{\tilde{\mathbf{F}}}_m(z,v_\|,\mu)\,d\hat{v}_\|\,d\hat{\mu}\right\rangle
\end{multline*} 
As this result does not yet lead to an expression for
$\langle\Phi\rangle$, we now make the assumption 
\begin{displaymath}
  \left\langle\mathcal{P}_{\mathrm{ions}}''\cdot\tilde{\mathbf{\Phi}}(k_m,z)
  \right\rangle=\left\langle\mathcal{P}_{\mathrm{ions}}''\right\rangle\cdot\left\langle\tilde{\mathbf{\Phi}}(k_m,z)\right\rangle
\end{displaymath}
This leads then to 
\begin{displaymath}
\left\langle\mathcal{P}_{\mathrm{ions}}''\right\rangle\cdot\left\langle\hat{\tilde{\mathbf{\Phi}}}(k_m,z)\right\rangle
  = \left\langle\sum_j \frac{e_j}{e} \frac{\refj{n}}{n_\rf} \pi\hat{B}\int 
  \mathcal{M}(\bar{k}_{2,m},z,\mu)\cdot\hat{\tilde{\mathbf{F}}}_m(z,v_\|,\mu)\,d\hat{v}_\|\,d\hat{\mu}\right\rangle
 \end{displaymath}
which can be solved for $\langle\Phi\rangle$.
Having $\langle\Phi\rangle$, we can now rewrite the quasineutrality equation eq.~(\ref{eq:qnfull}) and get
\begin{eqnarray*}
  \left[\frac{n_{0\mathrm{e}}}{n_\rf} \frac{T_\rf}{T_{0\mathrm{e}}} \mathcal{L}
  +\mathcal{P}_{\mathrm{ions}}''\right]\cdot \hat{\tilde{\mathbf{\Phi}}}(k_m,z)    
& = & \sum_j \frac{e_j}{e} \frac{\refj{n}}{n_\rf} \pi\hat{B}\int 
  \mathcal{M}(\bar{k}_{2,m},z,\mu)\cdot\hat{\tilde{\mathbf{F}}}_m(z,v_\|,\mu)\,d\hat{v}_\|\,d\hat{\mu} \\
& &  +\frac{n_{0\mathrm{e}}}{n_\rf} \frac{T_\rf}{T_{0\mathrm{e}}} \mathcal{L}\cdot
   \hat{\tilde{\mathbf{\langle\Phi\rangle}}}
\end{eqnarray*} 
which can be solved either by iteration or inversion.

\section{Base functions in radial direction}
\label{sec:basefunc}
Until now, the base functions, that we used in $x$ direction where
totally general. We just assumed that one can represent all functions
in series of these base functions. One possibility of base functions
are splines, but for splines, the coefficients and the grid values of
the function are different, so that one needs to solve a linear system
of equations to come from the grid values to the coefficients. This
system is expensive to solve on a parallelized radial grid. So we are
looking for other set of base functions, where the point values and
the coefficients of the base function representation are
equal. 

Instead of extending the support of the polynomial (as done for
the splines), one could also keep the support to only two grid
cells. If one uses only polynoms of degree one (linear interpolation),
we come to the hat function, but in principle one can use all degrees,
but then one has to impose some more conditions on the base functions
to determine all coefficients of the polynoms.
The function is represented as
\begin{displaymath}
  f(x)=\sum_n\left[ f_n H_n(x) + f_n' G_n(x)\right]
\end{displaymath}
where the values of the function at the grid points together with the
values of the derivatives of the function at the grid points are
known. Then we have the conditions for the two polynomial functions
$H_n(x)$ and $G_n(x)$:
\begin{eqnarray*}
  H_n(x_j)=\delta_{nj}\quad\mbox{and}\quad G_n(x_j)=0\\
  H'_n(x_j)=0\quad\mbox{and}\quad G'_n(x_j)=\delta_{nj}
\end{eqnarray*}

These four conditions lead to the possibility to use cubic
polynoms. So making the ansatz
\begin{displaymath}
  H_n(x)=a_3x^3+a_2x^2+a_1x+a_0\qquad G_n(x)=b_3x^3+b_2x^2+b_1x+b_0
\end{displaymath}
we come to the following linear system of equations for the two different
sides a three point interval $[x_{j-1},x_{j+1}]$.
\begin{align*}
  x_{j\pm1}^3 &a_3 &+x_{j\pm1}^2&a_2 &+x_{j\pm1}&a_1&+a_0 &=&0\\
  x_{j}^3   &a_3 &+x_{j}^2&a_2  &+x_{j}&a_1&+a_0 &=&1\\
  3x_{j\pm1}^2&a_3 &+2x_{j\pm1}&a_2&+&a_1 &&=& 0\\
  3x_j^2&a_3    &+2x_j&a_2&+&a_1& &=& 0
\end{align*}
We now normalize the interval to the interval $[-\Delta x_n,0]$ for the left
half and $[0,\Delta x_n]$ for the right half. The system of equations then
become
\begin{align*}
  \pm\Delta x_n^3 &a_3 &+\Delta x_n^2&a_2 &\pm\Delta x_n&a_1&+a_0 &=&0\\
   & &&  &&&a_0 &=&1\\
  3\Delta x_n^2&a_3 &\pm2\Delta x_n&a_2&+&a_1 &&=& 0\\
  &    &&&&a_1& &=& 0
\end{align*}


These four equations for four unknows do have a solution for the coefficients
of the polynomial function.
\begin{eqnarray*}
  a_3 = -\frac{2}{(x_j-x_{j\pm1})^3}
  \qquad
  a_2 = \frac{3 (x_j+x_{j\pm1})}{(x_j-x_{j\pm1})^3}
  \qquad
  a_1 = \frac{6 x_j x_{j\pm1}}{(x_{j\pm1}-x_j)^3}
  \qquad
  a_0 = \frac{x_{j\pm1}^2 (x_{j\pm1}-3x_j)}{(x_{j\pm1}-x_j)^3}
\end{eqnarray*}
One therefore gets a cubic polynom for the left and one for the right half of
the above defined interval.
For an equally spaced $x$-grid, we can write for the left half $[x_{j-1},x_j]$
\begin{eqnarray*}
  a_3 = -\frac{2}{\Delta x^3}
  \qquad
  a_2 = \frac{3 (x_j+x_{j-1})}{\Delta x^3}
  \qquad
  a_1 = -\frac{6 x_j x_{j-1}}{\Delta x^3}
  \qquad
  a_0 = -\frac{x_{j-1}^2 (x_{j-1}-3x_j)}{\Delta x^3}
\end{eqnarray*}


If one want to use higher degree polynoms, we have to use
also higher derivatives. So we take into account the second derivative
and find
\begin{displaymath}
  f(x)=\sum_n\left[ f_n H_n(x) + f_n' G_n(x) + f_n'' Q_n\right]
\end{displaymath}
This lead to the conditions
\begin{eqnarray*}
  H_n(x_j)=\delta_{nj}\qquad G_n(x_j)=0\qquad Q_n(x_j)=0\\
  H'_n(x_j)=0\qquad G'_n(x_j)=\delta_{nj}\qquad Q'_n(x_j)=0\\
  H''_n(x_j)=0\qquad G''_n(x_j)=0\qquad Q''_n(x_j)=\delta_{nj}
\end{eqnarray*}



\section{Moments}
The definition of the moments is (taken for example from my fast
particle paper) the following but now we do not use the Bessel
functions but try to write the moments expressions in terms of
averaging $\langle\cdot\rangle$ operators. Again we use $r=2q$ as we
only use even moments of $v_\bot$. $j$ is the species index.
\begin{eqnarray*}
  M_{kr}(\mathbf{x}) &=& \int v_\|^kv_\bot^{2q} f_1(\mathbf{x})\,d^3v
  =\frac{2^qB_0^{q+1}}{m_j^{q+1}}\int v_\|^k \mu^q f_1(\mathbf{x})\,dv_\|\,d\mu\,d\theta\\
  &=&\frac{2^qB_0^{q+1}}{m_j^{q+1}}
  \int \delta(\mathbf{X}+\mathbf{r}-\mathbf{x}) v_\|^k \mu^q T^*F_1(\mathbf{X})\,d\mathbf{X}\,dv_\|\,d\mu\,d\theta\\
  &=&\frac{2^qB_0^{q+1}}{m_j^{q+1}}
  \int v_\|^k \mu^q
  T^*F_1(\mathbf{x}-\mathbf{r})\,dv_\|\,d\mu\,d\theta
\end{eqnarray*}
Now we want to do the gyroaveraging which gives the following
expression
\begin{eqnarray*}
  M_{kr}(\mathbf{x}) &=&\frac{2^qB_0^{q+1}}{m_j^{q+1}}
  \int v_\|^k \mu^q \langle T^*F_1(\mathbf{x}-\mathbf{r})\rangle\,dv_\|\,d\mu\,d\theta\\
  &=&2\pi\frac{2^qB_0^{q+1}}{m_j^{q+1}}
  \int v_\|^k \mu^q \Bigg[
    \left\langle
      F_1(\mathbf{x}-\mathbf{r})
    \right\rangle
    +\left\langle
      \left(
        \frac{e_j}{m_jc}\frac{\partial F_0}{\partial v_\|}
        -\frac{e_jv_\|}{cB_0}\frac{\partial F_0}{\partial \mu}
      \right)\left(
        A_{\|1}(\mathbf{x})
        -\bar{A}_{\|1}(\mathbf{x}-\mathbf{r})
      \right)
    \right\rangle\\
    &&+\left\langle
      \frac{e_j}{B_0}\frac{\partial F_0}{\partial\mu}\left(
        \Phi_1(\mathbf{x})
        -\bar{\Phi}_1(\mathbf{x}-\mathbf{r})
      \right)
    \right\rangle
  \Bigg]\,dv_\|\,d\mu\\
  &=&2\pi\frac{2^qB_0^{q+1}}{m_j^{q+1}}
  \int v_\|^k \mu^q \Bigg[
    \left\langle
      F_1(\mathbf{x}-\mathbf{r})
    \right\rangle
    +\left(
      \frac{e_j}{m_jc}\frac{\partial F_0}{\partial v_\|}
      -\frac{e_jv_\|}{cB_0}\frac{\partial F_0}{\partial \mu}
    \right)\left\langle
      A_{\|1}(\mathbf{x})
      -\bar{A}_{\|1}(\mathbf{x}-\mathbf{r})
    \right\rangle\\
    &&+\frac{e_j}{B_0}\frac{\partial F_0}{\partial\mu}\left\langle
      \left(
        \Phi_1(\mathbf{x})
        -\bar{\Phi}_1(\mathbf{x}-\mathbf{r})
      \right)
    \right\rangle
  \Bigg]\,dv_\|\,d\mu
\end{eqnarray*}
The last expression can now be written in terms of the operators
defined earlier.
\begin{eqnarray*}
  M_{kr}(x_1,x_2,x_3) &=&2\pi\frac{2^qB_0^{q+1}}{m_j^{q+1}}
  \int v_\|^k \mu^q \Bigg[
    \sum_m e^{ik_mx_2}\sum_nF_{n,m}(x_3)\mathcal{M}_{n,m}(\rho)\\
    &&+\left(
      \frac{e_j}{m_jc}\frac{\partial F_0}{\partial v_\|}
      -\frac{e_jv_\|}{cB_0}\frac{\partial F_0}{\partial \mu}
    \right)
    \sum_m e^{ik_mx_2}\sum_n A_{\|,n,m}(x_3)\left(
      \Lambda_n(x_1)-\mathcal{P}_{n,m}(\rho)
    \right)\\
    &&+\frac{e_j}{B_0}\frac{\partial F_0}{\partial\mu}
    \sum_me^{ik_mx_2}\sum_n\Phi_{n,m}(x_3)\left(
      \Lambda_n(x_1)-\mathcal{P}_{n,m}(\rho)
    \right)
  \Bigg]\,dv_\|\,d\mu\\
  &=&\sum_m e^{ik_mx_2} \,2\pi\frac{2^qB_0^{q+1}}{m_j^{q+1}}
  \int v_\|^k \mu^q \Bigg[
    \sum_nF_{n,m}(x_3)\mathcal{M}_{n,m}(\rho)\\
    &&+\left(
      \frac{e_j}{m_jc}\frac{\partial F_0}{\partial v_\|}
      -\frac{e_jv_\|}{cB_0}\frac{\partial F_0}{\partial \mu}
    \right)
    \sum_n A_{\|,n,m}(x_3)\left(
      \Lambda_n(x_1)-\mathcal{P}_{n,m}(\rho)
    \right)\\
    &&+\frac{e_j}{B_0}\frac{\partial F_0}{\partial\mu}
    \sum_n\Phi_{n,m}(x_3)\left(
      \Lambda_n(x_1)-\mathcal{P}_{n,m}(\rho)
    \right)
  \Bigg]\,dv_\|\,d\mu\\
\end{eqnarray*}
Fourier transforming the left hand side also, we get an expression for
each Fourier mode of the flux
\begin{eqnarray*}
  M_{kr}(x_1,k_m,x_3) &=& 2\pi\frac{2^qB_0^{q+1}}{m_j^{q+1}}
  \int v_\|^k \mu^q \Bigg[
    \sum_nF_{n,m}(x_3)\mathcal{M}_{n,m}(\rho)\\
    &&+\left(
      \frac{e_j}{m_jc}\frac{\partial F_0}{\partial v_\|}
      -\frac{e_jv_\|}{cB_0}\frac{\partial F_0}{\partial \mu}
    \right)
    \sum_n A_{\|,n,m}(x_3)\left(
      \Lambda_n(x_1)-\mathcal{P}_{n,m}(\rho)
    \right)\\
    &&+\frac{e_j}{B_0}\frac{\partial F_0}{\partial\mu}
    \sum_n\Phi_{n,m}(x_3)\left(
      \Lambda_n(x_1)-\mathcal{P}_{n,m}(\rho)
    \right)
  \Bigg]\,dv_\|\,d\mu
\end{eqnarray*}

\subsubsection{Normalization}
Using the normalization described in Section~\ref{sec:normalization}
\begin{eqnarray*}
  F_{0j}(x^1,x^3,v_\|,\mu) &=& \frac{\refj{n}}{v_{0j}^3}\hat{F}_{0j}(x^1,x^3,v_\|,\mu)\\
  F_{n,m}(x^3,v_\|,\mu) &=&
  \frac{\refj{n}}{v_{0j}^3}\frac{\rho_\rf}{L_\rf}\hat{F}_{n,m}(x^3,v_\|,\mu)\\
  A_{\|,n,m}(x^3) &=& \rho_\rf B_\rf \frac{\rho_\rf}{L_\rf}\hat{A}_{\|,n,m}(x^3)\\
  \Phi_{n,m}(x^3) &=& \frac{\rho_\rf}{L_\rf} \frac{T_\rf}{e} \hat{\Phi}_{n,m}(x^3)
\end{eqnarray*}
we arrive at
\begin{eqnarray}
  M_{kr}(x_1,k_m,x_3) &=& 2\pi\frac{2^qB_0^{q+1}}{m_j^{q+1}} v_{0j}^{k+1}
  \frac{\refj{T}^{q+1}}{B_\rf^{q+1}}
  \int \hat{v}_\|^k \hat{\mu}^q \sum_n \Bigg[
     \frac{\refj{n}}{v_{0j}^3} \frac{\rho_\rf}{L_\rf}\hat{F}_{n,m}(x_3)\mathcal{M}_{n,m}(\rho) \nn \\
    &&+\frac{e_j}{m_jc} \frac{\refj{n}}{v_{0j}^4} \left(
      \frac{\partial \hat{F}_0}{\partial \hat{v}_\|}
      -\frac{m_j v_{0j}^2}{B_\rf}\frac{B_\rf}{\refj{T}} \frac{\hat{v}_\|}{\hat{B}}\frac{\partial 
        \hat{F}_0}{\partial \hat{\mu}}
    \right)
    B_\rf \frac{\rho_\rf^2}{L_\rf} \hat{A}_{\|,n,m}(x_3)\left(
      \Lambda_n(x_1)-\mathcal{P}_{n,m}(\rho) \right) \nn \\
    &&+\frac{e_j}{B_0} \frac{\refj{n}}{v_{0j}^3}\frac{B_\rf}{\refj{T}} \frac{\rho_\rf}{L_\rf} \frac{T_\rf}{e}
    \frac{\partial\hat{F}_0}{\partial\hat{\mu}}
    \hat{\Phi}_{n,m}(x_3)\left(
      \Lambda_n(x_1)-\mathcal{P}_{n,m}(\rho)
    \right)
  \Bigg]\,d\hat{v}_\|\,d\hat{\mu} \nn \\
& = & \refj{n} \frac{\rho_\rf}{L_\rf} v_{0j}^{2q+k} \pi\hat{B}^{q+1} 
   \int \hat{v}_\|^k \hat{\mu}^q \sum_n \Bigg[
     \hat{F}_{n,m}(x_3)\mathcal{M}_{n,m}(\rho)\nn \\
     &&+\frac{\hat{e}_j c_\rf}{\hat{m}_j v_{0j}} \left(
       \frac{\partial \hat{F}_0}{\partial \hat{v}_\|}
       -2\frac{\hat{v}_\|}{\hat{B}}\frac{\partial\hat{F}_0}{\partial \hat{\mu}}
     \right)
     \hat{A}_{\|,n,m}(x_3)\left(\Lambda_n(x_1)-\mathcal{P}_{n,m}(\rho)
     \right) \nn \\
     &&+\frac{\hat{e}_j}{\hat{B}}\frac{T_\rf}{\refj{T}}
     \frac{\partial\hat{F}_0}{\partial\hat{\mu}}
     \hat{\Phi}_{n,m}(x_3)\left(
       \Lambda_n(x_1)-\mathcal{P}_{n,m}(\rho)
     \right)
   \Bigg]\,d\hat{v}_\|\,d\hat{\mu} \nn \\
& \equiv & \left\{\refj{n} \frac{\rho_\rf}{L_\rf} v_{0j}^{2q+k}\right\} \cdot \hat{M}_{kr} \label{eq:mom_norm}
\end{eqnarray}
 
\subsubsection{Symmetric Maxwellian as equilibrium distribution function}
The moment calculation can be further simplified by assuming an unshifted maxwellian
as equilibrium distribution function, e.g. 
$F_{0j}=\pi^{-\frac{3}{2}} \frac{n_{0j}}{v_{T_j}^3}{\rm e}^{-\frac{m_jv_\parallel^2/2+\mu B_0}{T_{0j}}}$.
Now, Eq.~(\ref{eq:mom_norm}) reads
\begin{eqnarray}
\hat{M}_{kr} & = & \pi\hat{B}^{q+1}
   \int \hat{v}_\|^k \hat{\mu}^q \sum_n \Bigg[
     \hat{F}_{n,m}(x_3)\mathcal{M}_{n,m}(\rho)\nn \\
      &&-\frac{\hat{e}_j}{\Tpj}\frac{T_\rf}{\refj{T}}\hat{F}_0
     \hat{\Phi}_{n,m}(x_3)\left(
       \Lambda_n(x_1)-\mathcal{P}_{n,m}(\rho)
     \right)
   \Bigg]\,d\hat{v}_\|\,d\hat{\mu}. \nn
\end{eqnarray}
The second term can be further evaluated using 
\begin{eqnarray*}
\Upsilon(k)=\frac{1}{\sqrt{\pi}}\int_{-\infty}^{\infty}x^k {\rm e}^{-x^2} dx =
\begin{cases} 
0, & k \mbox{ odd} \\ 
1, & k=0 \\
\frac{1\cdot3\cdots(k-1)}{\sqrt{2}^k} & k \mbox{ even}
\end{cases}
\end{eqnarray*}
to calculate the $v_\parallel$-integral analytically
\begin{eqnarray*}
\int \hat{v}^k_\| \hat{F}_0(\hat{v}_\|)\,d\hat{v}_\|
& = & \frac{{\rm e}^{-\hat{\mu}\frac{\hat{B}}{\hat{T}_{0j}}}}
  {(\pi \Tpj)^{\frac{3}{2}}} \npj \Tpj^{\frac{k}{2}}
  \int_{-\infty}^{\infty}\frac{\hat{v}^k_\|}{\Tpj^{\frac{k}{2}}} {\rm e}^{-\frac{\hat{v}^2_\|}
  {\Tpj}}\,d\hat{v}_\| \nn \\
& = & \frac{1}{\pi} {\rm e}^{-\hat{\mu}\frac{\hat{B}}{\Tpj}} 
  \npj \Tpj^{\frac{k}{2}-1} \Upsilon(k).
\end{eqnarray*}
We now have
\begin{eqnarray}
\hat{M}_{kr} & = & \pi\hat{B}^{q+1}  \sum_n \Bigg[
   \int \hat{v}_\|^k \hat{\mu}^q
     \hat{F}_{n,m}(x_3)\mathcal{M}_{n,m}(\rho)\,d\hat{v}_\|\,d\hat{\mu} \nn \\
      &&-\frac{\hat{e}_j}{\pi} \npj \Tpj^{\frac{k}{2}-2}\frac{T_\rf}{\refj{T}} 
      \Upsilon(k) \hat{\Phi}_{n,m}(x_3) 
      \int \hat{\mu}^q {\rm e}^{-\hat{\mu}\frac{\hat{B}}{\hat{T}_{0j}}} 
     \left(\Lambda_n(x_1)-\mathcal{P}_{n,m}(\rho)
     \right)\,d\hat{\mu}\Bigg] \nn
\end{eqnarray}
which can be simplified to
\begin{eqnarray}
\hat{M}_{kr} & = & \pi\hat{B}^{q+1}  \sum_n \Bigg[
   \int \hat{v}_\|^k \hat{\mu}^q
     \hat{F}_{n,m}(x_3)\mathcal{M}_{n,m}(\rho)\,d\hat{v}_\|\,d\hat{\mu} \nn \\
      &&-\frac{\hat{e}_j}{\pi}\frac{T_\rf}{\refj{T}} \npj\Tpj^{\frac{k}{2}-2} \Upsilon(k) q! 
      \left(\frac{\Tpj}{\hat{B}}\right)^{q+1}
      \hat{\Phi}_{n,m}(x_3) \Lambda_n(x_1)\\
      &&+\frac{\hat{e}_j}{\pi}\frac{T_\rf}{\refj{T}} \npj\Tpj^{\frac{k}{2}-2} 
      \Upsilon(k) \hat{\Phi}_{n,m}(x_3) 
      \int \hat{\mu}^q {\rm e}^{-\hat{\mu}\frac{\hat{B}}{\Tpj}} 
     \mathcal{P}_{n,m}(\rho)\,d\hat{\mu}\Bigg]. \nn
\end{eqnarray}
Changing to matrix form yields
\begin{eqnarray*}
\hat{\mathbf{M}}_{kr}(k_m,x_3) &=& \pi\,\diag\{\hat{B}^{q+1}(x_1,x^3),\ldots,
   \hat{B}(x_N,x^3)\}\cdot\\
&& \int \hat{v}_\|^k \hat{\mu}^q
   \hat{\mathcal{M}}(k_{2,m},x^3,\hat{\mu})\cdot
   \hat{\mathbf{F}}(k_{2,m},x^3,\hat{v}_\|,\hat{\mu})\,d\hat{v}_\|\,d\hat{\mu} \\
&& -\hat{e}_j\frac{T_\rf}{\refj{T}} q!\Upsilon(k)\, 
   \diag\{\npj(x_1)\Tpj^{q+\frac{k}{2}-1}(x_1),\ldots,\npj(x_N)\Tpj^{q+\frac{k}{2}-1}(x_N)\}\cdot
   \mathcal{L} \cdot\hat{\mathbf{\Phi}}(k_{2,m},x^3)\\
&& +\hat{e}_j\frac{T_\rf}{\refj{T}}\Upsilon(k) \int \hat{\mu}^q 
   \diag\left\{\npj(x_1)\Tpj^{\frac{k}{2}-2}(x_1)\hat{B}^{q+1}(x_1,x^3)
   \exp{\left(-\hat{\mu}\frac{\hat{B}(x_1,x^3)}{\Tpj(x_1)}\right)},\ldots,\right.\\
&& \phantom{e_j\int}\left.\npj(x_N)\Tpj^{\frac{k}{2}-2}(x_N)
   \hat{B}^{q+1}(x_N,x^3)
      \exp{\left(-\hat{\mu}\frac{\hat{B}(x_N,x^3)}{\Tpj(x_N)}\right)}\right\}\cdot
   \mathcal{P}(k_{2,m},x^3,\hat{\mu})\cdot\\
&& \phantom{e_j\int} \hat{\mathbf{\Phi}}(k_{2,m},x^3)\,d\hat{\mu}
\end{eqnarray*}
with $\mathcal{L}_{rq}=\Lambda_q(x_r)$. Using also the abbreviation
$\hat{\mathcal{B}}(x^3)=\diag\{\hat{B}(x_1,x^3),\ldots,\hat{B}(x_N,x^3)\}$
we can write the equation in the compact form
\begin{eqnarray*}
\hat{\mathbf{M}}_{kr}(k_m,x_3)  &=&  \pi\hat{\mathcal{B}}(x^3)
   \cdot \int \hat{v}_\|^k \hat{\mu}^q
   \hat{\mathcal{M}}(k_{2,m},x^3,\hat{\mu})\cdot
   \hat{\mathbf{F}}(k_{2,m},x^3,\hat{v}_\|,\hat{\mu})\,d\hat{v}_\|\,d\hat{\mu} \\
&& -\hat{e}_j\frac{T_\rf}{\refj{T}} \Upsilon(k) q!\, \diag\{\npj\Tpj^{q+\frac{k}{2}-1}\}\cdot
   \mathcal{L} \cdot\hat{\mathbf{\Phi}}(k_{2,m},x^3)\\
&& +\hat{e}_j\frac{T_\rf}{\refj{T}} \Upsilon(k) \int \hat{\mu}^q 
   \diag\{\npj\Tpj^{\frac{k}{2}-2}\hat{B}^{q+1}(x^3)\exp{\{-\hat{\mu}
       \frac{\hat{B}(x^3)}{\Tpj}\}}\}\cdot
   \mathcal{P}(k_{2,m},x^3,\hat{\mu})\,d\hat{\mu}\\
&& \cdot\hat{\mathbf{\Phi}}(k_{2,m},x^3)
\end{eqnarray*}

\subsubsection{Properties of the (0,0)-moment}
If one calculates the following quantity
\begin{displaymath}
  \sum_j\frac{e_j}{e} \mathbf{M}_{00}(k_m,x_3)
\end{displaymath}
we find this quantity to be zero. This means the density of
electrons and ions in a two species run are equal. This should also be
reflected in the \texttt{nrg} file and is a good further check of the
calculation.
\begin{eqnarray*}
   \sum_j\hat{e}_j \mathbf{M}_{00}(k_m,x_3)
   &=& \sum_j\hat{e}_j
   \pi\hat{\mathcal{B}}(x^3)\cdot \int\mathcal{M}(k_{2,m},x^3,\mu)\cdot\mathbf{F}(k_{2,m},x^3,v_\|,\mu)\,dv_\|\,d\mu\\
   && -\sum_j\hat{e}_j^2\frac{T_\rf}{\refj{T}}\diag\{\frac{\npj}{\Tpj}\}\cdot\mathcal{L}\cdot\mathbf{\Phi}(k_{2,m},x^3)\\
   &&+\sum_j\hat{e}_j^2\frac{T_\rf}{\refj{T}}\int
   \diag\{\frac{\npj}{\Tpj^2}\hat{B}(x^3)\exp\{-\frac{\mu \hat{B}}{\Tpj}\}\}
   \cdot\mathcal{P}(k_{2,m},x^3,\mu)\,d\mu\,\cdot\mathbf{\Phi}(k_{2,m},x^3)
\end{eqnarray*}
Using the quasineutrality equation, we can replace the first summand
by $\mathcal{P}''(k_m,x_3)\mathbf{\Phi}_m(x_3)$ and get then
\begin{multline*}
  \sum_j\hat{e}_j \mathbf{M}_{00}(k_m,x_3)
  = \mathcal{P}''(k_m,x_3)\mathbf{\Phi}_m(x_3)
  -\sum_j\hat{e}_j^2\frac{T_\rf}{\refj{T}}\diag\{\frac{\npj}{\Tpj}\}\\\cdot\left(\mathcal{L}
    -\int\diag\{\frac{\hat{B}(x^3)}{\Tpj}\exp\{-\frac{\mu \hat{B}}{\Tpj}\}\}
    \cdot\mathcal{P}(k_{2,m},x^3,\mu)\,d\mu
  \right)\cdot\mathbf{\Phi}(k_{2,m},x^3)
\end{multline*}
The second summand is exactly the definition of $\mathcal{P}''$. So in
the end the so calculated sum is equal to zero.



\section{Transport fluxes}
\label{sec:transflux_global}

\subsection{{\bf ExB} velocity}
The ${\bf E_\chi\times B}$ velocity is needed in every flux calculation. 
Therefore, we first take a closer look at this quantity. The definition is 
\begin{displaymath}
\mathbf{v}_{\bf E_\chi\times B} = c\frac{\mathbf{B}_0\times\nabla\chi}{B_0^2}
\end{displaymath}
with $\chi = \Phi(\mathbf{x})-\frac{1}{c}v_\parallel A_\parallel(\mathbf{x})$ 
(neglecting $B_\parallel$ fluctuations). Since we are only interested in
radial fluxes we consider the projection on the {\em contravariant} basis 
vector
\begin{eqnarray*}
v_E & \equiv & \mathbf{v}_{\bf E_\chi\times B}\cdot \mathbf{e}^1 \\
& = & \frac{c}{B_3 B^3}\frac{\varepsilon^{ij1}}{J} E_{\chi,i} B_j.
\end{eqnarray*}
Using $J=B_{\rm ref}/B^3$, $E_{\chi,i}=-\partial\chi/\partial u^i$ and $B_j=B_3\delta_{3j}$
yields
\begin{eqnarray*}
v_E & \equiv & -\frac{c}{B_3 B_\rf} \varepsilon^{231} \frac{\partial\chi}{\partial y} B_3\\
 & = & - \frac{c}{B_\rf} \frac{\partial\chi }{\partial y}
\end{eqnarray*}



\subsection{Particle flux}
\label{sec:partflux}

The aim of this subsection is to derive an expression for the particle
flux in a global simulation. We start as usual from
\begin{displaymath}
  \Gamma(\mathbf{x})=\int v_E(\mathbf{x})f_1(\mathbf{x},\mathbf{v})\,d^3v
\end{displaymath}

\subsubsection{Electrostatic particle flux}
First we are only interested in the electrostatic particle flux, so we
use only the electrostatic contribution to the radial component of the
$E\times B$ velocity.
\begin{displaymath}
  \Gamma_\mathrm{es}(\mathbf{x})=-\frac{c}{B_\rf}\frac{\partial\Phi(\mathbf{x})}{\partial y} 
  \int f_1(\mathbf{x},\mathbf{v})\,d^3v
  =-\frac{c}{B_\rf}\frac{\partial\Phi(\mathbf{x})}{\partial y} M_{00}(\mathbf{x})
\end{displaymath}
Normalization leads to 
\begin{displaymath}
  \Gamma_\mathrm{es}(\mathbf{x})
  =-\frac{c}{B_\rf\rho_\rf}\frac{T_\rf}{e}\frac{\rho_\rf}{L_\rf}
  \frac{\partial\hat{\Phi}(\mathbf{x})}{\partial \hat{y}} \refj{n}\frac{\rho_\rf}{L_\rf}
   \hat{M}_{00}(\mathbf{x})
  =-\frac{T_\rf}{m_\rf c_\rf}\refj{n}\frac{\rho_\rf^2}{L_\rf^2}
  \frac{\partial\hat{\Phi}(\mathbf{x})}{\partial \hat{y}} \hat{M}_{00}(\mathbf{x})
\end{displaymath}
Normalizing the flux to units of $n_\rf c_\rf \rho_\rf^2/L_\rf^2$ we arrive
at
\begin{displaymath}
  \hat{\Gamma}_\mathrm{es}(x^1,x^2,x^3)
  =-\frac{\refj{n}}{n_\rf}\frac{\partial\hat{\Phi}(x^1,x^2,x^3)}{\partial x^2} \hat{M}_{00}(x^1,x^2,x^3)
\end{displaymath}
What we have in the code are the Fourier transformed quantities in
$x^2$ direction, so they must come into play somewhere. Representing
the right hand side quantities by the associated Fourier series we
come to
\begin{eqnarray*}
\sum_q\Gamma_{\mathrm{es},q}(x^1,x^3)e^{ik_qx^2}
 & = & -\left(\sum_m\Phi_m(x^1,x^3)ik_m e^{ik_mx^2}\right)
   \left(\sum_r M_{00,r}(x^1,x^3)e^{ik_rx^2}\right)\\
 & = & -\sum_m\sum_rik_m\Phi_m(x^1,x^3)M_{00,r}(x^1,x^3) e^{i(k_m+k_r)x^2}
\end{eqnarray*}
For the contributions of the different wave numbers to the total flux
we can write
\begin{displaymath}
  \Gamma_{\mathrm{es},q}(x^1,x^3)
  =-\sum_m ik_m\Phi_m(x^1,x^3)M_{00,q-m}(x^1,x^3)
\end{displaymath}
For the \texttt{diag\_nrg} routine, we want to calculate the average flux
in the simulation volume
$\Gamma_\mathrm{es} = \frac{1}{V} \int \Gamma_\mathrm{es}(x^1,x^2,x^3)J(x^3)\,dx^1\,dx^2\,dx^3$
where the volume $V=\int J(x^3)\,dx^1\,dx^2\,dx^3\equiv L_x L_y \int J(x^3)\,dx^3$.
This is done as usual as the sum over all
contributions. As we are also averaging in $y$ direction, this means
that we are only taking into account the $k_q=0$ contribution to $\Gamma_\mathrm{es}$.
\begin{eqnarray*}
  \Gamma_\mathrm{es} &=& 
  \int \Gamma_{\mathrm{es},0}\,dx^1\,dx^3\\
  &=& -\frac{1}{L_x}\frac{1}{\int J(x^3)\,dx^3}
  \sum_{m=-N/2}^{N/2} ik_m \int\Phi_m(x^1,x^3)  M_{00,-m}(x^1,x^3) \,dx^1\,dx^3\\
  &=& -\frac{1}{L_x}\frac{1}{\int J(x^3)\,dx^3}\int
  \sum_{m=1}^{N/2} 2\Re\{ik_m\Phi_m(x^1,x^3)  M_{00,m}^*(x^1,x^3)\} \,dx^1\,dx^3
\end{eqnarray*}

\subsubsection{Check for ambipolarity of the electrostatic particle flux}
We check now for ambipolarity of the electrostatic particle flux,
which is important to be fulfilled. To get the right result, I have to
use at some point the quasineutrality equation from a previous
section. We start by writing the total flux for a given wave number
$k_q$ as
\begin{eqnarray*}
  \sum_j\frac{e_j}{e}\Gamma_{\mathrm{es},q} 
  &=&-\frac{T_0}{m_0v_0^2}\sum_j\frac{e_j}{e}\frac{1}{\hat{B}(x^1,x^3)}\sum_m ik_m\Phi_m(x^1,x^3)M_{00,q-m}(x^1,x^3)
\end{eqnarray*}
Written in matrix vector form
\begin{eqnarray*}
  \sum_j\frac{e_j}{e}\mathbf{\Gamma}_{\mathrm{es},q} 
  &=&-\frac{T_0}{m_0v_0^2}\sum_j\frac{e_j}{e}\frac{1}{\hat{B}(x^1,x^3)}\sum_m ik_m\diag\{\Phi_m\}\mathbf{M}_{00,q-m}(x^1,x^3)
\end{eqnarray*}
Using the above expression for the first moment.
\begin{eqnarray*}
  \sum_j\frac{e_j}{e}\mathbf{\Gamma}_{\mathrm{es},q} 
  &=&-\frac{T_0}{m_0v_0^2}\sum_j\frac{e_j}{e}\frac{1}{\hat{B}(x^1,x^3)}\\
  &&\sum_m ik_m\diag\{\Phi_m\}\pi\hat{\mathcal{B}}(x^3)\cdot \int \Bigg[
  \mathcal{M}(k_{2,q-m},x^3,\mu)\cdot\mathbf{F}(k_{2,q-m},x^3,v_\|,\mu)\\
  &&-\frac{e_j}{e}\frac{T_0}{T_{0j}}\mathcal{F}_0(x^3,v_\|,\mu)\cdot
  \left(
    \mathcal{L}-\mathcal{P}(k_{2,q-m},x^3,\mu)
  \right)\cdot\mathbf{\Phi}(k_{2,q-m},x^3)
  \Bigg]\,dv_\|\,d\mu\\
  &=&-\frac{T_0}{m_0v_0^2}\sum_j\frac{e_j}{e}\frac{1}{\hat{B}(x^1,x^3)}\\
  &&\sum_m ik_m\diag\{\Phi_m\} \Bigg[
  \pi\hat{\mathcal{B}}(x^3)\cdot \int\mathcal{M}(k_{2,q-m},x^3,\mu)\cdot\mathbf{F}(k_{2,q-m},x^3,v_\|,\mu)\,dv_\|\,d\mu\\
  &&-\pi\hat{\mathcal{B}}(x^3)\cdot \int\frac{e_j}{e}\frac{T_0}{T_{0j}}\mathcal{F}_0(x^3,v_\|,\mu)\cdot
  \left(
    \mathcal{L}-\mathcal{P}(k_{2,q-m},x^3,\mu)
  \right)\cdot\mathbf{\Phi}(k_{2,q-m},x^3)
  \,dv_\|\,d\mu\Bigg]\\
  &=&-\frac{T_0}{m_0v_0^2}\frac{1}{\hat{B}(x^1,x^3)}\\
  &&\sum_m ik_m\diag\{\Phi_m\} \Bigg[
  \sum_j\frac{e_j}{e}\pi\hat{\mathcal{B}}(x^3)\cdot \int\mathcal{M}(k_{2,q-m},x^3,\mu)\cdot\mathbf{F}(k_{2,q-m},x^3,v_\|,\mu)\,dv_\|\,d\mu\\
  &&-\sum_j\frac{e_j^2}{e^2}\frac{T_0}{T_{0j}}\pi\hat{\mathcal{B}}(x^3)\cdot \int\mathcal{F}_0(x^3,v_\|,\mu)\cdot
  \left(
    \mathcal{L}-\mathcal{P}(k_{2,q-m},x^3,\mu)
  \right)\cdot\mathbf{\Phi}(k_{2,q-m},x^3)
  \,dv_\|\,d\mu\Bigg]\\
  &=&-\frac{T_0}{m_0v_0^2}\frac{1}{\hat{B}(x^1,x^3)}\\
  &&\sum_m ik_m\diag\{\Phi_m\} \Bigg[
  \sum_j\frac{e_j}{e}\pi\hat{\mathcal{B}}(x^3)\cdot \int\mathcal{M}(k_{2,q-m},x^3,\mu)\cdot\mathbf{F}(k_{2,q-m},x^3,v_\|,\mu)\,dv_\|\,d\mu\\
  &&-\sum_j\frac{e_j^2}{e^2}\frac{T_0}{T_{0j}}\pi\hat{\mathcal{B}}(x^3)\cdot \int
  \diag\{\frac{n_{0j}}{n_0}\frac{v_{0j}^3}{v_{Tj}^3}\pi^{-3/2}\exp\{-v_\|^2v_{0j}^2/v_{Tj}^2-\mu T_0\hat{B}/T_{0j}\}\}\cdot\\
  &&\left(
    \mathcal{L}-\mathcal{P}(k_{2,q-m},x^3,\mu)
  \right)\cdot\mathbf{\Phi}(k_{2,q-m},x^3)
  \,dv_\|\,d\mu\Bigg]
\end{eqnarray*}
Doing the $v_\|$ integration
\begin{eqnarray*}
  \sum_j\frac{e_j}{e}\mathbf{\Gamma}_{\mathrm{es},q} 
  &=&-\frac{T_0}{m_0v_0^2}\frac{1}{\hat{B}(x^1,x^3)}\\
  &&\sum_m ik_m\diag\{\Phi_m\} \Bigg[
  \sum_j\frac{e_j}{e}\pi\hat{\mathcal{B}}(x^3)\cdot \int\mathcal{M}(k_{2,q-m},x^3,\mu)\cdot\mathbf{F}(k_{2,q-m},x^3,v_\|,\mu)\,dv_\|\,d\mu\\
  &&-\int \sum_j\frac{e_j^2}{e^2}
  \cdot\diag\{\frac{T_0}{T_{0j}}\hat{B}\frac{n_{0j}}{n_0}\frac{T_0}{T_{0j}}\exp\{-\mu T_0\hat{B}/T_{0j}\}\}
  \cdot\mathcal{L}\cdot\mathbf{\Phi}(k_{2,q-m},x^3)\,d\mu\\
  &&+\int\sum_j\frac{e_j^2}{e^2}
  \cdot\diag\{\frac{T_0}{T_{0j}}\hat{B}\frac{n_{0j}}{n_0}\frac{T_0}{T_{0j}}\exp\{-\frac{\mu T_0\hat{B}}{T_{0j}}\}\}
  \cdot\mathcal{P}(k_{2,q-m},x^3,\mu)\cdot\mathbf{\Phi}(k_{2,q-m},x^3) \,d\mu
  \Bigg]\\
  &=&-\frac{T_0}{m_0v_0^2}\frac{1}{\hat{B}(x^1,x^3)}\\
  &&\sum_m ik_m\diag\{\Phi_m\} \Bigg[
  \sum_j\frac{e_j}{e}\pi\hat{\mathcal{B}}(x^3)\cdot \int\mathcal{M}(k_{2,q-m},x^3,\mu)\cdot\mathbf{F}(k_{2,q-m},x^3,v_\|,\mu)\,dv_\|\,d\mu\\
  &&-\sum_j\frac{e_j^2}{e^2}
  \cdot\diag\{\frac{T_0}{T_{0j}}\frac{n_{0j}}{n_0}\}\}
  \cdot\mathcal{L}\cdot\mathbf{\Phi}(k_{2,q-m},x^3)\\
  &&+\int\sum_j\frac{e_j^2}{e^2}
  \cdot\diag\{\frac{T_0}{T_{0j}}\hat{B}\frac{n_{0j}}{n_0}\frac{T_0}{T_{0j}}\exp\{-\frac{\mu T_0\hat{B}}{T_{0j}}\}\}
  \cdot\mathcal{P}(k_{2,q-m},x^3,\mu)\cdot\mathbf{\Phi}(k_{2,q-m},x^3) \,d\mu
  \Bigg]
\end{eqnarray*}
If one uses the quasineutrality equation, we can show that the total
flux is equal to zero. So ambipolarity is fulfilled, but only if the
methods to calculate the fluxes and to solve the field equations are
identical.

\subsubsection{Electromagnetic particle flux}
Now we consider the contribution of the electromagnetic part of the $E_\chi\times B$ velocity.
Neglecting $B_\|$-fluctuations we get
\begin{eqnarray*}
\Gamma_\mathrm{em}(\mathbf{x})&=&\frac{1}{B_\rf}\frac{\partial A_\|(\mathbf{x})}{\partial y}
  \int v_\| \, f_1(\mathbf{x},\mathbf{v})\,d^3v\\
&=&\frac{1}{B_\rf}\frac{\partial A_\|(\mathbf{x})}{\partial y} M_{10}(\mathbf{x})\\
&=&\frac{1}{B_\rf}\frac{B_\rf\rho_\rf}{L_\rf}\frac{\partial \hat{A}_\|(\mathbf{x})}{\partial \hat{y}}
 \refj{n} \frac{\rho_\rf}{L_\rf} v_{0j} \hat{M}_{10}(\mathbf{x}).
\end{eqnarray*}
The particle flux shall be normalized (species independent) to 
$\frac{c_\rf \rho_\rf^2}{L_\rf}\frac{n_\rf}{L_\rf}$. Therefore,
\begin{equation}
\hat{\Gamma}_\mathrm{em}(\mathbf{x})=\frac{\refj{n}}{n_\rf}
 \sqrt{\frac{2\refj{T}/T_\rf}{\hat{m}_j}}\hat{M}_{10}(\mathbf{x})
\label{eq:Gem_norm}
\end{equation}


\subsection{Heat flux}
\label{sec:heatflux}
The heat flux is defined as
\begin{equation}
Q(\mathbf{x}) = \int \frac{1}{2}m_jv^2 v_E(\mathbf{x})f_1(\mathbf{x},\mathbf{v})\,d^3v
\label{eq:Q_basic}
\end{equation}

\subsubsection{Electrostatic heat flux}
First we again consider only the electrostatic contribution
\begin{eqnarray*}
Q_\mathrm{es}(\mathbf{x})&=&-\frac{m_jc}{2B_\rf} \frac{\partial\Phi(\mathbf{x})}{\partial y} 
 \frac{B_0}{m_j} \int \left(v_\|^2+\frac{2\mu B_0}{m_j}\right) 
 f_1(\mathbf{x},\mathbf{v})\,dv_\| \,d\mu \,d\theta\\
&=& -\frac{m_jc}{2B_\rf} \frac{\partial\Phi(\mathbf{x})}{\partial y} 
 \left(M_{20}(\mathbf{x})+M_{02}(\mathbf{x})\right)
\end{eqnarray*}
Normalizing as described in Section~\ref{sec:normalization} and Eq.~\ref{eq:mom_norm} 
yields
\begin{eqnarray}
Q_\mathrm{es}(\mathbf{x})
&=& -\frac{1}{2}\frac{m_j c}{e B_\rf}\frac{T_\rf}{\rho_\rf}\frac{\rho_\rf}{L_\rf}
 \frac{\partial\hat{\Phi}(\mathbf{x})}{\partial \hat{y}}
 \refj{n} \frac{\rho_\rf}{L_\rf} v_{0j}^2 \left(\hat{M}_{20}(\mathbf{x})+
 \hat{M}_{02}(\mathbf{x})\right)\nn \\
&=& - \frac{c_\rf \rho_\rf^2}{L_\rf}\frac{p_\rf}{L_\rf} 
 \frac{\refj{n}}{n_\rf}\frac{\refj{T}}{T_\rf}\left(\hat{M}_{20}(\mathbf{x})+
 \hat{M}_{02}(\mathbf{x})\right)\nn \\
&\equiv & \frac{c_\rf \rho_\rf^2}{L_\rf}\frac{p_\rf}{L_\rf} \hat{Q}_\mathrm{es}(\mathbf{x})
\label{eq:Qes_norm}
\end{eqnarray}

\subsubsection{Electromagnetic heat flux}
The electromagnetic heat flux (without $B_\|$ fluctuations) is given by
\begin{eqnarray*}
Q_\mathrm{em}(\mathbf{x})&=&\frac{m_j}{2B_\rf}\frac{\partial A_\| (\mathbf{x})}{\partial y} 
 \int \left(v_\|^2+\frac{2\mu B_0}{m_j}\right)\, v_\| \,
 f_1(\mathbf{x},\mathbf{v})\,dv_\| \,d\mu \,d\theta\\
&=& \frac{m_j}{2B_\rf}\frac{\partial A_\| (\mathbf{x})}{\partial y}
 \left( M_{30}(\mathbf{x}) + M_{12}(\mathbf{x})\right).
\end{eqnarray*}
Using the same normalization as for $Q_{\rm es}$, 
namely $\frac{c_\rf \rho_\rf^2}{L_\rf}\frac{p_\rf}{L_\rf}$, we define
\begin{equation}
\hat{Q}_\mathrm{em}(\mathbf{x}) = \frac{\refj{n}}{n_\rf}\frac{\refj{T}}{T_\rf}
  \sqrt{\frac{2\refj{T}/T_\rf}{m_j}}
  \left( \hat{M}_{30}(\mathbf{x}) + \hat{M}_{12}(\mathbf{x})\right)
\end{equation}


\section{Further quantities of interest}
\label{sec:moremom}
Besides the transport fluxes {\sc GENE} calculates and writes 
further quantities like e.g. $T_\|$, $T_\perp$, $u_\|$, etc.
which shall be shortly derived in the following.

\subsection{Density}
The density is defined as
\begin{eqnarray*}
n_j(\mathbf{x}) & = & \int f_1(\mathbf{x},\mathbf{v})\,d^3v\\
&=& M_{00}(\mathbf{x})
\end{eqnarray*}
At the moment, a density normalization of $\refj{n} \frac{\rho_\rf}{L_\rf}$
is used in {\sc GENE} (maybe we later on want $n_\rf \frac{\rho_\rf}{L_\rf}$?).
Thus
\begin{eqnarray*}
\hat{n}_j(\mathbf{x}) & = & \hat{M}_{00}(\mathbf{x}).
\end{eqnarray*}

\subsection{Parallel velocity moment}
The definition is
\begin{eqnarray*}
n_{0j}(\mathbf{x}) u_\| (\mathbf{x}) &=& \int v_\| f_1(\mathbf{x},\mathbf{v})\,d^3v 
 =M_{10}(\mathbf{x}).
\end{eqnarray*}
Using the usual normalizations we get
\begin{eqnarray*}
u_\| (\mathbf{x}) &=& c_\rf\frac{\rho_\rf}{L_\rf} \frac{1}{\npj} 
  \sqrt{\frac{2\refj{T}/T_\rf}{\hat{m}_j}}\hat{M}_{10}(\mathbf{x}) \\
& \equiv & c_\rf\frac{\rho_\rf}{L_\rf} \hat{u}_\| (\mathbf{x}).
\end{eqnarray*}

\subsection{Parallel temperature}
For symmetric equilibrium distribution functions in velocity space where
$u_{\| 0} = 0$ we can simply define
\begin{eqnarray*}
n_{0j}(\mathbf{x}) T_{\| 1} (\mathbf{x}) &=& p_{\| 1}(\mathbf{x})
 -n_1(\mathbf{x})T_{\| 0}(\mathbf{x})\\
&=& m_j \int \left(v_\|-u_{\| 1}\right)^2 f_1(\mathbf{x},\mathbf{v})\,d^3v 
 -n_1(\mathbf{x})T_{\| 0}(\mathbf{x})\\
&\approx& m_j \int v_\|^2 f_1(\mathbf{x},\mathbf{v})\,d^3v 
 -n_1(\mathbf{x})T_{\| 0}(\mathbf{x})
\end{eqnarray*}
where squared and higher perturbed terms were neglected in the last step.
Using the moment abbreviations and identifying $T_{\| 0}=T_{0j}$ yields
\begin{eqnarray*}
n_{0j}(\mathbf{x}) T_{\| 1} (\mathbf{x}) &=& m_j M_{20} - M_{00}T_{0j}\\
&=& \refj{n} \frac{\rho_\rf}{L_\rf} \left(m_j \frac{2\refj{T}}{m_j} \hat{M}_{20}
    - \hat{M}_{00}T_{0j}\right).
\end{eqnarray*}
Currently, {\sc GENE} uses $\refj{T}\frac{\rho_\rf}{L_\rf}$ as temperature
normalization, therefore 
\begin{eqnarray*}
\hat{T}_{\| 1} (\mathbf{x}) &=& 
\frac{1}{\npj} \left(2\hat{M}_{20}- \Tpj\hat{M}_{00}\right).
\end{eqnarray*}

\subsection{Perpendicular temperature}
Starting from the definition
\begin{eqnarray*}
n_{0j}(\mathbf{x}) T_{\perp 1} (\mathbf{x}) &=& p_{\perp 1}(\mathbf{x})
 -n_1(\mathbf{x})T_{\perp 0}(\mathbf{x})\\
&=& \frac{m_j}{2} \int v_{\perp}^2 f_1(\mathbf{x},\mathbf{v})\,d^3v 
 -n_1(\mathbf{x})T_{\| 0}(\mathbf{x})
\end{eqnarray*}
the perpendicular temperature can be written as
\begin{eqnarray*}
n_{0j}(\mathbf{x}) T_{\perp 1} (\mathbf{x}) &=& \frac{m_j}{2} M_{02}-T_{0j} M_{00}
\end{eqnarray*}
and if normalized to $\refj{T}\frac{\rho_\rf}{L_\rf}$
\begin{eqnarray*}
\hat{T}_{\perp 1} (\mathbf{x}) &=& \frac{1}{\npj} \left(\hat{M}_{02}-\Tpj\hat{M}_{00}\right).
\end{eqnarray*}

\subsection{Parallel heat current density}
The parallel heat current density is defined by
\begin{eqnarray*}
q_{\|}(\mathbf{x}) &=& \frac{m_j}{2} \int \left(v_{\|}-u_{\|}\right)^3 
  f(\mathbf{x},\mathbf{v})\,d^3v.
\end{eqnarray*}
Keeping only linear perturbed terms and using $u_{\| 0}=0$ we arrive at
\begin{eqnarray*}
q_{\| 1}(\mathbf{x}) &=& \frac{m_j}{2} \int v_{\|}^3 f_1(\mathbf{x},\mathbf{v})\,d^3v
 - \frac{3m_j}{2} \int v_{\|}^2 f_0(\mathbf{x},\mathbf{v})\,d^3v\,u_{\| 1}\\
&=& \frac{m_j}{2} M_{30} - \frac{3}{2} m_j \pi^{-\frac{3}{2}}\frac{n_{0j}}{v_{T_j}^3}2\pi\frac{B_0}{m_j}
 \int_{-\infty}^{\infty} v_{\|}^2{\rm e}^{-\frac{v_\|^2}{v_{T_j}^2}}\,dv_\| 
 \int_{0}^{\infty} {\rm e}^{-\mu\frac{B_0}{T_{0j}}}d\mu\, u_{\| 1} \\
&=& \frac{m_j}{2} M_{30} - \frac{3}{2} \frac{n_{0j}}{v_{Tj}^3}\frac{2B_0}{\sqrt{\pi}} 
 \frac{\sqrt{\pi}v_{T_j}^3}{2} \frac{T_{0j}}{B_0}u_{\| 1}\\
&=& \frac{m_j}{2} M_{30} - \frac{3}{2} p_{0j} u_{\| 1}
\end{eqnarray*}
Inserting the normalizations yields
\begin{eqnarray*}
q_{\| 1}(\mathbf{x}) &=& \frac{m_j}{2} \refj{n} \frac{\rho_\rf}{L_\rf} v_{0j}^3 \hat{M}_{30} 
  - 3 p_{0j} c_\rf \frac{\rho_\rf}{L_\rf}\hat{u}_{\| 1}\\
&=& \refj{p} c_\rf \frac{\rho_\rf}{L_\rf} \left(\sqrt{\frac{2\refj{T}/T_\rf}{\hat{m}_j}}\hat{M}_{30} 
  - \frac{3}{2}\npj\Tpj\hat{u}_{\| 1}\right)
\end{eqnarray*}
{\bf Note:} For historical reasons {\sc GENE} does not write out $q_\|$ but 
$\tilde{M}_{30}=\sqrt{\frac{2\refj{T}/T_\rf}{\hat{m}_j}}\hat{M}_{30}$.

\subsection{Perpendicular heat current density}
The definition of the perpendicular heat current density is
\begin{eqnarray*}
q_{\perp}(\mathbf{x}) &=& \frac{m_j}{2} \int v_{\perp}^2 \left(v_{\|}-u_{\|}\right)^2
  f(\mathbf{x},\mathbf{v})\,d^3v.
\end{eqnarray*}
Neglecting all except for first order terms gives
\begin{eqnarray*}
q_{\perp 1}(\mathbf{x})&=& \frac{m_j}{2} \int v_{\perp}^2 v_{\|} f_1(\mathbf{x},\mathbf{v})\,d^3v-
 \frac{m_j}{2}u_{\| 1} \int v_{\perp}^2 f_0(\mathbf{x},\mathbf{v})\,d^3v\\
&=& \frac{m_j}{2} M_{12} - \frac{m_j}{2}u_{\| 1} \pi^{-\frac{3}{2}}\frac{n_{0j}}{v_{T_j}^3}2\pi\frac{2B_0^2}{m_j^2}
 \int_{-\infty}^{\infty} {\rm e}^{-\frac{v_\|^2}{v_{T_j}^2}}\,dv_\| 
 \int_{0}^{\infty} \mu \, {\rm e}^{-\mu\frac{B_0}{T_{0j}}}d\mu \\
&=& \frac{m_j}{2} M_{12} - \frac{m_j}{2}u_{\| 1} \pi^{-\frac{3}{2}}\frac{n_{0j}}{v_{T_j}^3}2\pi\frac{2B_0^2}{m_j^2}
 \sqrt{\pi} v_{T_j} \frac{T_{0j}^2}{B_0^2}\\
&=& \frac{m_j}{2} M_{12} - p_{0j} u_{\| 1}
\end{eqnarray*}
Using the same normalization as for $q_\|$, namely $\refj{p} c_\rf \frac{\rho_\rf}{L_\rf}$, we get
\begin{eqnarray*}
\hat{q}_{\perp}(\mathbf{x})&=& \sqrt{\frac{2\refj{T}/T_\rf}{\hat{m}_j}} \hat{M}_{12} - \npj\Tpj u_{\| 1}
\end{eqnarray*}
{\bf Note:} For historical reasons {\sc GENE} does not write out $q_\perp$ but 
$\tilde{M}_{12}=\sqrt{\frac{2\refj{T}/T_\rf}{\hat{m}_j}}\hat{M}_{12}$.

%%% Local Variables:
%%% mode: latex
%%% TeX-master: "globalgene"
%%% End:

\chapter{Sources and Sinks}
This chapter treats the various sources and sinks which are currently implemented in \gene.

\section{Krook operator}

\section{Krook type heat source}

Similar to B.F. McMillan et al., {\em Long global gyrokinetic simulations: source terms and particle noise control}, Phys.~Plasmas {\bf 15} (2008) 052308.

\bea
\frac{\D g_{1\spec}}{\D t} = \mathcal{S}_{\rm KH} 
= - \gamma_{\rm KH} \left[\avg{f_{1\spec}^{\rm symm}}_{\rm FS} - \avg{F_{0\spec}}_{\rm FS} \frac{\avg{\int\D^3v\avg{f_{1\spec}^{\rm symm}}_{\rm FS}}_{\rm FS}}{\avg{\int\D^3v\avg{F_{0\spec}}_{\rm FS}}_{\rm FS}} \right] \label{eq:krookheat_def}
\eea

Here, $f_{1\spec}^{\rm symm} \equiv f_{1\spec}(\mvec{X},\abs{v_\|},\mu) \equiv (f_{1\spec}(\mvec{X},v_\|,\mu)+f_{1\spec}(\mvec{X},-v_\|,\mu))/2$

\section{Krook type particle source}

\begin{itemize}
\item Daniel's suggestion:
\bea
\frac{\D g_\spec}{\D t} = \mathcal{S}^{(1)}_{\rm KP} 
= - \gamma_{\rm KP} \left[\avg{f_{1\spec}^{\rm symm}}_{\rm FS} - \avg{F_{0\spec}}_{\rm FS} \frac{\sum_s q_s \avg{\int\D^3v\avg{f_{1s}^{\rm symm}}_{\rm FS}}_{\rm FS}/n_{\rm spec}}{q_\spec \avg{\int\D^3v\avg{F_{0\spec}}_{\rm FS}}_{\rm FS}} \right] \label{eq:krookpart1_def}
\eea

Heat being injected by this operator is compensated by dynamically adapting the Krook type heat source strength $\gamma_{\rm KH}$ (see above).

\item Minor modification (additional flux surface averages) Ben's suggestion for {\sc Orb}5/{\sc Gene} benchmark:
%Sp_i = -\gamma_p f_0i  \frac{ \Int_P \Sum_j \delta n_j  }{ \Int_P \Sum_j f_0j }

\bea
\frac{\D g_\spec}{\D t} = \mathcal{S}^{(2)}_{\rm KP} 
= - \gamma_{\rm KP} \avg{F_{0\spec}}_{\rm FS} \frac{\sum_s \avg{\int\D^3v \avg{f_{1s}^{\rm symm}}_{\rm FS}}_{\rm FS}}
{\sum_s \avg{\int\D^3v\avg{F_{0s}}_{\rm FS}}_{\rm FS}} \label{eq:krookpart2_def}
\eea

\end{itemize}


\section{Localized heat source}

The following section is closely following Y.~Sarazin et al., {\em Large scale dynamics in flux driven gyrokinetic turbulence}, Nucl.~Fusion {\bf 50} (2010), 054004. However, the normalization, for instance, slightly differs from {\sc Gysela}.\\

The heat source is added to the right hand side of the Vlasov equation as follows:

\bea
\frac{\D g}{\D t} = \mathcal{S}_H = \mathcal{S}_0 \hat{\mathcal{S}}_x \hat{\mathcal{S}}_E \label{eq:heat_src_def}
\eea
with the normalized radial source profile $\hat{\mathcal{S}}_x$, the normalized energy source term $\hat{\mathcal{S}}_E$, and an amplitude $\mathcal{S}_0$ which is given in units of $\frac{n_{0\spec}(\xp)}{v_{T\spec}^3(\xp)}\roL\frac{c_\rf}{L_\rf}$ as can be seen when normalizing Eq.~\ref{eq:heat_src_def}. Using the definition
\bea
E = \frac{1}{2}m_\spec v_\|^2 + \mu B_0 = T_{0\spec}(\xp) \left(\hat{v}_\|^2+\hat{\mu}\hat{B}_0\right) \equiv T_{0\spec}(\xp) \hat{E}
\eea
allows for evaluating the total injected power as follows
\bea
P_{\rm add} & = & \mathcal{S}_0 \int\!\D^3x \int\!\D^3v E \, \hat{\mathcal{S}}_x \hat{\mathcal{S}}_E \nonumber \\
&=& \mathcal{S}_0 \int\delta(\mvec{X}+\mvec{r}-\mvec{x}) E \, \hat{\mathcal{S}}_x \hat{\mathcal{S}}_E \frac{B_{0\|}^*}{m_\spec} \,\D^3X \D\vpar \D\mu \D\theta.
\eea
Since fluctuating quantities are absent, it is possible to identify $\mvec{r}\rightarrow 0$ and thus immediately perform the gyroangle integration which yields
\bea
P_{\rm add} & = & \frac{2\pi}{m_\spec} \mathcal{S}_0  \int\!\D^3x\,\hat{\mathcal{S}}_x \int\D\vpar\D\mu\, E \hat{\mathcal{S}}_E B_{0\|}^*
\eea
or, in the low-$\beta$ limit where $B_{0\|}^*\approx B_0$, 
\bea
P_{\rm add} & = & \frac{2\pi}{m_\spec}\mathcal{S}_0 \int\!\D^3x\,\hat{\mathcal{S}}_x B_0 \int\D\vpar\D\mu\, E \hat{\mathcal{S}}_E. \label{eq:p_low_beta}
\eea

The energy term is set to
\bea
\hat{S}_E = \frac{1}{\mathcal{N}_E} \left(\frac{\hat{E}}{\hat{T}_{p\spec}}-\frac{3}{2}\right) \hat{F}_{0\spec}
\eea
to represent a pure heat but no particle or momentum source as can be seen by computing the according moments
\bea
\int\D\hat{v}_\|\D\hat{\mu}\,\hat{S}_E & = & 0 \nonumber \\
\int\D\hat{v}_\|\D\hat{\mu}\,\hat{v}_\|\hat{S}_E & = & 0 \nonumber.
\eea
The normalization factor $\mathcal{N}_E$ is chosen such that
\bea
\pi \hat{B}_0 \hat{p}_{0\spec}(\xp)\, \int\D\hat{v}_\|\D\hat{\mu}\, \hat{E} \hat{S}_E = 1
\eea
which gives
\bea
\hat{S}_E = \frac{2}{3}\frac{1}{\hat{p}_{0\spec}(x)} \left(\frac{\hat{E}}{\hat{T}_{p\spec}}-\frac{3}{2}\right) \hat{F}_{0\spec}.
\eea
This result can be employed in the normalized version of Eq.~\ref{eq:p_low_beta},
\bea
P_{\rm add} & = & n_\rf T_\rf \rho_\rf^3\frac{c_\rf}{L_\rf}\,  \hat{\mathcal{S}}_0 \int\!\D^3\hat{x}\, \hat{\mathcal{S}}_x(\hat{x}) \hat{J}(\hat{x},\hat{z}) \underbrace{\pi \hat{B}_0\,\hat{p}_{0\spec}(\xp) \int\D\hat{\vpar}\D\hat{\mu}\, \hat{E} \hat{\mathcal{S}}_E}_{=1}.
\eea
In a next step, the radial profile is normalized such that
\bea
\int\!\D^3\hat{x}\, \hat{\mathcal{S}}_x(\hat{x}) \hat{J}(\hat{x},\hat{z}) = 1
\eea
which implies
\bea
\hat{\mathcal{S}}_x(\hat{x}) & = & \mathcal{S}_{x,in}(\hat{x})\,/\int\!\D^3\hat{x}\, \hat{\mathcal{S}}_{x,in}(\hat{x}) \hat{J}(\hat{x},\hat{z}) \nonumber \\
& \sim & \mathcal{S}_{x,in}(\hat{x}) \cdot \left[\frac{\hat{L}_x}{N_{x}} n_0 \hat{L}_y \frac{2\pi}{N_z} \sum_x \sum_z \hat{\mathcal{S}}_{x,in}(\hat{x}) \hat{J}(\hat{x},\hat{z})\right]^{-1}
\eea
Note, that we consider the whole flux surface, i.e. the total box length in the $y$ direction is $n_0 L_y$!
With this choice, the final relation between $P_{\rm add}$ and the source amplitude $\mathcal{S}_0$ evaluates to
\bea
P_{\rm add} & = &\hat{\mathcal{S}}_0 n_\rf T_\rf \rho_\rf^3\frac{c_\rf}{L_\rf}
\eea
which can alternatively be written as
\bea
P_{\rm add} & = &\hat{\mathcal{S}}_0 \cdot 1.6726\cdot 10^{-5} \cdot n_{e19} T_{\rf ,\mathrm{keV}}^3 \frac{m_\rf}{m_p}\frac{1}{B_{\rf,T}^3}\frac{1}{L_{\rf,m}}\, \mathrm{MW}
\eea
with electron density $n_{e19}$ in $10^{19}\mathrm{m}^{-3}$, $T_{\rf ,\mathrm{keV}}$ in units of $\mathrm{keV}$, $B_{\rf,\mathrm{T}}$ in Tesla, and the reference length $L_{\rf,\mathrm{m}}$ in $\mathrm{m}$. \\
\\
For an {\sc ITER}-like deuterium plasma with $n_\rf = 5\cdot 10^{19}\,\mathrm{m}^{-3}$, $T_\rf=5\,\mathrm{keV}$ (taken at mid-minor radius), $m_\rf/m_p=2$, $B_\rf = 5\,\mathrm{T}$ (taken at magnetic axis), and $L_\rf = R_0 = 6.21\,\mathrm{m}$, $\hat{\mathcal{S}}_0$ should be about $3.7\cdot 10^{5}$ to achieve a power injection of about $10\,\mathrm{MW}$.\\
For an {\sc Asdex}-Upgrade deuterium plasma, we have $n_\rf \approx 4\cdot 10^{19}\,\mathrm{m}^{-3}$, $T_\rf\approx 3\,\mathrm{keV}$ (taken at mid-minor radius), $m_\rf/m_p=2$, $B_\rf \approx 2.5\,\mathrm{T}$ (taken at magnetic axis), and $L_\rf = R_0 = 1.65\,\mathrm{m}$, $\hat{\mathcal{S}}_0$ should thus be about $7.1\cdot 10^{4}$ for a $10\,\mathrm{MW}$ power injection.

%%% Local Variables:
%%% mode: latex
%%% TeX-master: "globalgene"
%%% End:



\chapter{Datastructures and Implementation}
\label{sec:implementation}
In this chapter, we will describe in more detail some implementation
details like datastructures and algorithms, which are used in the
global gene version. The content of this chapter mainly concerns the
$x$-global version of \gene.

\section{Datastructures}
\label{sec:datastructures}

\subsection{\texttt{Matrix}}
\label{sec:matrix}
The \texttt{Matrix} type is used as a kind of abstract class for a
matrix. It defines the interfaces for the operations which can be done
with a matrix. The real storage of the data and the real
implementation of the functions is done in the
\texttt{StoreFullMatrixObject} type. The idea of this splitting was,
that in the future, when object-oriented structures are possible in
Fortran (Fortran 2003 and further in Fortran 2008), we can rewrite
these types as real classes with real relationship. Then
\texttt{Matrix} will be the only class which is used from outside and
all the storage details and even the fact if the matrix is Banded,
Diagonal or something else is hidden from the user, but determined
automatically by the class itself.


\subsection{\texttt{BandedMatrix}}
At the moment we need to tell in advance from the user side if a
matrix can be treated as banded or as full. This type is mainly used
for the gyromatrix in $x$-direction. This matrix is a band matrix with
a bandwidth determined by the gyroradius-to-gridspace ratio. For the
ions this is always much larger than for the electrons due to their
larger mass and therefore gyroradius.

This is like \texttt{Matrix} just an abstract interface class, the
storage and distribution details are hidden in the
\texttt{StoreBandedMatrixObject} type.

\subsection{\texttt{Vector}}
Like the other two types \texttt{Matrix} and \texttt{BandedMatrix}
this type is just for defining the interface. Storage and calculation
details are in the \texttt{StoreVectorObject} type.

\subsection{\texttt{StoreFullMatrixObject}}
A full matrix is stored in this object. As we want to be able to use
ScaLapack routines at some point in the code, we setup the storage
format in a way usable be these routines. This means, we are using the
twodimensional cyclic block distribution. This concerns the
distribution over the processes and the local storage of the local
parts of the matrix. 

First a processgrid is initialized, where in principle one could choose between three
possibilities of how to distribute the matrix over the processes. First
is to store blocks of columns (\texttt{MAT\_BLOCKS\_OF\_COLS}) on the processes, second is to store
blocks of rows (\texttt{MAT\_BLOCKS\_OF\_ROWS}) on the processes, and the third one to make subblocks
of the matrix (\texttt{MAT\_BOTH}) and distribute them over the processes. The last
possibility is not yet implemented, the first one is not tested, so
the default is to use blocks of rows.

The global view of the distribution is then:
\begin{equation}
  \label{eq:matrix}
  \left(\begin{array}{cccccccc}
    a_{11} &a_{12} &a_{13}& a_{14} & a_{15} & a_{16} & a_{17} & a_{18}\\
    a_{21} &a_{22} &a_{23}& a_{24} & a_{25} & a_{26} & a_{27} & a_{28}\\
    a_{31} &a_{32} &a_{33}& a_{34} & a_{35} & a_{36} & a_{37} & a_{38}\\
    a_{41} &a_{42} &a_{43}&a_{44}& a_{45} & a_{46} & a_{47} & a_{48}\\\hline
    a_{51} &a_{52} &a_{53}&a_{54} &a_{55}& a_{56} & a_{57} & a_{58}\\
    a_{61} &a_{62} &a_{63}&a_{64} &a_{65} &a_{66}& a_{67} & a_{68}\\
    a_{71} &a_{72} &a_{73}&a_{74} &a_{75} &a_{76} &a_{77}& a_{78}\\
    a_{81} &a_{82} &a_{83}&a_{84} &a_{85} &a_{86}  &a_{87} &a_{88}
  \end{array}\right)
\end{equation}

How is this matrix stored locally?
Although the ScaLapack storage format is quite complicated, it is
simplified in our case by the fact, that we use only a one-dimensional
process grid. This means in the case of blocks of rows (the default),
the storage is exactly as shown in the global view. But we can
nevertheless make use of the ScaLapack routines for the field
solver. These ScaLapack routines does not scale optimal in our
distribution, and for an extension to larger processor number in the
$x$-direction or in the 3D case, we have to rethink the distribution.

\subsection{\texttt{StoreBandedMatrixObject}}
For the storage of the \texttt{BandedMatrix} a self-defined storage
format has been chosen, as this was the easiest way to have all the
features we need. The data is distributed by columns, the number of
columns per process is given by the attribute
\texttt{ColsPerBlock}. The banded structure of the Matrix is
represented by the \texttt{UpperBandwidth} and \texttt{LowerBandwidth}
attribute. None of these includes the diagonal. The local 2D array in
which the data is stored has the dimensions
\begin{displaymath}
  \mathtt{NumberOfStoredRows} \times\mathtt{ColsPerBlock} 
\end{displaymath}
where
\texttt{NumberOfStoredRows}=\texttt{UpperBandwidth}+\texttt{LowerBandwidth}+1. How
an element from the global matrix maps to the local storage is given
by the following graph for a $8\times8$ matrix with \texttt{UpperBandwidth}=2 and
\texttt{LowerBandwidth}=1.
\begin{equation}
  \label{eq:matrix}
  \left(\begin{array}{cccc|cccc}
    a_{11} &a_{12} &a_{13}& 0\\
    a_{21} &a_{22} &a_{23}& a_{24} & 0\\
    0 &a_{32} &a_{33}& a_{34} & a_{35} & 0\\
    0 & 0 &a_{43} &a_{44}& a_{45} & a_{46} & 0\\
    0 & 0 &0 &a_{54} &a_{55}& a_{56} & a_{57} & 0\\
    0 & 0 & 0 &0 &a_{65} &a_{66}& a_{67} & a_{68}\\
    0 & 0 & 0 & 0 &0 &a_{76} &a_{77}& a_{78}\\
    0 & 0 & 0 & 0 & 0 &0 &a_{87} &a_{88}
  \end{array}\right)
\end{equation}
This matrix is stored in an $4\times8$ array if there is no
distribution over processes.
\begin{displaymath}
  \begin{array}{c}
    -\mathtt{UpperBandwidth}\\
    \vdots\\
    0\\
    \mathtt{LowerBandwidth}
  \end{array}
\left[
  \begin{array}{cccc|cccc}
    * & * & a_{13} &  a_{24} & a_{35} & a_{46} & a_{57} & a_{68}\\
    * & a_{12} & a_{23} & a_{34} & a_{45} & a_{56} & a_{67} & a_{78}\\
    a_{11} & a_{22} & a_{33} & a_{44} & a_{55} & a_{66} & a_{77} & a_{88}\\
    a_{21} & a_{32} & a_{43} & a_{54} & a_{65} & a_{76} & a_{87} & *\\
  \end{array}
\right]
\end{displaymath}
If one has a distribution over two processes the array is separated at
the indicated line. Then the local array just has an extent of
$4\times4$.

To convert global indices $(r,c)$ into indices $(i,j)$ in the
\texttt{localMatrix}, we have to perform the following transformation
\begin{displaymath}
  i=r-c\qquad\mbox{and}\qquad j=((c-1)\mod\mathtt{ColsPerBlock})+1
\end{displaymath}
The usual storage order in memory is column-major as usual in
Fortran. This means the values of one column are contiguous in
memory. But as one can imagine this is not ideal for dot product with
a vector. There a row-major storage order would be beneficial. Hence,
for improving performance, the user can store a matrix in row-major
order if it is known that the matrix will be used mainly for
matrix-vector multiplication as the gyromatrix. This attribute of the
matrix is given at initialization time of the matrix as an additional
argument.

If the matrix is stored in row-major order, the storage layout of the
matrix, defined in eq. (\ref{eq:matrix}) is
\begin{displaymath}
  \begin{array}{c}
    -\mathtt{LowerBandwidth}\\
    \vdots\\
    0\\
    \mathtt{UpperBandwidth}
  \end{array}
\left[
  \begin{array}{cccc|cccc}
    *     & a_{21} & a_{32} & a_{43} & a_{54} & a_{65} & a_{76} & a_{87}\\
    a_{11} & a_{22} & a_{33} & a_{44} & a_{55} & a_{66} & a_{77} & a_{88}\\
    a_{12} & a_{23} & a_{34} & a_{45} & a_{56} & a_{67} & a_{78} & *\\
    a_{13} & a_{24} & a_{35} & a_{46} & a_{57} & a_{68} & * & *
  \end{array}
\right]
\end{displaymath}
Here the transformation rules from global $(r,c)$ indices to
\texttt{localMatrix} indices $(i,j)$ are
\begin{displaymath}
  i=c-r\qquad\mbox{and}\qquad j=((r-1)\mod\mathtt{ColsPerBlock})+1
\end{displaymath}


\subsection{StoreVectorObject}
A vector is stored as a one dimensional array. It is trivially
distributed over the rows.
\begin{displaymath}
  \left(
  \begin{array}{c}
    c_1\\
    c_2\\
    c_3\\
    c_4\\
    \hline
    c_5\\
    c_6\\
    c_7\\
    c_8
  \end{array}\right)\Longrightarrow
\left[
  \begin{array}{cccc|cccc}
    c_1 & c_2 & c_3 & c_4 & c_5 & c_6 & c_7 & c_8
  \end{array}
\right]
\end{displaymath}

\subsection{Matrix-Vector multiplication}
This is the most relevant operation for the gyro-average and has to be
done very often. Therefore we optimized it. The gyromatrix is stored
in row-major order, so a row of the matrix is always local to one
processor. 
For a matrix-vector multiplication, the vector is first gathered on
all processes and then multiplied with the local rows. The resulting
vector is already correctly distributed over the processes.

If the matrix is not transposed, first the local parts of the
matrix-vector multiplication are calculated on all processes in
parallel. Then a reduction is done over the processes.

\subsection{Matrix-Matrix multiplication}
For the matrix-matrix multiplication there have to be distinguished
several cases, which are different in the implementation.
\begin{enumerate}
\item Matrix-Matrix multiplication with two full matrices
\item BandedMatrix-BandedMatrix multiplication
  At the moment only the case with first matrix in row-major order and
  second matrix in column-major order is implemented.
\item BandedMatrix-Matrix multiplication
  Here also only the case with the BandedMatrix in row-major order and
  the Matrix in column-major order is implemented. The result is
  stored in column-major order.
\end{enumerate}

\subsubsection{BandedMatrix-BandedMatrix multiplication}
First, we will have a look on this independent of the storage
format. The operation is 
\begin{displaymath}
  A\cdot B = R
\end{displaymath}
or with the matrices written explicitly, assuming a dimension of 6. We
also use for matrix $A$ \texttt{UpperBandwidth}=2 and
\texttt{LowerBandwidth}=1, for matrix $B$ \texttt{UpperBandwidth}=1
and \texttt{LowerBandwidth}=0.
\begin{equation}
  \label{eq:bmatbmatbmat}
  \left(\begin{array}{cccccc}
    r_{11} &r_{12} &r_{13}& r_{14} & 0 & 0\\
    r_{21} &r_{22} &r_{23}& r_{24} & r_{25} & 0\\
    0 &r_{32} &r_{33}& r_{34} & r_{35} & r_{36}\\\hline
    0 & 0 &r_{43}& r_{44} & r_{45} & r_{46} \\
    0 & 0 & 0 & r_{54} & r_{55} & r_{56} \\
    0 & 0 & 0 & 0 & r_{65} & r_{66}\\
  \end{array}\right) =
  \left(\begin{array}{cccccc}
    a_{11} &a_{12} &a_{13}& 0\\
    a_{21} &a_{22} &a_{23}& a_{24} & 0\\
    0 &a_{32} &a_{33}& a_{34} & a_{35} & 0\\\hline
    0 & 0 &a_{43} &a_{44}& a_{45} & a_{46} \\
    0 & 0 &0 &a_{54} &a_{55}& a_{56}\\
    0 & 0 & 0 &0 &a_{65} &a_{66}
  \end{array}\right)
\cdot
  \left(\begin{array}{cccccc}
    b_{11} &b_{12} & 0 & 0\\
    0 &b_{22} &b_{23}& 0 & 0\\
    0 & 0 &b_{33}& b_{34} & 0 & 0\\\hline
    0 & 0 & 0 &b_{44}& b_{45} & 0 \\
    0 & 0 & 0 & 0 &b_{55}& b_{56}\\
    0 & 0 & 0 & 0 & 0 &b_{66}
  \end{array}\right)
\end{equation}
The upper and lower bandwidths of the matrices $A$ and $B$ just add
and give the upper and lower bandwidth of the result matrix. In our
example case $R$ has \texttt{UpperBandwidth}=3 and
\texttt{LowerBandwidth}=1. 

The usual definition of the matrix-matrix multiplication in index form
is
\begin{displaymath}
  r_{ij}=\sum_{k=1}^N a_{ik} b_{kj}\qquad i,j\in\{1,\ldots,N\}
\end{displaymath}
We divide the full index space of $\{1,\ldots,N\}$ into $N_p$
intervals, one for each processor.
\begin{displaymath}
  I_p=\{(p-1)\frac{N}{N_p}+1,\ldots,p\frac{N}{N_p}\}\qquad p\in\{1,\ldots,N_p\}
\end{displaymath}
Then we can separate the sum into sums over the intervals and sum the
partial sums.
\begin{displaymath}
  r_{ij}=\sum_{p=1}^{N_p}\sum_{k\in I_p} a_{ik} b_{kj}\qquad i,j\in\{1,\ldots,N\}
\end{displaymath}
We also separate the different rows of the result matrix in the
intervals
\begin{equation}
  \label{eq:distr_mat}
  r_{ij}^{(s)}=\sum_{p=1}^{N_p}\sum_{k\in I_p} a_{ik}^{(s)}
  b_{kj}^{(p)}\qquad i\in I_s \wedge j\in\{1,\ldots,N\}
\end{equation}
where the upper index indicates the number of the processor where the
data is local.
Hence, we can only do calculations with data which is on the same
processor, otherwise we need communication.

We expand the $p$ sum in eq. (\ref{eq:distr_mat}) and get
\begin{equation}
  \label{eq:distr_mat_exp}
  r_{ij}^{(s)}=\sum_{k\in I_1} a_{ik}^{(s)}b_{kj}^{(1)}
  +\sum_{k\in I_2} a_{ik}^{(s)}b_{kj}^{(2)}
  +\ldots
  +\sum_{k\in I_{N_p}} a_{ik}^{(s)}b_{kj}^{({N_p})}
  \qquad i\in I_s \wedge j\in\{1,\ldots,N\}
\end{equation}
Only one summand of this expansion can be computed directly, due to
the locality of the data. For all other computations, we first have to
transfer a block of the $B$ matrix to the other processors.

In the computation of the dot product of two blocks (one subsum term in the
above sum) we have now to take into account the banded structure of
both matrices. This is done by constraining the index intervals to the
indices of the band. Therefore we define the column index set of band
matrix $M$ in row $i$ to be
\begin{equation}
  \label{eq:colindexset}
  \mathcal{B}_i(M)=\{i-\mathtt{LowerBandwidth}(M),i+\mathtt{UpperBandwidth}(M)\}
\end{equation}
and the row index set of a band matrix $M$ in column $j$ as
\begin{equation}
  \label{eq:rowindexset}
  \bar{\mathcal{B}}_j(M)=\{j-\mathtt{UpperBandwidth}(M),j+\mathtt{LowerBandwidth}(M)\}.
\end{equation}
The sum can then be rewritten by
\begin{equation}
  \label{eq:distr_mat}
  r_{ij}^{(s)}=\sum_{p=1}^{N_p}\sum_{k\in J_p} a_{ik}^{(s)}
  b_{kj}^{(p)}\qquad i\in J_s \wedge j\in\mathcal{B}_i(R)
\end{equation}
with the reduced index set
\begin{displaymath}
  J_p=I_p\cap\mathcal{B}_i(A)\cap\bar{\mathcal{B}}_j(B)
\end{displaymath}
In our example, we have for $p=1, N=8, N_p=2$
\begin{eqnarray*}
  I_1&=&\{1,2,3,4\}\\
  \mathcal{B}_1(A) &=& \{1-1,\ldots,1+2\}=\{0,1,2,3\}\\
  \bar{\mathcal{B}}_1(B) &=& \{1-1,\ldots,1+0\}=\{0,1\}\\
  J_1 &=& \{1,2,3,4\}\cap\{0,1,2,3\}\cap\{0,1\}=\{1\}
\end{eqnarray*}
and for $p=2$
\begin{eqnarray*}
  I_2&=&\{5,6,7,8\}\\
  \mathcal{B}_1(A) &=& \{1-1,\ldots,1+2\}=\{0,1,2,3\}\\
  \bar{\mathcal{B}}_1(B) &=& \{1-1,\ldots,1+0\}=\{0,1\}\\
  J_2 &=& \{5,6,7,8\}\cap\{0,1,2,3\}\cap\{0,1\}=\{\}
\end{eqnarray*}
Then we have for the $r_{11}$ element of the result matrix the
expression
\begin{displaymath}
  r_{11}^{(1)}=a_{11}^{(1)}b_{11}^{(1)}
\end{displaymath}

I will elaborate a bit more about the case where all three involved
matrices are stored in row-major (``C'') order. 
For each matrix, the rows are stored contiguously. The stored arrays
of the example matrices are then
\begin{displaymath}
  \mathtt{localMatrix}(A)=  \begin{array}{c}
    -1\\
    0\\
    1\\
    2
  \end{array}
\left[
  \begin{array}{ccc|ccc}
    *     & a_{21} & a_{32} & a_{43} & a_{54} & a_{65} \\
    a_{11} & a_{22} & a_{33} & a_{44} & a_{55} & a_{66} \\
    a_{12} & a_{23} & a_{34} & a_{45} & a_{56} & * \\
    a_{13} & a_{24} & a_{35} & a_{46} & * & * 
  \end{array}
\right]
\end{displaymath}
and
\begin{displaymath}
  \mathtt{localMatrix}(B)=  \begin{array}{c}
    0\\
    1
  \end{array}
\left[
  \begin{array}{ccc|ccc}
    b_{11} & b_{22} & b_{33} & b_{44} & b_{55} & b_{66} \\
    b_{12} & b_{23} & b_{34} & b_{45} & b_{56} & * 
  \end{array}
\right]
\end{displaymath}
We continously have to multiply the local block of matrix $A$ with the
received blocks from matrix $B$. We start with 
\begin{displaymath}
  \left(\begin{array}{cccccc}
    r_{11} &r_{12} &r_{13}& r_{14} & 0 & 0\\
    r_{21} &r_{22} &r_{23}& r_{24} & r_{25} & 0\\
    0 &r_{32} &r_{33}& r_{34} & r_{35} & r_{36}
  \end{array}\right) \stackrel{+}{\longleftarrow}
  \left(\begin{array}{ccc}
      a_{11} &a_{12} &a_{13} \\
      a_{21} &a_{22} &a_{23}\\
      0 &a_{32} &a_{33}
  \end{array}\right)
\cdot
  \left(\begin{array}{cccccc}
    b_{11} &b_{12} & 0 & 0 & 0 & 0\\
    0 &b_{22} &b_{23}& 0 & 0 & 0\\
    0 & 0 &b_{33}& b_{34} & 0 & 0
  \end{array}\right)
\end{displaymath}
and then
\begin{displaymath}
  \left(\begin{array}{cccccc}
    r_{11} &r_{12} &r_{13}& r_{14} & 0 & 0\\
    r_{21} &r_{22} &r_{23}& r_{24} & r_{25} & 0\\
    0 &r_{32} &r_{33}& r_{34} & r_{35} & r_{36}
  \end{array}\right) \stackrel{+}{\longleftarrow}
  \left(\begin{array}{ccc}
      0      & 0 & 0\\
      a_{24} & 0 & 0\\
      a_{34} & a_{35} & 0
  \end{array}\right)
\cdot
  \left(\begin{array}{cccccc}
    0 & 0 & 0 &b_{44}& b_{45} & 0 \\
    0 & 0 & 0 & 0 &b_{55}& b_{56}\\
    0 & 0 & 0 & 0 & 0 &b_{66}
  \end{array}\right)
\end{displaymath}
Or written with the stored blocks:
\begin{eqnarray*}
  p=0, I_0=\{1,2,3\}\\
  \left(\begin{array}{cccccc}
    r_{11} &r_{12} &r_{13}& r_{14} & 0 & 0\\
    r_{21} &r_{22} &r_{23}& r_{24} & r_{25} & 0\\
    0 &r_{32} &r_{33}& r_{34} & r_{35} & r_{36}
  \end{array}\right) \stackrel{+}{\longleftarrow}
\left[
  \begin{array}{ccc}
    *     & a_{21} & a_{32}  \\
    a_{11} & a_{22} & a_{33} \\
    a_{12} & a_{23} & a_{34}\\
    a_{13} & a_{24} & a_{35} 
  \end{array}
\right]*
\left[
  \begin{array}{ccc}
    b_{11} & b_{22} & b_{33} \\
    b_{12} & b_{23} & b_{34} 
  \end{array}
\right]
\end{eqnarray*}
and then
\begin{eqnarray*}
  p=1, I_1=\{4,5,6\}\\
  \left(\begin{array}{cccccc}
    r_{11} &r_{12} &r_{13}& r_{14} & 0 & 0\\
    r_{21} &r_{22} &r_{23}& r_{24} & r_{25} & 0\\
    0 &r_{32} &r_{33}& r_{34} & r_{35} & r_{36}
  \end{array}\right) \stackrel{+}{\longleftarrow}
\left[
  \begin{array}{ccc}
    *     & a_{21} & a_{32}  \\
    a_{11} & a_{22} & a_{33} \\
    a_{12} & a_{23} & a_{34} \\
    a_{13} & a_{24} & a_{35}
  \end{array}
\right]*
\left[
  \begin{array}{ccc}
    b_{44} & b_{55} & b_{66} \\
    b_{45} & b_{56} & * 
  \end{array}
\right]
\end{eqnarray*}

\subsubsection{BandedMatrix (column-major)-BandedMatrix(row-major) multiplication}
Another multiplication scheme, which involves a different algorithm is
the multiplication of a BandedMatrix in column-major format with a
BandedMatrix in row-major format.
The matrix equation for the given example is the
\begin{equation}
  \label{eq:bmatcbmatrbmatr}
  \left(\begin{array}{cccccc}
      r_{11} &r_{12} &r_{13}& r_{14} & 0 & 0\\
      r_{21} &r_{22} &r_{23}& r_{24} & r_{25} & 0\\
      0 &r_{32} &r_{33}& r_{34} & r_{35} & r_{36}\\\hline
      0 & 0 &r_{43}& r_{44} & r_{45} & r_{46} \\
      0 & 0 & 0 & r_{54} & r_{55} & r_{56} \\
      0 & 0 & 0 & 0 & r_{65} & r_{66}\\
    \end{array}\right) =
  \left(\begin{array}{ccc|ccc}
      a_{11} &a_{12} &a_{13}& 0\\
      a_{21} &a_{22} &a_{23}& a_{24} & 0\\
      0 &a_{32} &a_{33}& a_{34} & a_{35} & 0\\
      0 & 0 &a_{43} &a_{44}& a_{45} & a_{46} \\
      0 & 0 &0 &a_{54} &a_{55}& a_{56}\\
      0 & 0 & 0 &0 &a_{65} &a_{66}
    \end{array}\right)
  \cdot
  \left(\begin{array}{cccccc}
      b_{11} &b_{12} & 0 & 0\\
      0 &b_{22} &b_{23}& 0 & 0\\
      0 & 0 &b_{33}& b_{34} & 0 & 0\\\hline
      0 & 0 & 0 &b_{44}& b_{45} & 0 \\
      0 & 0 & 0 & 0 &b_{55}& b_{56}\\
      0 & 0 & 0 & 0 & 0 &b_{66}
    \end{array}\right)
\end{equation}
The only difference is the different parallelization of the $A$
matrix. But this difference has consequences in the algorithm.

Again we can write for the elements of the result matrix the matrix
product in index writing:
\begin{equation}
  \label{eq:distr_mat}
  r_{ij}^{(s)}=\sum_{p=1}^{N_p}\sum_{k\in I_p} a_{ik}^{(p)}
  b_{kj}^{(p)}\qquad i\in I_s \wedge j\in\{1,\ldots,N\}
\end{equation}

The algorithm for this multiplication is to multiply the respective
local submatrices on the processors and then reduce the results to the
correct processors (the first three rows to the first processor, the
second three rows to the second processor in the example). For the
subsums we have to take into account the banded structure of the
matrices as before. This is again done by using the reduced index set
$J_p$. The local result matrix is a bandedmatrix 
\begin{equation}
  \label{eq:distr_mat}
  r_{ij}^{(s)}=\sum_{p=1}^{N_p}\sum_{k\in J_p} a_{ik}^{(p)}
  b_{kj}^{(p)}\qquad i\in I_s \wedge j\in\{1,\ldots,N\}
\end{equation}
Written with the stored \texttt{localmatrix} array, we have
\begin{displaymath}
  \mathtt{localMatrix}(A)=  \begin{array}{c}
    -2\\
    -1\\
    0\\
    1
  \end{array}
\left[
  \begin{array}{ccc|ccc}
    * & * & a_{13} &  a_{24} & a_{35} & a_{46} \\
    * & a_{12} & a_{23} & a_{34} & a_{45} & a_{56} \\
    a_{11} & a_{22} & a_{33} & a_{44} & a_{55} & a_{66} \\
    a_{21} & a_{32} & a_{43} & a_{54} & a_{65} & * 
  \end{array}
\right]
\end{displaymath}
and
\begin{displaymath}
  \mathtt{localMatrix}(B)=  \begin{array}{c}
    0\\
    1
  \end{array}
\left[
  \begin{array}{ccc|ccc}
    b_{11} & b_{22} & b_{33} & b_{44} & b_{55} & b_{66} \\
    b_{12} & b_{23} & b_{34} & b_{45} & b_{56} & * 
  \end{array}
\right]
\end{displaymath}

\subsubsection{BandedMatrix (row-major)-BandedMatrix(column-major) $\rightarrow$ BandedMatrix(column-major) multiplication}
Another multiplication scheme, which involves a different algorithm is
the multiplication of a BandedMatrix in row-major format with a
BandedMatrix in column-major format and the result is written in column-major format.
The matrix equation for the given example is the
\begin{equation}
  \label{eq:bmatrbmatcbmatc}
  \left(\begin{array}{ccc|ccc}
      r_{11} &r_{12} &r_{13}& r_{14} & 0 & 0\\
      r_{21} &r_{22} &r_{23}& r_{24} & r_{25} & 0\\
      0 &r_{32} &r_{33}& r_{34} & r_{35} & r_{36}\\
      0 & 0 &r_{43}& r_{44} & r_{45} & r_{46} \\
      0 & 0 & 0 & r_{54} & r_{55} & r_{56} \\
      0 & 0 & 0 & 0 & r_{65} & r_{66}\\
    \end{array}\right) =
  \left(\begin{array}{cccccc}
      a_{11} &a_{12} &a_{13}& 0\\
      a_{21} &a_{22} &a_{23}& a_{24} & 0\\
      0 &a_{32} &a_{33}& a_{34} & a_{35} & 0\\ \hline
      0 & 0 &a_{43} &a_{44}& a_{45} & a_{46} \\
      0 & 0 &0 &a_{54} &a_{55}& a_{56}\\
      0 & 0 & 0 &0 &a_{65} &a_{66}
    \end{array}\right)
  \cdot
  \left(\begin{array}{ccc|ccc}
      b_{11} &b_{12} & 0 & 0\\
      0 &b_{22} &b_{23}& 0 & 0\\
      0 & 0 &b_{33}& b_{34} & 0 & 0\\
      0 & 0 & 0 &b_{44}& b_{45} & 0 \\
      0 & 0 & 0 & 0 &b_{55}& b_{56}\\
      0 & 0 & 0 & 0 & 0 &b_{66}
    \end{array}\right)
\end{equation}

We start again with the definition of the dot product of two matrices:
\begin{displaymath}
  r_{ij}=\sum_{k=1}^N a_{ik}b_{kj}\qquad i,j\in\{1,\ldots,N\}
\end{displaymath}
But now we do not divide the $k$ sum, as for one $i$ all $a_{ik}$ and
for one $j$ all $b_{kj}$ are local in the given storage formats. We
now have to separate the calculation of different $r_{ij}$. With the
usual division of the total index space $\{1,\ldots,N\}$ into $N_p$
subintervals $I_p$ with
\begin{displaymath}
  I_p=\{(p-1)\frac{N}{N_p}+1,\ldots,p\frac{N}{N_p}\}\qquad p\in\{1,\ldots,N_p\}
\end{displaymath}
we can write
\begin{displaymath}
  r_{ij}^{(s)}=\sum_{k=1}^N a_{ik}^{(p)}b_{kj}^{(s)}\qquad
  i\in I_p\wedge j\in I_s
\end{displaymath}
Only for $s=p$ we can do the calculation without communication, for
all other cases, we have to transfer parts of matrix $A$ from process
$p$ to process $s$. 

For the multiplication of the blocks, we have again to take into
account the banded structure of the matrices. Hence, we use again the
definitions (\ref{eq:colindexset}) and (\ref{eq:rowindexset}).
We get then
\begin{displaymath}
  r_{ij}^{(s)}=\sum_{k\in K_{ij}}^N a_{ik}^{(p)}b_{kj}^{(s)}\qquad
  i\in I_p\wedge j\in I_s
\end{displaymath}
with 
\begin{displaymath}
  K_{ij}=\{1,\ldots,N\}\cap\mathcal{B}_i(A)\cap\bar{\mathcal{B}}_j(B).
\end{displaymath}
And now additionally we can constrain the indices $i,j$ for the result
matrix, as we know already which entries in $R$ are zero just due to
the structure of the matrices $A$ and $B$.
There are two equivalent possibilities to write this, first
\begin{displaymath}
  r_{ij}^{(s)}=\sum_{k\in K_{ij}}^N a_{ik}^{(p)}b_{kj}^{(s)}\qquad
  i\in I_p\wedge j\in I_s\cap\mathcal{B}_i(R)
\end{displaymath}
and second
\begin{displaymath}
  r_{ij}^{(s)}=\sum_{k\in K_{ij}}^N a_{ik}^{(p)}b_{kj}^{(s)}\qquad
  j\in I_s\wedge i\in I_p\cap\bar{\mathcal{B}}_j(R)
\end{displaymath}

In a simplified algorithm, we can gather the matrix $A$ on all
processes and can then write the last equation as
\begin{displaymath}
  r_{ij}^{(s)}=\sum_{k\in K_{ij}}^N a_{ik}b_{kj}^{(s)}\qquad
  j\in I_s\wedge i\in\{1,\ldots,N\}\cap\bar{\mathcal{B}}_j(R)
\end{displaymath}


Written with the stored \texttt{localmatrix} array, we have
\begin{displaymath}
  \mathtt{localMatrix}(A)=  \begin{array}{c}
    -1\\
    0\\
    1\\
    2
  \end{array}
\left[
  \begin{array}{ccc|ccc}
    *     & a_{21} & a_{32} & a_{43} & a_{54} & a_{65} \\
    a_{11} & a_{22} & a_{33} & a_{44} & a_{55} & a_{66} \\
    a_{12} & a_{23} & a_{34} & a_{45} & a_{56} & * \\
    a_{13} & a_{24} & a_{35} & a_{46} & *     & * 
  \end{array}
\right]
\end{displaymath}
and
\begin{displaymath}
  \mathtt{localMatrix}(B)=  \begin{array}{c}
    -1\\
    0
  \end{array}
\left[
  \begin{array}{ccc|ccc}
    *     & b_{12} & b_{23} & b_{34} & b_{45} & b_{56} \\ 
    b_{11} & b_{22} & b_{33} & b_{44} & b_{55} & b_{66} \\
  \end{array}
\right]
\end{displaymath}
and
\begin{displaymath}
  \mathtt{localMatrix}(R)=  \begin{array}{c}
    -3\\
    -2\\
    -1 \\
    0\\
    1
  \end{array}
\left[
  \begin{array}{ccc|ccc}
    *     & *      & *     & r_{14} & r_{25} & r_{36} \\ 
    *     & *      & r_{13} & r_{24} & r_{35} & r_{46} \\ 
    *     & r_{12} & r_{23} & r_{34} & r_{45} & r_{56} \\ 
    r_{11} & r_{22} & r_{33} & r_{44} & r_{55} & r_{66} \\
    r_{21} & r_{32} & r_{43} & r_{54} & r_{65} & * \\ 
  \end{array}
\right]
\end{displaymath}


\subsubsection{BandedMatrix-Matrix multiplication}
The last case will be derived here in more detail.
\begin{equation}
  \label{eq:bmatmatmat}
  \left(\begin{array}{cccccc}
    r_{11} &r_{12} &r_{13}& r_{14} & r_{15} & r_{16}\\
    r_{21} &r_{22} &r_{23}& r_{24} & r_{25} & r_{26}\\
    r_{31} &r_{32} &r_{33}& r_{34} & r_{35} & r_{36}\\\hline
    r_{41} &r_{42} &r_{43}& r_{44} & r_{45} & r_{46} \\
    r_{51} &r_{52} &r_{53}& r_{54} & r_{55} & r_{56} \\
    r_{61} &r_{62} &r_{63}& r_{64} & r_{65} & r_{66}\\
  \end{array}\right) =
  \left(\begin{array}{cccccc}
    a_{11} &a_{12} &a_{13}& 0\\
    a_{21} &a_{22} &a_{23}& a_{24} & 0\\
    0 &a_{32} &a_{33}& a_{34} & a_{35} & 0\\\hline
    0 & 0 &a_{43} &a_{44}& a_{45} & a_{46} \\
    0 & 0 &0 &a_{54} &a_{55}& a_{56}\\
    0 & 0 & 0 &0 &a_{65} &a_{66}
  \end{array}\right)
\cdot
  \left(\begin{array}{cccccccc}
    b_{11} &b_{12} &b_{13}& b_{14} & b_{15} & b_{16}\\
    b_{21} &b_{22} &b_{23}& b_{24} & b_{25} & b_{26}\\
    b_{31} &b_{32} &b_{33}& b_{34} & b_{35} & b_{36}\\\hline
    b_{41} &b_{42} &b_{43}& b_{44} & b_{45} & b_{46}\\
    b_{51} &b_{52} &b_{53}& b_{54} & b_{55} & b_{56}\\
    b_{61} &b_{62} &b_{63}& b_{64} & b_{65} & b_{66}\\
  \end{array}\right)
\end{equation}
The usual definition of the matrix-matrix multiplication in index form
is
\begin{displaymath}
  r_{ij}=\sum_{k=1}^N a_{ik} b_{kj}\qquad i,j\in\{1,\ldots,N\}
\end{displaymath}
We divide the full index space of $\{1,\ldots,N\}$ into $N_p$
intervals, one for each processor.
\begin{displaymath}
  I_p=\{(p-1)\frac{N}{N_p}+1,\ldots,p\frac{N}{N_p}\}\qquad p\in\{1,\ldots,N_p\}
\end{displaymath}
Then we can separate the sum into sums over the intervals and sum the
partial sums.
\begin{displaymath}
  r_{ij}=\sum_{p=1}^{N_p}\sum_{k\in I_p} a_{ik} b_{kj}\qquad i,j\in\{1,\ldots,N\}
\end{displaymath}
We also separate the different rows of the result matrix in the
intervals
\begin{equation}
  \label{eq:distr_mat}
  r_{ij}^{(s)}=\sum_{p=1}^{N_p}\sum_{k\in I_p} a_{ik}^{(s)}
  b_{kj}^{(p)}\qquad i\in I_s \wedge j\in\{1,\ldots,N\}
\end{equation}
where the upper index indicates the number of the processor where the
data is local.
Hence, we can only do calculations with data which is on the same
processor, otherwise we need communication.

We expand the $p$ sum in eq. (\ref{eq:distr_mat}) and get
\begin{equation}
  \label{eq:distr_mat_exp}
  r_{ij}^{(s)}=\sum_{k\in I_1} a_{ik}^{(s)}b_{kj}^{(1)}
  +\sum_{k\in I_2} a_{ik}^{(s)}b_{kj}^{(2)}
  +\ldots
  +\sum_{k\in I_{N_p}} a_{ik}^{(s)}b_{kj}^{({N_p})}
  \qquad i\in I_s \wedge j\in\{1,\ldots,N\}
\end{equation}
Only one summand of this expansion can be computed directly, due to
the locality of the data. For all other computations, we first have to
transfer a block of the $b$ matrix to the other processors.

To overlay computation with communication we adopt the following
algorithm. Always initiate a non-blocking receive for the next block,
then send non-blocking the local block to the previous processor. Then
while waiting for the communication, do the local computation.

%%% Local Variables:
%%% mode: latex
%%% TeX-master: "globalgene"
%%% End:

\appendix
\chapter{Standard coordinate systems}

\section{Cartesian coordinates $x,y,z$}
First we start with a Cartesian  system, where the metric tensor it just the
unit matrix. With the usage of $x^1=x_1=x$, $x^2=x_2=y$ and $x^3=x_3=z$, we arrive at the Lagrangian
\begin{displaymath}
  L(\mathbf{x},\mathbf{v}) =
  mv_j\dot x^j
  +\frac{e}{c}A_j(\mathbf{x})\dot x^j
  -\frac{m}{2}v_jv^j
  -e\Phi(\mathbf{x})
\end{displaymath}
and the equations of motions
\begin{eqnarray*}
  m\dot v_r &=&
  \frac{e}{c}\left[
    \partial_rA_j
    -\partial_jA_r
  \right]\dot x^j
  -e\partial_r\Phi\\
  \dot x_r &=& v_r
\end{eqnarray*}


\section{Cylindrical coordinates $r,\varphi,z$}
First we have to calculate the metric tensor of a cylindrical coordinate
system. This task is to be done with explicitly writing down the
transformation equations from a cartesian to a cylindrical system and then
calculate the metric tensor out of this transformation equations.
The transformation is done with
\begin{eqnarray*}
  x^1=r=\sqrt{x^2+y^2}\qquad x^2=\varphi=\arctan\frac{y}{x}\qquad x^3=z=z
\end{eqnarray*}
For the unit vectors in the three directions we get
\begin{eqnarray*}
  \mathbf{e}^1&=&\nabla x^1
  =\frac{\partial x^1}{\partial x}\nabla x
  +\frac{\partial x^1}{\partial y}\nabla y
  +\frac{\partial x^1}{\partial z}\nabla z
  =\frac{x}{r}\mathbf{e}^x+\frac{y}{r}\mathbf{e}^y\\
  \mathbf{e}^2&=&\nabla x^2
  =\frac{\partial x^2}{\partial x}\nabla x
  +\frac{\partial x^2}{\partial y}\nabla y
  +\frac{\partial x^2}{\partial z}\nabla z
  =-\frac{y}{r^2}\mathbf{e}^x+\frac{x}{r^2}\mathbf{e}^y\\
  \mathbf{e}^3&=&\nabla x^3=\mathbf{e}^z
\end{eqnarray*}
From this we get the components of the metric tensor as
\begin{eqnarray*}
  g^{11} &=& \mathbf{e}^1\cdot\mathbf{e}^1
  =\left(\frac{x}{r}\mathbf{e}^x+\frac{y}{r}\mathbf{e}^y\right)
  \cdot\left(\frac{x}{r}\mathbf{e}^x+\frac{y}{r}\mathbf{e}^y\right)
  =\frac{x^2}{r^2}+\frac{y^2}{r^2}=1\\
  g^{12} &=& \mathbf{e}^1\cdot\mathbf{e}^2=0\\
  g^{13} &=& \mathbf{e}^1\cdot\mathbf{e}^3=0\\
  g^{22} &=& \mathbf{e}^2\cdot\mathbf{e}^2
  =\left(-\frac{y}{r^2}\mathbf{e}^x+\frac{x}{r^2}\mathbf{e}^y\right)\cdot
  \left(-\frac{y}{r^2}\mathbf{e}^x+\frac{x}{r^2}\mathbf{e}^y\right)
  =\frac{y^2+x^2}{r^4}=\frac{1}{r^2}\\
  g^{23} &=& \mathbf{e}^2\cdot\mathbf{e}^3=0\\
  g^{33} &=& \mathbf{e}^3\cdot\mathbf{e}^3=1
\end{eqnarray*}

Having the metric tensor, we also have its inverse
\begin{displaymath}
  (g_{ij})=\left(
    \begin{array}{ccc}
      1 & 0 & 0\\0 & r^2 & 0\\ 0 & 0 & 1
    \end{array}
  \right)
\end{displaymath}
The Lagrangian is independent of the choice of the coordinate system, so it is
still
\begin{displaymath}
  L(x,\dot x) =
  \frac{m}{2}\dot x_j\dot x^j
  +\frac{e}{c}A_j(\mathbf{x})\dot x^j
  -e\Phi(\mathbf{x})
\end{displaymath}
The equations of motion are 
\begin{eqnarray*}
  \ddot r
  &=& r(\dot \varphi)^2
  +\frac{e}{mc}r\left[
    B^3\dot \varphi
    -B^2\dot z
  \right]
  -\frac{e}{m}\frac{\partial \Phi(x)}{\partial r}\\
  r^2\ddot \varphi
  &=& -2r\dot \varphi\dot r
  +\frac{e}{mc}r\left[
    B^1\dot z
    -B^3\dot r
  \right]
  -\frac{e}{m}\frac{\partial \Phi(x)}{\partial \varphi}\\
  \ddot z
  &=&\frac{e}{mc}r\left[
    B^2\dot r
    -B^1\dot \varphi
  \right]
  -\frac{e}{m}\frac{\partial \Phi(x)}{\partial z}
\end{eqnarray*}

As an example, we make the assumption of a constant magnetic field
along the $z$ axis. So we use  $A_1=A_3=0$ and $A_2=Br^2/2$ in the
Lagrangian and get
\begin{displaymath}
  L(x,\dot x) =
  \frac{m}{2}(\dot r^2+r^2\dot\varphi^2+\dot z^2)
  +\frac{eB}{2c} r^2\dot \varphi
\end{displaymath}
With this ansatz, we see directly that $z$ and $\varphi$ are cyclic
variables. 


\section{Toroidal coordinate system}
\label{sec:torsys}
This subsection is only introduced to have a place where I can
calculate what I need for the description of this widely used
coordinate system. The coordinates are the radial coordinate $r$, the
poloidal angle $\theta$ and the toroidal angle $\phi$. The
transformation equations from the cylindrical system $(R,\phi_c,Z)$
are given by
\begin{displaymath}
  x^1=r=\sqrt{(R-R_0)^2+Z^2}
  \qquad x^2=\theta=\arctan\frac{Z}{R-R_0}
  \qquad x^3=\phi=-\phi_c
\end{displaymath}
From these equations one can easily get the new unit vectors
\begin{eqnarray*}
  \nabla r&=&\frac{\partial r}{\partial R}\nabla R
  +\frac{\partial r}{\partial Z}\nabla Z\\
  &=&\frac{R-R_0}{\sqrt{(R-R_0)^2+Z^2}}\nabla R
  +\frac{Z}{\sqrt{(R-R_0)^2+Z^2}}\nabla Z
  =\frac{R-R_0}{r}\nabla R+\frac{Z}{r}\nabla Z\\
                                %
  \nabla \theta&=&\frac{\partial\theta}{\partial R}\nabla R
  +\frac{\partial\theta}{\partial Z}\nabla Z\\
  &=&-\frac{Z}{\left(Z^2+(R-R_0)^2\right)}\nabla R
  +\frac{1}{(R-R_0) \left(\frac{Z^2}{(R-R_0)^2}+1\right)}\nabla Z
  =-\frac{Z}{r^2}\nabla R
  +\frac{(R-R_0)}{r^2}\nabla Z\\
  \nabla\phi &=&-\nabla\phi_c
\end{eqnarray*}
From these base vectors, we can find the components of the metric
tensor by building scalar product of the base vectors.
\begin{eqnarray*}
  g^{11} &=& \nabla r\cdot\nabla r \\
  &=&\frac{(R-R_0)^2}{r^2}\nabla R\cdot\nabla R
  +\frac{Z(R-R_0)}{r^2}\nabla R\cdot\nabla Z
  +\frac{Z(R-R_0)}{r^2}\nabla Z\cdot\nabla R
  +\frac{Z^2}{r^2}\nabla Z\cdot\nabla Z
  = 1\\
  g^{12} &=& 0\\
  g^{13} &=& 0\\
  g^{23} &=& 0\\
  g^{22} &=& \frac{1}{r^2}\\
  g^{33} &=& \tilde{g}^{22}=\frac{1}{R^2}=\frac{1}{(R_0+r\cos\theta)^2}
\end{eqnarray*}
The metric tensor for the covariant components is then the inverse
\begin{displaymath}
  g_{ij}=\left(
    \begin{array}{ccc}
      1 & 0 & 0\\
      0 & r^2 & 0\\
      0 & 0 & R^2
    \end{array}\right)
\end{displaymath}
The Jacobian in this coordinate system is then
\begin{displaymath}
  J^{-1}=\sqrt{|g_{ij}|}=\sqrt{r^2R^2}=rR
\end{displaymath}

\section{Some characteristics of the magnetic field in an
  axisymmetric case}
\label{sec:charar_genBfield}

For the magnetic field, the homogeneous Maxwell equation must be
fulfilled. This equation can be written in the flux surface
coordinates as
\begin{eqnarray*}
  0&=&\div \mathbf{B}=\frac{1}{J(\rho,\theta)}\left(
    \frac{\partial}{\partial \rho} (JB^\rho)
    +\frac{\partial}{\partial \theta} (JB^\theta)
    +\frac{\partial}{\partial \phi} (JB^\phi)
  \right)\\
  &=&\frac{1}{J(\rho,\theta)}\left(
    \frac{\partial}{\partial \rho} (JB^\rho)
    +\frac{\partial}{\partial \theta} (JB^\theta)
  \right)
\end{eqnarray*}
This leads to the condition
\begin{displaymath}
  \frac{\partial}{\partial \rho} (JB^\rho)
  =-\frac{\partial}{\partial \theta} (JB^\theta)
\end{displaymath}
We can write the magnetic field also as
\begin{displaymath}
  \mathbf{B}=B^\rho\mathbf{e}_\rho+B^\theta\mathbf{e}_\theta
  +B^\phi\mathbf{e}_\phi
\end{displaymath}
But with the definition of $\rho$ as a flux surface label, we have 
\begin{displaymath}
  \mathbf{B}\cdot\nabla\rho=B^\rho=0
\end{displaymath}
which simplifies the above equations and we get for the magnetic field
\begin{displaymath}
  \mathbf{B}=B^\theta\mathbf{e}_\theta+B^\phi\mathbf{e}_\phi
\end{displaymath}
and from the divergence equation
\begin{displaymath}
  \frac{\partial}{\partial \theta} (JB^\theta)=0
\end{displaymath}
we find that $JB^\theta$ is also a flux surface function.

The magnetic field can further be written as
\begin{eqnarray*}
  \mathbf{B} &=& JB^\theta \mathbf{e}^\phi\times \mathbf{e}^\rho
  +JB^\phi \mathbf{e}^\rho\times\mathbf{e}^\theta\\
  &=& \nabla\rho\times\left(
    JB^\phi \nabla\theta
    -JB^\theta \nabla\phi
  \right)
\end{eqnarray*}
Now introducing a function $\nu(\rho,\theta,\phi)$ by the defining
equations
\begin{displaymath}
  \frac{\partial \nu}{\partial\theta}=JB^\phi
  \qquad  \frac{\partial \nu}{\partial\phi}=-JB^\theta
\end{displaymath}
we can write the magnetic field in the simple Clebsch form
\begin{displaymath}
  \mathbf{B}=\nabla\rho\times\nabla\nu
\end{displaymath}

\bigskip
For the function $\nu$ we can further process
\begin{eqnarray*}
  \frac{\partial \nu}{\partial\phi}=-JB^\theta=-f(\rho)
  \qquad\Longrightarrow\quad
  \nu(\rho,\theta,\phi) = -JB^\theta \phi+c_1(\rho,\theta)
\end{eqnarray*}
We know further that $JB^\phi$ is a physical quantity, so it must be
periodic in the both angles $\theta$ and $\phi$. From this we can
deduce that the dependence of $\nu$ on $\theta$ can only be of the
general form
\begin{displaymath}
  \nu(\rho,\theta,\phi)= -JB^\theta \phi+A(\rho)\theta+\tilde\nu(\rho,\theta)
\end{displaymath}
where $\tilde\nu(\rho,\theta)$ is a periodic function in $\theta$.
To find a physical expression for the prefactor $A(\rho)$, we can
write the toroidal flux function
\begin{eqnarray*}
  \psi_\mathrm{tor}(\rho)&=&\frac{1}{2\pi}\int_0^{2\pi}\int_0^{2\pi}\int_0^\rho JB^\phi\,d\rho'\,d\theta\,d\phi
  = \int_0^\rho\int_0^{2\pi} \frac{\partial\nu}{\partial\theta}\,d\theta\,d\rho'\\
  &=&\int_0^\rho (\nu(\rho',2\pi,\phi)-\nu(\rho',0,\phi))\,d\rho'
  = \int_0^\rho 2\pi A(\rho')\,d\rho'
\end{eqnarray*}
So we can write
\begin{displaymath}
  A(\rho) = \frac{\psi_\mathrm{tor}'}{2\pi} = \psi_t'
\end{displaymath}
This leads to the general form for axisymmetric magnetic field for the
generating function $\nu$
\begin{displaymath}
  \nu(\rho,\theta,\phi)=-JB^\theta \phi+\psi_t'(\rho)\theta+\tilde\nu(\rho,\theta)
\end{displaymath}
With the same thoughts as we did for the toroidal flux, we can find
for the poloidal flux the expression
\begin{displaymath}
  \psi_p'=\frac{\psi_\mathrm{pol}'}{2\pi}=JB^\theta
\end{displaymath}
so we can also write for $\nu$
\begin{displaymath}
  \nu(\rho,\theta,\phi)=\psi_t'\theta-\psi_p'\phi+\tilde\nu(\rho,\theta)
  =\psi_p'\left(
    q(\rho)\theta-\phi
  \right)+\tilde\nu(\rho,\theta)
\end{displaymath}


\section{Oblique angle coordinate system in two dimensions}
\label{sec:gengeom}

In this section, I will investigate some geometry related
problems. The first one arises, if one wants to do the gyroaveraging
in two dimensions, with one dimension Fourier transformed, as we do in
gene10. The problem is, that the twodimensional coordinate system, in
which we work, the $x$-$y$ coordinate system of the code is
non-orthogonal in general. This means that $g^{12}\neq0$. The question
is then, if and how one can do a Fourier transform in one direction and
treating the other direction by other methods. 

I start from a Cartesian system with the coordinates $x$ and $y$. It
is true
\begin{displaymath}
  x=x^1=x_1\qquad
  y=x^2=x_2\qquad
  \mathbf{e}^1=\nabla x = (1,0)^\top=\mathbf{e}_1\qquad
  \mathbf{e}^2=\nabla y = (0,1)^\top=\mathbf{e}_2
\end{displaymath}
This coordinate system is valid only at the outboard midplane for
$\hat{s}\neq0$. If we follow the magnetic field line, the $y$ axis
does not change in the turned coordinate system, but the angle between
$x$ and $y$ axis does change. We name this angle here $\theta_0$. This
also has the consequence, that the covariant and contravariant base
vectors in $x$ direction are not parallel. So the contravariant base
vector $\mathbf{\bar e}^1=\nabla \bar x^1$ is still orthogonal to the
flux surface, but the covariant base vector $\mathbf{\bar e}_1$, which
denotes the coordinate axis direction is not. The transformation of the
contravariant components of a vector from the old Cartesian coordinate system
($x,y$) to the new non-orthogonal $(\bar x,\bar y)$ coordinate system is given by the
following equations.
\begin{eqnarray*}
  \bar x^j\longrightarrow x^j &\qquad& x^1=\bar x^1\sin\theta_0\qquad\qquad
  x^2=\bar x^2 + \bar x^1\cos\theta_0\\
  x^j \longrightarrow \bar x^j &\qquad&  \bar x^1=\frac{x^1}{\sin\theta_0}\qquad\qquad
  \bar x^2 = x^2-\frac{x^1}{\tan\theta_0}
\end{eqnarray*}
For the base vectors in the new system we get
\begin{eqnarray*}
  \mbox{contravariant:}&&\qquad\mathbf{\bar e}^1=\frac{1}{\sin\theta_0}\mathbf{e}^1\qquad
  \mathbf{\bar e}^2=-\frac{1}{\tan\theta_0}\mathbf{e}^1+\mathbf{e}^2\\
  \mbox{covariant:}&&\qquad\mathbf{\bar e}_1=\mathbf{e}_1\sin\theta_0+\mathbf{e}_2\cos\theta_0\qquad
  \mathbf{\bar e}_2=\mathbf{e}_2
\end{eqnarray*}
The metric tensor is given as
\begin{displaymath}
  g^{ij}=\left(
\begin{array}{ll}
  \frac{1}{\sin^2(\theta_0)} & -\frac{\cos\theta_0}{\sin^2(\theta_0)} \\
 -\frac{\cos\theta_0}{\sin^2(\theta_0)} & \frac{1}{\sin^2(\theta_0)}
\end{array}
\right)\qquad
  g_{ij}=\left(
    \begin{array}{cc}
      1 & \cos\theta_0\\
      \cos\theta_0 & 1
    \end{array}\right)
\end{displaymath}


\chapter{B-Splines as base functions}
\label{sec:bsplines}

As base functions on the grid we want to use B-splines. They are
defined by the recurrence relation:
\begin{eqnarray*}
  B_{jk}=\omega_{jk}B_{j,k-1}+(1-\omega_{j+1,k})B_{j+1,k-1}\\
  \mbox{with}\quad\omega_{jk}=\frac{x-t_j}{t_{j+k-1}-t_j}
\end{eqnarray*}
or put together
\begin{eqnarray*}
  B_{jk}=\frac{x-t_j}{t_{j+k-1}-t_j}\,B_{j,k-1}
  +\frac{t_{j+k}-x}{t_{j+k}-t_{j+1}}\,B_{j+1,k-1}
\end{eqnarray*}
and the starting B-spline 
\begin{displaymath}
  B_{j1}=
  \begin{cases}
    1 & x\in[t_j,t_{j+1})\\
    0 & x\notin[t_j,t_{j+1})
  \end{cases}
\end{displaymath}

Therefrom we get for the different orders with the following
calculations the following results.
\begin{eqnarray*}
  B_{j2}&=&\frac{x-t_j}{t_{j+1}-t_j} B_{j,1}
  +\left(1-\frac{x-t_{j+1}}{t_{j+2}-t_{j+1}}\right)B_{j+1,1}\\
  &=&
  \begin{cases}
    \frac{x-t_j}{t_{j+1}-t_j} & x\in[t_j,t_{j+1})\\
    \frac{t_{j+2}-x}{t_{j+2}-t_{j+1}} & x\in[t_{j+1},t_{j+2})\\
    0 & \mbox{else}
  \end{cases}
\end{eqnarray*}
This is exactly the hat function.

The next order gives quadratic (parabolic) B-splines.
\begin{eqnarray*}
  B_{j3}&=&\frac{x-t_j}{t_{j+2}-t_j}\,B_{j,2}
  +\frac{t_{j+3}-x}{t_{j+3}-t_{j+1}}\,B_{j+1,2}\\
  &=&
  \begin{cases}
    \frac{x-t_j}{t_{j+2}-t_j}\,\frac{x-t_j}{t_{j+1}-t_j} & x\in[t_j,t_{j+1})\\
    \frac{x-t_j}{t_{j+2}-t_j}\,\frac{t_{j+2}-x}{t_{j+2}-t_{j+1}}
    +\frac{t_{j+3}-x}{t_{j+3}-t_{j+1}}\,\frac{x-t_{j+1}}{t_{j+2}-t_{j+1}} & x\in[t_{j+1},t_{j+2})\\
    \frac{t_{j+3}-x}{t_{j+3}-t_{j+1}}\,\frac{t_{j+3}-x}{t_{j+3}-t_{j+2}} & x\in[t_{j+2},t_{j+3})\\
    0 & \mbox{else}
  \end{cases}\\
  &=&
  \begin{cases}
    \frac{(x-t_j)^2}{(t_{j+2}-t_j)(t_{j+1}-t_j)} & x\in[t_j,t_{j+1})\\
    \frac{x-t_j}{t_{j+2}-t_j}\,\frac{t_{j+2}-x}{t_{j+2}-t_{j+1}}
    +\frac{t_{j+3}-x}{t_{j+3}-t_{j+1}}\,\frac{x-t_{j+1}}{t_{j+2}-t_{j+1}} & x\in[t_{j+1},t_{j+2})\\
    \frac{(t_{j+3}-x)^2}{(t_{j+3}-t_{j+1})(t_{j+3}-t_{j+2})} & x\in[t_{j+2},t_{j+3})\\
    0 & \mbox{else}
  \end{cases}
\end{eqnarray*}

We want to go one order further to the well-known cubic B-splines. The
above discussed definition leads to
\begin{eqnarray*}
  B_{j4} &=& \frac{x-t_j}{t_{j+3}-t_j}\,B_{j,3}
  +\frac{t_{j+4}-x}{t_{j+4}-t_{j+1}}\,B_{j+1,3}\\
  &=&
  \begin{cases}
    \frac{x-t_j}{t_{j+3}-t_j}\,\frac{(x-t_j)^2}{(t_{j+2}-t_j)(t_{j+1}-t_j)}
    & x\in[t_j,t_{j+1})\\
    % 
    \frac{x-t_j}{t_{j+3}-t_j}\,\left(
      \frac{x-t_j}{t_{j+2}-t_j}\,\frac{t_{j+2}-x}{t_{j+2}-t_{j+1}}
      +\frac{t_{j+3}-x}{t_{j+3}-t_{j+1}}\,\frac{x-t_{j+1}}{t_{j+2}-t_{j+1}}
    \right)
    +\frac{t_{j+4}-x}{t_{j+4}-t_{j+1}}\,\left(
      \frac{(x-t_{j+1})^2}{(t_{j+3}-t_{j+1})(t_{j+2}-t_{j+1})}
    \right) & x\in[t_{j+1},t_{j+2})\\
    % 
    \frac{x-t_j}{t_{j+3}-t_j}\,\left(
      \frac{(t_{j+3}-x)^2}{(t_{j+3}-t_{j+1})(t_{j+3}-t_{j+2})}
    \right)
    +\frac{t_{j+4}-x}{t_{j+4}-t_{j+1}}\,\left(
      \frac{x-t_{j+1}}{t_{j+3}-t_{j+1}}\,\frac{t_{j+3}-x}{t_{j+3}-t_{j+2}}
      +\frac{t_{j+4}-x}{t_{j+4}-t_{j+2}}\,\frac{x-t_{j+2}}{t_{j+3}-t_{j+2}}
  \right)
    & x\in[t_{j+2},t_{j+3})\\
    % 
    \frac{t_{j+4}-x}{t_{j+4}-t_{j+1}}\,\left(
      \frac{(t_{j+4}-x)^2}{(t_{j+4}-t_{j+2})(t_{j+4}-t_{j+3})}
    \right) & x\in[t_{j+3},t_{j+4})\\
    0 & \mbox{else}
  \end{cases}\\
  &=&
  \begin{cases}
    \frac{(x-t_j)^3}{(t_{j+3}-t_j)(t_{j+2}-t_j)(t_{j+1}-t_j)}
    & x\in[t_j,t_{j+1})\\
    % 
    \frac{x-t_j}{t_{j+3}-t_j}\,\left(
      \frac{x-t_j}{t_{j+2}-t_j}\,\frac{t_{j+2}-x}{t_{j+2}-t_{j+1}}
      +\frac{t_{j+3}-x}{t_{j+3}-t_{j+1}}\,\frac{x-t_{j+1}}{t_{j+2}-t_{j+1}}
    \right)
    +\frac{(t_{j+4}-x)(x-t_{j+1})^2}{(t_{j+4}-t_{j+1})(t_{j+3}-t_{j+1})(t_{j+2}-t_{j+1})}
     & x\in[t_{j+1},t_{j+2})\\
    % 
    \frac{(x-t_j)(t_{j+3}-x)^2}{(t_{j+3}-t_j)(t_{j+3}-t_{j+1})(t_{j+3}-t_{j+2})}
    +\frac{t_{j+4}-x}{t_{j+4}-t_{j+1}}\,\left(
      \frac{x-t_{j+1}}{t_{j+3}-t_{j+1}}\,\frac{t_{j+3}-x}{t_{j+3}-t_{j+2}}
      +\frac{t_{j+4}-x}{t_{j+4}-t_{j+2}}\,\frac{x-t_{j+2}}{t_{j+3}-t_{j+2}}
  \right)
    & x\in[t_{j+2},t_{j+3})\\
    % 
    \frac{(t_{j+4}-x)^3}{(t_{j+4}-t_{j+1})(t_{j+4}-t_{j+2})(t_{j+4}-t_{j+3})}
    & x\in[t_{j+3},t_{j+4})\\
    0 & \mbox{else}
  \end{cases}
\end{eqnarray*}
On an equidistant mesh with gridsize $\Delta x$ this can be written as
\begin{eqnarray*}
  B_{j4} &=&
  \begin{cases}
    \frac{(x-t_j)^3}{6\Delta x^3}
    & x\in[t_j,t_{j+1})\\
    % 
    \frac{x-t_j}{6\Delta x^3}\,\left(
      (x-t_j)(t_{j+2}-x)
      +(t_{j+3}-x)(x-t_{j+1})
    \right)
    +\frac{(t_{j+4}-x)(x-t_{j+1})^2}{6\dx^3}
     & x\in[t_{j+1},t_{j+2})\\
    % 
    \frac{(x-t_j)(t_{j+3}-x)^2}{6\dx^3}
    +\frac{t_{j+4}-x}{6\dx^3}\,\left(
      (x-t_{j+1})(t_{j+3}-x)
      +(t_{j+4}-x)(x-t_{j+2})
  \right)
    & x\in[t_{j+2},t_{j+3})\\
    % 
    \frac{(t_{j+4}-x)^3}{6\dx^3}
    & x\in[t_{j+3},t_{j+4})\\
    0 & \mbox{else}
  \end{cases}
\end{eqnarray*}

To calculate the coefficients of a representation of a function in
this basis, one has to solve a linear system of equations. In one
dimension one has on a grid
\begin{eqnarray*}
  f(x_n) = \sum_j \psi_j B_j(x_n)\quad\forall n=1,\ldots,N
\end{eqnarray*}
Or written as a system of equations
\begin{displaymath}
  \mathcal{B}\cdot\mathbf{\psi}=\mathbf{f}=
  \left(
    \begin{array}{ccccc}
      B_1(x_1) & B_2(x_1) & \ldots & B_N(x_1)\\
      B_1(x_2) & \ldots & \ldots & B_N(x_2)\\
      \vdots & \ddots & \ddots & \vdots\\
      B_1(x_N) & \ldots & \ldots & B_N(x_N)
    \end{array}
  \right)\cdot\left(
    \begin{array}{c}
      \psi_1\\\psi_2\\\vdots\\\psi_N
    \end{array}
  \right)
  =\left(
    \begin{array}{c}
      f_1\\f_2\\\vdots\\f_N
    \end{array}
  \right)
\end{displaymath}
The matrix can be evaluated in advance, as we need only the values of
the splines on the grid points. From the definition above we get for a
equidistant mesh
\begin{eqnarray*}
  B_j(t_j) = 0\qquad B_j(t_{j+1})=\frac{1}{6}
  \qquad B_j(t_{j+2}) = \frac{2}{3}\qquad
  B_j(t_{j+3}) = \frac{1}{6}
\end{eqnarray*}
so that we get for the matrix
\begin{displaymath}
  \mathcal{B}=\left(
    \begin{array}{ccccccc}
      0 & \ldots & \ldots & 0 & \frac{1}{6} & \frac{2}{3} & \frac{1}{6}\\
      \frac{1}{6} & 0 & \ldots & \ldots & 0 & \frac{1}{6} & \frac{2}{3} \\
      \frac{2}{3} & \frac{1}{6} & 0 &  \ldots &\ldots & 0 & \frac{1}{6} \\
      \frac{1}{6} & \frac{2}{3} & \frac{1}{6} & 0 & \ldots & \ldots & 0 \\
      0 & \frac{1}{6} & \frac{2}{3} & \frac{1}{6} & 0 & \ldots & 0\\
      \vdots & \ddots & \ddots & \ddots & \ddots & \ddots & \vdots\\
      0 & \ldots & 0 & \frac{1}{6} & \frac{2}{3} & \frac{1}{6} & 0 
    \end{array}
  \right)
\end{displaymath}


In two dimensions the things change. We want to represent the two
dimensions separately, that we get
\begin{displaymath}
  f(x_n,y_m)=\sum_{i,j}\psi_{ij}B_i(x_n)B_j(y_m)
\end{displaymath}
To determine the coefficients, we have to rewrite this equations in
matrix-vector form and then solve the linear system of equations. The
matrix has now the dimension $N\cdot M\times N\cdot M$, for each
point, we have one column and one row. Therefore we can write the
system as
\begin{displaymath}
  \left(
    \begin{array}{ccccccc}
      B_{1,1}(1,1) & B_{2,1}(1,1) & \ldots & B_{N,1}(1,1) &
      B_{1,2}(1,1) & \ldots & B_{N,M}(1,1)\\
      B_{1,1}(2,1) & B_{2,1}(2,1) & \ldots & B_{N,1}(2,1) &
      B_{1,2}(2,1) & \ldots & B_{N,M}(2,1)\\
      \vdots & &&&&& \vdots\\
      B_{1,1}(N,M) & B_{2,1}(N,M) & \ldots & B_{N,1}(N,M) &
      B_{1,2}(N,M) & \ldots & B_{N,M}(N,M)\\
    \end{array}
  \right)\cdot\left(
    \begin{array}{c}
      \psi_{1,1}\\\psi_{2,1}\\\vdots\\\psi_{N,1}\\
      \psi_{1,2}\\\psi_{2,2}\\\vdots\\\psi_{N,2}\\
      \vdots\\
      \psi_{1,M}\\\psi_{2,M}\\\vdots\\\psi_{N,M}
    \end{array}
  \right) = \left(
    \begin{array}{c}
      f_{1,1}\\f_{2,1}\\\vdots\\f_{N,1}\\
      f_{1,2}\\f_{2,2}\\\vdots\\f_{N,2}\\
      \vdots\\
      f_{1,M}\\f_{2,M}\\\vdots\\f_{N,M}
    \end{array}
  \right)
\end{displaymath}
With periodic boundary conditions, we find that in each line of the
matrix, only 9 elements are nonzero. In the first line these are the
elements
\begin{eqnarray*}
  B_{N-2,M-2}=\frac{1}{36}\qquad
  B_{N-1,M-2}=\frac{1}{9}\qquad
  B_{N,M-2} =\frac{1}{36}\\
  B_{N-2,M-1}=\frac{1}{9}\qquad
  B_{N-1,M-1}=\frac{4}{9}\qquad
  B_{N,M-1} =\frac{1}{9}\\
  B_{N-2,M}=\frac{1}{36}\qquad
  B_{N-1,M}=\frac{1}{9}\qquad
  B_{N,M} =\frac{1}{36}
\end{eqnarray*}

\end{document}

%%% Local Variables: 
%%% mode: latex
%%% TeX-master: t
%%% End: 
