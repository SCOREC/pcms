\chapter{Gyroaveraging and field equations}
\label{chap:numerics}
In the last chapter, we focused on the Vlasov equation. In the present
one, the field equations and the closely linked gyro-averaging are
treated. 



\section{Gyroaveraging}
\label{sec:gyroaveraging}

In the local version of the GENE code, we did the gyroaveraging by
Fourier transforming in both perpendicular directions and multiplying
with a Bessel function $J_0$. This is further not possible 
as we can not Fourier transform in radial ($x$) direction. So we have
to go back to the definition of the gyroaveraging. Gyro-averaging of a
function $\Phi$ just means to integrate the function along a gyro
trajectory. This operation can be written as
\begin{eqnarray*}
  \avg{\Phi(\mathbf{X}+\mathbf{r})} 
  = \frac{1}{2\pi}\int\limits_0^{2\pi} \Phi(\mathbf{X}+\mathbf{r}(\theta))\,d\theta
\end{eqnarray*}
with the gyroangle $\theta$, $\mathbf{X}$ the gyrocenter position and
$\mathbf{r}$ the vector pointing from the gyro center to the particle
position $\mathbf{x}$. So we have $\mathbf{x}=\mathbf{X}+\mathbf{r}$. 

We do not have the potential $\Phi$ in real space, but as Fourier
series in $k_y$ with the definition
\begin{displaymath}
  \Phi(\mathbf{x}) = \Phi(x,y,z) = \sum_m \Phi_m(x,z) e^{ik_m y}
\end{displaymath}
Putting both together, we arrive at
\begin{displaymath}
  \avg{\Phi(\mathbf{X}+\mathbf{r})} 
  = \frac{1}{2\pi}\int\limits_0^{2\pi} \sum_m
  \Phi_m(X_1+r_1,X_3+r_3) e^{ik_m(X_2+r_2)}\,d\theta
  = \sum_m e^{ik_mX_2} \frac{1}{2\pi}\int\limits_0^{2\pi}
  \Phi_m(X_1+r_1,X_3) e^{ik_m r_2}\,d\theta.
\end{displaymath}
In the last step, we also assumed that the gyroaveraging takes place
in the plane perpendicular to the magnetic field. Hence, $r_3$ is zero
then, as the gyro-orbit is only in the $x_1-x_2$-plane.
Now using $r_1 = \rho\cos\theta$ and $r_2=\rho\sin\theta$ leads to the
final expression
\begin{displaymath}
  \avg{\Phi(\mathbf{X}+\mathbf{r})} 
  = \sum_m e^{ik_mX_2} \frac{1}{2\pi}\int\limits_0^{2\pi}
  \Phi_m(X_1+\rho\cos\theta,X_3) e^{ik_m \rho\sin\theta}\,d\theta.
\end{displaymath}

To discretize in radial direction we use finite elements
$\Lambda_n(x)$. For the electrostatic potential, we can then write
\begin{displaymath}
  \Phi(x,k_y,z) = \sum_n \Phi_n(k_y,z)\Lambda_n(x).
\end{displaymath}
Using this in the above expression leads to
\begin{eqnarray*}
  \avg{\Phi(\mathbf{X}+\mathbf{r})} 
  &=& \sum_m e^{ik_mX_2} \frac{1}{2\pi}\int\limits_0^{2\pi}
  \sum_n \Phi_{n}(k_m,X_3)\Lambda_n(X_1+\rho\cos\theta) e^{ik_m \rho\sin\theta}\,d\theta\\
  &=& \sum_n \sum_m e^{ik_mX_2} \Phi_{n}(k_mX_3) \frac{1}{2\pi}\int\limits_0^{2\pi}
  \Lambda_n(X_1+\rho\cos\theta) e^{ik_m \rho\sin\theta}\,d\theta\\
  &=& \sum_m e^{ik_mX_2} \sum_n \Phi_{n}(k_m,X_3) \mathcal{M}'_{n}(X_1,k_m,\rho)
\end{eqnarray*}
with
\begin{displaymath}
  \mathcal{M}'_{n}(X_1,k_m,\rho)= \frac{1}{2\pi}\int\limits_0^{2\pi}
  \Lambda_n(X_1+\rho\cos\theta) e^{ik_m \rho\sin\theta}\,d\theta
\end{displaymath}
which can all be calculated in the initialization phase to arbitrary
accuracy by choosing as many $\theta$ points as wanted.
The equation can also be written in a matrix-vector form in the
following way.
\begin{displaymath}
  \bar{\mathbf{\Phi}}(k_m,x_3) = \mathcal{M}'(k_m,\rho)\mathbf{\Phi}(k_m,x_3) 
\end{displaymath}
with the matrix $\mathcal{M}'$ which is composed of the elements 
\begin{displaymath}
  \mathcal{M}'_{in}(k_m,x_3,\rho,\sigma) = \frac{1}{2\pi}\int\limits_0^{2\pi}
  \Lambda_n(X_{1,i}+\rho(X_{1,i},x_3,\sigma)\cos\theta) e^{ik_m \rho(X_{1,i},x_3,\sigma)\sin\theta}\,d\theta
\end{displaymath}
where $\sigma$ is the species index. A bold letter indicates a vector
in $x_1$ direction.

This quantity is only used temporarily, in the code it never comes up
as we have a grid on particle coordinates, not on gyrocenter
coordinates. What we really have are the following expressions.
\begin{eqnarray*}
  \avg{F(\mathbf{x}-\mathbf{r})} 
  &=& \frac{1}{2\pi}\int\limits_0^{2\pi} F(\mathbf{x}-\mathbf{r})\,d\theta
  = \frac{1}{2\pi}\int\limits_0^{2\pi} \sum_m
  F_m(x_1-r_1,x_3-r_3)\,e^{ik_m(x_2-r_2)}\,d\theta\\
  &=& \sum_m e^{ik_mx_2} \frac{1}{2\pi}\int\limits_0^{2\pi} 
  F_m(x_1-r_1,x_3)\,e^{-ik_m r_2}\,d\theta\\
  &=& \sum_m e^{ik_mx_2} \frac{1}{2\pi}\int\limits_0^{2\pi} 
  F_m(x_1-\rho\cos\theta,x_3)\,e^{-ik_m \rho\sin\theta}\,d\theta\\
  &=& \sum_m e^{ik_mx_2} \sum_n F_{n,m}(x_3) \frac{1}{2\pi}\int\limits_0^{2\pi} 
  \Lambda_n(x_1-\rho\cos\theta)\,e^{-ik_m \rho\sin\theta}\,d\theta\\
  &=& \sum_m e^{ik_mx_2} \sum_n F_{n,m}(x_3) \mathcal{M}_{n}(x_1,k_m,x_3,\rho)\\
\end{eqnarray*}
with 
\begin{displaymath}
  \mathcal{M}_{n}(x_1,k_m,x_3,\rho,\sigma)=\frac{1}{2\pi}\int\limits_0^{2\pi} 
  \Lambda_n(x_1-\rho(x_1,x_3,\sigma)\cos\theta)\,e^{-ik_m \rho(x_1,x_3,\sigma)\sin\theta}\,d\theta
\end{displaymath}
In matrix-vector form we get the analog to the result shown above.


\subsection{Gyro-averaging in non-orthogonal coordinates}
In the last section, we derived formulas for the $J_0$ operator
calculated in mixed twodimensional space. The gyroaveraging 
has been done by really sampling over the gyroorbit. In the last
section, we assumed a circular gyroorbit in the given coordinates $x$
and $y$, which is only true if $x$ and $y$ are orthogonal. In general
this is not fulfilled, especially if we go along the field line where
the metric element $g^{12}$ becomes nonzero. In this subsection, I
will calculate the gyroorbit in non-orthogonal coordinates. I will
assume only the knowledge of the metric tensor in the new coordinates,
which is denoted by $g^{ij}$. The starting coordinate system will
be the orthonormal Cartesian one (here the metric tensor is clearly
the unit matrix). In principle what we show in this subsection is how
to come to the transformation rules between two coordinate systems, if
one only has the metric tensors of both systems. The only
simplification done here is to start from a Cartesian coordinate system.

I want to represent the vector $\mathbf{v}$ with the old 
coordinates $v_j$ in the new system, where it has the 
coordinates $\bar v_j$. Both vectors represent the same physical
point, so we can expand the vector in the two different bases:
\begin{equation}
  \label{eq:vecrepr}
  \mathbf{v}=v_j\mathbf{e}^j=\bar v_j\bar{\mathbf{e}}^j
\end{equation}
For the new contravariant base vectors, we can write
\begin{displaymath}
  \bar{\mathbf{e}}^j = \nabla\bar x^j
  =\frac{\partial\bar x^j}{\partial x^r}\mathbf{e}^r
\end{displaymath}
with the transformation rule for the contravariant components of a
vector $\bar{x}^j=\bar{x}^j(x^r)$. 
From this we get a system of equations for the metric coefficients.
\begin{equation}
  \label{eq:metricnew}
  g^{ij} = \bar{\mathbf{e}}^i\cdot\bar{\mathbf{e}}^j
\end{equation}
As the metric tensor of the old Cartesian coordinate system is already
known to be the unit matrix and is therefore never explicitly used as
symbol, we denote the new metric tensor without an overbar to simplify
notation. Written eq.~(\ref{eq:metricnew}) in explicit form, we get
\begin{eqnarray*}
  \left(
    \frac{\partial\bar x^1}{\partial x^1}
  \right)^2+\left(
    \frac{\partial\bar x^1}{\partial x^2}
  \right)^2 &=& g^{11}\\
  \left(
    \frac{\partial\bar x^1}{\partial x^1}
  \right)\left(
    \frac{\partial\bar x^2}{\partial x^1}
  \right)
  + \left(
    \frac{\partial\bar x^1}{\partial x^2}
  \right)\left(
    \frac{\partial\bar x^2}{\partial x^2}
  \right) &=& g^{12}\\
  \left(
    \frac{\partial\bar x^2}{\partial x^1}
  \right)^2
  +\left(
    \frac{\partial\bar x^2}{\partial x^2}
  \right)^2 &=& g^{22}
\end{eqnarray*}
These are three equations for four unknowns. To determine the
derivatives we have to add another restriction. In general the
question is, what determines the transformation rules if one has the
metric of a new coordinate system. From the last equations we can see,
that there is still one degree of freedom. This degree is given by the
fact, that the metric is identical if one turns the whole coordinate
system around the origin or if one translates the origin. The metric
only describes ``inner'' relations between the base vectors. So to get
the real transformation rules, we can choose the direction of one base
vector (the origin is always the same for all coordinate systems). 

We choose arbitrarily that the first contravariant base vector of the
new system should be parallel to the first contravariant base vector
of the old system, and should point in the same direction. In formulas
\begin{displaymath}
  \bar{\mathbf{e}}^1=a \mathbf{e}^1
\end{displaymath}
with a proportionality constant $a>0$.
This can be written explicitly as
\begin{eqnarray*}
  a\mathbf{e}^1 = \bar{\mathbf{e}}^1 = \frac{\partial\bar{x}^1}{\partial x^1}\,\mathbf{e}^1
  + \frac{\partial\bar{x}^1}{\partial x^2}\,\mathbf{e}^2
\end{eqnarray*}
Therefore we can deduce
\begin{displaymath}
  a=\frac{\partial\bar{x}^1}{\partial x^1}\qquad\mbox{and}\qquad 
  \frac{\partial\bar{x}^1}{\partial x^2}=0.
\end{displaymath}

Using these expression, we get for the equations shown above
\begin{eqnarray*}
  \left(
    \frac{\partial\bar x^1}{\partial x^1}
  \right)^2 &=& g^{11}\\
  \left(
    \frac{\partial\bar x^1}{\partial x^1}
  \right)\left(
    \frac{\partial\bar x^2}{\partial x^1}
  \right) &=& g^{12}\\
  \left(
    \frac{\partial\bar x^2}{\partial x^1}
  \right)^2
  +\left(
    \frac{\partial\bar x^2}{\partial x^2}
  \right)^2 &=& g^{22}
\end{eqnarray*}
Hence, the solution is 
\begin{eqnarray*}
  \frac{\partial\bar x^1}{\partial x^1} = \sqrt{g^{11}}\qquad
  \frac{\partial\bar x^2}{\partial x^1} = \frac{g^{12}}{\sqrt{g^{11}}}\qquad
  \frac{\partial\bar x^2}{\partial x^2}
  = \pm\sqrt{g^{22}-\frac{(g^{12})^2}{g^{11}}}
\end{eqnarray*}
Using this result, we can write for the new contravariant base vectors
\begin{eqnarray*}
  \bar{\mathbf{e}}^1&=&\sqrt{g^{11}}\,\mathbf{e}^1\\
  \bar{\mathbf{e}}^2&=&\frac{g^{12}}{\sqrt{g^{11}}}\,\mathbf{e}^1
  \pm\sqrt{\frac{g}{g^{11}}}\,\mathbf{e}^2
\end{eqnarray*}
where we used $g=\mbox{det}(g^{ij})=g^{11}g^{22}-(g^{12})^2$. We have
still to choose the sign in the second expression. As one can see,
this sign stands in front of the Jacobian of the coordinate system. We
want that the Jacobian of the new system has the same sign as the
Jacobian of the old system, this means that the ``handedness''
(H\"andigkeit) is the same in both systems. Therefore we choose the
sign to be positive.

To come to the explicit transformation rules for the vector
components, we can expand eq.~(\ref{eq:vecrepr}) and get
\begin{displaymath}
  v_1\mathbf{e}^1+v_2\mathbf{e}^2=\bar v_1\bar{\mathbf{e}}^1+\bar v_2\bar{\mathbf{e}}^2
\end{displaymath}
This equation is now multiplied once with $\bar{\mathbf{e}}^1$ and
once with $\bar{\mathbf{e}}^2$ to get the following two equations for
two unknowns.
\begin{eqnarray*}
  v_1\mathbf{e}^1\cdot\bar{\mathbf{e}}^1
  +v_2\mathbf{e}^2\cdot\bar{\mathbf{e}}^1
  &=& \bar v_1g^{11}+\bar v_2g^{12}\\
  v_1\mathbf{e}^1\cdot\bar{\mathbf{e}}^2
  +v_2\mathbf{e}^2\cdot\bar{\mathbf{e}}^2
  &=& \bar v_1g^{12}+\bar v_2g^{22}
\end{eqnarray*}
Further the dot products on the left hand sides can be written explicitly to lead
to
\begin{eqnarray*}
  v_1 \frac{\partial\bar x^1}{\partial x^1}
  +v_2\frac{\partial\bar x^1}{\partial x^2}
  &=& \bar v_1g^{11}+\bar v_2g^{12}\\
  v_1 \frac{\partial\bar x^2}{\partial x^1}
  +v_2\frac{\partial\bar x^2}{\partial x^2}
  &=& \bar v_1g^{12}+\bar v_2g^{22}
\end{eqnarray*}
The two unknowns are the barred vector components, so written in usual
matrix vector form we get
\begin{displaymath}
  \left(
    \begin{array}{cc}
      g^{11} & g^{12}\\
      g^{21} & g^{22}
    \end{array}
  \right)\cdot \left(
    \begin{array}{c}
      \bar v_1\\ \bar{v}_2
    \end{array}
  \right)
  = \left(
    \begin{array}{cc}
      \frac{\partial\bar x^1}{\partial x^1} &
      \frac{\partial\bar x^1}{\partial x^2}\\
      \frac{\partial\bar x^2}{\partial x^1} &
      \frac{\partial\bar x^2}{\partial x^2}
    \end{array}\right)\cdot\left(
    \begin{array}{c}
      v_1 \\
      v_2
    \end{array}
  \right)
\end{displaymath}
Bringing the metric tensor to the other side, we have then
\begin{displaymath}
   \left(
    \begin{array}{c}
      \bar v_1\\ \bar{v}_2
    \end{array}
  \right)
  = \frac{1}{g}\left(
    \begin{array}{cc}
      g^{22} & -g^{12}\\
      -g^{21} & g^{11}
    \end{array}
  \right)\cdot\left(
    \begin{array}{cc}
      \sqrt{g^{11}} &
      0\\
      \frac{g^{12}}{\sqrt{g^{11}}} &
      \sqrt{\frac{g}{g^{11}}}
    \end{array}\right)\cdot\left(
    \begin{array}{c}
      v_1 \\
      v_2
    \end{array}
  \right).
\end{displaymath}
The two matrices can be multiplied which yields
\begin{displaymath}
   \left(
    \begin{array}{c}
      \bar v_1\\ \bar{v}_2
    \end{array}
  \right)
  = \frac{1}{\sqrt{g^{11}}}\left(
    \begin{array}{cc}
      1 & -\frac{g^{12}}{\sqrt{g}}\\
      0 & \frac{g^{11}}{\sqrt{g}}
    \end{array}
  \right)\cdot\left(
    \begin{array}{c}
      v_1 \\
      v_2
    \end{array}
  \right)
\end{displaymath}

\subsubsection{Final transformation system}
The final transformation rules are put together in this subsection.
\begin{align*}
  v_j\rightarrow\bar v_j: && \bar v_1 &= \frac{1}{\sqrt{g^{11}}}\left(
    v_1 
    -\frac{g^{12}}{\sqrt{g}} v_2
  \right) &
  \bar v_2 &= \frac{\sqrt{g^{11}}}{\sqrt{g}}v_2\\
  % 
  v^j\rightarrow \bar v^j: && \bar v^1 &= \sqrt{g^{11}} v^1 &
  \bar v^2 &= \frac{g^{21}}{\sqrt{g^{11}}} v^1 
  +\frac{\sqrt{g}}{\sqrt{g^{11}}} v^2\\
  \bar v^j\rightarrow v^j: &&
  v^1 &= \frac{\bar v^1}{\sqrt{g^{11}}} &
  v^2 &= -\frac{g^{12}}{\sqrt{g}\sqrt{g^{11}}}\bar v^1
  +\frac{\sqrt{g^{11}}}{\sqrt{g}}\bar v^2\\
  \bar v_j\rightarrow v_j: && 
  v_1 &= \sqrt{g^{11}}\bar v_1
  +\frac{g^{12}}{\sqrt{g^{11}}}\bar v_2 &
  v_2 &= \frac{\sqrt{g}}{\sqrt{g^{11}}}\bar v_2
\end{align*}
and the base vectors
\begin{align*}
  \mbox{contravariant:} && \mathbf{\bar e}^1 &= \sqrt{g^{11}}\mathbf{e}^1 &
  \mathbf{\bar e}^2 &= \frac{g^{21}}{\sqrt{g^{11}}}\mathbf{e}^1
  +\frac{\sqrt{g}}{\sqrt{g^{11}}}\mathbf{e}^2\\
  % 
  \mbox{covariant:} && \mathbf{\bar e}_1 &= \frac{1}{\sqrt{g^{11}}}\mathbf{e}_1
  -\frac{g^{12}}{\sqrt{g}\sqrt{g^{11}}}\mathbf{e}_2 &
  %
  \mathbf{\bar e}_2 &= \frac{\sqrt{g^{11}}}{\sqrt{g}}\mathbf{e}_2.
\end{align*}

\subsubsection{Application to the gyroaveraging}
Having this transformation rules, we now want to apply these results
to the gyro-averaging case. First we have a look on how Fourier
transforms are applied in one direction in a non-orthogonal coordinate
system. Our new coordinate system has been chosen in a way that the
$\bar{\mathbf{e}}^1 \| \mathbf{e}^1$ which leads also to
$\bar{\mathbf{e}}_2\|\mathbf{e}_2$, due to the orthogonality condition
$\bar{\mathbf{e}}_i\cdot\bar{\mathbf{e}}^j=
\delta_i^j$. For the Fourier transform in the ``2'' direction we need
the base functions, which are the Fourier exponentials
$e^{i\mathbf{k}\mathbf{x}}$. The wave front of these plane waves is
always parallel to $\bar{\mathbf{e}}_1$, so that the wave vector shows
in the $\bar{\mathbf{e}}^2$ direction. This leads to 
\begin{displaymath}
  \mathbf{k}=\bar{k}_2\bar{\mathbf{e}}^2
\end{displaymath}
without a component in the $\bar{\mathbf{e}}^1$ direction. 
Hence, we can represent a function of $\mathbf{x}$ in a Fourier series
in the second direction by the following rule
\begin{displaymath}
  \Phi(\bar{\mathbf{x}})=\sum_{m=-N_y/2+1}^{N_y/2}\Phi_m(\bar{x}^1)\,e^{i\bar{k}_{2,m}\bar{x}^2}
\end{displaymath}

Now we will use this rule for the application to gyro-averaging.
The idea behind the application to gyroaveraging is that the
coordinate system is sheared along the field line and becomes
non-orthogonal. But to calculate the gyroorbit, we have to go back to
the orthogonal one because only there the gyroorbit is a real
circle. To describe the gyroorbit in the new non-orthogonal 
coordinate system, we have to use the transformation rules. The
gyro-averaging is given by
\begin{eqnarray*}
  \avg{F(\mathbf{\bar x}-\mathbf{\bar r})} 
  &=& \frac{1}{2\pi}\int\limits_0^{2\pi} F(\mathbf{\bar
    x}-\mathbf{\bar r})\,d\theta
  = \frac{1}{2\pi}\int\limits_0^{2\pi} \sum_m
  F_m(\bar x^1-\bar r^1)\,e^{i\bar k_{2,m}(\bar x^2-\bar r^2)}\,d\theta\\
  &=& \sum_m e^{i\bar k_{2,m}\bar x^2} \frac{1}{2\pi}\int\limits_0^{2\pi} 
  F_m(\bar x^1-\bar r^1)\,e^{-i\bar k_{2,m} \bar r^2}\,d\theta\\
  &=& \sum_m e^{i\bar k_{2,m}\bar x^2} \frac{1}{2\pi}\int\limits_0^{2\pi} 
  F_m(\bar x^1-\sqrt{g^{11}} r^1)\,\exp\left\{
    -i\bar k_{2,m} \left(
      \frac{g^{21}}{\sqrt{g^{11}}} r^1 
      +\frac{\sqrt{g}}{\sqrt{g^{11}}} r^2
    \right)
  \right\}\,d\theta\\
  &=& \sum_m e^{i\bar k_{2,m}\bar x^2} \frac{1}{2\pi}\int\limits_0^{2\pi} 
  F_m(\bar x^1-\sqrt{g^{11}} \rho\cos\theta)\,\exp\left\{
    -i\bar k_{2,m} \left(
      \frac{g^{21}}{\sqrt{g^{11}}} \rho\cos\theta
      +\frac{\sqrt{g}}{\sqrt{g^{11}}} \rho\sin\theta
    \right)
  \right\}\,d\theta\\
  &=& \sum_m e^{i\bar k_{2,m}\bar x^2} \sum_n
  F_{n,m}\frac{1}{2\pi}\int\limits_0^{2\pi} 
  \Lambda_n(\bar x^1-\sqrt{g^{11}} \rho\cos\theta)\,e^{
    -i\bar k_{2,m} \left(
      g^{21} \rho\cos\theta
      +\sqrt{g} \rho\sin\theta
    \right)/\sqrt{g^{11}}
  }\,d\theta\\
  &=& \sum_m e^{i\bar k_{2,m}\bar x^2} \mathcal{M}(\bar{k}_{2,m},z,\mu)\cdot\tilde{\mathbf{F}}_m
\end{eqnarray*}
again with a gyro averaging matrix $\mathcal{M}$, whose elements are
\begin{displaymath}
  \mathcal{M}_{in}(\bar{k}_{2,m},z,\mu,\sigma)=\frac{1}{2\pi}\int\limits_0^{2\pi} 
  \Lambda_n(\bar x^1_i-\sqrt{g^{11}} \rho_\sigma(\bar{x}^1_i,z,\mu)\cos\theta)\,e^{
    -i\bar k_{2,m} \rho_\sigma(\bar{x}^1_i,z,\mu)\left(
      g^{21} \cos\theta
      +\sqrt{g} \sin\theta
    \right)/\sqrt{g^{11}}
  }\,d\theta
\end{displaymath}
Here, tilded quantities stand for base function coefficients, while untilded
for point values. The gyroradius is given by
\begin{displaymath}
  \rho_\sigma^2(z,\mu)=\frac{v_\bot^2}{\Omega_\sigma^2(z)}
  =\frac{v_\bot^2}{c_\rf^2}\frac{\Omega_\rf^2}{\Omega_\sigma^2(z)}\frac{c_\rf^2}{\Omega_\rf^2}
  =\frac{2\mu B_0(z)}{m_\sigma c_\rf^2}\frac{\Omega_\rf^2m_\sigma^2c^2}{e_\sigma^2B_0^2(z)}\rho_\rf^2
  =2\frac{\mu B_\rf}{\refj[\sigma]{T}}\frac{\refj[\sigma]{T}}{T_\rf} \frac{e^2}{e_\sigma^2} \frac{m_\sigma}{m_\rf}
  \frac{B_\rf}{B_0(z)}\rho_\rf^2.
\end{displaymath}


\subsubsection{Efficient calculation of the gyro matrix}
To efficiently calculate the elements of the last matrix we can write
(using $\rho=\rho_\sigma(x^1,z,\mu)$ and $\bar{r}^2(\theta)=\rho \left(
      g^{21} \cos\theta+\sqrt{g} \sin\theta
    \right)/\sqrt{g^{11}}$ and $\bar{r}^1(\theta)=\sqrt{g^{11}} \rho\cos\theta$)
\begin{eqnarray*}
  \mathcal{M}(\bar{k}_{2,m},z,\mu) &=&\frac{1}{2\pi}\int\limits_0^{2\pi} 
  \Lambda_n(\bar x^1-\bar{r}^1(\theta))\,e^{
    -i\bar k_{2,m}\bar{r}^2(\theta)
  }\,d\theta\\
  &=&\frac{1}{2\pi}\int\limits_0^{\pi} 
  \Lambda_n(\bar x^1-\bar{r}^1(\theta))\,e^{
    -i\bar k_{2,m}\bar{r}^2(\theta)
  }\,d\theta
  +\frac{1}{2\pi}\int\limits_\pi^{2\pi} 
  \Lambda_n(\bar x^1-\bar{r}^1(\theta))\,e^{
    -i\bar k_{2,m}\bar{r}^2(\theta)
  }\,d\theta\\
  &=&\frac{1}{2\pi}\int\limits_0^{\pi} 
  \Lambda_n(\bar x^1-\bar{r}^1(\theta))\,e^{
    -i\bar k_{2,m}\bar{r}^2(\theta)
  }\,d\theta
  +\frac{1}{2\pi}\int\limits_{-\pi}^{0} 
  \Lambda_n(\bar x^1-\bar{r}^1(\theta))\,e^{
    -i\bar k_{2,m}\bar{r}^2(\theta)
  }\,d\theta\\
  &=&\frac{1}{2\pi}\int\limits_0^{\pi} 
  \Lambda_n(\bar x^1-\bar{r}^1(\theta))\,e^{
    -i\bar k_{2,m}\bar{r}^2(\theta)
  }\,d\theta
  -\frac{1}{2\pi}\int\limits_0^{-\pi}
  \Lambda_n(\bar x^1-\bar{r}^1(\theta))\,e^{
    -i\bar k_{2,m}\bar{r}^2(\theta)
  }\,d\theta
\end{eqnarray*}
Now me make a variable substitution from $\theta$ to
$\theta'=-\theta$ in the second integral. This leaves the $\cos$ unchanged and changes the
sign of the sine.
\begin{eqnarray*}
  \mathcal{M}(\bar{k}_{2,m},z,\mu) 
  &=&\frac{1}{2\pi}\int\limits_0^{\pi} 
  \Lambda_n(\bar x^1-\bar{r}^1(\theta))\,e^{
    -i\bar k_{2,m}\rho \left(
      g^{21} \cos\theta
      +\sqrt{g} \sin\theta
    \right)/\sqrt{g^{11}}
  }\,d\theta\\
  &&+\frac{1}{2\pi}\int\limits_0^\pi
  \Lambda_n(\bar x^1-\bar{r}^1(\theta'))\,e^{
    -i\bar k_{2,m}\rho \left(
      g^{21} \cos\theta'
      -\sqrt{g} \sin\theta'
    \right)/\sqrt{g^{11}}
  }\,d\theta'\\
  &=&\frac{1}{2\pi}\int\limits_0^{\pi} 
  \Lambda_n(\bar x^1-\bar{r}^1(\theta))\left[
    \,e^{
      -i\bar k_{2,m}\rho \left(
        g^{21} \cos\theta
        +\sqrt{g} \sin\theta
      \right)/\sqrt{g^{11}}
    }
    +e^{
      -i\bar k_{2,m}\rho \left(
        g^{21} \cos\theta
        -\sqrt{g} \sin\theta
      \right)/\sqrt{g^{11}}
    }
  \right]\,d\theta\\
  &=&\frac{1}{2\pi}\int\limits_0^{\pi} 
  \Lambda_n(\bar x^1-\bar{r}^1(\theta))e^{
      -i\bar k_{2,m}g^{21} \rho\cos\theta/\sqrt{g^{11}}
    }\left[
    e^{
      -i\bar k_{2,m}\sqrt{g} \rho\sin\theta/\sqrt{g^{11}}
    }
    +e^{
      i\bar k_{2,m} \sqrt{g} \rho\sin\theta/\sqrt{g^{11}}
    }
  \right]\,d\theta\\
  &=&\frac{1}{\pi}\int\limits_0^{\pi} 
  \Lambda_n(\bar x^1-\bar{r}^1(\theta))e^{
      -i\bar k_{2,m}g^{21} \rho\cos\theta/\sqrt{g^{11}}
    }\cos\left(
      \bar k_{2,m}\sqrt{g} \rho\sin\theta/\sqrt{g^{11}}
    \right)\,d\theta
\end{eqnarray*}
Now discretizing the interval $[0,\pi]$ with $N+1$ points, with
boundary points on $0$ and $\pi$:
\begin{displaymath}
  \theta_j=j\cdot\frac{\pi}{N}\qquad j=\{0,\ldots,N\}
\end{displaymath}
The integral is then discretized with a sum as follows.
\begin{eqnarray*}
  I &=&\frac{1}{\pi}\int\limits_0^{\pi} 
  \Lambda_n(\bar x^1-\sqrt{g^{11}} \rho\cos\theta)e^{
    -i\bar k_{2,m}g^{21} \rho\cos\theta/\sqrt{g^{11}}
  }\cos\left(
    \bar k_{2,m}\sqrt{g} \rho\sin\theta/\sqrt{g^{11}}
  \right)\,d\theta\\
  &=& \frac{1}{\pi}\sum\limits_{j=0}^{N} 
  \Lambda_n(\bar x^1-\sqrt{g^{11}} \rho\cos\theta_j) e^{
    -i\bar k_{2,m}g^{21} \rho\cos\theta_j/\sqrt{g^{11}}
  }\cos\left(
    \bar k_{2,m}\sqrt{g} \rho\sin\theta_j/\sqrt{g^{11}}
  \right)\,\Delta\theta_j\\
  &=& \frac{1}{\pi}
  \Lambda_n(\bar x^1-\sqrt{g^{11}} \rho\cos\theta_0) e^{
    -i\bar k_{2,m}g^{21} \rho\cos\theta_0/\sqrt{g^{11}}
  }\cos\left(
    \bar k_{2,m}\sqrt{g} \rho\sin\theta_0/\sqrt{g^{11}}
  \right)\,\frac{\pi}{2N}\\
  &&+ \frac{1}{\pi}
  \Lambda_n(\bar x^1-\sqrt{g^{11}} \rho\cos\theta_N) e^{
    -i\bar k_{2,m}g^{21} \rho\cos\theta_N/\sqrt{g^{11}}
  }\cos\left(
    \bar k_{2,m}\sqrt{g} \rho\sin\theta_N/\sqrt{g^{11}}
  \right)\,\frac{\pi}{2N}\\
  &&+ \frac{1}{\pi}\sum\limits_{j=1}^{N-1} 
  \Lambda_n(\bar x^1-\sqrt{g^{11}} \rho\cos\theta_j) e^{
    -i\bar k_{2,m}g^{21} \rho\cos\theta_j/\sqrt{g^{11}}
  }\cos\left(
    \bar k_{2,m}\sqrt{g} \rho\sin\theta_j/\sqrt{g^{11}}
  \right)\,\frac{\pi}{N}\\
  &=& \frac{1}{2N}
  \Lambda_n(\bar x^1-\sqrt{g^{11}} \rho) e^{
    -i\bar k_{2,m}g^{21} \rho/\sqrt{g^{11}}
  }
  + \frac{1}{2N}
  \Lambda_n(\bar x^1+\sqrt{g^{11}} \rho) e^{
    i\bar k_{2,m}g^{21} \rho/\sqrt{g^{11}}
  }\\
  &&+ \frac{1}{N}\sum\limits_{j=1}^{N-1} 
  \Lambda_n(\bar x^1-\sqrt{g^{11}} \rho\cos\theta_j) e^{
    -i\bar k_{2,m}g^{21} \rho\cos\theta_j/\sqrt{g^{11}}
  }\cos\left(
    \bar k_{2,m}\sqrt{g} \rho\sin\theta_j/\sqrt{g^{11}}
  \right)
\end{eqnarray*}

The parallelization over all dimensions beside $x$ is
straightforward. If parallelized over $x$, one has to calculate a
matrix-vector multiplication with divided fields and matrices, which
can be done with \texttt{petsc}. At the moment is seems as if one
needs the node vector of the $x$ nodes on all processors in $x$
direction to set up the gyro-matrix. But this also should be
possible. 


\subsection{Efficient calculation of the gyro-average}
In the last subsection, we derived an efficient way to set up the gyro
matrix. This is important for convenience, but not so important for
production runs. There the gyro-averaging procedure becomes much more
important, as it is called quite often during a simulation. To make
the matrix-vector multiplication for the gyroaveraging efficient, we
must use the sparse (banded) structure of the matrix. In general the
matrix-vector multiplication of a matrix $\mathcal{M}$ with the
elements $m_{ij}$ with a vector $v$ with the elements $v_j$ can be
written as
\begin{equation}
  \label{eq:matvec}
  r_i=\sum_{j=1}^N m_{ij}\,v_j.
\end{equation}
But we know that the matrix is banded, so we save only the band and
leave out all of the zeros. Therefore we define a matrix, which has as
columns the band diagonals of the original matrix. 
\begin{displaymath}
  a_{r,k}=m_{r,r+k}\quad\forall\,k\in\{-b,\ldots,0,\ldots,b\}\quad\mbox{and}\quad\forall r\in\{1,\ldots,N\}
\end{displaymath}
where $b$ is the number of neighboring diagonals. Hence the total
number of diagonals is $2b+1$, which is also the number of columns of
the new matrix $\mathcal{A}$. So do we come back from the entries of
$\mathcal{A}$ to the entries of $\mathcal{M}$? The following relation
will do it.
\begin{displaymath}
  m_{ij}=
  \begin{cases}
    a_{i,j-i} & j\in[i-b,i+b]\\
    0 & \mbox{else}
  \end{cases}
\end{displaymath}
Using this relation in eq.~(\ref{eq:matvec}) and come to
\begin{displaymath}
  r_i=\sum_{j=i-b}^{i+b} a_{i,j-i}\,v_j
\end{displaymath}
Substituting the index $j$ by the rule $k=j-i$, we come to
\begin{displaymath}
  r_i=\sum_{k=-b}^{b} a_{i,k}\,v_{k+i}
\end{displaymath}





\section{Gyro-Mapping and the field equations}
\label{sec:gyromapping}

For the field equations we don't need a gyro-averaging operator but an
operator which does the gyro-mapping. The derivation of the field
equations needs an expression for the particle density and the
particle current. 

\subsection{Quasineutrality}
\label{sec:quasineutrality}
We have (as already often derived) for the density
\begin{eqnarray*}
  n_{1j}(\mathbf{x}) &=& \frac{B_0(x,z)}{m_j}\int f_{1j}\,dv_\|\,d\mu\,d\theta\\
  &=& \frac{B_0(x,z)}{m_j}\int \delta(\mathbf{X}+\mathbf{r}-\mathbf{x})\,T^*F_{1j}\,dv_\|\,d\mu\,d\theta\\
  &=& \frac{B_0(x,z)}{m_j}\int
  \delta(\mathbf{X}+\mathbf{r}-\mathbf{x})\,\left[
    F_{j1}(\mathbf{X})
    -\frac{e_j}{T_{0j}(x)}\tilde\Phi(\mathbf{X}+\mathbf{r}) F_{0j}(x,z)
  \right]\,d\mathbf{X}\,dv_\|\,d\mu\,d\theta\\
  &=& \frac{B_0(x,z)}{m_j}\int
  \delta(\mathbf{X}+\mathbf{r}-\mathbf{x})\,\Bigg[
    F_{j1}(\mathbf{X})\\
    &&-\frac{e_j}{T_{0j}(x)}\left(\Phi(\mathbf{X}+\mathbf{r})-\langle\Phi(\mathbf{X}+\mathbf{r})\rangle\right) F_{0j}(x,z)
  \Bigg]\,d\mathbf{X}\,dv_\|\,d\mu\,d\theta\\
  &=& \frac{B_0(x,z)}{m_j}\int F_{j1}(\mathbf{x}-\mathbf{r})\,dv_\|\,d\mu\,d\theta\\
  &&-\frac{e_j}{T_{0j}(x)}\left(
    \Phi(\mathbf{x}) n_{0j}(x)
    -\frac{B_0(x,z)}{m_j}\int\delta(\mathbf{X}+\mathbf{r}-\mathbf{x})\,\langle\Phi(\mathbf{X}+\mathbf{r})\rangle
    F_{0j}(x,z)\,d\mathbf{X}\,dv_\|\,d\mu\,d\theta
  \right) 
\end{eqnarray*}
This expression is now again gyro-averaged which leads to
\begin{eqnarray*}
  n_{1j}(\mathbf{x}) &=&\frac{1}{2\pi}\int\limits_0^{2\pi}
  \Bigg[
    \frac{B_0(x,z)}{m_j}\int F_{j1}(\mathbf{x}-\mathbf{r}'')\,dv_\|\,d\mu\,d\theta\\
    &&-\frac{e_j}{T_{0j}(x)}\left(
      \Phi(\mathbf{x}) n_{0j}(x)
      -\frac{B_0(x,z)}{m_j}\int\delta(\mathbf{X}+\mathbf{r}''-\mathbf{x})\,\langle\Phi(\mathbf{X}+\mathbf{r}')\rangle
      F_{0j}(x,z)\,d\mathbf{X}\,dv_\|\,d\mu\,d\theta
    \right) 
  \Bigg]\,d\theta''\\
  &=& 
  \frac{B_0(x,z)}{m_j}\int \frac{1}{2\pi}\int\limits_0^{2\pi} F_{j1}(\mathbf{x}-\mathbf{r}'')\,d\theta''\,dv_\|\,d\mu\,d\theta\\
  &&-\frac{e_j}{T_{0j}(x)}\Bigg(
    \Phi(\mathbf{x}) n_{0j}(x)\\
    &&-\frac{B_0(x,z)}{m_j}\int 
    \frac{1}{2\pi}\int\limits_0^{2\pi}\delta(\mathbf{X}+\mathbf{r}''-\mathbf{x})\,\langle\Phi(\mathbf{X}+\mathbf{r}')\rangle
    \,d\theta''\,d\mathbf{X} F_{0j}(x,z)\,dv_\|\,d\mu\,d\theta
  \Bigg)\\
  &=& \frac{B_0(x,z)}{m_j}\int \mathcal{G}[F_{j1}](\mathbf{x},v_\|,\mu)\,dv_\|\,d\mu\,d\theta\\
  &&-\frac{e_j}{T_{0j}(x)}\left(
    \Phi(\mathbf{x}) n_{0j}(x)
    -\frac{B_0(x,z)}{m_j}\int \mathfrak{G}[\Phi](\mathbf{x},\mu) F_{0j}(x,z)\,dv_\|\,d\mu\,d\theta
  \right) 
\end{eqnarray*}
with $\mathfrak{G}[\Phi]=\int
d\mathbf{X}\frac{1}{2\pi}\int\limits_0^{2\pi}d\theta''
\delta(\mathbf{X}+\mathbf{r}''-\mathbf{x})\,\langle\Phi(\mathbf{X}+\mathbf{r}')\rangle$. 
So we have to evaluate the two integrals. The first one is already
calculated in Sec.~\ref{sec:gyroaveraging}, the second will be
calculated in the present section. We start with 
\begin{eqnarray*}
  \mathfrak{G}[\Phi](\bar{\mathbf{x}},\mu) &=&
  \avg{\int\delta(\bar{\mathbf{X}}+\bar{\mathbf{r}}-\bar{\mathbf{x}})\bar{\Phi}(\bar{\mathbf{X}})\,d\bar{\mathbf{X}}}
  =\avg{\int\delta(\bar{\mathbf{X}}+\bar{\mathbf{r}}-\bar{\mathbf{x}})
    \avg{\Phi(\bar{\mathbf{X}}+\bar{\mathbf{r}})}\,d\bar{\mathbf{X}}} \\
  &=&\avg{\int\delta(\bar{\mathbf{X}}+\bar{\mathbf{r}}-\bar{\mathbf{x}})
    \sum_m e^{i\bar{k}_{2,m}\bar{X}^2}\sum_n\mathcal{M}_n(\bar{X}^1,\bar{k}_{2,m},z,\mu,\sigma)\tilde{\Phi}_n(\bar{k}_{2,m},z)\,d\mathbf{X}} \\
  &=&\frac{1}{2\pi}\int\limits_0^{2\pi}
  \sum_m e^{i\bar{k}_{2,m}(\bar{x}^2-\bar{r}^2)}
  \sum_n\mathcal{M}_n(\bar{x}^1-\bar{r}^1,\bar{k}_{2,m},z,\mu,\sigma)\tilde{\Phi}_n(\bar{k}_{2,m},z)\,d\theta \\
  &=&\sum_m e^{i\bar{k}_{2,m}\bar{x}^2} \sum_n \left[
    \frac{1}{2\pi}\int\limits_0^{2\pi}
    \mathcal{M}_n(\bar{x}^1-\bar{r}^1,\bar{k}_{2,m},z,\mu,\sigma)
    e^{-i\bar{k}_{2,m}\bar{r}^2}\,d\theta
  \right]\tilde{\Phi}_n(\bar{k}_{2,m},z)
\end{eqnarray*}
The term in the square brackets is a function of the radial coordinate
and can therefore be expanded in a series of base functions.
\begin{displaymath}
  \mathcal{M}_n(\bar{x}^1-\bar{r}^1,\bar{k}_{2,m},z,\mu,\sigma)
  =\sum_s \mathcal{M}_{sn}(\bar{k}_{2,m},z,\mu,\sigma)\Lambda_s(\bar{x}^1-\bar{r}^1)
\end{displaymath}
Using this, we can further write
\begin{eqnarray*}
  \mathfrak{G}[\Phi](\bar{\mathbf{x}},\mu)
  &=&\sum_m e^{i\bar{k}_{2,m}\bar{x}^2} \sum_n \left[
    \sum_s \mathcal{M}_{sn}(\bar{k}_{2,m},z,\mu,\sigma)
    \frac{1}{2\pi}\int\limits_0^{2\pi}\Lambda_s(\bar{x}^1-\bar{r}^1)
    e^{-i\bar{k}_{2,m}\bar{r}^2}\,d\theta
  \right]\tilde{\Phi}_n(\bar{k}_{2,m},z)\\
  &=&\sum_m e^{i\bar{k}_{2,m}\bar{x}^2} \sum_n \left[
    \sum_s \mathcal{M}_s(\bar{k}_{2,m},z,\mu,\sigma)\cdot\mathcal{M}_{sn}(\bar{k}_{2,m},z,\mu,\sigma)
  \right]\tilde{\Phi}_n(\bar{k}_{2,m},z)
\end{eqnarray*}
In a matrix vector form, this can also be written as
\begin{eqnarray*}
  \pmb{\mathfrak{G}}[\Phi](\bar{\mathbf{x}},\mu)
  &=&\sum_m e^{i\bar{k}_{2,m}\bar{x}^2} \mathcal{M}^2(\bar{k}_{2,m},z,\mu,\sigma)\cdot\tilde{\mathbf{\Phi}}(\bar{k}_{2,m},z)
\end{eqnarray*}
where bold quantities indicate a vector in $\bar{x}^1$ direction.

Another way to derive the double gyro-averaging operator is the
following, where we explicitly calculate the $\mathcal{P}$
matrix. This matrix is identical to the $\mathcal{M}^2$ matrix, as can
also be shown numerically. In the \gene\  code, we calculate the
gyro-matrix $\mathcal{M}$ and then square it to get $\mathcal{P}$. 
\begin{eqnarray*}
  \mathfrak{G}[\Phi]&=&\avg{\int\delta(\mathbf{X}+\mathbf{r}-\mathbf{x})\bar{\Phi}(\mathbf{X})\,d\mathbf{X}} 
  =\avg{\int\delta(\mathbf{X}+\mathbf{r}-\mathbf{x})\avg{\Phi(\mathbf{X}+\mathbf{r})}\,d\mathbf{X}} \\
  &=& \Bigg\langle
    \int\delta(\mathbf{X}+\mathbf{r}-\mathbf{x}) 
    \sum_m e^{i\bar k_{2,m}\bar X^2} \sum_n \Phi_{n,m}\\
    &&\cdot\,\frac{1}{2\pi}\int\limits_0^{2\pi}
    \Lambda_n(\bar{X}^1+\sqrt{g^{11}}\rho\cos\theta')
    e^{i\bar{k}_{2,m} \rho\left(
        g^{12}\cos\theta'+g\sin\theta'
      \right)/\sqrt{g^{11}}
    }\,d\theta'\,d\mathbf{X}
  \Bigg\rangle \\
  &=& \frac{1}{2\pi}\int\limits_0^{2\pi} 
    \int\delta(\mathbf{X}+\mathbf{r}-\mathbf{x}) 
    \sum_m e^{i\bar k_{2,m}\bar X^2} \sum_n \Phi_{n,m}\\
    &&\cdot\,\frac{1}{2\pi}\int\limits_0^{2\pi}
    \Lambda_n(\bar{X}^1+\sqrt{g^{11}}\rho\cos\theta')
    e^{i\bar{k}_{2,m} \rho\left(
        g^{12}\cos\theta'+g\sin\theta'
      \right)/\sqrt{g^{11}}
    }\,d\theta'\,d\mathbf{X}
  \,d\theta \\
  &=& \frac{1}{2\pi}\int\limits_0^{2\pi} 
  \sum_m e^{i\bar k_{2,m}(\bar x^2-\bar r^2)} 
    \sum_n \Phi_{n,m}\\
    &&\cdot\,\frac{1}{2\pi}\int\limits_0^{2\pi}
    \Lambda_n(\bar{x}^1-\bar{r}^1+\sqrt{g^{11}}\rho\cos\theta')
    e^{i\bar{k}_{2,m} \rho\left(
        g^{12}\cos\theta'+g\sin\theta'
      \right)/\sqrt{g^{11}}
    }\,d\theta' \,d\theta \\
  &=& \sum_m e^{i\bar k_{2,m}\bar x^2} \sum_n \Phi_{n,m}
  \frac{1}{2\pi}\int\limits_0^{2\pi} e^{-i\bar k_{2,m}\rho(g^{12}\cos\theta+g\sin\theta)/\sqrt{g^{11}}} \\
  &&\cdot\,\frac{1}{2\pi}\int\limits_0^{2\pi}
  \Lambda_n(\bar{x}^1-\sqrt{g^{11}}\rho\cos\theta+\sqrt{g^{11}}\rho\cos\theta')
  e^{i\bar{k}_{2,m} \rho\left(
      g^{12}\cos\theta'+g\sin\theta'
    \right)/\sqrt{g^{11}}
  }\,d\theta' \,d\theta \\
  &=& \sum_m e^{i\bar k_{2,m}\bar x^2} \sum_n \Phi_{n,m}
  \frac{1}{2\pi}\int\limits_0^{2\pi} e^{-i\bar k_{2,m}\rho(g^{12}\cos\theta+g\sin\theta)/\sqrt{g^{11}}} \\
  &&\cdot\,\frac{1}{2\pi}\int\limits_0^{2\pi}
  \Lambda_n(\bar{x}^1-\sqrt{g^{11}}\rho(\cos\theta-\cos\theta'))
  e^{i\bar{k}_{2,m} \rho\left(
      g^{12}\cos\theta'+g\sin\theta'
    \right)/\sqrt{g^{11}}
  }\,d\theta' \,d\theta \\
  &=& \sum_m e^{i\bar k_{2,m}\bar x^2} \sum_n \mathcal{P}(\bar k_{2,m},z,\mu) \Phi_{n,m}
\end{eqnarray*}
with
\begin{eqnarray*}
  \mathcal{P}(\bar{k}_{2,m},z,\mu) &=& 
  \frac{1}{2\pi}\int\limits_0^{2\pi} e^{-i\bar k_{2,m}\rho(g^{12}\cos\theta+g\sin\theta)/\sqrt{g^{11}}} \\
  &&\cdot\,\frac{1}{2\pi}\int\limits_0^{2\pi}
  \Lambda_n(\bar{x}^1-\sqrt{g^{11}}\rho(\cos\theta-\cos\theta'))
  e^{i\bar{k}_{2,m} \rho\left(
      g^{12}\cos\theta'+g\sin\theta'
    \right)/\sqrt{g^{11}}
  }\,d\theta' \,d\theta 
\end{eqnarray*}
This operator accords to $J_0^2$ operator in the double Fourier case.
If we also take the moment of this operator weighted with $F_0$ we
arrive at the correspondent to the $\Gamma_0$ function in the double
Fourier case. We can therefore write
\begin{eqnarray*}
  G[\Phi](\mathbf{x}) &=& \frac{B_0(x,z)}{m_j}\int
  \mathfrak{G}[\Phi](\mathbf{x},\mu) F_{0j}(x,z,v_\|,\mu)\,dv_\|\,d\mu\,d\theta\\
  &=& \frac{2n_{0j}(x)}{v_{Tj}^2(x)}\frac{B_0(x,z)}{m_j}\int\mathfrak{G}[\Phi](\mathbf{x},\mu)
  \,e^{-\mu B_0(x,z)/T_{0j}(x)}\,d\mu\\
  &=& \frac{2n_{0j}(x)}{v_{Tj}^2(x)}\frac{B_0(x,z)}{m_j}
  \int \sum_m e^{i\bar k_{2,m}\bar x^2} \sum_n \mathcal{P}_n(\bar{x}^1,\bar{k}_{2,m},z,\mu) \tilde{\Phi}_{n}(\bar{k}_{2,m},z)
  \,e^{-\mu B_0(x,z)/T_{0j}(x)}\,d\mu
\end{eqnarray*}
This can be written in matrix-vector form
\begin{eqnarray*}
  \mathbf{G}[\Phi](\bar{k}_{2,m},z) &=& 
  \frac{2n_{0j}(x)}{v_{Tj}^2(x)}\frac{B_0(x,z)}{m_j}\int \mathcal{P}(\bar k_{2,m},z,\mu)\cdot\tilde{\mathbf{\Phi}}_m
  \,e^{-\mu B_0(x,z)/T_{0j}(x)}\,d\mu\\
  &=& \left[\frac{2n_{0j}(x)}{v_{Tj}^2(x)}\frac{B_0(x,z)}{m_j}\int \mathcal{P}(\bar{k}_{2,m},z,\mu)
    \,e^{-\mu B_0(x,z)/T_{0j}(x)}\,d\mu\right] \cdot\tilde{\mathbf{\Phi}}_m
\end{eqnarray*}
where the expression in the brackets can also be calculated during the
initialization. The tilded boldface $\Phi$ stands for a vector of
base function coefficients, representing the electrostatic potential. Values
without a tilde are values at the grid points.
With this notation, we arrive at an expression for the density given
by
\begin{eqnarray*}
  n_{1j}(\mathbf{x}) &=&
  \frac{B_0(x,z)}{m_j}\int \mathcal{G}[F_{j1}](\mathbf{x},v_\|,\mu)\,dv_\|\,d\mu\,d\theta
  -\frac{e_j}{T_{0j}(x)}\left(
    \Phi(\mathbf{x}) n_{0j}(x)
    -G[\Phi](\mathbf{x})
  \right) 
\end{eqnarray*}

Using the quasineutrality equation, we can write for the first field
equation in unnormalized units
\begin{eqnarray*}
  0 &=&\sum_j e_jn_{1j}(\mathbf{x})\\
  &=&\sum_j e_j \left(
    \frac{B_0(x,z)}{m_j}\int \mathcal{G}[F_{j1}](\mathbf{x},v_\|,\mu)\,dv_\|\,d\mu\,d\theta
    -\frac{e_j}{T_{0j}(x)}\left(
      \Phi(\mathbf{x}) n_{0j}(x)
      -G[\Phi](\mathbf{x})
    \right)  
  \right)
\end{eqnarray*}
Reordered we come to
\begin{displaymath}
  \sum_j\frac{e_j^2}{T_{0j}(x)}\left(
    \Phi(\mathbf{x}) n_{0j}(x)
    -G[\Phi](\mathbf{x})
  \right) 
  = \sum_je_j\frac{B_0(x,z)}{m_j}\int \mathcal{G}[F_{j1}](\mathbf{x},v_\|,\mu)\,dv_\|\,d\mu\,d\theta
\end{displaymath}
To further normalize this equation, we use the expressions, derived
above to come to
\begin{multline*}
  \sum_j\frac{e_j^2}{T_{0j}(x)}\Bigg(
    \sum_m e^{i\bar k_{2,m}\bar x^2} \Phi(\bar{x}^1,\bar{k}_{2,m},z) n_{0j}(x)\\
    -\frac{2n_{0j}(x)}{v_{Tj}^2(x)}\frac{B_0(x,z)}{m_j}
    \int \sum_m e^{i\bar k_{2,m}\bar x^2} \sum_n \mathcal{P}_n(\bar{x}^1,\bar{k}_{2,m},z,\mu) \tilde{\Phi}_{n}(\bar{k}_{2,m},z)
    \,e^{-\mu B_0(x,z)/T_{0j}(x)}\,d\mu
  \Bigg) \\
  = \sum_j e_j\frac{B_0(x,z)}{m_j}\int \sum_m e^{i\bar{k}_{2,m}\bar{x}^2}
  \sum_n\mathcal{M}_n(\bar{x}^1,\bar{k}_{2,m},z,\mu)\tilde{F}_n(\bar{k}_{2,m},z,v_\|,\mu)\,dv_\|\,d\mu\,d\theta
\end{multline*}
Reordering
\begin{multline*}
  \sum_m e^{i\bar k_{2,m}\bar x^2} \sum_j\frac{e_j^2}{T_{0j}(x)}\Bigg(
    \Phi(\bar{x}^1,\bar{k}_{2,m},z) n_{0j}(x)\\
    -\frac{2n_{0j}(x)}{v_{Tj}^2(x)}\frac{B_0(x,z)}{m_j}
    \int  \sum_n \mathcal{P}_n(\bar{x}^1,\bar{k}_{2,m},z,\mu) \tilde{\Phi}_{n}(\bar{k}_{2,m},z)
    \,e^{-\mu B_0(x,z)/T_{0j}(x)}\,d\mu
  \Bigg) \\
  = \sum_m e^{i\bar{k}_{2,m}\bar{x}^2} \sum_j e_j\frac{B_0(x,z)}{m_j}\int 
  \sum_n\mathcal{M}_n(\bar{x}^1,\bar{k}_{2,m},z,\mu)\tilde{F}_n(\bar{k}_{2,m},z,v_\|,\mu)\,dv_\|\,d\mu\,d\theta
\end{multline*}
As both sides of the field equation are Fourier sums in the second
coordinate, we can write the field equation for all modes
($\forall m$)
\begin{multline*}
  \sum_j\frac{e_j^2}{T_{0j}(x)}\Bigg(
    \Phi(\bar{x}^1,\bar{k}_{2,m},z) n_{0j}(x)\\
    -\frac{2n_{0j}(x)}{v_{Tj}^2(x)}\frac{B_0(x,z)}{m_j}
    \int  \sum_n \mathcal{P}_n(\bar{x}^1,\bar{k}_{2,m},z,\mu) \tilde{\Phi}_{n}(\bar{k}_{2,m},z)
    \,e^{-\mu B_0(x,z)/T_{0j}(x)}\,d\mu
  \Bigg) \\
  = \sum_j 2\pi e_j\frac{B_0(x,z)}{m_j}\int 
  \sum_n\mathcal{M}_n(\bar{x}^1,\bar{k}_{2,m},z,\mu)\tilde{F}_n(\bar{k}_{2,m},z,v_\|,\mu)\,dv_\|\,d\mu
\end{multline*}
Now switching to the matrix-vector form, we can write
\begin{multline*}
  \sum_je_j^2 \diag\{\frac{n_{0j}(x)}{T_{0j}(x)}\}\Bigg(
    \mathcal{L}\\
    -\frac{2}{m_j}\diag\{\frac{B_0(x,z)}{v_{Tj}^2(x)}\}
    \int \mathcal{P}(\bar{k}_{2,m},z,\mu)
    \cdot\diag\{e^{-\mu B_0(x,z)/T_{0j}(x)}\}\,d\mu
  \Bigg)\cdot\tilde{\mathbf{\Phi}}(\bar{k}_{2,m},z) \\
  = \sum_j \frac{2\pi e_j}{m_j}\diag\{B_0(x,z)\}\int 
  \mathcal{M}(\bar{k}_{2,m},z,\mu)\cdot\tilde{\mathbf{F}}(\bar{k}_{2,m},z,v_\|,\mu)\,dv_\|\,d\mu
\end{multline*}
where we used a base function representation for the electrostatic potential also for
the first occurrence with
$\mathbf{\Phi}(\bar{k}_{2,m},z)=\mathcal{L}\cdot\tilde{\mathbf{\Phi}}(\bar{k}_{2,m},z)$.
We can reorganize the equation to introduce dimensionless quantities.
\begin{multline*}
  \frac{e}{T_\rf}n_\rf\sum_j\frac{e_j^2}{e^2} \diag\{\frac{n_{0j}(x)}{n_\rf}\frac{T_\rf}{T_{0j}(x)}\}\Bigg(
    \mathcal{L}\\
    -\diag\{\frac{B_0(x,z)}{B_\rf}\frac{\refj{T}}{T_{0j}(x)}\}\frac{B_\rf}{\refj{T}}\int \mathcal{P}(\bar{k}_{2,m},z,\mu)
    \cdot\diag\{e^{-\mu B_0(x,z)/T_{0j}(x)}\}\,d\mu
  \Bigg)\cdot\tilde{\mathbf{\Phi}}(\bar{k}_{2,m},z) \\
  = 2\pi\sum_j \frac{1}{m_j}\frac{e_j}{e}\diag\{B_0(x,z)\}\int 
  \mathcal{M}(\bar{k}_{2,m},z,\mu)\cdot\tilde{\mathbf{F}}(\bar{k}_{2,m},z,v_\|,\mu)\,dv_\|\,d\mu
\end{multline*}
Using 
\begin{multline*}
  \mathcal{P}''(\bar{k}_{2,m},z) = \sum_j\frac{e_j^2}{e^2}
  \diag\{\frac{n_{0j}(x)}{n_\rf}\frac{T_\rf}{T_{0j}(x)}\}\cdot\\
  \left(
    \mathcal{L}
    -\diag\{\frac{B_0(x,z)}{B_\rf}\frac{\refj{T}}{T_{0j}(x)}\}\frac{B_\rf}{\refj{T}}\int \mathcal{P}(\bar{k}_{2,m},z,\mu)
    \cdot\diag\{e^{-\mu B_0(x,z)/T_{0j}(x)}\}\,d\mu
  \right)
\end{multline*}
we can write the quasineutrality equation in the following form
\begin{multline*}
  \frac{e}{T_\rf} \mathcal{P}''(\bar{k}_{2,m},z)\cdot\tilde{\mathbf{\Phi}}(\bar{k}_{2,m},z) \\
  = \frac{2\pi}{n_\rf}\sum_j \frac{1}{m_j}\frac{e_j}{e}\diag\{B_0(x,z)\}\int 
  \mathcal{M}(\bar{k}_{2,m},z,\mu)\cdot\tilde{\mathbf{F}}(\bar{k}_{2,m},z,v_\|,\mu)\,dv_\|\,d\mu
\end{multline*}
with the dimensionless matrices $\mathcal{P}''$ and $\mathcal{M}$.

The matrix $\mathcal{P}''$ can be calculated during initialization and
then in each timestep, we have either to solve the resulting system of
equations or we invert the matrix in the initialization and multiply
just the inverted matrix to the right hand side. Which method is
faster, has to be seen.
As the $\mathcal{P}''$ matrix is also independent of the species (it is already a sum
over all species) and independent of velocity space variables, we can
combine its inverse with the gyro-averaging matrix and get the final
field equation (still unnormalized)
\begin{multline*}
  \frac{e}{T_\rf} \tilde{\mathbf{\Phi}}(\bar{k}_{2,m},z) \\
  = \frac{2\pi}{n_\rf}\sum_j \frac{1}{m_j}\frac{e_j}{e}\diag\{B_0(x,z)\}\int 
  \left[\mathcal{P}''(\bar{k}_{2,m},z)\right]^{-1}\cdot\mathcal{M}(\bar{k}_{2,m},z,\mu)\cdot\tilde{\mathbf{F}}(\bar{k}_{2,m},z,v_\|,\mu)\,dv_\|\,d\mu
\end{multline*}
This combined matrix can be calculated during initialization.

\subsubsection{Normalization}
\label{sec:norm_field1}
We are normalizing as described in sec.~\ref{sec:normalization}. The
$\mathcal{P}''$ matrix is already dimensionless, but it is calculated
in normalized units as follows.
\begin{multline*}
  \mathcal{P}''(\bar{k}_{2,m},z) = \sum_j\frac{e_j^2}{e^2}
  \diag\{\frac{n_{0j}(x)}{n_\rf}\frac{T_\rf}{T_{0j}(x)}\}\cdot\\
  \left(
    \mathcal{L}
    -\diag\{\frac{B_0(x,z)}{B_\rf}\frac{\refj{T}}{T_{0j}(x)}\} \int \mathcal{P}(\bar{k}_{2,m},z,\mu)
    \cdot\diag\{e^{-\hat{\mu}\refj{T}/T_{0j}(x)\, B_0(x,z)/B_\rf}\}\,d\hat{\mu}
  \right)
\end{multline*}
We leave the hats and use only normalized quantities in what follows in this
subsection. 
\begin{multline*}
  \tilde{\mathbf{\Phi}}(\bar{k}_{2,m},z)
  = \pi\sum_j \frac{e_j}{e}\frac{\refj{n}}{n_\rf}
  \diag\{\frac{B_0(x,z)}{B_\rf}\}\int \left[\mathcal{P}''(\bar{k}_{2,m},z)\right]^{-1}\cdot\mathcal{M}(\bar{k}_{2,m},z,\mu)
  \cdot\tilde{\mathbf{F}}(\bar{k}_{2,m},z,v_\|,\mu)\,dv_\|\,d\mu
\end{multline*}
For a more efficient implementation we can write
\begin{multline*}
  \tilde{\mathbf{\Phi}}(\bar{k}_{2,m},z) =\\
  \pi\sum_j \frac{e_j}{e}\frac{\refj{n}}{n_\rf}
  \diag\{\frac{B_0(x,z)}{B_\rf}\}\int \left[\mathcal{P}''(\bar{k}_{2,m},z)\right]^{-1}\cdot\mathcal{M}(\bar{k}_{2,m},z,\mu)
  \cdot\left[
    \int\tilde{\mathbf{F}}(\bar{k}_{2,m},z,v_\|,\mu)\,dv_\|
  \right]\,d\mu
\end{multline*}
Or with the $\mathcal{P}''$ matrix on the left hand side:
\begin{multline*}
  \mathcal{P}''(\bar{k}_{2,m},z)\cdot\tilde{\mathbf{\Phi}}(\bar{k}_{2,m},z) =\\
  \pi\sum_j \frac{e_j}{e}\frac{\refj{n}}{n_\rf}
  \diag\{\frac{B_0(x,z)}{B_\rf}\}\int \mathcal{M}(\bar{k}_{2,m},z,\mu)
  \cdot\left[
    \int\tilde{\mathbf{F}}(\bar{k}_{2,m},z,v_\|,\mu)\,dv_\|
  \right]\,d\mu
\end{multline*}


\subsection{Amp\`ere's law}
\label{sec:field2}
What we need for Amp\`ere's law is the parallel current density
$j_{\|1j}$. It is calculated according to
\begin{eqnarray*}
  j_{\|1j}(\mathbf{x}) &=& e_j\int v_\|f_1(\mathbf{x},v_\|,v_\bot,\theta)\,d^3v
  =e_j\frac{B_0}{m_j}\int v_\| f_1(\mathbf{x},v_\|,\mu,\theta)\,dv_\|\,d\mu\,d\theta\\
  &=&e_j\frac{B_0}{m_j}\int \delta(\mathbf{X}+\mathbf{r}-\mathbf{x}) v_\|
  T^*F_1(\mathbf{X},v_\|,\mu)\,d\mathbf{X}\,dv_\|\,d\mu\,d\theta\\
  &=&e_j\frac{B_0}{m_j}\int \delta(\mathbf{X}+\mathbf{r}-\mathbf{x}) v_\|
  \left(
    F_{j1}(\mathbf{X})
    -\frac{e_j}{T_{0j}}\tilde\Phi(\mathbf{X}+\mathbf{r}) F_{0j}
  \right)\,d\mathbf{X}\,dv_\|\,d\mu\,d\theta\\
  &=&e_j\frac{B_0}{m_j}\int \delta(\mathbf{X}+\mathbf{r}-\mathbf{x}) v_\|
  \left(
    F_{j1}(\mathbf{X})
    -\frac{e_j}{T_{0j}}\left(\Phi(\mathbf{X}+\mathbf{r})-\langle\Phi(\mathbf{X}+\mathbf{r})\rangle\right) F_{0j}
  \right)\,d\mathbf{X}\,dv_\|\,d\mu\,d\theta\\
  &=& e_j\frac{B_0}{m_j}\int \delta(\mathbf{X}+\mathbf{r}-\mathbf{x}) v_\|F_{j1}(\mathbf{X})\,d\mathbf{X}\,dv_\|\,d\mu\,d\theta\\
  &&-\frac{e_j}{T_{0j}}e_j\frac{B_0}{m_j}
  \int \delta(\mathbf{X}+\mathbf{r}-\mathbf{x}) v_\|\left(
    \Phi(\mathbf{X}+\mathbf{r})
    -\langle\Phi(\mathbf{X}+\mathbf{r})\rangle
  \right) F_{0j}\,d\mathbf{X}\,dv_\|\,d\mu\,d\theta\\
  &=& e_j\frac{B_0}{m_j}\int \delta(\mathbf{X}+\mathbf{r}-\mathbf{x}) v_\|F_{j1}(\mathbf{X})\,d\mathbf{X}\,dv_\|\,d\mu\,d\theta
\end{eqnarray*}
The second integral vanishes due to the $v_\|$ symmetry of
$F_{0j}$. At this point, we now use the modified distribution function
$g_j$ to replace
$F_{1j}(\mathbf{X})=g_j(\mathbf{X})-\frac{2e_jv_\|}{m_jcv_{Tj}^2}
F_{0j}\frac{1}{2\pi}\int A_{\|1}(\mathbf{X}+\mathbf{r})\,d\theta$. We
arrive at
\begin{eqnarray*}
  j_{\|1j}(\mathbf{x}) 
  &=& \frac{1}{2\pi}\int e_j\frac{B_0}{m_j}\int \delta(\mathbf{X}+\mathbf{r}''-\mathbf{x})
  v_\|\left[
    g_j(\mathbf{X})-\frac{2e_jv_\|}{m_jcv_{Tj}^2}
    F_{0j}\frac{1}{2\pi}\int A_{\|1}(\mathbf{X}+\mathbf{r}')\,d\theta'
  \right]\,d\mathbf{X}\,dv_\|\,d\mu\,d\theta\,d\theta''\\
  &=& \frac{1}{2\pi}\int e_j\frac{B_0}{m_j}\int \delta(\mathbf{X}+\mathbf{r}''-\mathbf{x})
  v_\|g_j(\mathbf{X})\,d\mathbf{X}\,dv_\|\,d\mu\,d\theta\,d\theta''\\
  &&-\frac{1}{2\pi}\int e_j\frac{B_0}{m_j}\int \delta(\mathbf{X}+\mathbf{r}''-\mathbf{x})
  v_\|\frac{2e_jv_\|}{m_jcv_{Tj}^2}
  F_{0j}\frac{1}{2\pi}\int A_{\|1}(\mathbf{X}+\mathbf{r}')\,d\theta'\,d\mathbf{X}\,dv_\|\,d\mu\,d\theta\,d\theta''\\
  &=& \frac{1}{2\pi}\int\limits_0^{2\pi} e_j
  \frac{B_0}{m_j}\int v_\|g_j(\mathbf{x}-\mathbf{r}'',v_\|,\mu)\,dv_\|\,d\mu\,d\theta\,d\theta''\\
  &&-\frac{2e_j^2}{m_jcv_{Tj}^2}\frac{1}{2\pi}\int\limits_0^{2\pi} \frac{B_0}{m_j}\int \delta(\mathbf{X}+\mathbf{r}''-\mathbf{x})
  v_\|^2 F_{0j}\frac{1}{2\pi}\int\limits_0^{2\pi} 
  A_{\|1}(\mathbf{X}+\mathbf{r}')\,d\theta'\,d\mathbf{X}\,dv_\|\,d\mu\,d\theta\,d\theta''\\
  &=&  2\pi e_j\frac{B_0}{m_j}\int v_\|
  \frac{1}{2\pi}\int\limits_0^{2\pi}g_j(\mathbf{x}-\mathbf{r}'',v_\|,\mu)\,d\theta''\,dv_\|\,d\mu\\
  &&-\frac{2e_j^2}{m_jcv_{Tj}^2}n_{0j} \frac{B_0}{m_j}\int e^{-\mu B_0/T_{0j}}
  \frac{1}{2\pi}\int\limits_0^{2\pi} \delta(\mathbf{X}+\mathbf{r}''-\mathbf{x})
  \frac{1}{2\pi}\int\limits_0^{2\pi} A_{\|1}(\mathbf{X}+\mathbf{r}')\,d\theta'\,d\mathbf{X}\,d\mu\,d\theta''\\
  &=&  2\pi e_j\frac{B_0}{m_j}\int v_\| \mathcal{G}[g_j](\mathbf{x},v_\|,\mu)\,dv_\|\,d\mu
  -\frac{2e_j^2}{m_jcv_{Tj}^2}n_{0j} \frac{B_0}{m_j}\int e^{-\mu B_0/T_{0j}}
  \mathfrak{G}[A_{\|1}](\mathbf{x},\mu)\,d\mu\\
  &=&  2\pi e_j\frac{B_0}{m_j}\int v_\| \mathcal{G}[g_j](\mathbf{x},v_\|,\mu)\,dv_\|\,d\mu
  -\frac{e_j^2}{m_jc} G[A_{\|1}](\mathbf{x})
\end{eqnarray*}
This is put into the field equation
\begin{eqnarray*}
  -\nabla_\bot^2 A_{\|1}(\mathbf{x}) 
  = \frac{4\pi}{c}\sum_j \left[
    2\pi e_j\frac{B_0}{m_j}\int v_\| \mathcal{G}[g_j](\mathbf{x},v_\|,\mu)\,dv_\|\,d\mu
    -\frac{e_j^2}{m_jc} G[A_{\|1}](\mathbf{x})
  \right]
\end{eqnarray*}
Reordering
\begin{eqnarray*}
  -\nabla_\bot^2 A_{\|1}(\mathbf{x}) 
  +\frac{4\pi}{c}\sum_j\frac{e_j^2}{m_jc} G[A_{\|1}](\mathbf{x})
  = \frac{4\pi}{c}\sum_j 2\pi e_j\frac{B_0}{m_j}\int v_\| \mathcal{G}[g_j](\mathbf{x},v_\|,\mu)\,dv_\|\,d\mu
\end{eqnarray*}
Representing the electromagnetic potential in terms of Fourier and
spline basis functions, we can write
\begin{displaymath}
  A_{\|1}(\mathbf{x})=\sum_m A_m(\bar x^1,z)\,e^{i\bar{k}_{2,m}\bar x^2}
  =\sum_m \sum_n \tilde{A}_{nm}(z)\Lambda_n(\bar x^1)\,e^{i\bar{k}_{2,m}\bar x^2}
\end{displaymath}
Now how to calculate the Laplacian? In general coordinates we have for
the $-\nabla_\bot^2$ part
\begin{eqnarray*}
  -\nabla_\bot^2 A_{\|1}(\mathbf{x}) &=& -\left(
    g^{11}\partial_1^2+2g^{12}\partial_1\partial_2+g^{22}\partial_2^2
  \right) \sum_m \sum_n \tilde{A}_{nm}(z)\Lambda_n(\bar x^1)\,e^{i\bar{k}_{2,m}\bar x^2}\\
  &=&-\sum_m \sum_n \tilde{A}_{nm}(z) \left(
    g^{11}\frac{\partial^2\Lambda_n(\bar x^1)}{\partial(\bar x^1)^2}\,
    +2i\bar{k}_{2,m} g^{12}\frac{\partial\Lambda_n(\bar x^1)}{\partial\bar x^1}
    -\Lambda_n(\bar x^1) g^{22}\bar{k}_{2,m}^2
  \right) \,e^{i\bar{k}_{2,m}\bar x^2}
\end{eqnarray*}
Put this result into the field equation, we come to
\begin{multline*}
  -\sum_m \sum_n \tilde{A}_{nm}(z) \left(
    g^{11}\frac{\partial^2\Lambda_n(\bar x^1)}{\partial(\bar x^1)^2}\,
    +2i\bar{k}_{2,m} g^{12}\frac{\partial\Lambda_n(\bar x^1)}{\partial\bar x^1}
    -\Lambda_n(\bar x^1) g^{22}\bar{k}_{2,m}^2
  \right) \,e^{i\bar{k}_{2,m}\bar x^2}\\
  +\frac{4\pi}{c}\sum_j\frac{e_j^2}{m_jc} G[A_{\|1}](\mathbf{x})\\
  = \frac{4\pi}{c}\sum_j 2\pi e_j\frac{B_0}{m_j}\int v_\| \mathcal{G}[g_j](\mathbf{x},v_\|,\mu)\,dv_\|\,d\mu
\end{multline*}
Using now the expressions for the gyro-averaging and the gyro-mapping,
we come to
\begin{multline*}
  -\sum_m \sum_n \tilde{A}_{nm}(z) \left(
    g^{11}\frac{\partial^2\Lambda_n(\bar x^1)}{\partial(\bar x^1)^2}\,
    +2i\bar{k}_{2,m} g^{12}\frac{\partial\Lambda_n(\bar x^1)}{\partial\bar x^1}
    -\Lambda_n(\bar x^1) g^{22}\bar{k}_{2,m}^2
  \right) \,e^{i\bar{k}_{2,m}\bar x^2}\\
  +\frac{4\pi}{c}\sum_j\frac{e_j^2}{m_jc} \sum_m e^{i\bar k_{2,m}\bar x^2} 
  \left[\frac{2n_{0j}}{v_{Tj}^2}\frac{B_0}{m_j}\int \mathcal{P}(\bar{k}_{2,m},z,\mu)
    \,e^{-\mu B_0/T_{0j}}\,d\mu\right] \cdot\tilde{\mathbf{A}}_m\\
  = \frac{4\pi}{c}\sum_j 2\pi e_j \frac{B_0}{m_j}\int v_\| 
  \sum_m e^{i\bar k_{2,m}\bar x^2} \mathcal{M}(\bar{k}_{2,m},z,\mu)\cdot\tilde{\mathbf{g}}_m(z,v_\|,\mu)\,dv_\|\,d\mu
\end{multline*}
Simplifying and reordering leads to
\begin{multline*}
  -\sum_m e^{i\bar{k}_{2,m}\bar x^2} \sum_n \tilde{A}_{nm}(z) \left(
    g^{11}\frac{\partial^2\Lambda_n(\bar x^1)}{\partial(\bar x^1)^2}\,
    +2i\bar{k}_{2,m} g^{12}\frac{\partial\Lambda_n(\bar x^1)}{\partial\bar x^1}
    -\Lambda_n(\bar x^1) g^{22}\bar{k}_{2,m}^2
  \right) \\
  +\sum_m e^{i\bar k_{2,m}\bar x^2} \frac{4\pi}{c}\sum_j\frac{e_j^2}{m_jc} 
  \left[\frac{2n_{0j}}{v_{Tj}^2}\frac{B_0}{m_j}\int \mathcal{P}(\bar{k}_{2,m},z,\mu)
    \,e^{-\mu B_0/T_{0j}}\,d\mu\right] \cdot\tilde{\mathbf{A}}_m\\
  = \sum_m e^{i\bar k_{2,m}\bar x^2}\frac{4\pi}{c}\sum_j 2\pi e_j \frac{B_0}{m_j}\int v_\| 
   \mathcal{M}(\bar{k}_{2,m},z,\mu)\cdot\tilde{\mathbf{g}}_m(z,v_\|,\mu)\,dv_\|\,d\mu
\end{multline*}
Two Fouriersums are only equivalent if all Fourier coefficients are
equal. Hence, we come to
\begin{multline*}
  -\sum_n \tilde{A}_{nm}(z) \left(
    g^{11}\frac{\partial^2\Lambda_n(\bar x^1)}{\partial(\bar x^1)^2}\,
    +2i\bar{k}_{2,m} g^{12}\frac{\partial\Lambda_n(\bar x^1)}{\partial\bar x^1}
    -\Lambda_n(\bar x^1) g^{22}\bar{k}_{2,m}^2
  \right) \\
  + \frac{4\pi}{c}\sum_j\frac{e_j^2}{m_jc} 
  \left[\frac{2n_{0j}}{v_{Tj}^2}\frac{B_0}{m_j}\int \mathcal{P}(\bar{k}_{2,m},z,\mu)
    \,e^{-\mu B_0/T_{0j}}\,d\mu\right] \cdot\tilde{\mathbf{A}}_m\\
  = \frac{4\pi}{c}\sum_j 2\pi e_j \frac{B_0}{m_j}\int v_\| 
   \mathcal{M}(\bar{k}_{2,m},z,\mu)\cdot\tilde{\mathbf{g}}_m(z,v_\|,\mu)\,dv_\|\,d\mu
\end{multline*}
Furthermore we can write the Laplacian as a matrix $\mathcal{D}(\bar
k_{2,m},z)$ which is defined as
\begin{displaymath}
  \mathcal{D}_{rn}(\bar k_{2,m},z) = 
  g^{11}(z)\frac{\partial^2\Lambda_n(\bar x^1)}{\partial(\bar x^1)^2}\Bigg|_{\bar x^1_r}
  +2i\bar{k}_{2,m} g^{12}(z)\frac{\partial\Lambda_n(\bar x^1)}{\partial\bar x^1}\Bigg|_{\bar x^1_r}
  -\Lambda_n(\bar x^1_r) g^{22}(z)\bar{k}_{2,m}^2.
\end{displaymath}
With this we can write the field equation 
\begin{multline*}
  -\mathcal{D}(\bar k_{2,m},z)\cdot \tilde{\mathbf{A}}_m(z)
  + \frac{4\pi}{c}\sum_j\frac{e_j^2}{m_jc} 
  \left[\frac{2n_{0j}}{v_{Tj}^2}\frac{B_0}{m_j}\int \mathcal{P}(\bar{k}_{2,m},z,\mu)
    \,e^{-\mu B_0/T_{0j}}\,d\mu\right] \cdot\tilde{\mathbf{A}}_m\\
  = \frac{4\pi}{c}\sum_j 2\pi e_j \frac{B_0}{m_j}\int v_\| 
   \mathcal{M}(\bar{k}_{2,m},z,\mu)\cdot\tilde{\mathbf{g}}_m(z,v_\|,\mu)\,dv_\|\,d\mu
\end{multline*}
And at last we combine the left hand side to have
\begin{multline*}
  \left[
    -\mathcal{D}(\bar k_{2,m},z)
    + \frac{4\pi}{c}\sum_j\frac{e_j^2}{m_jc} \frac{2n_{0j}}{v_{Tj}^2}\frac{B_0}{m_j}\int \mathcal{P}(\bar{k}_{2,m},z,\mu)
    \,e^{-\mu B_0/T_{0j}}\,d\mu
  \right] \cdot\tilde{\mathbf{A}}_m(z)\\
  = \frac{4\pi}{c}\sum_j 2\pi e_j \frac{B_0}{m_j}\int v_\| 
   \mathcal{M}(\bar{k}_{2,m},z,\mu)\cdot\tilde{\mathbf{g}}_m(z,v_\|,\mu)\,dv_\|\,d\mu
\end{multline*}


\subsubsection{Normalization}
\label{sec:norm_amp}
This second field equation has also to be normalized. We use the
normalization, defined in Section~\ref{sec:normalization}.
\begin{multline*}
  \frac{\rho_\rf}{L_\rf}B_\rf\rho_\rf\left[
    -\frac{1}{\rho_\rf^2}\hat{\mathcal{D}}(\bar k_{2,m},z)
    +\frac{4\pi}{c}\sum_j\frac{e_j^2}{m_jc} \frac{2n_{0j}}{v_{Tj}^2}
    \frac{\refj{T}}{B_\rf}\frac{B_0}{m_j}\int \mathcal{P}(\bar{k}_{2,m},z,\mu)
    \,e^{-\hat{\mu} B_0/B_\rf\,\refj{T}/T_{0j}}\,d\hat{\mu}
  \right] \cdot\hat{\tilde{\mathbf{A}}}_m(z)\\
  = \frac{\rho_\rf}{L_\rf}\frac{4\pi}{c}
  \sum_j 2\pi e_j \frac{\refj{n}}{v_{0j}^3}\frac{B_0}{m_j}v_{0j}^2\frac{\refj{T}}{B_\rf} \int \hat{v}_\| 
   \mathcal{M}(\bar{k}_{2,m},z,\mu)\cdot\hat{\tilde{\mathbf{g}}}_m(z,v_\|,\mu)\,d\hat{v}_\|\,d\hat{\mu}
\end{multline*}
Simplifications lead to
\begin{multline*}
  \left[
    -\hat{\mathcal{D}}(\bar k_{2,m},z)
    +\frac{\beta}{2}\sum_j\frac{e_j^2}{e^2}\frac{n_{0j}}{n_\mathrm{ref}}\frac{m_\rf}{m_j} 
    \frac{\refj{T}}{T_{0j}}\frac{B_0}{B_\rf}\int \mathcal{P}(\bar{k}_{2,m},z,\mu)
    \,e^{-\hat{\mu} B_0/B_\rf\,\refj{T}/T_{0j}}\,d\hat{\mu}
  \right] \cdot\hat{\tilde{\mathbf{A}}}_m(z)\\
  = \frac{\beta}{2}\sum_j \frac{e_j}{e} \frac{v_{0j}}{c_\rf}\pi\frac{B_0}{B_\rf}\int \hat{v}_\| 
   \mathcal{M}(\bar{k}_{2,m},z,\mu)\cdot\hat{\tilde{\mathbf{g}}}_m(z,v_\|,\mu)\,d\hat{v}_\|\,d\hat{\mu}
\end{multline*}
with $\beta = \frac{8\pi n_\rf T_\rf}{B_\rf^2}$.

\subsubsection{Efficient implementation}
The operator on the left hand side of the last field equation can be
calculated in advance, to give a combined matrix operator
$\mathcal{C}(\bar{k}_{2,m},z)$ which not any more species dependent,
as it is a sum over all species. The field equation is then
\begin{displaymath}
  \mathcal{C}(\bar{k}_{2,m},z) \cdot\hat{\tilde{\mathbf{A}}}_m(z)
  = \frac{\beta}{2}\sum_j \frac{e_j}{e} \frac{v_{0j}}{c_\rf}\pi\frac{B_0}{B_\mathrm{ref}}\int \hat{v}_\| 
  \mathcal{M}(\bar{k}_{2,m},z,\mu)\cdot\hat{\tilde{\mathbf{g}}}_m(z,v_\|,\mu)\,d\hat{v}_\|\,d\hat{\mu}
\end{displaymath}
This combined matrix operator can be brought to the right hand side by
multiplying with the inverse from the left. As the operator is species
and $\mu$ independent, we can combine it further with the matrix
operator on the right hand side $\mathcal{M}$. 
\begin{displaymath}
  \hat{\tilde{\mathbf{A}}}_m(z)
  = \frac{\beta}{2}\sum_j \frac{e_j}{e} \frac{v_{0j}}{c_\rf}\pi\frac{B_0}{B_\mathrm{ref}}\int \hat{v}_\| 
  \mathcal{C}(\bar{k}_{2,m},z)^{-1}\cdot\mathcal{M}(\bar{k}_{2,m},z,\mu)\cdot\hat{\tilde{\mathbf{g}}}_m(z,v_\|,\mu)\,d\hat{v}_\|\,d\hat{\mu}
\end{displaymath}
The then whole matrix operator is independent of $v_\|$, so we can
first do the integration over $v_\|$, then let the matrix operate on
the result and then do the $\mu$ integration.
\begin{displaymath}
  \hat{\tilde{\mathbf{A}}}_m(z)
  = \frac{\beta}{2}\sum_j \frac{e_j}{e} \frac{v_{0j}}{c_\rf}\frac{B_0}{B_\mathrm{ref}}\int
  \mathcal{C}(\bar{k}_{2,m},z)^{-1}\cdot\mathcal{M}(\bar{k}_{2,m},z,\mu)\cdot\left[
    \pi\int\hat{v}_\|\hat{\tilde{\mathbf{g}}}_m(z,v_\|,\mu)\,d\hat{v}_\|\right]\,d\hat{\mu}
\end{displaymath}


\subsection{Adiabatic electrons}
\label{sec:adiabatic_electrons}
The two field equations derived in the to preceeding sections are
valid for two species simulations. If we want to treat the electrons
adiabatically, we have to replace the expression for the perturbed
electron density by the following unnormalized expression:
\begin{displaymath}
  \frac{n_{1\mathrm{e}}}{n_{0\mathrm{e}}(x)} 
  = \frac{e\Phi}{T_{0\mathrm{e}}(x)}-\frac{e}{T_{0\mathrm{e}}(x)}\langle\Phi\rangle
\end{displaymath}
where $\langle\cdot\rangle$ is the flux surface average of the argument.
Using again the quasineutrality condition
\begin{eqnarray*}
  0&=&\sum_j e_jn_{1j}
  =-en_{1\mathrm{e}}+\sum_{j=\mbox{ions}}e_jn_{1j}\\
  &=&-\frac{e^2}{T_{0\mathrm{e}}(x)}n_{0\mathrm{e}}(x)\left[
    \Phi
    -\langle\Phi\rangle
  \right]
  +\sum_{j}e_j\left[
    \frac{B_0}{m_j}\int \mathcal{G}[F_{j1}](\mathbf{x},v_\|,\mu)\,dv_\|\,d\mu\,d\theta
    -\frac{e_j}{T_{0j}}\left(
      \Phi(\mathbf{x}) n_{0j}
      -G[\Phi](\mathbf{x})
    \right) 
  \right]
\end{eqnarray*}
We again reorder to have all field dependent parts on the left hand
side and the parts dependent on the distribution function on the right
hand side.
\begin{eqnarray*}
  \frac{e^2}{T_{0\mathrm{e}}}n_{0\mathrm{e}}\left[
    \Phi
    -\langle\Phi\rangle
  \right] 
  +\sum_{j}\frac{e_j^2}{T_{0j}}\left(
    \Phi(\mathbf{x}) n_{0j}
    -G[\Phi](\mathbf{x})
  \right) 
  &=& \sum_{j}e_j\frac{B_0}{m_j}\int \mathcal{G}[F_{j1}](\mathbf{x},v_\|,\mu)\,dv_\|\,d\mu\,d\theta
\end{eqnarray*}
The same procedure as for the nonadiabatic case is followed and in the
end we come to
\begin{multline*}
  \frac{e^2}{T_{0\mathrm{e}}}n_{0\mathrm{e}}\mathcal{L}\cdot\left[
    \tilde{\mathbf{\Phi}}(k_m,z)
    -\tilde{\mathbf{\langle\Phi\rangle}}
  \right] \\
  +\sum_{j}\frac{e_j^2}{T_{0j}}n_{0j}\left(
    \mathcal{L} 
    -\frac{B_0}{T_{0j}}\int \mathcal{P}(\bar{k}_{2,m},z,\mu)
    \,e^{-\mu B_0/T_{0j}}\,d\mu
  \right)\cdot\tilde{\mathbf{\Phi}}(k_m,z) \\
  = \sum_j e_j 2\pi\frac{B_0}{m_j}\int 
  \mathcal{M}(\bar{k}_{2,m},z,\mu)\cdot\tilde{\mathbf{F}}_m(z,v_\|,\mu)\,dv_\|\,d\mu
\end{multline*}
Normalization yields to
\begin{multline*}
  \frac{T_\rf}{e}e^2\frac{n_{0\mathrm{e}}}{T_{0\mathrm{e}}} \mathcal{L}\cdot\left[
    \hat{\tilde{\mathbf{\Phi}}}(k_m,z)-\hat{\tilde{\mathbf{\langle\Phi\rangle}}}\right]\\
  +\frac{T_\rf}{e} e^2 \sum_{j}\frac{e_j^2}{e^2}\frac{n_{0j}}{T_{0j}}\left(
    \mathcal{L}-\frac{B_0}{B_\rf}\frac{\refj{T}}{T_{0j}}\int \mathcal{P}(\bar{k}_{2,m},z,\mu)
    \,e^{-\hat{\mu} B_0/B_\rf \refj{T}/T_{0j}}\,d\hat{\mu}
  \right)\cdot\hat{\tilde{\mathbf{\Phi}}}(k_m,z) \\
  = \sum_j e_j 2\pi\frac{B_0}{m_j}\frac{\refj{n}}{v_{0j}^2}\frac{\refj{T}}{B_\rf}\int 
  \mathcal{M}(\bar{k}_{2,m},z,\mu)\cdot\hat{\tilde{\mathbf{F}}}_m(z,v_\|,\mu)\,d\hat{v}_\|\,d\hat{\mu}
\end{multline*} 
which can be simplified to
\begin{multline}
  \label{eq:qnfull}
  \frac{n_{0\mathrm{e}}}{n_\rf} \frac{T_\rf}{T_{0\mathrm{e}}} \mathcal{L}\cdot\left[
    \hat{\tilde{\mathbf{\Phi}}}(k_m,z)-\hat{\tilde{\mathbf{\langle\Phi\rangle}}}\right]
  +\mathcal{P}_{\mathrm{ions}}''\cdot\hat{\tilde{\mathbf{\Phi}}}(k_m,z) \\
  = \sum_j \frac{e_j}{e} \frac{\refj{n}}{n_\rf} \pi\hat{B}\int 
  \mathcal{M}(\bar{k}_{2,m},z,\mu)\cdot\hat{\tilde{\mathbf{F}}}_m(z,v_\|,\mu)\,d\hat{v}_\|\,d\hat{\mu}
\end{multline} 
using a similar abbreviation as before
\begin{multline*}
    \mathcal{P}_{\mathrm{ions}}'' = \sum_{j=\mathrm{ions}}\frac{e_j^2}{e^2}
  \diag\{\frac{n_{0j}(x)}{n_\rf}\frac{T_\rf}{T_{0j}(x)}\}\cdot\\
  \left(\mathcal{L}
    - \int \diag\{\frac{B_0(x,z)}{B_\rf}\frac{\refj{T}}{T_{0j}(x)} 
    e^{-\hat{\mu}\refj{T}/T_{0j}(x)\, B_0(x,z)/B_\rf}\} \cdot 
    \mathcal{P}(\bar{k}_{2,m},z,\mu)\,d\hat{\mu}\right).
\end{multline*}
To get an expression for the flux surface average of the electrostatic
potential, we calculate the flux surface average of the whole
equation which gives
\begin{multline*}
\left\langle\mathcal{P}_{\mathrm{ions}}''\cdot\hat{\tilde{\mathbf{\Phi}}}(k_m,z)\right\rangle
  = \left\langle\sum_j \frac{e_j}{e} \frac{\refj{n}}{n_\rf} \pi\hat{B}\int 
  \mathcal{M}(\bar{k}_{2,m},z,\mu)\cdot\hat{\tilde{\mathbf{F}}}_m(z,v_\|,\mu)\,d\hat{v}_\|\,d\hat{\mu}\right\rangle
\end{multline*} 
As this result does not yet lead to an expression for
$\langle\Phi\rangle$, we now make the assumption 
\begin{displaymath}
  \left\langle\mathcal{P}_{\mathrm{ions}}''\cdot\tilde{\mathbf{\Phi}}(k_m,z)
  \right\rangle=\left\langle\mathcal{P}_{\mathrm{ions}}''\right\rangle\cdot\left\langle\tilde{\mathbf{\Phi}}(k_m,z)\right\rangle
\end{displaymath}
This leads then to 
\begin{displaymath}
\left\langle\mathcal{P}_{\mathrm{ions}}''\right\rangle\cdot\left\langle\hat{\tilde{\mathbf{\Phi}}}(k_m,z)\right\rangle
  = \left\langle\sum_j \frac{e_j}{e} \frac{\refj{n}}{n_\rf} \pi\hat{B}\int 
  \mathcal{M}(\bar{k}_{2,m},z,\mu)\cdot\hat{\tilde{\mathbf{F}}}_m(z,v_\|,\mu)\,d\hat{v}_\|\,d\hat{\mu}\right\rangle
 \end{displaymath}
which can be solved for $\langle\Phi\rangle$.
Having $\langle\Phi\rangle$, we can now rewrite the quasineutrality equation eq.~(\ref{eq:qnfull}) and get
\begin{eqnarray*}
  \left[\frac{n_{0\mathrm{e}}}{n_\rf} \frac{T_\rf}{T_{0\mathrm{e}}} \mathcal{L}
  +\mathcal{P}_{\mathrm{ions}}''\right]\cdot \hat{\tilde{\mathbf{\Phi}}}(k_m,z)    
& = & \sum_j \frac{e_j}{e} \frac{\refj{n}}{n_\rf} \pi\hat{B}\int 
  \mathcal{M}(\bar{k}_{2,m},z,\mu)\cdot\hat{\tilde{\mathbf{F}}}_m(z,v_\|,\mu)\,d\hat{v}_\|\,d\hat{\mu} \\
& &  +\frac{n_{0\mathrm{e}}}{n_\rf} \frac{T_\rf}{T_{0\mathrm{e}}} \mathcal{L}\cdot
   \hat{\tilde{\mathbf{\langle\Phi\rangle}}}
\end{eqnarray*} 
which can be solved either by iteration or inversion.

\section{Base functions in radial direction}
\label{sec:basefunc}
Until now, the base functions, that we used in $x$ direction where
totally general. We just assumed that one can represent all functions
in series of these base functions. One possibility of base functions
are splines, but for splines, the coefficients and the grid values of
the function are different, so that one needs to solve a linear system
of equations to come from the grid values to the coefficients. This
system is expensive to solve on a parallelized radial grid. So we are
looking for other set of base functions, where the point values and
the coefficients of the base function representation are
equal. 

Instead of extending the support of the polynomial (as done for
the splines), one could also keep the support to only two grid
cells. If one uses only polynoms of degree one (linear interpolation),
we come to the hat function, but in principle one can use all degrees,
but then one has to impose some more conditions on the base functions
to determine all coefficients of the polynoms.
The function is represented as
\begin{displaymath}
  f(x)=\sum_n\left[ f_n H_n(x) + f_n' G_n(x)\right]
\end{displaymath}
where the values of the function at the grid points together with the
values of the derivatives of the function at the grid points are
known. Then we have the conditions for the two polynomial functions
$H_n(x)$ and $G_n(x)$:
\begin{eqnarray*}
  H_n(x_j)=\delta_{nj}\quad\mbox{and}\quad G_n(x_j)=0\\
  H'_n(x_j)=0\quad\mbox{and}\quad G'_n(x_j)=\delta_{nj}
\end{eqnarray*}

These four conditions lead to the possibility to use cubic
polynoms. So making the ansatz
\begin{displaymath}
  H_n(x)=a_3x^3+a_2x^2+a_1x+a_0\qquad G_n(x)=b_3x^3+b_2x^2+b_1x+b_0
\end{displaymath}
we come to the following linear system of equations for the two different
sides a three point interval $[x_{j-1},x_{j+1}]$.
\begin{align*}
  x_{j\pm1}^3 &a_3 &+x_{j\pm1}^2&a_2 &+x_{j\pm1}&a_1&+a_0 &=&0\\
  x_{j}^3   &a_3 &+x_{j}^2&a_2  &+x_{j}&a_1&+a_0 &=&1\\
  3x_{j\pm1}^2&a_3 &+2x_{j\pm1}&a_2&+&a_1 &&=& 0\\
  3x_j^2&a_3    &+2x_j&a_2&+&a_1& &=& 0
\end{align*}
We now normalize the interval to the interval $[-\Delta x_n,0]$ for the left
half and $[0,\Delta x_n]$ for the right half. The system of equations then
become
\begin{align*}
  \pm\Delta x_n^3 &a_3 &+\Delta x_n^2&a_2 &\pm\Delta x_n&a_1&+a_0 &=&0\\
   & &&  &&&a_0 &=&1\\
  3\Delta x_n^2&a_3 &\pm2\Delta x_n&a_2&+&a_1 &&=& 0\\
  &    &&&&a_1& &=& 0
\end{align*}


These four equations for four unknows do have a solution for the coefficients
of the polynomial function.
\begin{eqnarray*}
  a_3 = -\frac{2}{(x_j-x_{j\pm1})^3}
  \qquad
  a_2 = \frac{3 (x_j+x_{j\pm1})}{(x_j-x_{j\pm1})^3}
  \qquad
  a_1 = \frac{6 x_j x_{j\pm1}}{(x_{j\pm1}-x_j)^3}
  \qquad
  a_0 = \frac{x_{j\pm1}^2 (x_{j\pm1}-3x_j)}{(x_{j\pm1}-x_j)^3}
\end{eqnarray*}
One therefore gets a cubic polynom for the left and one for the right half of
the above defined interval.
For an equally spaced $x$-grid, we can write for the left half $[x_{j-1},x_j]$
\begin{eqnarray*}
  a_3 = -\frac{2}{\Delta x^3}
  \qquad
  a_2 = \frac{3 (x_j+x_{j-1})}{\Delta x^3}
  \qquad
  a_1 = -\frac{6 x_j x_{j-1}}{\Delta x^3}
  \qquad
  a_0 = -\frac{x_{j-1}^2 (x_{j-1}-3x_j)}{\Delta x^3}
\end{eqnarray*}


If one want to use higher degree polynoms, we have to use
also higher derivatives. So we take into account the second derivative
and find
\begin{displaymath}
  f(x)=\sum_n\left[ f_n H_n(x) + f_n' G_n(x) + f_n'' Q_n\right]
\end{displaymath}
This lead to the conditions
\begin{eqnarray*}
  H_n(x_j)=\delta_{nj}\qquad G_n(x_j)=0\qquad Q_n(x_j)=0\\
  H'_n(x_j)=0\qquad G'_n(x_j)=\delta_{nj}\qquad Q'_n(x_j)=0\\
  H''_n(x_j)=0\qquad G''_n(x_j)=0\qquad Q''_n(x_j)=\delta_{nj}
\end{eqnarray*}



\section{Moments}
The definition of the moments is (taken for example from my fast
particle paper) the following but now we do not use the Bessel
functions but try to write the moments expressions in terms of
averaging $\langle\cdot\rangle$ operators. Again we use $r=2q$ as we
only use even moments of $v_\bot$. $j$ is the species index.
\begin{eqnarray*}
  M_{kr}(\mathbf{x}) &=& \int v_\|^kv_\bot^{2q} f_1(\mathbf{x})\,d^3v
  =\frac{2^qB_0^{q+1}}{m_j^{q+1}}\int v_\|^k \mu^q f_1(\mathbf{x})\,dv_\|\,d\mu\,d\theta\\
  &=&\frac{2^qB_0^{q+1}}{m_j^{q+1}}
  \int \delta(\mathbf{X}+\mathbf{r}-\mathbf{x}) v_\|^k \mu^q T^*F_1(\mathbf{X})\,d\mathbf{X}\,dv_\|\,d\mu\,d\theta\\
  &=&\frac{2^qB_0^{q+1}}{m_j^{q+1}}
  \int v_\|^k \mu^q
  T^*F_1(\mathbf{x}-\mathbf{r})\,dv_\|\,d\mu\,d\theta
\end{eqnarray*}
Now we want to do the gyroaveraging which gives the following
expression
\begin{eqnarray*}
  M_{kr}(\mathbf{x}) &=&\frac{2^qB_0^{q+1}}{m_j^{q+1}}
  \int v_\|^k \mu^q \langle T^*F_1(\mathbf{x}-\mathbf{r})\rangle\,dv_\|\,d\mu\,d\theta\\
  &=&2\pi\frac{2^qB_0^{q+1}}{m_j^{q+1}}
  \int v_\|^k \mu^q \Bigg[
    \left\langle
      F_1(\mathbf{x}-\mathbf{r})
    \right\rangle
    +\left\langle
      \left(
        \frac{e_j}{m_jc}\frac{\partial F_0}{\partial v_\|}
        -\frac{e_jv_\|}{cB_0}\frac{\partial F_0}{\partial \mu}
      \right)\left(
        A_{\|1}(\mathbf{x})
        -\bar{A}_{\|1}(\mathbf{x}-\mathbf{r})
      \right)
    \right\rangle\\
    &&+\left\langle
      \frac{e_j}{B_0}\frac{\partial F_0}{\partial\mu}\left(
        \Phi_1(\mathbf{x})
        -\bar{\Phi}_1(\mathbf{x}-\mathbf{r})
      \right)
    \right\rangle
  \Bigg]\,dv_\|\,d\mu\\
  &=&2\pi\frac{2^qB_0^{q+1}}{m_j^{q+1}}
  \int v_\|^k \mu^q \Bigg[
    \left\langle
      F_1(\mathbf{x}-\mathbf{r})
    \right\rangle
    +\left(
      \frac{e_j}{m_jc}\frac{\partial F_0}{\partial v_\|}
      -\frac{e_jv_\|}{cB_0}\frac{\partial F_0}{\partial \mu}
    \right)\left\langle
      A_{\|1}(\mathbf{x})
      -\bar{A}_{\|1}(\mathbf{x}-\mathbf{r})
    \right\rangle\\
    &&+\frac{e_j}{B_0}\frac{\partial F_0}{\partial\mu}\left\langle
      \left(
        \Phi_1(\mathbf{x})
        -\bar{\Phi}_1(\mathbf{x}-\mathbf{r})
      \right)
    \right\rangle
  \Bigg]\,dv_\|\,d\mu
\end{eqnarray*}
The last expression can now be written in terms of the operators
defined earlier.
\begin{eqnarray*}
  M_{kr}(x_1,x_2,x_3) &=&2\pi\frac{2^qB_0^{q+1}}{m_j^{q+1}}
  \int v_\|^k \mu^q \Bigg[
    \sum_m e^{ik_mx_2}\sum_nF_{n,m}(x_3)\mathcal{M}_{n,m}(\rho)\\
    &&+\left(
      \frac{e_j}{m_jc}\frac{\partial F_0}{\partial v_\|}
      -\frac{e_jv_\|}{cB_0}\frac{\partial F_0}{\partial \mu}
    \right)
    \sum_m e^{ik_mx_2}\sum_n A_{\|,n,m}(x_3)\left(
      \Lambda_n(x_1)-\mathcal{P}_{n,m}(\rho)
    \right)\\
    &&+\frac{e_j}{B_0}\frac{\partial F_0}{\partial\mu}
    \sum_me^{ik_mx_2}\sum_n\Phi_{n,m}(x_3)\left(
      \Lambda_n(x_1)-\mathcal{P}_{n,m}(\rho)
    \right)
  \Bigg]\,dv_\|\,d\mu\\
  &=&\sum_m e^{ik_mx_2} \,2\pi\frac{2^qB_0^{q+1}}{m_j^{q+1}}
  \int v_\|^k \mu^q \Bigg[
    \sum_nF_{n,m}(x_3)\mathcal{M}_{n,m}(\rho)\\
    &&+\left(
      \frac{e_j}{m_jc}\frac{\partial F_0}{\partial v_\|}
      -\frac{e_jv_\|}{cB_0}\frac{\partial F_0}{\partial \mu}
    \right)
    \sum_n A_{\|,n,m}(x_3)\left(
      \Lambda_n(x_1)-\mathcal{P}_{n,m}(\rho)
    \right)\\
    &&+\frac{e_j}{B_0}\frac{\partial F_0}{\partial\mu}
    \sum_n\Phi_{n,m}(x_3)\left(
      \Lambda_n(x_1)-\mathcal{P}_{n,m}(\rho)
    \right)
  \Bigg]\,dv_\|\,d\mu\\
\end{eqnarray*}
Fourier transforming the left hand side also, we get an expression for
each Fourier mode of the flux
\begin{eqnarray*}
  M_{kr}(x_1,k_m,x_3) &=& 2\pi\frac{2^qB_0^{q+1}}{m_j^{q+1}}
  \int v_\|^k \mu^q \Bigg[
    \sum_nF_{n,m}(x_3)\mathcal{M}_{n,m}(\rho)\\
    &&+\left(
      \frac{e_j}{m_jc}\frac{\partial F_0}{\partial v_\|}
      -\frac{e_jv_\|}{cB_0}\frac{\partial F_0}{\partial \mu}
    \right)
    \sum_n A_{\|,n,m}(x_3)\left(
      \Lambda_n(x_1)-\mathcal{P}_{n,m}(\rho)
    \right)\\
    &&+\frac{e_j}{B_0}\frac{\partial F_0}{\partial\mu}
    \sum_n\Phi_{n,m}(x_3)\left(
      \Lambda_n(x_1)-\mathcal{P}_{n,m}(\rho)
    \right)
  \Bigg]\,dv_\|\,d\mu
\end{eqnarray*}

\subsubsection{Normalization}
Using the normalization described in Section~\ref{sec:normalization}
\begin{eqnarray*}
  F_{0j}(x^1,x^3,v_\|,\mu) &=& \frac{\refj{n}}{v_{0j}^3}\hat{F}_{0j}(x^1,x^3,v_\|,\mu)\\
  F_{n,m}(x^3,v_\|,\mu) &=&
  \frac{\refj{n}}{v_{0j}^3}\frac{\rho_\rf}{L_\rf}\hat{F}_{n,m}(x^3,v_\|,\mu)\\
  A_{\|,n,m}(x^3) &=& \rho_\rf B_\rf \frac{\rho_\rf}{L_\rf}\hat{A}_{\|,n,m}(x^3)\\
  \Phi_{n,m}(x^3) &=& \frac{\rho_\rf}{L_\rf} \frac{T_\rf}{e} \hat{\Phi}_{n,m}(x^3)
\end{eqnarray*}
we arrive at
\begin{eqnarray}
  M_{kr}(x_1,k_m,x_3) &=& 2\pi\frac{2^qB_0^{q+1}}{m_j^{q+1}} v_{0j}^{k+1}
  \frac{\refj{T}^{q+1}}{B_\rf^{q+1}}
  \int \hat{v}_\|^k \hat{\mu}^q \sum_n \Bigg[
     \frac{\refj{n}}{v_{0j}^3} \frac{\rho_\rf}{L_\rf}\hat{F}_{n,m}(x_3)\mathcal{M}_{n,m}(\rho) \nn \\
    &&+\frac{e_j}{m_jc} \frac{\refj{n}}{v_{0j}^4} \left(
      \frac{\partial \hat{F}_0}{\partial \hat{v}_\|}
      -\frac{m_j v_{0j}^2}{B_\rf}\frac{B_\rf}{\refj{T}} \frac{\hat{v}_\|}{\hat{B}}\frac{\partial 
        \hat{F}_0}{\partial \hat{\mu}}
    \right)
    B_\rf \frac{\rho_\rf^2}{L_\rf} \hat{A}_{\|,n,m}(x_3)\left(
      \Lambda_n(x_1)-\mathcal{P}_{n,m}(\rho) \right) \nn \\
    &&+\frac{e_j}{B_0} \frac{\refj{n}}{v_{0j}^3}\frac{B_\rf}{\refj{T}} \frac{\rho_\rf}{L_\rf} \frac{T_\rf}{e}
    \frac{\partial\hat{F}_0}{\partial\hat{\mu}}
    \hat{\Phi}_{n,m}(x_3)\left(
      \Lambda_n(x_1)-\mathcal{P}_{n,m}(\rho)
    \right)
  \Bigg]\,d\hat{v}_\|\,d\hat{\mu} \nn \\
& = & \refj{n} \frac{\rho_\rf}{L_\rf} v_{0j}^{2q+k} \pi\hat{B}^{q+1} 
   \int \hat{v}_\|^k \hat{\mu}^q \sum_n \Bigg[
     \hat{F}_{n,m}(x_3)\mathcal{M}_{n,m}(\rho)\nn \\
     &&+\frac{\hat{e}_j c_\rf}{\hat{m}_j v_{0j}} \left(
       \frac{\partial \hat{F}_0}{\partial \hat{v}_\|}
       -2\frac{\hat{v}_\|}{\hat{B}}\frac{\partial\hat{F}_0}{\partial \hat{\mu}}
     \right)
     \hat{A}_{\|,n,m}(x_3)\left(\Lambda_n(x_1)-\mathcal{P}_{n,m}(\rho)
     \right) \nn \\
     &&+\frac{\hat{e}_j}{\hat{B}}\frac{T_\rf}{\refj{T}}
     \frac{\partial\hat{F}_0}{\partial\hat{\mu}}
     \hat{\Phi}_{n,m}(x_3)\left(
       \Lambda_n(x_1)-\mathcal{P}_{n,m}(\rho)
     \right)
   \Bigg]\,d\hat{v}_\|\,d\hat{\mu} \nn \\
& \equiv & \left\{\refj{n} \frac{\rho_\rf}{L_\rf} v_{0j}^{2q+k}\right\} \cdot \hat{M}_{kr} \label{eq:mom_norm}
\end{eqnarray}
 
\subsubsection{Symmetric Maxwellian as equilibrium distribution function}
The moment calculation can be further simplified by assuming an unshifted maxwellian
as equilibrium distribution function, e.g. 
$F_{0j}=\pi^{-\frac{3}{2}} \frac{n_{0j}}{v_{T_j}^3}{\rm e}^{-\frac{m_jv_\parallel^2/2+\mu B_0}{T_{0j}}}$.
Now, Eq.~(\ref{eq:mom_norm}) reads
\begin{eqnarray}
\hat{M}_{kr} & = & \pi\hat{B}^{q+1}
   \int \hat{v}_\|^k \hat{\mu}^q \sum_n \Bigg[
     \hat{F}_{n,m}(x_3)\mathcal{M}_{n,m}(\rho)\nn \\
      &&-\frac{\hat{e}_j}{\Tpj}\frac{T_\rf}{\refj{T}}\hat{F}_0
     \hat{\Phi}_{n,m}(x_3)\left(
       \Lambda_n(x_1)-\mathcal{P}_{n,m}(\rho)
     \right)
   \Bigg]\,d\hat{v}_\|\,d\hat{\mu}. \nn
\end{eqnarray}
The second term can be further evaluated using 
\begin{eqnarray*}
\Upsilon(k)=\frac{1}{\sqrt{\pi}}\int_{-\infty}^{\infty}x^k {\rm e}^{-x^2} dx =
\begin{cases} 
0, & k \mbox{ odd} \\ 
1, & k=0 \\
\frac{1\cdot3\cdots(k-1)}{\sqrt{2}^k} & k \mbox{ even}
\end{cases}
\end{eqnarray*}
to calculate the $v_\parallel$-integral analytically
\begin{eqnarray*}
\int \hat{v}^k_\| \hat{F}_0(\hat{v}_\|)\,d\hat{v}_\|
& = & \frac{{\rm e}^{-\hat{\mu}\frac{\hat{B}}{\hat{T}_{0j}}}}
  {(\pi \Tpj)^{\frac{3}{2}}} \npj \Tpj^{\frac{k}{2}}
  \int_{-\infty}^{\infty}\frac{\hat{v}^k_\|}{\Tpj^{\frac{k}{2}}} {\rm e}^{-\frac{\hat{v}^2_\|}
  {\Tpj}}\,d\hat{v}_\| \nn \\
& = & \frac{1}{\pi} {\rm e}^{-\hat{\mu}\frac{\hat{B}}{\Tpj}} 
  \npj \Tpj^{\frac{k}{2}-1} \Upsilon(k).
\end{eqnarray*}
We now have
\begin{eqnarray}
\hat{M}_{kr} & = & \pi\hat{B}^{q+1}  \sum_n \Bigg[
   \int \hat{v}_\|^k \hat{\mu}^q
     \hat{F}_{n,m}(x_3)\mathcal{M}_{n,m}(\rho)\,d\hat{v}_\|\,d\hat{\mu} \nn \\
      &&-\frac{\hat{e}_j}{\pi} \npj \Tpj^{\frac{k}{2}-2}\frac{T_\rf}{\refj{T}} 
      \Upsilon(k) \hat{\Phi}_{n,m}(x_3) 
      \int \hat{\mu}^q {\rm e}^{-\hat{\mu}\frac{\hat{B}}{\hat{T}_{0j}}} 
     \left(\Lambda_n(x_1)-\mathcal{P}_{n,m}(\rho)
     \right)\,d\hat{\mu}\Bigg] \nn
\end{eqnarray}
which can be simplified to
\begin{eqnarray}
\hat{M}_{kr} & = & \pi\hat{B}^{q+1}  \sum_n \Bigg[
   \int \hat{v}_\|^k \hat{\mu}^q
     \hat{F}_{n,m}(x_3)\mathcal{M}_{n,m}(\rho)\,d\hat{v}_\|\,d\hat{\mu} \nn \\
      &&-\frac{\hat{e}_j}{\pi}\frac{T_\rf}{\refj{T}} \npj\Tpj^{\frac{k}{2}-2} \Upsilon(k) q! 
      \left(\frac{\Tpj}{\hat{B}}\right)^{q+1}
      \hat{\Phi}_{n,m}(x_3) \Lambda_n(x_1)\\
      &&+\frac{\hat{e}_j}{\pi}\frac{T_\rf}{\refj{T}} \npj\Tpj^{\frac{k}{2}-2} 
      \Upsilon(k) \hat{\Phi}_{n,m}(x_3) 
      \int \hat{\mu}^q {\rm e}^{-\hat{\mu}\frac{\hat{B}}{\Tpj}} 
     \mathcal{P}_{n,m}(\rho)\,d\hat{\mu}\Bigg]. \nn
\end{eqnarray}
Changing to matrix form yields
\begin{eqnarray*}
\hat{\mathbf{M}}_{kr}(k_m,x_3) &=& \pi\,\diag\{\hat{B}^{q+1}(x_1,x^3),\ldots,
   \hat{B}(x_N,x^3)\}\cdot\\
&& \int \hat{v}_\|^k \hat{\mu}^q
   \hat{\mathcal{M}}(k_{2,m},x^3,\hat{\mu})\cdot
   \hat{\mathbf{F}}(k_{2,m},x^3,\hat{v}_\|,\hat{\mu})\,d\hat{v}_\|\,d\hat{\mu} \\
&& -\hat{e}_j\frac{T_\rf}{\refj{T}} q!\Upsilon(k)\, 
   \diag\{\npj(x_1)\Tpj^{q+\frac{k}{2}-1}(x_1),\ldots,\npj(x_N)\Tpj^{q+\frac{k}{2}-1}(x_N)\}\cdot
   \mathcal{L} \cdot\hat{\mathbf{\Phi}}(k_{2,m},x^3)\\
&& +\hat{e}_j\frac{T_\rf}{\refj{T}}\Upsilon(k) \int \hat{\mu}^q 
   \diag\left\{\npj(x_1)\Tpj^{\frac{k}{2}-2}(x_1)\hat{B}^{q+1}(x_1,x^3)
   \exp{\left(-\hat{\mu}\frac{\hat{B}(x_1,x^3)}{\Tpj(x_1)}\right)},\ldots,\right.\\
&& \phantom{e_j\int}\left.\npj(x_N)\Tpj^{\frac{k}{2}-2}(x_N)
   \hat{B}^{q+1}(x_N,x^3)
      \exp{\left(-\hat{\mu}\frac{\hat{B}(x_N,x^3)}{\Tpj(x_N)}\right)}\right\}\cdot
   \mathcal{P}(k_{2,m},x^3,\hat{\mu})\cdot\\
&& \phantom{e_j\int} \hat{\mathbf{\Phi}}(k_{2,m},x^3)\,d\hat{\mu}
\end{eqnarray*}
with $\mathcal{L}_{rq}=\Lambda_q(x_r)$. Using also the abbreviation
$\hat{\mathcal{B}}(x^3)=\diag\{\hat{B}(x_1,x^3),\ldots,\hat{B}(x_N,x^3)\}$
we can write the equation in the compact form
\begin{eqnarray*}
\hat{\mathbf{M}}_{kr}(k_m,x_3)  &=&  \pi\hat{\mathcal{B}}(x^3)
   \cdot \int \hat{v}_\|^k \hat{\mu}^q
   \hat{\mathcal{M}}(k_{2,m},x^3,\hat{\mu})\cdot
   \hat{\mathbf{F}}(k_{2,m},x^3,\hat{v}_\|,\hat{\mu})\,d\hat{v}_\|\,d\hat{\mu} \\
&& -\hat{e}_j\frac{T_\rf}{\refj{T}} \Upsilon(k) q!\, \diag\{\npj\Tpj^{q+\frac{k}{2}-1}\}\cdot
   \mathcal{L} \cdot\hat{\mathbf{\Phi}}(k_{2,m},x^3)\\
&& +\hat{e}_j\frac{T_\rf}{\refj{T}} \Upsilon(k) \int \hat{\mu}^q 
   \diag\{\npj\Tpj^{\frac{k}{2}-2}\hat{B}^{q+1}(x^3)\exp{\{-\hat{\mu}
       \frac{\hat{B}(x^3)}{\Tpj}\}}\}\cdot
   \mathcal{P}(k_{2,m},x^3,\hat{\mu})\,d\hat{\mu}\\
&& \cdot\hat{\mathbf{\Phi}}(k_{2,m},x^3)
\end{eqnarray*}

\subsubsection{Properties of the (0,0)-moment}
If one calculates the following quantity
\begin{displaymath}
  \sum_j\frac{e_j}{e} \mathbf{M}_{00}(k_m,x_3)
\end{displaymath}
we find this quantity to be zero. This means the density of
electrons and ions in a two species run are equal. This should also be
reflected in the \texttt{nrg} file and is a good further check of the
calculation.
\begin{eqnarray*}
   \sum_j\hat{e}_j \mathbf{M}_{00}(k_m,x_3)
   &=& \sum_j\hat{e}_j
   \pi\hat{\mathcal{B}}(x^3)\cdot \int\mathcal{M}(k_{2,m},x^3,\mu)\cdot\mathbf{F}(k_{2,m},x^3,v_\|,\mu)\,dv_\|\,d\mu\\
   && -\sum_j\hat{e}_j^2\frac{T_\rf}{\refj{T}}\diag\{\frac{\npj}{\Tpj}\}\cdot\mathcal{L}\cdot\mathbf{\Phi}(k_{2,m},x^3)\\
   &&+\sum_j\hat{e}_j^2\frac{T_\rf}{\refj{T}}\int
   \diag\{\frac{\npj}{\Tpj^2}\hat{B}(x^3)\exp\{-\frac{\mu \hat{B}}{\Tpj}\}\}
   \cdot\mathcal{P}(k_{2,m},x^3,\mu)\,d\mu\,\cdot\mathbf{\Phi}(k_{2,m},x^3)
\end{eqnarray*}
Using the quasineutrality equation, we can replace the first summand
by $\mathcal{P}''(k_m,x_3)\mathbf{\Phi}_m(x_3)$ and get then
\begin{multline*}
  \sum_j\hat{e}_j \mathbf{M}_{00}(k_m,x_3)
  = \mathcal{P}''(k_m,x_3)\mathbf{\Phi}_m(x_3)
  -\sum_j\hat{e}_j^2\frac{T_\rf}{\refj{T}}\diag\{\frac{\npj}{\Tpj}\}\\\cdot\left(\mathcal{L}
    -\int\diag\{\frac{\hat{B}(x^3)}{\Tpj}\exp\{-\frac{\mu \hat{B}}{\Tpj}\}\}
    \cdot\mathcal{P}(k_{2,m},x^3,\mu)\,d\mu
  \right)\cdot\mathbf{\Phi}(k_{2,m},x^3)
\end{multline*}
The second summand is exactly the definition of $\mathcal{P}''$. So in
the end the so calculated sum is equal to zero.



\section{Transport fluxes}
\label{sec:transflux_global}

\subsection{{\bf ExB} velocity}
The ${\bf E_\chi\times B}$ velocity is needed in every flux calculation. 
Therefore, we first take a closer look at this quantity. The definition is 
\begin{displaymath}
\mathbf{v}_{\bf E_\chi\times B} = c\frac{\mathbf{B}_0\times\nabla\chi}{B_0^2}
\end{displaymath}
with $\chi = \Phi(\mathbf{x})-\frac{1}{c}v_\parallel A_\parallel(\mathbf{x})$ 
(neglecting $B_\parallel$ fluctuations). Since we are only interested in
radial fluxes we consider the projection on the {\em contravariant} basis 
vector
\begin{eqnarray*}
v_E & \equiv & \mathbf{v}_{\bf E_\chi\times B}\cdot \mathbf{e}^1 \\
& = & \frac{c}{B_3 B^3}\frac{\varepsilon^{ij1}}{J} E_{\chi,i} B_j.
\end{eqnarray*}
Using $J=B_{\rm ref}/B^3$, $E_{\chi,i}=-\partial\chi/\partial u^i$ and $B_j=B_3\delta_{3j}$
yields
\begin{eqnarray*}
v_E & \equiv & -\frac{c}{B_3 B_\rf} \varepsilon^{231} \frac{\partial\chi}{\partial y} B_3\\
 & = & - \frac{c}{B_\rf} \frac{\partial\chi }{\partial y}
\end{eqnarray*}



\subsection{Particle flux}
\label{sec:partflux}

The aim of this subsection is to derive an expression for the particle
flux in a global simulation. We start as usual from
\begin{displaymath}
  \Gamma(\mathbf{x})=\int v_E(\mathbf{x})f_1(\mathbf{x},\mathbf{v})\,d^3v
\end{displaymath}

\subsubsection{Electrostatic particle flux}
First we are only interested in the electrostatic particle flux, so we
use only the electrostatic contribution to the radial component of the
$E\times B$ velocity.
\begin{displaymath}
  \Gamma_\mathrm{es}(\mathbf{x})=-\frac{c}{B_\rf}\frac{\partial\Phi(\mathbf{x})}{\partial y} 
  \int f_1(\mathbf{x},\mathbf{v})\,d^3v
  =-\frac{c}{B_\rf}\frac{\partial\Phi(\mathbf{x})}{\partial y} M_{00}(\mathbf{x})
\end{displaymath}
Normalization leads to 
\begin{displaymath}
  \Gamma_\mathrm{es}(\mathbf{x})
  =-\frac{c}{B_\rf\rho_\rf}\frac{T_\rf}{e}\frac{\rho_\rf}{L_\rf}
  \frac{\partial\hat{\Phi}(\mathbf{x})}{\partial \hat{y}} \refj{n}\frac{\rho_\rf}{L_\rf}
   \hat{M}_{00}(\mathbf{x})
  =-\frac{T_\rf}{m_\rf c_\rf}\refj{n}\frac{\rho_\rf^2}{L_\rf^2}
  \frac{\partial\hat{\Phi}(\mathbf{x})}{\partial \hat{y}} \hat{M}_{00}(\mathbf{x})
\end{displaymath}
Normalizing the flux to units of $n_\rf c_\rf \rho_\rf^2/L_\rf^2$ we arrive
at
\begin{displaymath}
  \hat{\Gamma}_\mathrm{es}(x^1,x^2,x^3)
  =-\frac{\refj{n}}{n_\rf}\frac{\partial\hat{\Phi}(x^1,x^2,x^3)}{\partial x^2} \hat{M}_{00}(x^1,x^2,x^3)
\end{displaymath}
What we have in the code are the Fourier transformed quantities in
$x^2$ direction, so they must come into play somewhere. Representing
the right hand side quantities by the associated Fourier series we
come to
\begin{eqnarray*}
\sum_q\Gamma_{\mathrm{es},q}(x^1,x^3)e^{ik_qx^2}
 & = & -\left(\sum_m\Phi_m(x^1,x^3)ik_m e^{ik_mx^2}\right)
   \left(\sum_r M_{00,r}(x^1,x^3)e^{ik_rx^2}\right)\\
 & = & -\sum_m\sum_rik_m\Phi_m(x^1,x^3)M_{00,r}(x^1,x^3) e^{i(k_m+k_r)x^2}
\end{eqnarray*}
For the contributions of the different wave numbers to the total flux
we can write
\begin{displaymath}
  \Gamma_{\mathrm{es},q}(x^1,x^3)
  =-\sum_m ik_m\Phi_m(x^1,x^3)M_{00,q-m}(x^1,x^3)
\end{displaymath}
For the \texttt{diag\_nrg} routine, we want to calculate the average flux
in the simulation volume
$\Gamma_\mathrm{es} = \frac{1}{V} \int \Gamma_\mathrm{es}(x^1,x^2,x^3)J(x^3)\,dx^1\,dx^2\,dx^3$
where the volume $V=\int J(x^3)\,dx^1\,dx^2\,dx^3\equiv L_x L_y \int J(x^3)\,dx^3$.
This is done as usual as the sum over all
contributions. As we are also averaging in $y$ direction, this means
that we are only taking into account the $k_q=0$ contribution to $\Gamma_\mathrm{es}$.
\begin{eqnarray*}
  \Gamma_\mathrm{es} &=& 
  \int \Gamma_{\mathrm{es},0}\,dx^1\,dx^3\\
  &=& -\frac{1}{L_x}\frac{1}{\int J(x^3)\,dx^3}
  \sum_{m=-N/2}^{N/2} ik_m \int\Phi_m(x^1,x^3)  M_{00,-m}(x^1,x^3) \,dx^1\,dx^3\\
  &=& -\frac{1}{L_x}\frac{1}{\int J(x^3)\,dx^3}\int
  \sum_{m=1}^{N/2} 2\Re\{ik_m\Phi_m(x^1,x^3)  M_{00,m}^*(x^1,x^3)\} \,dx^1\,dx^3
\end{eqnarray*}

\subsubsection{Check for ambipolarity of the electrostatic particle flux}
We check now for ambipolarity of the electrostatic particle flux,
which is important to be fulfilled. To get the right result, I have to
use at some point the quasineutrality equation from a previous
section. We start by writing the total flux for a given wave number
$k_q$ as
\begin{eqnarray*}
  \sum_j\frac{e_j}{e}\Gamma_{\mathrm{es},q} 
  &=&-\frac{T_0}{m_0v_0^2}\sum_j\frac{e_j}{e}\frac{1}{\hat{B}(x^1,x^3)}\sum_m ik_m\Phi_m(x^1,x^3)M_{00,q-m}(x^1,x^3)
\end{eqnarray*}
Written in matrix vector form
\begin{eqnarray*}
  \sum_j\frac{e_j}{e}\mathbf{\Gamma}_{\mathrm{es},q} 
  &=&-\frac{T_0}{m_0v_0^2}\sum_j\frac{e_j}{e}\frac{1}{\hat{B}(x^1,x^3)}\sum_m ik_m\diag\{\Phi_m\}\mathbf{M}_{00,q-m}(x^1,x^3)
\end{eqnarray*}
Using the above expression for the first moment.
\begin{eqnarray*}
  \sum_j\frac{e_j}{e}\mathbf{\Gamma}_{\mathrm{es},q} 
  &=&-\frac{T_0}{m_0v_0^2}\sum_j\frac{e_j}{e}\frac{1}{\hat{B}(x^1,x^3)}\\
  &&\sum_m ik_m\diag\{\Phi_m\}\pi\hat{\mathcal{B}}(x^3)\cdot \int \Bigg[
  \mathcal{M}(k_{2,q-m},x^3,\mu)\cdot\mathbf{F}(k_{2,q-m},x^3,v_\|,\mu)\\
  &&-\frac{e_j}{e}\frac{T_0}{T_{0j}}\mathcal{F}_0(x^3,v_\|,\mu)\cdot
  \left(
    \mathcal{L}-\mathcal{P}(k_{2,q-m},x^3,\mu)
  \right)\cdot\mathbf{\Phi}(k_{2,q-m},x^3)
  \Bigg]\,dv_\|\,d\mu\\
  &=&-\frac{T_0}{m_0v_0^2}\sum_j\frac{e_j}{e}\frac{1}{\hat{B}(x^1,x^3)}\\
  &&\sum_m ik_m\diag\{\Phi_m\} \Bigg[
  \pi\hat{\mathcal{B}}(x^3)\cdot \int\mathcal{M}(k_{2,q-m},x^3,\mu)\cdot\mathbf{F}(k_{2,q-m},x^3,v_\|,\mu)\,dv_\|\,d\mu\\
  &&-\pi\hat{\mathcal{B}}(x^3)\cdot \int\frac{e_j}{e}\frac{T_0}{T_{0j}}\mathcal{F}_0(x^3,v_\|,\mu)\cdot
  \left(
    \mathcal{L}-\mathcal{P}(k_{2,q-m},x^3,\mu)
  \right)\cdot\mathbf{\Phi}(k_{2,q-m},x^3)
  \,dv_\|\,d\mu\Bigg]\\
  &=&-\frac{T_0}{m_0v_0^2}\frac{1}{\hat{B}(x^1,x^3)}\\
  &&\sum_m ik_m\diag\{\Phi_m\} \Bigg[
  \sum_j\frac{e_j}{e}\pi\hat{\mathcal{B}}(x^3)\cdot \int\mathcal{M}(k_{2,q-m},x^3,\mu)\cdot\mathbf{F}(k_{2,q-m},x^3,v_\|,\mu)\,dv_\|\,d\mu\\
  &&-\sum_j\frac{e_j^2}{e^2}\frac{T_0}{T_{0j}}\pi\hat{\mathcal{B}}(x^3)\cdot \int\mathcal{F}_0(x^3,v_\|,\mu)\cdot
  \left(
    \mathcal{L}-\mathcal{P}(k_{2,q-m},x^3,\mu)
  \right)\cdot\mathbf{\Phi}(k_{2,q-m},x^3)
  \,dv_\|\,d\mu\Bigg]\\
  &=&-\frac{T_0}{m_0v_0^2}\frac{1}{\hat{B}(x^1,x^3)}\\
  &&\sum_m ik_m\diag\{\Phi_m\} \Bigg[
  \sum_j\frac{e_j}{e}\pi\hat{\mathcal{B}}(x^3)\cdot \int\mathcal{M}(k_{2,q-m},x^3,\mu)\cdot\mathbf{F}(k_{2,q-m},x^3,v_\|,\mu)\,dv_\|\,d\mu\\
  &&-\sum_j\frac{e_j^2}{e^2}\frac{T_0}{T_{0j}}\pi\hat{\mathcal{B}}(x^3)\cdot \int
  \diag\{\frac{n_{0j}}{n_0}\frac{v_{0j}^3}{v_{Tj}^3}\pi^{-3/2}\exp\{-v_\|^2v_{0j}^2/v_{Tj}^2-\mu T_0\hat{B}/T_{0j}\}\}\cdot\\
  &&\left(
    \mathcal{L}-\mathcal{P}(k_{2,q-m},x^3,\mu)
  \right)\cdot\mathbf{\Phi}(k_{2,q-m},x^3)
  \,dv_\|\,d\mu\Bigg]
\end{eqnarray*}
Doing the $v_\|$ integration
\begin{eqnarray*}
  \sum_j\frac{e_j}{e}\mathbf{\Gamma}_{\mathrm{es},q} 
  &=&-\frac{T_0}{m_0v_0^2}\frac{1}{\hat{B}(x^1,x^3)}\\
  &&\sum_m ik_m\diag\{\Phi_m\} \Bigg[
  \sum_j\frac{e_j}{e}\pi\hat{\mathcal{B}}(x^3)\cdot \int\mathcal{M}(k_{2,q-m},x^3,\mu)\cdot\mathbf{F}(k_{2,q-m},x^3,v_\|,\mu)\,dv_\|\,d\mu\\
  &&-\int \sum_j\frac{e_j^2}{e^2}
  \cdot\diag\{\frac{T_0}{T_{0j}}\hat{B}\frac{n_{0j}}{n_0}\frac{T_0}{T_{0j}}\exp\{-\mu T_0\hat{B}/T_{0j}\}\}
  \cdot\mathcal{L}\cdot\mathbf{\Phi}(k_{2,q-m},x^3)\,d\mu\\
  &&+\int\sum_j\frac{e_j^2}{e^2}
  \cdot\diag\{\frac{T_0}{T_{0j}}\hat{B}\frac{n_{0j}}{n_0}\frac{T_0}{T_{0j}}\exp\{-\frac{\mu T_0\hat{B}}{T_{0j}}\}\}
  \cdot\mathcal{P}(k_{2,q-m},x^3,\mu)\cdot\mathbf{\Phi}(k_{2,q-m},x^3) \,d\mu
  \Bigg]\\
  &=&-\frac{T_0}{m_0v_0^2}\frac{1}{\hat{B}(x^1,x^3)}\\
  &&\sum_m ik_m\diag\{\Phi_m\} \Bigg[
  \sum_j\frac{e_j}{e}\pi\hat{\mathcal{B}}(x^3)\cdot \int\mathcal{M}(k_{2,q-m},x^3,\mu)\cdot\mathbf{F}(k_{2,q-m},x^3,v_\|,\mu)\,dv_\|\,d\mu\\
  &&-\sum_j\frac{e_j^2}{e^2}
  \cdot\diag\{\frac{T_0}{T_{0j}}\frac{n_{0j}}{n_0}\}\}
  \cdot\mathcal{L}\cdot\mathbf{\Phi}(k_{2,q-m},x^3)\\
  &&+\int\sum_j\frac{e_j^2}{e^2}
  \cdot\diag\{\frac{T_0}{T_{0j}}\hat{B}\frac{n_{0j}}{n_0}\frac{T_0}{T_{0j}}\exp\{-\frac{\mu T_0\hat{B}}{T_{0j}}\}\}
  \cdot\mathcal{P}(k_{2,q-m},x^3,\mu)\cdot\mathbf{\Phi}(k_{2,q-m},x^3) \,d\mu
  \Bigg]
\end{eqnarray*}
If one uses the quasineutrality equation, we can show that the total
flux is equal to zero. So ambipolarity is fulfilled, but only if the
methods to calculate the fluxes and to solve the field equations are
identical.

\subsubsection{Electromagnetic particle flux}
Now we consider the contribution of the electromagnetic part of the $E_\chi\times B$ velocity.
Neglecting $B_\|$-fluctuations we get
\begin{eqnarray*}
\Gamma_\mathrm{em}(\mathbf{x})&=&\frac{1}{B_\rf}\frac{\partial A_\|(\mathbf{x})}{\partial y}
  \int v_\| \, f_1(\mathbf{x},\mathbf{v})\,d^3v\\
&=&\frac{1}{B_\rf}\frac{\partial A_\|(\mathbf{x})}{\partial y} M_{10}(\mathbf{x})\\
&=&\frac{1}{B_\rf}\frac{B_\rf\rho_\rf}{L_\rf}\frac{\partial \hat{A}_\|(\mathbf{x})}{\partial \hat{y}}
 \refj{n} \frac{\rho_\rf}{L_\rf} v_{0j} \hat{M}_{10}(\mathbf{x}).
\end{eqnarray*}
The particle flux shall be normalized (species independent) to 
$\frac{c_\rf \rho_\rf^2}{L_\rf}\frac{n_\rf}{L_\rf}$. Therefore,
\begin{equation}
\hat{\Gamma}_\mathrm{em}(\mathbf{x})=\frac{\refj{n}}{n_\rf}
 \sqrt{\frac{2\refj{T}/T_\rf}{\hat{m}_j}}\hat{M}_{10}(\mathbf{x})
\label{eq:Gem_norm}
\end{equation}


\subsection{Heat flux}
\label{sec:heatflux}
The heat flux is defined as
\begin{equation}
Q(\mathbf{x}) = \int \frac{1}{2}m_jv^2 v_E(\mathbf{x})f_1(\mathbf{x},\mathbf{v})\,d^3v
\label{eq:Q_basic}
\end{equation}

\subsubsection{Electrostatic heat flux}
First we again consider only the electrostatic contribution
\begin{eqnarray*}
Q_\mathrm{es}(\mathbf{x})&=&-\frac{m_jc}{2B_\rf} \frac{\partial\Phi(\mathbf{x})}{\partial y} 
 \frac{B_0}{m_j} \int \left(v_\|^2+\frac{2\mu B_0}{m_j}\right) 
 f_1(\mathbf{x},\mathbf{v})\,dv_\| \,d\mu \,d\theta\\
&=& -\frac{m_jc}{2B_\rf} \frac{\partial\Phi(\mathbf{x})}{\partial y} 
 \left(M_{20}(\mathbf{x})+M_{02}(\mathbf{x})\right)
\end{eqnarray*}
Normalizing as described in Section~\ref{sec:normalization} and Eq.~\ref{eq:mom_norm} 
yields
\begin{eqnarray}
Q_\mathrm{es}(\mathbf{x})
&=& -\frac{1}{2}\frac{m_j c}{e B_\rf}\frac{T_\rf}{\rho_\rf}\frac{\rho_\rf}{L_\rf}
 \frac{\partial\hat{\Phi}(\mathbf{x})}{\partial \hat{y}}
 \refj{n} \frac{\rho_\rf}{L_\rf} v_{0j}^2 \left(\hat{M}_{20}(\mathbf{x})+
 \hat{M}_{02}(\mathbf{x})\right)\nn \\
&=& - \frac{c_\rf \rho_\rf^2}{L_\rf}\frac{p_\rf}{L_\rf} 
 \frac{\refj{n}}{n_\rf}\frac{\refj{T}}{T_\rf}\left(\hat{M}_{20}(\mathbf{x})+
 \hat{M}_{02}(\mathbf{x})\right)\nn \\
&\equiv & \frac{c_\rf \rho_\rf^2}{L_\rf}\frac{p_\rf}{L_\rf} \hat{Q}_\mathrm{es}(\mathbf{x})
\label{eq:Qes_norm}
\end{eqnarray}

\subsubsection{Electromagnetic heat flux}
The electromagnetic heat flux (without $B_\|$ fluctuations) is given by
\begin{eqnarray*}
Q_\mathrm{em}(\mathbf{x})&=&\frac{m_j}{2B_\rf}\frac{\partial A_\| (\mathbf{x})}{\partial y} 
 \int \left(v_\|^2+\frac{2\mu B_0}{m_j}\right)\, v_\| \,
 f_1(\mathbf{x},\mathbf{v})\,dv_\| \,d\mu \,d\theta\\
&=& \frac{m_j}{2B_\rf}\frac{\partial A_\| (\mathbf{x})}{\partial y}
 \left( M_{30}(\mathbf{x}) + M_{12}(\mathbf{x})\right).
\end{eqnarray*}
Using the same normalization as for $Q_{\rm es}$, 
namely $\frac{c_\rf \rho_\rf^2}{L_\rf}\frac{p_\rf}{L_\rf}$, we define
\begin{equation}
\hat{Q}_\mathrm{em}(\mathbf{x}) = \frac{\refj{n}}{n_\rf}\frac{\refj{T}}{T_\rf}
  \sqrt{\frac{2\refj{T}/T_\rf}{m_j}}
  \left( \hat{M}_{30}(\mathbf{x}) + \hat{M}_{12}(\mathbf{x})\right)
\end{equation}


\section{Further quantities of interest}
\label{sec:moremom}
Besides the transport fluxes {\sc GENE} calculates and writes 
further quantities like e.g. $T_\|$, $T_\perp$, $u_\|$, etc.
which shall be shortly derived in the following.

\subsection{Density}
The density is defined as
\begin{eqnarray*}
n_j(\mathbf{x}) & = & \int f_1(\mathbf{x},\mathbf{v})\,d^3v\\
&=& M_{00}(\mathbf{x})
\end{eqnarray*}
At the moment, a density normalization of $\refj{n} \frac{\rho_\rf}{L_\rf}$
is used in {\sc GENE} (maybe we later on want $n_\rf \frac{\rho_\rf}{L_\rf}$?).
Thus
\begin{eqnarray*}
\hat{n}_j(\mathbf{x}) & = & \hat{M}_{00}(\mathbf{x}).
\end{eqnarray*}

\subsection{Parallel velocity moment}
The definition is
\begin{eqnarray*}
n_{0j}(\mathbf{x}) u_\| (\mathbf{x}) &=& \int v_\| f_1(\mathbf{x},\mathbf{v})\,d^3v 
 =M_{10}(\mathbf{x}).
\end{eqnarray*}
Using the usual normalizations we get
\begin{eqnarray*}
u_\| (\mathbf{x}) &=& c_\rf\frac{\rho_\rf}{L_\rf} \frac{1}{\npj} 
  \sqrt{\frac{2\refj{T}/T_\rf}{\hat{m}_j}}\hat{M}_{10}(\mathbf{x}) \\
& \equiv & c_\rf\frac{\rho_\rf}{L_\rf} \hat{u}_\| (\mathbf{x}).
\end{eqnarray*}

\subsection{Parallel temperature}
For symmetric equilibrium distribution functions in velocity space where
$u_{\| 0} = 0$ we can simply define
\begin{eqnarray*}
n_{0j}(\mathbf{x}) T_{\| 1} (\mathbf{x}) &=& p_{\| 1}(\mathbf{x})
 -n_1(\mathbf{x})T_{\| 0}(\mathbf{x})\\
&=& m_j \int \left(v_\|-u_{\| 1}\right)^2 f_1(\mathbf{x},\mathbf{v})\,d^3v 
 -n_1(\mathbf{x})T_{\| 0}(\mathbf{x})\\
&\approx& m_j \int v_\|^2 f_1(\mathbf{x},\mathbf{v})\,d^3v 
 -n_1(\mathbf{x})T_{\| 0}(\mathbf{x})
\end{eqnarray*}
where squared and higher perturbed terms were neglected in the last step.
Using the moment abbreviations and identifying $T_{\| 0}=T_{0j}$ yields
\begin{eqnarray*}
n_{0j}(\mathbf{x}) T_{\| 1} (\mathbf{x}) &=& m_j M_{20} - M_{00}T_{0j}\\
&=& \refj{n} \frac{\rho_\rf}{L_\rf} \left(m_j \frac{2\refj{T}}{m_j} \hat{M}_{20}
    - \hat{M}_{00}T_{0j}\right).
\end{eqnarray*}
Currently, {\sc GENE} uses $\refj{T}\frac{\rho_\rf}{L_\rf}$ as temperature
normalization, therefore 
\begin{eqnarray*}
\hat{T}_{\| 1} (\mathbf{x}) &=& 
\frac{1}{\npj} \left(2\hat{M}_{20}- \Tpj\hat{M}_{00}\right).
\end{eqnarray*}

\subsection{Perpendicular temperature}
Starting from the definition
\begin{eqnarray*}
n_{0j}(\mathbf{x}) T_{\perp 1} (\mathbf{x}) &=& p_{\perp 1}(\mathbf{x})
 -n_1(\mathbf{x})T_{\perp 0}(\mathbf{x})\\
&=& \frac{m_j}{2} \int v_{\perp}^2 f_1(\mathbf{x},\mathbf{v})\,d^3v 
 -n_1(\mathbf{x})T_{\| 0}(\mathbf{x})
\end{eqnarray*}
the perpendicular temperature can be written as
\begin{eqnarray*}
n_{0j}(\mathbf{x}) T_{\perp 1} (\mathbf{x}) &=& \frac{m_j}{2} M_{02}-T_{0j} M_{00}
\end{eqnarray*}
and if normalized to $\refj{T}\frac{\rho_\rf}{L_\rf}$
\begin{eqnarray*}
\hat{T}_{\perp 1} (\mathbf{x}) &=& \frac{1}{\npj} \left(\hat{M}_{02}-\Tpj\hat{M}_{00}\right).
\end{eqnarray*}

\subsection{Parallel heat current density}
The parallel heat current density is defined by
\begin{eqnarray*}
q_{\|}(\mathbf{x}) &=& \frac{m_j}{2} \int \left(v_{\|}-u_{\|}\right)^3 
  f(\mathbf{x},\mathbf{v})\,d^3v.
\end{eqnarray*}
Keeping only linear perturbed terms and using $u_{\| 0}=0$ we arrive at
\begin{eqnarray*}
q_{\| 1}(\mathbf{x}) &=& \frac{m_j}{2} \int v_{\|}^3 f_1(\mathbf{x},\mathbf{v})\,d^3v
 - \frac{3m_j}{2} \int v_{\|}^2 f_0(\mathbf{x},\mathbf{v})\,d^3v\,u_{\| 1}\\
&=& \frac{m_j}{2} M_{30} - \frac{3}{2} m_j \pi^{-\frac{3}{2}}\frac{n_{0j}}{v_{T_j}^3}2\pi\frac{B_0}{m_j}
 \int_{-\infty}^{\infty} v_{\|}^2{\rm e}^{-\frac{v_\|^2}{v_{T_j}^2}}\,dv_\| 
 \int_{0}^{\infty} {\rm e}^{-\mu\frac{B_0}{T_{0j}}}d\mu\, u_{\| 1} \\
&=& \frac{m_j}{2} M_{30} - \frac{3}{2} \frac{n_{0j}}{v_{Tj}^3}\frac{2B_0}{\sqrt{\pi}} 
 \frac{\sqrt{\pi}v_{T_j}^3}{2} \frac{T_{0j}}{B_0}u_{\| 1}\\
&=& \frac{m_j}{2} M_{30} - \frac{3}{2} p_{0j} u_{\| 1}
\end{eqnarray*}
Inserting the normalizations yields
\begin{eqnarray*}
q_{\| 1}(\mathbf{x}) &=& \frac{m_j}{2} \refj{n} \frac{\rho_\rf}{L_\rf} v_{0j}^3 \hat{M}_{30} 
  - 3 p_{0j} c_\rf \frac{\rho_\rf}{L_\rf}\hat{u}_{\| 1}\\
&=& \refj{p} c_\rf \frac{\rho_\rf}{L_\rf} \left(\sqrt{\frac{2\refj{T}/T_\rf}{\hat{m}_j}}\hat{M}_{30} 
  - \frac{3}{2}\npj\Tpj\hat{u}_{\| 1}\right)
\end{eqnarray*}
{\bf Note:} For historical reasons {\sc GENE} does not write out $q_\|$ but 
$\tilde{M}_{30}=\sqrt{\frac{2\refj{T}/T_\rf}{\hat{m}_j}}\hat{M}_{30}$.

\subsection{Perpendicular heat current density}
The definition of the perpendicular heat current density is
\begin{eqnarray*}
q_{\perp}(\mathbf{x}) &=& \frac{m_j}{2} \int v_{\perp}^2 \left(v_{\|}-u_{\|}\right)^2
  f(\mathbf{x},\mathbf{v})\,d^3v.
\end{eqnarray*}
Neglecting all except for first order terms gives
\begin{eqnarray*}
q_{\perp 1}(\mathbf{x})&=& \frac{m_j}{2} \int v_{\perp}^2 v_{\|} f_1(\mathbf{x},\mathbf{v})\,d^3v-
 \frac{m_j}{2}u_{\| 1} \int v_{\perp}^2 f_0(\mathbf{x},\mathbf{v})\,d^3v\\
&=& \frac{m_j}{2} M_{12} - \frac{m_j}{2}u_{\| 1} \pi^{-\frac{3}{2}}\frac{n_{0j}}{v_{T_j}^3}2\pi\frac{2B_0^2}{m_j^2}
 \int_{-\infty}^{\infty} {\rm e}^{-\frac{v_\|^2}{v_{T_j}^2}}\,dv_\| 
 \int_{0}^{\infty} \mu \, {\rm e}^{-\mu\frac{B_0}{T_{0j}}}d\mu \\
&=& \frac{m_j}{2} M_{12} - \frac{m_j}{2}u_{\| 1} \pi^{-\frac{3}{2}}\frac{n_{0j}}{v_{T_j}^3}2\pi\frac{2B_0^2}{m_j^2}
 \sqrt{\pi} v_{T_j} \frac{T_{0j}^2}{B_0^2}\\
&=& \frac{m_j}{2} M_{12} - p_{0j} u_{\| 1}
\end{eqnarray*}
Using the same normalization as for $q_\|$, namely $\refj{p} c_\rf \frac{\rho_\rf}{L_\rf}$, we get
\begin{eqnarray*}
\hat{q}_{\perp}(\mathbf{x})&=& \sqrt{\frac{2\refj{T}/T_\rf}{\hat{m}_j}} \hat{M}_{12} - \npj\Tpj u_{\| 1}
\end{eqnarray*}
{\bf Note:} For historical reasons {\sc GENE} does not write out $q_\perp$ but 
$\tilde{M}_{12}=\sqrt{\frac{2\refj{T}/T_\rf}{\hat{m}_j}}\hat{M}_{12}$.

%%% Local Variables:
%%% mode: latex
%%% TeX-master: "globalgene"
%%% End:
